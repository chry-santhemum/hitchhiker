\documentclass[11pt]{amsart}
\usepackage{scorpius}
\usepackage{stmaryrd}
\usepackage{multicol}
\renewcommand{\theprb}{\arabic{prb}}

\title{Number Theory}
\author{Notes by Atticus Wang}
\date{18.785 was taught in Fall 2022 by Bjorn Poonen; 18.786 in Spring 2023 by Andrew Sutherland}

\begin{document}

\maketitle


\setcounter{tocdepth}{1}
\begin{multicols}{2}
\tableofcontents
\end{multicols}

\newpage

\section{Lecture, Sep. 7}

(Absolute values and equivalence relation, valuations, discrete valuation rings)

\begin{defn}
A (real-valued) \emph{absolute value} on a field $k$ is a map $|\ |: k \to \RR_{\ge 0}$ such that:
\begin{itemize}
    \item $|x| = 0 \iff x = 0$;
    \item $|xy| = |x||y|$;
    \item $|x+y| \le |x| + |y|$.
\end{itemize}
If the stronger condition that $|x+y| \le \max(|x|,|y|)$ is satisfied, the absolute value is called \emph{nonarchimedean}; otherwise it is \emph{archimedean}.
\end{defn}

Examples of absolute values:
\begin{itemize}
    \item The usual absolute value $|\ |$ on $\CC$, and the inherited absolute values on $\RR$, $\QQ$.
    \item The trivial absolute value on any field.
    \item The $p$-adic absolute value on $\QQ$: $|x|_p = p^{-v_p(x)}$.
\end{itemize}

An absolute value induces a metric on $k$, hence a topology. Two absolute values are \emph{equivalent} if they induce the same topology. 

\begin{prop}
Two absolute values $|\ |_1$ and $|\ |_2$ are equivalent if and only if $|\ |_2 = |\ |_1^s$ for some real $s>0$.
\end{prop}

\begin{proof}
Consider the image of the homomorphism $f: k^* \to \RR^2$ by $x \mapsto (\log |x|_1, \log |x|_2)$.

Case 1: the image does not intersect the second quadrant. Then it must be a subset of a line with positive slope, and therefore $|\ |_2 = |\ |_1^s$ for some positive $s$. Since these induce the same open balls, they have the same topology as well. So in this case both statements are true.

Case 2: the image intersects the second quadrant. Then there exists $x\in k$ such that $|x|_1 < 1$ and $|x|_2 > 1$. In this case, the sequence $x, x^2, x^3,\dots$ converge in the first topology but diverges in the second, so the two absolute values induce different topologies. So in this case both statements are false.
\end{proof}

Corollary: if two absolute values $|\ |_1$ and $|\ |_2$ on $k$ are inequivalent, then there exists $x\in k$ such that $|x|_1 < 1$ and $|x|_2 > 1$.

\begin{defn}
A \emph{valuation} on a field $k$ is a homomorphism $v:k^* \to \RR$ such that $v(x+y) \ge \min(v(x), v(y))$. We usually extend this to a map on the whole $k$ by $v(0) = +\infty$.
\end{defn}

If $v$ is a valuation, $c \in (0,1)$, then $|x| = c^{v(x)}$ is a nonarchimedean absolute value. The image of $v$ is called the \emph{value group}. Let $A = \{x\in k: v(x) \ge 0\}$, then $A$ is called a \emph{valuation ring}. If the valuation group is discrete (which is scaled to $\ZZ$), then $v$ is called a \emph{discrete valuation} and $A$ is a \emph{discrete valuation ring}.

The concept of (discrete) valuation rings can be defined more algebraically.

\begin{defn}
A \emph{valuation ring} $A$ is an integral domain such that for any $x\in K = \Frac(A)$, either $x\in A$ or $x^{-1} \in A$ (or both).
\end{defn}

\begin{prop}
Let $A$ be a valuation ring of $K = \Frac(A)$. Then:
\begin{itemize}
    \item $A$ is local, with the set of nonunits as its maximal ideal;
    \item $A$ is integrally closed.
\end{itemize}
\end{prop}

Let $U$ be the group of units in $A$, and let $\Gamma = K^{\times}/U$. Then the projection $v: K^{\times} \to \Gamma$ is a valuation: $\Gamma$ is a totally ordered abelian group under the relation $v(x) \ge v(y) \iff xy^{-1} \in A$. Conversely, given a valuation $v: K^{\times} \to \Gamma$ where $\Gamma$ is an arbitrary totally ordered abelian group, we can recover $A$ as the elements $x$ with $v(x) \ge 0$.

\begin{prop}
Let $A$ be a Noetherian local domain of dimension 1, with maximal ideal $\fm$ and residue field $k$. Then the following are equivalent:
\begin{itemize}
    \item $A$ is a DVR;
    \item $A$ is integrally closed;
    \item $\fm$ is a principal ideal;
    \item $\dim_k (\fm/\fm^2) = 1$;
    \item Every nonzero ideal is a power of $\fm$;
    \item There exists a \emph{uniformizer} $\pi$ such that every nonzero ideal is of form $(\pi^k)$.
\end{itemize}
\end{prop}


\section{Lecture, Sep. 9}

(Discrete valuation rings, regular local rings)

Examples of DVRs:
\begin{itemize}
    \item Consider $v_p: \QQ\to \ZZ \cup \{\infty\}$, then its valuation ring is $A = \ZZ_{(p)}$ ($\ZZ$ localized at $(p)$).
    \item Consider $v: k((t)) \to \ZZ\cup \{\infty\}$ mapping each formal Laurent series to the lowest degree whose coefficient is nonzero. Then $A = k[[t]]$.
    \item For a connected open $U\subseteq \CC$, let $\sM(U)$ be the field of meromorphic functions on $U$. For $V\subset U$ open, there is a restriction map $\sM(U) \to \sM(V)$ that is injective (because of analytic continuation). Let 
    \[\sM = \varinjlim_{U\ni 0} \sM(U).\]
    This is the field of germs of meromorphic functions at 0. Consider $v: \sM \to \ZZ\cup \{\infty\}$ mapping $f$ to the order of vanishing of $f$ at 0. Then $A$ is the ring of germs of holomorphic functions at 0.
\end{itemize}

DVRs are the simplest commutative rings after fields. There is the following tower of inclusions:
\[
\begin{tikzcd}[column sep=tiny]
\text{Noetherian, dim 1} \arrow[d, dash] & \text{integrally closed} \arrow[d, dash] & \\
\text{Dedekind} \arrow[d, dash] \arrow[ru, dash] & \text{UFD} \arrow[d, dash] & \text{local} \\
\text{PID} \arrow[d, dash] \arrow[ru, dash] & \text{regular local ring} \arrow[ru, dash] & \\
\text{field, DVR} \arrow[ru, dash] & &
\end{tikzcd}
\]
Furthermore, the following reverse implications hold:
\begin{itemize}
    \item Noetherian, dim 1 + integrally closed $\implies$ Dedekind;
    \item Dedekind + UFD $\implies$ PID;
    \item Dedekind + local $\implies$ field or DVR.
\end{itemize}

\begin{defn}
For a Noetherian local ring $A$ with maximal ideal $\fm$ and residue field $k$, it is called a \emph{regular local ring} if $\dim_k \fm/\fm^2 = \dim A$ (in general $\dim_k \fm/\fm^2 \ge \dim A$).
\end{defn}

Geometrically, a regular local ring corresponds to a curve being nonsingular at a point.

\begin{exm}
Consider the Noetherian local ring $A = \CC[[x,y]]/(y^2-x^3)$. The curve $y^2-x^3$ has a singularity at the origin. Correspondingly, $A$ is not a regular local ring for any of the following reasons:
\begin{itemize}
    \item $\fm = (x,y)$ is not principal;
    \item $\dim_{\CC} \fm/\fm^2 = 2$, while $\dim A = 1$;
    \item $A$ is not integrally closed: consider the injection $A \into \CC[[t]]$ by $x\mapsto t^2$, $y\mapsto t^3$. Then $A$ maps isomorphically to the subring of $\CC[[t]]$ consisting of power series in which the coefficient of $t$ is 0. This ring is not integrally closed since $t = t^3/t^2\in \Frac(A)$ is integral over $A$ but not in $A$. 
\end{itemize}
\end{exm}

\section{Lecture, Sep. 12}

(Integral closure)

\begin{prop}
Let $A$ be an integrally closed domain, $K = \Frac(A)$, $L/K$ a finite extension. Then $\alpha\in L$ is integral over $A$ if and only if its minimal polynomial in $K$ has coefficients in $A$.
\end{prop}

\begin{proof}
Suppose $\alpha\in L$ is integral over $A$. Let $g \in A[x]$ be monic such that $g(\alpha) = 0$, and let $f\in K[x]$ be the minimal polynomial of $\alpha$ in $K$. Consider an algebraic closure $\bar{K} \supset L \supset K$, then in $\bar{K}[x]$, $f$ factors into linear factors $f(x) = \prod(x-\alpha_i)$. Then each $\alpha_i$ is also a root of $g$, hence integral over $A$. Therefore, the coefficients of $f$, being symmetric polynomials in $\alpha_i$, are elements in $K$ integral over $A$, so they are in $A$ themselves.
\end{proof}

\begin{exm}[Integral closure resolves codimension-1 singularities]
Let $A = k[x,y]/(y^2 - x^3)$. We saw last time that $A$ is not integrally closed by embedding $A \cong k[t^2, t^3] \into k[t]$. The integral closure of $A$ (in its fraction field) is $k[t]$. The map $A\into k[t]$ corresponds to the map between varieties from the affine line to the curve $y^2 - x^3 = 0$.
\end{exm}

\section{Lecture, Sep. 14}

(Localization, Dedekind domain)

The following properties are preserved by localization (by a set not containing 0):
\begin{itemize}
    \item Noetherian
    \item Integrally closed
    \item Integral domain
    \item PID
    \item UFD
    \item Exactness.
\end{itemize}

\begin{prop}
$\dim A = \sup \{\dim A_{\fp}: \fp \in \Spec A\}$. (easy)
\end{prop}

\begin{prop}
Let $A\subset K$ where $K$ is a field, let $M$ be an $A$-module such that $M$ injects into the vector space $V = M\otimes_A K$. Then 
\[M = \bigcap_{\fp\subset A \mathrm{\ prime}} M_{\fp} = \bigcap_{\fm\subset A \mathrm{\ maximal}} M_{\fm}.\]
\end{prop}

\begin{proof}
It suffices to show that if $x\in M_{\fm}$ for every $\fm$, then $x\in M$. Define the ideal
\[I = \{a\in A: ax\in M\}.\]
Since $x\in M_{\fm}$, there exists $s\notin \fm$ such that $s\in I$. Therefore, $I$ is not contained in any maximal ideal, so $I = A$, so $x\in M$.
\end{proof}

Remarks: 1) We require $M\into V$ to be injective because otherwise we cannot view $M$ as a submodule of $M_{\fm}$. 2) This proposition allows us to go from local to global.

\begin{prop}
Let $A$ be a Noetherian domain. The following are equivalent:

\begin{itemize}
    \item Each $A_{\fp}$ ($\fp\neq 0$) is a DVR;
    \item $\dim A\le 1$ and $A$ is integrally closed.
\end{itemize}

Such an $A$ is called a \emph{Dedekind domain}.
\end{prop}

\begin{proof}
(i) $\implies$ (ii): If $\fp\neq 0$, then $A_{\fp}$ is a DVR. If $\fp = 0$ then $A_{\fp} = \Frac(A)$ is a field. Therefore by Proposition 5.1, $\dim A \le 1$. Also, each $A_{\fp}$ is integrally closed, so by Proposition 5.2, $A = \bigcap A_{\fp}$, so it is integrally closed as well.

(ii) $\implies$ (i): easy.
\end{proof}

Examples of Dedekind domains:
\begin{itemize}
    \item Every PID is a Dedekind domain. In particular, $\ZZ$ and $k[x]$ are Dedekind domains.
    \item The ring of integers $\cO_K$ of any algebraic number field is a Dedekind domain.
    \item The coordinate ring of a nonsingular affine algebraic curve $C$ is a Dedekind domain.
\end{itemize}

\begin{defn}
Let $A$ be a Noetherian domain, $K = \Frac(A)$. A \emph{fractional ideal} of $A$ is a finitely generated $A$-submodule of $K$.
\end{defn}

The set of invertible fractional ideals forms an abelian group under multiplication. It is the \emph{ideal group} $\Div(A)$ of $A$. The set of principal fractional ideals forms a subgroup, and the quotient is called the \emph{class group} $\Cl(A)$.

Invertibility is a local property:

\begin{prop}
For a fractional ideal $M$, the following are equivalent:
\begin{enumerate}[(i)]
    \item $M$ is invertible;
    \item Each $M_{\fp}$ is invertible;
    \item Each $M_{\fm}$ is invertible.
\end{enumerate}
\end{prop}

\begin{cor}
In a Dedekind domain $A$, every nonzero fractional ideal is invertible.
\end{cor}

(Reduce to the local case, where everything is easy because it's DVR.)

\begin{prop}
Let $A$ be a Dedekind domain, then every nonzero $x\in A$ belongs to finitely many prime ideals.
\end{prop}

\begin{proof}
The map $I\mapsto (x)I^{-1}$ gives an order-reversing involution on the set of ideals between $(x)$ and $A$. Therefore, $A/(x)$ is an Artinian ring, so it has dimension 0 and has finitely many maximal ideals. Since every prime is maximal, it has finitely many prime ideals.
\end{proof}


\section{Lecture, Sep. 16}

(Prime factorization in Dedekind domains)

In what follows, assume $A$ is a Dedekind domain, and $K$ its field of fractions.

Let $I$ be a fractional ideal of $A$, then $I_{\fp}$ is a fractional ideal of $A_{\fp}$, so it is equal to $(\fp A_{\fp})^{n}$ for some unique $n\in\ZZ$. Define then $v_{\fp}(I) = n$.

\begin{prop}
(i) The map $v_{\fp}: \Div(A) \to \ZZ$ mapping $I\mapsto v_{\fp}(I)$ is a group homomorphism. 
(ii) Suppose $I$ is generated by $x_1,\dots,x_m$, thn $v_{\fp}(I) = \min v_{\fp}(x_i)$.
\end{prop}

\begin{cor}
For each $x\in K^{\times}$, there only exist finitely many $\fp \leq 0$ such that $v_{\fp}(x)\neq 0$.
\end{cor}

\begin{proof}
For any $x\in A$, it belongs to only finitely many primes, so for all other primes $\fp$, $x$ is invertible in $A_{\fp}$, so $v_{\fp}(x) = 0$. In general $r/s \in K^{\times}$, where $r,s\in A$.
\end{proof}

\begin{cor}
For any fractional ideal $I$ of $A$, there only exist finitely many $\fp \leq 0$ such that $v_{\fp}(I)\neq 0$.
\end{cor}

\begin{thm}
There is an isomorphism of abelian groups:
\begin{align*}
    \Div(A) &\cong \bigoplus_{\text{primes }\fp \neq 0} \ZZ \\
    I &\mapsto (\dots, v_{\fp}(I),\dots) \\
    \prod_{\fp} \fp^{e_{\fp}} &\mapsfrom (e_{\fp})_{\fp}
\end{align*}
\end{thm}

\begin{prop}
Let $I = \prod_{\fp} \fp^{e_{\fp}}$, $J = \prod_{\fp} \fp^{f_{\fp}}$. Then
\begin{itemize}
    \item $I \supset J \iff e_{\fp} \le f_{\fp}$ (to contain is to divide)
    \item $I + J = \prod_{\fp} \fp^{\min(e_{\fp}, f_{\fp})}$
    \item $I \cap J = \prod_{\fp} \fp^{\max(e_{\fp}, f_{\fp})}$
    \item $IJ = \prod_{\fp} \fp^{e_{\fp} + f_{\fp}}$
    \item $(I:J) = \prod_{\fp} \fp^{e_{\fp} - f_{\fp}}$
\end{itemize}
\end{prop}

\begin{thm}
For a Dedekind domain $A$, the following are all equivalent:
\begin{itemize}
    \item $\Cl(A)$ is trivial.
    \item $A$ is a PID;
    \item $A$ is a UFD;
\end{itemize}
\end{thm}

\begin{proof}
(iii) $\implies$ (i): Let $I$ be any fractional ideal. Because it factors into the product of prime powers, it suffices to show that any nonzero prime ideal $\fp$ is principal. Pick $a\neq 0$ in $\fp$, then we can uniquely factorize $a = \prod_p p$ where each $p$ is irreducible. Since $\fp\supseteq (a)$, $\fp \mid (a)$, so $\fp \mid \prod_p(p)$. Since $\fp$ is prime, $\fp$ must divide some $(p)$, but since $(p)$ is a prime ideal, $\fp = (p)$ is principal.
\end{proof}

More concepts: Let $A$ be the coordinate ring of a regular affine curve $X$ over an algebraically closed field $k$. (Then $X = \Spec A$, and $A$ is a Dedekind domain.)
\[
\begin{tabular}{c|c}
algebra & geometry \\
\hline
$K = \Frac(A)$ & function field on $X$ \\
nonzero primes $\fp\subset A$ & closed points $P$ of $X$ \\
nonzero fractional ideal $I = \prod_{\fp}\fp^{e_{\fp}}$ of $A$ & divisor $\sum_{P} e_{\fp} P$ on $X$ \\
integral ideal $I\subseteq A$ & effective divisor on $X$ \\
principal fractional ideal $(f)$ & principal divisor $(f)$ on $X$
\end{tabular}
\]


\section{Lecture, Sep. 19}

(Strong approximation theorem, separability)

\begin{thm}[Strong approximation theorem]
Let $A$ be a Dedekind domain, $K = \Frac(A)$. Suppose we have distinct nonzero prime ideals $\fp_1,\dots,\fp_n \subset A$, integers $e_1,\dots,e_n$, and elements $a_1,\dots,a_n\in K$. Then there exists $x\in K$, such that:
\begin{itemize}
    \item $v_{\fp_i}(x - a_i) \ge e_i$ (this is the ``weak'' part);
    \item $v_{\fq}(x) \ge 0$ for all prime ideals $\fq\neq 0$, $\fq \notin \{\fp_1,\dots,\fp_n\}$.
\end{itemize}
\end{thm}

\begin{proof}
Without loss of generality, assume all $e_i \ge 0$.

Case I: Suppose $a_1 \in A$, $a_2,\dots,a_n  = 0$. Because $\fp_1^{e_1} + \fp_2^{e_2}\cdots\fp_n^{e_n} = A$, there exists $y\in \fp_1^{e_1}$, $x\in \fp_2^{e_2}\cdots\fp_n^{e_n}$ such that $x + y = a_1$. Then $v_{\fp_1}(x - a_1) = v_{\fp_1}(-y) = v_{\fp_1}(y) \ge e_1$, and $v_{\fp_i}(x - a_i) = v_{\fp_i}(x) \ge e_i$ for every $i\neq 1$. Also, since $x \in A$, $v_{\fq}(x) \ge 0$ for all $\fq$.

Case II: Suppose $a_1,\dots,a_n\in A$. Then using Case I, we can choose $x_i$ satisfying that $v_{\fp_i}(x_i - a_i) \ge e_i$ and $v_{\fp_j}(x_i) \ge 0$ for $i\neq j$. Let $x = x_1+\dots+x_n$, then $v_{\fp_i}(x - a_i) \ge v_{\fp_i}(x_i - a_i) \ge e_i$, and $v_{\fq}(x) \ge 0$.

Case III: Suppose $a_1,\dots,a_n\in K$ in general. Take nonzero $t\in A$ such that $ta_1,\dots,ta_n \in A$. Then by Case II, there exists $x\in A$ such that $v_{\fp_i}(x - ta_i)\ge e_i + v_{\fp_i}(t)$, $v_{\fq}(x) \ge v_{\fq}(t)$ for those $\fq$ with $v_{\fq(t)\ge 0}$, and $v_{\fq}(x) \ge 0$ for all others. Then $x/t\in K$ satisfies the conditions.
\end{proof}

Remark: we can in fact force $v_{\fp_i}(x) = f_i$ for any collection of $f_i$: just take $a_i$ such that $v_{\fp_i}(a_i) = f_i$ and $e_i > f_i$, then any $x$ such that $v_{\fp_i}(x-a_i) \ge e_i$ satisfies $v_{\fp_i}(x) = f_i$.

\begin{cor}
\label{semilocalDedPID}
A semilocal Dedekind ring $A$ must be a PID.
\end{cor}

\begin{proof}
Let $\fp_1,\dots,\fp_n$ be the nonzero primes. Any ideal $I$ is $\fp_1^{e_1}\cdots\fp_n^{e_n}$. By SA, there exists $x\in K = \Frac(A)$ such that $v_{\fp_i}(x) = e_i$, so in fact $I = (x)$.
\end{proof}

Next, we review some field theory related to separability. Let $K$ be a field and $\bar{K}$ be an algebraic closure of $K$. 

\begin{lem}
Let $\alpha \in \bar{K}$, $L = K(\alpha)$. Then $[L:K] \ge |\Hom_{K}(L, \bar{K})|$ with equality iff $\alpha$ is separable, iff $L/K$ is separable.
\end{lem}

\begin{proof}
We have $L \cong K[x]/(f(x))$ for some irreducible $f(x) \in K[x]$. Any homomorphism $\sigma: L \to \bar{K}$ fixing $K$ must send $x$ to another root of $f$ in $\bar{K}$, so there are at most $\deg f$ choices, and there are exactly $\deg f$ choices if and only if all roots of $f$ are distinct.

Let $\beta\in L$ be any element, then $K(\beta)\subset K(\alpha)$. Since $\alpha$ is separable over both $K$ and $K(\beta)$, we then have $[K(\beta):K] = \frac{[K(\alpha):K]}{[K(\alpha):K(\beta)]} = \frac{\Hom_K(K(\alpha), \bar{K})}{\Hom_{K(\beta)}(K(\alpha), \bar{K})} = \Hom_K(K(\beta),\bar{K})$, which shows that $\beta$ is separable over $K$ as well. Therefore $L/K$ is separable.
\end{proof}

\begin{prop}
For a finite extension $L/K$, the following are equivalent:
\begin{itemize}
    \item $L$ is separable over $K$;
    \item $L = K(\alpha_1,\dots,\alpha_n)$ for some $\alpha_i$ separable over $K$;
    \item $L = K(\alpha)$ for some $\alpha$ separable over $K$;
    \item $[L:K] = |\Hom_K(L, \bar{K})|$.
\end{itemize}
\end{prop}

\begin{cor}
Let $M/L$, $L/K$ be finite separable extensions, then $M/K$ is separable as well.
\end{cor}

\begin{lem}
Let $L/K$ be a field extension, and let $F$ be the set of elements in $L$ separable over $K$. Then $F$ is a field between $L$ and $K$.
\end{lem}

\begin{proof}
It suffces to show that if $\alpha, \beta \in L$ are separable over $K$, then so are $\alpha + \beta, \alpha\beta$. Consider the tower of extensions
$K(\alpha, \beta) \supset K(\alpha) \supset K$.
By the above lemma, $[K(\alpha):K] = |\Hom_K(K(\alpha), \bar{K})|$ and $[K(\alpha,\beta):K(\alpha)] = |\Hom_{K(\alpha)}(K(\alpha,\beta), \bar{K})|$. So 
\[[K(\alpha,\beta):K] = |\Hom_K(K(\alpha), \bar{K})|\cdot |\Hom_{K(\alpha)}(K(\alpha,\beta), \bar{K})| = |\Hom_K(K(\alpha, \beta), \bar{K})|.\] 
By the primitive element theorem, there exists $\gamma \in K(\alpha,\beta)$ with $K(\gamma) = K(\alpha,\beta)$, then we conclude that $\gamma$ is separable over $K$. Thus $\alpha+\beta, \alpha\beta \in L$ are both separable.
\end{proof}

Then we call $[F:K] = [L:K]_s$ the \emph{separable degree} of $L/K$, and call $[L:F] = [L:K]_i$ the \emph{inseparable degree} of $L/K$. Call $L/K$ \emph{separable} if $F = L$, and \emph{purely inseparable} if $F = K$.

\begin{thm}[Primitive element theorem]
Let $L/K$ be a finite separable extension. Then $L = K(\alpha)$ for some element $\alpha\in L$.
\end{thm}

\begin{thm}[Normal basis theorem]
\label{normal_basis}
    Let $L/K$ be a finite Galois extension, with $G$ its Galois group. Then there exists $\beta\in L$, such that $\{\sigma \beta: \sigma\in G\}$ forms a $K$-basis of $L$.
\end{thm}

\begin{thm}[Purely inseparable extensions]
Let $K$ be a field of characteristic $p$. 
\begin{itemize}
    \item A extension $L/K$ of degree $p$ is purely inseparable iff $L = K(\alpha^{1/p})$ where $\alpha \in K$ is not a $p$-th power.
    \item Any purely inseparable extension is a tower of purely inseparable degree-$p$ extensions.
\end{itemize}
\end{thm}

\begin{prop}
The separable degree $[L:K]_s$ is equal to $|\Hom_K(L, \bar{K})|$.
\end{prop}

\begin{proof}
By definition, $[L:K]_s = |\Hom_K(F, \bar{K})|$ where $F$ is the separable closure of $K$ in $L$. But $\Hom_K(F, \bar{K})$ corresponds one-to-one with $|\Hom_K(L, \bar{K})|$ (use the above theorem and the fact that $p$th roots are unique in characteristic $p$).
\end{proof}

So the separable degree is multiplicative: for field extensions $M/L/K$, $[M:L]_s[L:K]_s = [M:K]_s$, and so does the inseparable degree.

\begin{defn}
A field $K$ is called \emph{perfect} if any finite extension of $K$ is separable. Equivalently, either $\characteristic K = 0$, or $\characteristic K = p$ and the Frobenius endomorphism $x\mapsto x^p$ is an automorphism.
\end{defn}

For example, any finite field $\FF_q$ is perfect, but $\FF_q(t)$ is not.

\begin{defn}
A field $K$ is called \emph{separably closed} if its only separable extension is $K$ itself.
\end{defn}

\section{Lecture, Sep. 21}

(\'Etale algebras, norm, trace, bilinear pairings)

\begin{defn}
Let $K$ be a field. An \emph{\'etale algebra} $L$ over $K$ is a finite product of finite separable extensions of $K$.
\end{defn}

Apparently, a $K$-algebra $A$ is \'etale if and only if the map $\Spec A\to \Spec K$ is an \'etale morphism.

\begin{prop}
Let $L$ be a commutative $K$-algebra with finite dimension, such that $\dim_K L < |K|$. TFAE:
\begin{itemize}
    \item $L$ is a fintie \'etale $K$-algebra;
    \item Every element of $L$ is separable over $K$;
    \item $L\otimes_K K'$ is reduced for every extension $K'/K$;
    \item $L\otimes_K K'$ is semisimple for every extension $K'/K$;
    \item $L = K[x]/(f)$ for some separable $f\in K[x]$.
\end{itemize}
\end{prop}

The advantage of working with \'etale algebras instead of separable field extensions is that they are preserved by extension of coefficients. In other words, let $K'/K$ be a field extension, and $L/K$ a finite separable extension, then $L\otimes_K K'$ is not necessarily a field. However:

\begin{prop}
Let $K'/K$ be a field extension, $L$ is an \'etale $K$-algebra, then $L\otimes_K K'$ is an \'etale $K'$-algebra.
\end{prop}

\begin{proof}
Because tensor products commute with finite products, WLOG assume $L/K$ is a finite separable extension. By the primitive element theorem, $L = K[x]/(f(x))$ for some irreducible separable polynomial $f$. Then $L\otimes_K K' = K'[x]/(f(x))$. 

In $K'[x]$, $f(x)$ factors into the product of irreducible separable polynomials $f_1(x)\cdots f_n(x)$. By the Chinese Remainder Theorem, $K'[x]/(f(x)) \cong \prod_{i = 1}^n K'[x]/(f_i(x))$ is a product of finite separable extensions over $K'$.
\end{proof}

\begin{prop}
\label{etaleSepClosed}
Let $L/K$ be an \'etale algebra, $\Omega$ a separably closed field containing $K$. Then
\begin{align*}
    L\otimes_K \Omega &\to \prod_{\sigma \in \Hom_K(L, \Omega)} \Omega \\
    \ell \otimes 1 &\mapsto (\dots, \sigma(\ell),\dots)
\end{align*}
is an isomorphism.
\end{prop}


\begin{proof}
Because $\Hom_K(\prod L_i, \Omega) = \coprod \Hom_K(L_i, \Omega)$, we may again assume $L/K$ is a finite separable extension, i.e. $L \cong K[x]/(f(x))$ for an irreducible separable polynomial $f$. Then $f(x) = (x - \alpha_1)(x - \alpha_2) \dots (x - \alpha_n)$ in $\Omega[x]$, so any $\sigma\in \Hom_K(L,\Omega)$ must send $x$ to one of $\alpha_i$. The map is therefore given by
\[L \xto{\ell \mapsto \ell\otimes 1} L\otimes_K \Omega = \frac{\Omega[x]}{(f(x))} = \prod_{i=1}^n \frac{\Omega[x]}{x-\alpha_i} \cong \prod_{\sigma\in \Hom_K(L,\Omega)} \Omega.\]
\end{proof}


Next, we review norm, trace, and bilinear mappings.

\begin{defn}
Let $A\subset B$ be commutative rings, such that $B$ is a free $A$-module of rank $n$. For $b\in B$, the map $B\xto{\times b} B$ is an $A$-linear map, so we may define
\[\N_{B/A}(b) = \det(B\xto{\times b} B),\]
\[\Tr_{B/A}(b) = \tr(B\xto{\times b}B).\]
\end{defn}

\begin{prop}
\label{normTrCoeff}
Let $A\to A'$ be any ring homomorphism, $A\subset B$ such that $B$ is a free $A$-module of rank $n$, and let $B' = B\otimes_A A'$ be a ring that is a free $A'$-module of rank $n$. Then
\[\N_{B/A}(b) = \N_{B'/A'}(b\otimes 1),\]
\[\Tr_{B/A}(b) = \Tr_{B'/A'}(b\otimes 1).\]
\end{prop}

\begin{thm}
Let $L$ be an \'etale $K$-algebra, $\Omega/K$ separably closed, and $\Sigma = \Hom_{K}(L,\Omega)$. Then
\[\N_{L/K}(b) = \prod_{\sigma\in\Sigma} \sigma(b),\]
\[\Tr_{L/K}(b) = \sum_{\sigma\in\Sigma} \sigma(b).\]
\end{thm}

\begin{proof}
We have $\N_{L/K}(b) = \N_{L\otimes_K \Omega/\Omega}(b\otimes 1) = \N_{\Omega\times\dots\times \Omega/\Omega}(\dots, \sigma(b),\dots)$, by propositions \ref{etaleSepClosed} and \ref{normTrCoeff}. But this is just the diagonal matrix with entries $\sigma(b)$, so the norm is $\prod_{\sigma\in\Sigma} \sigma(b)$. The situation is identical for the trace.
\end{proof}

\begin{prop}[Norm and trace for finite extensions]
Let $L/K$ be a finite extension, and fix an embedding $L\subset \bar{K}$. Let $\alpha\in L^{\times}$ have minimal polynomial $f(x)\in K[x]$. Suppose $f(x) = \prod_i (x-\alpha_i)$ in $\bar{K}[x]$, and let $e = [L:K(\alpha)]$. Then
\[\N_{L/K}(\alpha) = \prod_i \alpha_i^e,\quad \Tr_{L/K}(\alpha) = e\sum_i \alpha_i.\]
\end{prop}

\begin{thm}
Suppose $A\subseteq B \subseteq C$ are rings, such that $B$ is a free $A$-module of rank $n$, and $C$ is a free $B$-module of rank $m$. Then
\[\N_{C/A}(c) = \N_{B/A}(\N_{C/B}(c)),\]
\[\Tr_{C/A}(c) = \Tr_{B/A}(\Tr_{C/B}(c)).\]
\end{thm}

\begin{proof}
We refer to \url{https://stacks.math.columbia.edu/tag/0BIJ}.
\end{proof}

Let $k$ be a field, $V$ a finite dimensional $k$-vector space. Let $\langle-,-\rangle: V\times V\to k$ be a symmetric bilinear pairing. This induces a map $V\to V^{*}$ by
\[v \longmapsto (w\mapsto \langle v, w\rangle).\]
The left kernel (which is equal to the right kernel since the form is symmetric) is the set of $v\in V$ such that $\langle v, w\rangle = 0$ for all $w\in V$.

Fixing a basis $e_1,\dots,e_n$ of $V$ allows the definition of the \emph{discriminant}
\[\disc(\langle-,-\rangle, e_1,\dots,e_n) = \det(\langle e_i, e_j\rangle).\]
Applying a change-of-basis matrix $T$ multiplies the discriminant by a factor of $(\det T)^2$.

The symmetric bilinear form is called \emph{nondegenerate} (or a \emph{perfect pairing}) if the following equivalent conditions are met:
\begin{itemize}
    \item the induced $V\to V^*$ is an isomorphism;
    \item the left kernel is 0;
    \item the discriminant under any basis is nonzero.
\end{itemize}
Given a basis $e_1,\dots,e_n$ of $V$, there is a \emph{dual basis} $f_1,\dots,f_n$ of $V^*$ defined by $f_i(e_j) = \delta_{ij}$. If the pairing is perfect, then $f_i$ correspond to a dual basis $e_i'$ of $V$, satisfying $\langle e_i, e_j\rangle = \delta_{ij}$.


\section{Lecture, Sep. 26}

(Extension of Dedekind domain is Dedekind)

We work in the following setup. Let $A$ be a Dedekind domain, $K = \Frac(A)$, $L/K$ a finite separable extension, and $B$ the integral closure of $A$ in $L$. The main goal of this lecture is to show that $B$ is also a Dedekind domain.

\begin{prop}
For any element $\ell\in L$, there exists $s\in A$ such that $s\ell \in B$.
\end{prop}

Consequently, $L = \Frac(B)$.


\begin{prop}
If $b\in B$, then $\Tr_{L/K}(b) \in A$.
\end{prop}

We define the \emph{trace pairing}:
\begin{align*}
    L\times L &\to K \\
    (x,y) &\mapsto \Tr_{L/K}(xy).
\end{align*}

\begin{prop}
The trace pairing is nondegenerate.
\end{prop}

\begin{proof}
Let $\Sigma = \Hom_K(L, \Omega) = \{\sigma_1,\dots,\sigma_m\}$ where $\Omega$ is some separably closed extension of $K$.
Pick a basis $\beta_1,\dots,\beta_m$ of $L/K$. Then the discriminant is equal to
\[\det(\Tr(\beta_i\beta_j)) = \det\pr{\sum_{\sigma_k} \sigma_k(\beta_i)\sigma_k(\beta_j)} = \det(\sigma_k(\beta_i))\det(\sigma_k(\beta_j)) = \det(\sigma_k(\beta_i))^2.\]
So it suffices to show that $\sigma_k(\beta_i)$ are linearly independent over $\Omega$. But this is just the linear independence of characters (on the group $L^{\times}$).
\end{proof}

Given an $A$-module $M\subseteq L$, define its \emph{dual} $M^* = \{x\in L: \Tr(xm) \in A\  \forall m\in M\}$. This is order-reversing.

\begin{prop}
$B$ is a finitely generated module over $A$.
\end{prop}

\begin{proof}
Consider an arbitrary basis of $L/K$, then each basis element can be multiplied by some element in $A$ such that they lie in $B$. Call this basis $u_1,\dots,u_n$ and let $M\subseteq B$ be the $A$-module generated by these elements. Consider its dual, $M^*$, which is freely generated by the dual basis $v_i$ of $u_i$, $\Tr(v_iu_j) = \delta_{ij}$. So $B \subseteq B^* \subseteq M^*$, and $B$ is finitely generated (since $A$ is Noetherian).
\end{proof}


\begin{thm}
$B$ is also a Dedekind domain.
\end{thm}

\begin{proof}
Because $B$ is a Noetherian $A$-module, it is a Noetherian ring. By definition, $B$ is integrally closed. Because $B/A$ is integral, $\dim B = \dim A \le 1$. So $B$ is a integrally closed Noetherian domain with dimension at most 1, hence a Dedekind domain.
\end{proof}

\begin{cor}
$\cO_K$ is Dedekind.
\end{cor}

Finally, we mention the following notations: 
\begin{itemize}
    \item $\fq\mid \fp$ (lying over) for primes $\fq\subset B$, $\fp\subset A$ means that $\fq\cap A = \fp$;
    \item Given nonzero prime $\fp \subset A$, we can uniquely factor
\[\fp B = \prod_i \fq_i^{e_i}.\]
Call $e_i$ the \emph{ramification index} of $\fq_i$ over $\fp$;
    \item For $\fq\mid\fp$, $f_{\fq} = [B/\fq : A/\fp]$ is called the \emph{residue field degree}.
\end{itemize} 


\section{Lecture, Sep. 28}

(Factorization of primes in Dedekind extension)

We continue to work in the AKLB setup. Let $\fp\subset A$ be a prime ideal, then $\fp B = \prod \fq^{e_{\fq}}$ factors as a product of primes in $B$. For a prime $\fq\in B$, $\fq\mid \fp \iff \fq\cap A = \fp \iff \fq \supseteq \fp B \iff \fq$ appears in the factorization of $\fp B$.

\begin{prop}
$[B/\fp B : A/\fp] = [L:K] =: n$.
\end{prop}

\begin{proof}
Let $S = A-\fp$. Because $A/\fp \cong S^{-1}A/\fp(S^{-1}A)$ and $B/\fp B \cong S^{-1}B/\fp(S^{-1}B)$, we may WLOG replace $A$ with $S^{-1}A$ and $B$ by $S^{-1}B$. (Here we implicitly use the fact that localization commutes with integral closure.) But now since $S^{-1}A = A_{\fp}$ is a DVR, it is a PID, so $B$ is free over $A$ with the same rank as $[L:K]$. Consequenly, $[B/\fp B: A/\fp] = [L:K]$.
\end{proof}


\begin{prop}
Given $\fp\subset A$, $\sum_{\fq \mid \fp} e_{\fq} f_{\fq} = n$. 
\end{prop}


\begin{proof}
We count the dimension of $B/\fp B$ as a $A/\fp$-vector space. By the above proposition, this dimension is equal to $n$. On the other hand, by CRT, $B/\fp B = \prod_{\fq|\fp} B/\fq^{e_{\fq}}$. Consider the filtration of $B/\fq$-vector spaces:
\[B/\fq^{e_{\fq}} \supset \fq/\fq^{e_{\fq}} \supset\dots\supset \fq^{e_{\fq}-1}/\fq^{e_{\fq}} \supset 0.\]
Every step is equal to $\fq^i/\fq^{i+1}$, which is a 1-dimensional $B/\fq$-vector space, so $B/\fq^{e_{\fq}}$ is $e_{\fq}$-dimensional over $B/\fq$, which is in turn $f_{\fq}$-dimensional over $A/\fp$. So $\dim_{A/\fp} B/\fq^{e_{\fq}} = e_{\fp} f_{\fp}$, and we're done.
\end{proof}

\begin{cor}
There are at most $n$ primes lying over $\fp$.
\end{cor}

\begin{defn}
The extension $L/K$ is called:
\begin{itemize}
    \item \emph{totally ramified at $\fq$} if $e_{\fq} = n$, $f_{\fq} = 1$, and $\fq$ is the only prime lying over $\fp$.
    \item \emph{unramified at $\fq$} if $e_{\fq} = 1$ and $B/\fq$ is separable over $A/\fp$.
    \item \emph{unramified above $\fp$} if it is unramified at every prime above $\fp$. Equivalently, iff $B/\fp B$ is an \'etale $A/\fp$-algebra.
\end{itemize}
\end{defn}

\begin{defn}
A prime $\fp\subset A$:
\begin{itemize}
    \item is \emph{inert} if $\fq = \fp B$ is prime in $B$.
    \item \emph{splits completely} if all $e_{\fq} = f_{\fq} = 1$.
\end{itemize}
\end{defn}

\begin{defn}
A discrete valuation $w$ on $L$ is said to \emph{extend} the discrete valuation $v$ on $K$ if $w|_K = e\cdot v$ for some $e\in \ZZ_{+}$.
\end{defn}

\begin{prop}
Fix $\fp\subset A$. Then there is a bijection
\[\{\mathrm{primes\ } \fq\mid \fp\} \iff \{\mathrm{discrete\ valuations\ } w \mathrm{\ extending\ } v_{\fp}\}\]
given by $\fq \mapsto v_{\fq}$.
\end{prop}

\begin{proof}
First, we show that $v_{\fq}$ indeed extends $v_{\fp}$. Because for distinct primes in $A$, the sets of primes $\fq$ lying above them are disjoint, it is clear that $v_{\fq}(x) = e_{\fq}v_{\fp}(x)$. The hard part is to show that all discrete valuations extending $v_{\fp}$ are of this form. Let $w$ be such a discrete valuation, and let $W = \{x\in L: w(x) \ge 0\}$, which is a DVR. Let $\fm$ be the maximal ideal of $W$, and $\fq = \fm \cap B$. Since $\fq = \fm \cap B \supseteq \fm \cap A = \fp$, $\fq\mid \fp$. Because $L\neq W \supseteq B_{\fq}$, $W = B_{\fq}$ (since $B_{\fq}$ is a DVR), and $w = v_{\fq}$.
\end{proof}


\section{Lecture, Sep. 30}

(Dedekind-Kummer theorem, index of $A$-lattices)

Continuing last time's discussion, we wish to give some intuition of the $e_{\fq}$ and $f_{\fq}$'s. We continue to work in the AKLB setup.

\begin{thm}[Dedekind-Kummer]
Suppose $B = A[\alpha]$ for some $\alpha\in L$. Let $f(x)\in A[x]$ be the minimal polynomial of $\alpha$ in $K$, and suppose that $f(x) \mod \fp = \prod (g_i(x) \mod \fp)^{e_i}$. Then $\fp B = \prod \fq_i^{e_i}$, where $\fq_i = (\fp, g_i(\alpha)) \subset B$, $e_i = e_{\fq_i}$, and $f_i = f_{\fq_i} = [B/\fq_i :A/\fp] = \deg g_i(x)$.
\end{thm}

\begin{proof}
We have $B = A[x]/(f(x))$, so 
\begin{align*}
    B/\fp B &= (A/\fp)[x]/(f(x)\mod\fp) \\
    &= \prod (A/\fp)[x]/(g_i(x)\mod\fp)^{e_i} \\
    &= \prod A[x]/(\fp, g_i(x))^{e_i} \\
    &= \prod B/(\fp, g_i(\alpha))^{e_i}.
\end{align*}
By the uniqueness of factorizing an \'etale algebra into separable extensions, we conclude $\fp B = \prod (\fp, g_i(\alpha))^{e_i} = \prod \fq_i^{e_i}$. Furthermore, $f_i = [B/\fq_i: A/\fp] = [A[x]/(\fp, g_i(x)): A/\fp] = [(A/\fp)[x]/g_i(x) : A/\fp] = \deg g_i$.
\end{proof}

From this, counting the degree of $f$, we get a more intuitive epxlanation of $n = \sum_{\fq\mid \fp} e_{\fq}f_{\fq}$.

Geometric example:
\[
\begin{tabular}{c|c|c}
& algebra & geometry \\
\hline
Dedekind domain $A$ & $\ZZ$ & $\CC[z]$ \\
Field of fractions $K$ & $\QQ$ & $\CC(z)$ \\
Degree two separable extension $L$ & $\QQ(\sqrt{-5})$ & $\CC(\sqrt{z})$ \\
Integral closure $B$ & $\ZZ[\sqrt{-5}]$ & $\CC[\sqrt{z}]$ \\
Totally ramified ideal & $(2)$ & $(z)$ \\
Ideal that split completely & $(3)$ & $(z-z_0)$, $z_0\neq 0$
\end{tabular}
\]

We now change gears to the topic of $A$-lattices.

\begin{defn}
Let $V$ be an $r$-dimensional vector space over $K$. An \emph{$A$-lattice in $V$} is a finitely generated $A$-submodule of $V$ such that $V = MK$.
\end{defn}

Our goal in this lecture is to define the ``index'' of an $A$-lattice, which will be an ideal in $A$. This allows us to define the ideal norm.

First, consider a torsion module $M$ over $A$ of finite type. Since $A$ is a Dedekind domain, the simple torsion modules over $A$ are of the form $A/\fp$ for some prime ideal $\fp$. Then given any composition series
\[M = M_n \supset M_{n-1} \supset\dots\supset M_1\supset M_0,\]
with $M_i/M_{i-1} \cong A/\fp_i$, we define
\[\chi(M) = \fp_1\dots\fp_n.\]
By Jordan-H\"older theorem, $\chi(M)$ only depends on $M$, and not on the composition series chosen.

\begin{prop}
For fractional ideals $I\subseteq J$, $\chi(J/I) = IJ^{-1}$.
\end{prop}

\begin{proof}
Localize at each prime to assume $A$ is a DVR, where everything is easy. 
\end{proof}

\begin{cor}
If $I\subset A$ is an integral ideal, then $\chi(A/I) = I$.
\end{cor}

\begin{defn}
Let $M,N \subset V$ be $A$-lattices. 
\begin{itemize}
    \item If $M\supseteq N$, then $M/N$ is torsion. Define $(M:N)_A = \chi(M/N)$, which is an integral ideal in $A$.
    \item In general, for any two $A$-lattices $M,N$, there exists an $A$-lattice $P$ contained in $M$ and $N$, so we can define $(M:N)_A = \frac{(M:P)_A}{(N:P)_A}$.
\end{itemize}
\end{defn}

In particular, when $V = K$, for $I,J$ fractional ideals, $(J:I)_A = IJ^{-1}$.

It is important that everything we do here commutes with localization: for example, $((M:N)_A)_{\fp} = (\chi(M/N))_{\fp} = \chi((M/N)_{\fp}) = \chi(M_{\fp}/N_{\fp}) = (M_{\fp}: N_{\fp})_{A_{\fp}}$. Many arguments we have for the general AKLB setup start by immediately reducing to the DVR case using localization.

\begin{prop}
\label{idealNormDet}
Given $X\in \GL_n(K)$, $(A^n: X(A^n))_A = (\det X)$.
\end{prop}

\begin{proof}
Assume WLOG $A$ is a DVR, hence a PID, so $X$ has a Smith normal form, which is diagonal, so we just reduce to the case $n = 1$. But $(A: xA)_A = \chi(A/(x)) = (x)$.
\end{proof}

\section{Lecture, Oct. 3}
\label{dvrextensions}

(Inclusion and ideal norm; DVR extensions)

We continue to work in the AKLB setup.

\begin{defn}
Let $\cI_A, \cI_B$ be the ideal groups of $A$ and $B$. Define 
\begin{itemize}
    \item $i: \cI_A\to \cI_B$ by $I\mapsto IB$, the \emph{inclusion homomorphism}.
    \item $N: \cI_B \to \cI_A$ by $J\mapsto (B:J)_A$, the \emph{ideal norm}.
\end{itemize}
\end{defn}

\begin{prop}
The following two diagrams commute:
\[
\begin{tikzcd}
L^{\times} \arrow[r, "x \mapsto (x)"] & \cI_B \\
K^{\times} \arrow[u, hook] \arrow[r, "x\mapsto (x)"] & \cI_A, \arrow[u, "i"]
\end{tikzcd}
\quad
\begin{tikzcd}
L^{\times} \arrow[r, "x \mapsto (x)"] \arrow[d, "\N_{L/K}"] & \cI_B \arrow[d, "N"] \\
K^{\times} \arrow[r, "x\mapsto (x)"] & \cI_A.
\end{tikzcd}
\]
\end{prop}

\begin{proof}
The first one is trivial. For the second one, consider an element $x\in L^{\times}$, then $N((x)) = (B : (x))_A$. If $A$ is a DVR, then it is a PID, so $B$ is a free $A$-module, and by proposition \ref{idealNormDet}, $(B:(x))_A = (\det(L\xto{x} L)) = (\N_{L/K}(x))$. In general, localize at each prime $\fp$, and because $((B: (x))_A)_{\fp} = (B_{\fp}: (x)_{\fp})_{A_{\fp}} = (\N_{L/K}(x))_{\fp}$ at each $\fp$, $(B:(x))_A = (\N_{L/K}(x))$.
\end{proof}

\begin{prop}
$i$ and $N$ are group homomorphisms.
\end{prop}

\begin{proof}
This is clear for $i$. If $A$ is a DVR, hence a PID, $B$ must be a semilocal Dedekind domain, so it is a PID (corollary \ref{semilocalDedPID}). This means that the map $L^{\times} \to \cI_B$ is surjective, so $N$ is a homomorphism. In general, localize $A$ at each prime $\fp$. Then because localization commutes with $(:)_A$, the diagrams
\[
\begin{tikzcd}
\cI_B \arrow[r] \arrow[d, "N"] & \cI_{B_{\fp}} \arrow[d, "N_{\fp}"] \\
\cI_A \arrow[r] & \cI_{A_{\fp}}
\end{tikzcd}
\]
commute. Because the $N_{\fp}$'s on the right are group homomorphisms for every $\fp$, we conclude that $N: \cI_B \to \cI_A$ is a homomorphism as well.
\end{proof}

\begin{prop}
Let $\fp\subset A$ satisfy that $\fp B = \prod \fq^{e_{\fq}}$. Then:
\begin{itemize}
    \item $i(\fp) = \prod \fq^{e_{\fq}}$;
    \item For $\fq\mid \fp$, $N(\fq) = \fp^{f_{\fq}}$.
\end{itemize}
\end{prop}

\begin{proof}
For the second one, $N(\fq) = (B:\fq)_A = \chi(B/\fq) = \chi((A/\fp)^{\oplus f_{\fq}}) = \fp^{f_{\fq}}$.
\end{proof}

The geometric picture:
\[
\begin{tabular}{c|c}
algebra & geometry \\
\hline
Ring $A$ & Affine scheme $\Spec A$ \\
Dedekind domain & Nonsingular curve \\
Inclusion of Dedekind domains $A\into B$ & (possibly) Ramified cover $\Spec B \onto \Spec A$ \\
Ideal group $\cI$ & Divisor group $\Div$ \\
Inclusion homomorphism $i: \cI_A \to \cI_B$ & Inverse image/pullback $f^*: \Div X \to \Div Y$ \\
Ideal norm $N:\cI_B\to \cI_A$ & Image/pushforward $f_*: \Div Y \to \Div X $
\end{tabular}
\]

We now consider the following setup. Let $A$ be a DVR with maximal ideal $\fp = (\pi)$, $K = \Frac A$, $B = A[x]/(f(x))$ for some monic $f(x)\in A[x]$. In general, $B$ need not even be integrally closed.

\begin{lem}
Any maximal ideal of $B$ contains $\fp$.
\end{lem}

\begin{proof}
Let $\fm \subset B$ be maximal. Then if $\fp\not\subseteq \fm$, $\fm + \fp B = B$, so the image of $\fm$ generates $B/\fp B$. Applying Nakayama's lemma to the local ring $A$ and finitely generated $A$-module $B$, we see that $\fm$ generates $B$, a contradiction.
\end{proof}

\begin{cor}
Maximal ideals of $B$ are in bijection with maximal ideals of $B/\fp B = (A/\fp)[x]/(f)$, which are in bijection with irreducible factors of $f(x)$ mod $\fp$.
\end{cor}

Armed with this information, we consider two conditions on $f$ that would make $B$ not only Dedekind, but actually a DVR.

Case 1: Suppose $f$ is irreducible mod $\fp$. Then the only maximal ideal of $B$ is $\fp B = (\pi)B$, which is principal. So $B$ is a local Noetherian domain whose maximal ideal is principal, so $B$ is a DVR. Here, the ramification index $e = 1$, $f = n$, $\fp\subset A$ is inert, and unramified if $f$ mod $\fp$ is separable.

Case 2: Suppose $f$ is Eisenstein; this means that $f = x^n + a_{n-1}x^{n-1} + \dots + a_1x + a_0$, where $a_i \in \fp$ but $a_0\notin\fp^2$. (This actually implies $f$ is irreducible too.) In this case $f = x^n$ mod $\fp$, so there is also only one maximal ideal in $B$, corresponding to $(\fp, x) = (a_0, x)$. But since $a_0 = -(x^n + \dots + a_1x)$, $a_0 \in (x)$. So the unique maximal ideal is just $(x)$, so $B$ is also a DVR. Also, we check that $B/(x) = A/\fp$, so $f = 1$, $e = n$, and $\fp$ is totally ramified.



\section{Lecture, Oct. 5}

(More DVR extensions; Galois extensions, $efg = n$, decomposition group)

We now study the converse of what we considered in the previous lecture. Suppose in the AKLB setup, $[L:K] = n$, and we assume in addition that $A$ is a DVR. Then the following are true:

\begin{prop}
If $B$ is a DVR, with maximal ideal $\fm$, such that $[B/\fm: A/\fp] = n$, then $B \cong A[x]/(f(x))$ for some monic $f\in A[x]$ irreducible mod $\fp$.
\end{prop}

\begin{proof}
By the primitive element theorem, there exists $\bar{b}\in B/\fm$ that generates it over $A/\fp$, which is represented by $b\in B$. Let $f(x)\in A[x]$ be the characteristic polynomial of $b$ over $K$. We have $f(b) = 0$, so the image $\bar{f}$ of $f$ in $(A/\fp) [x]$ has $\bar{b}$ as a root. Since $\bar{b}$ is of degree $n$ over $A/\fp$, $\bar{f}$ is irreducible of degree $n$. So by the discussion last lecture, $A[x]/(f(x))$ is a DVR, and there is an inclusion $A[x]/(f(x)) \into B$ mapping $x\mapsto b$. Since $L = K(b)$, $L = \Frac A[x]/(f(x))$ as well, and because $B$ is an intermediate ring between a DVR and its field of fractions, and $B\neq L$, it must be that $B = A[x]/(f(x))$.
\end{proof}

\begin{prop}
If $B$ is a DVR, with the discrete valuation $w: L^{\times}\to \ZZ$, and $w$ extends the valuation $v$ on $A$ with index $n$, then $B \cong A[x]/(f(x))$ for some Eisenstein polynomial $f\in A[x]$.
\end{prop}

\begin{proof}
Pick $\beta\in B$ such that $w(\beta) = 1$. Let $f \in A[x]$ be the characteristic polynomial of $\beta$ in $K$. We wish to show that it is Eisenstein. Write $f(x) = x^n + a_{n-1}x^{n-1} + \dots + a_1x + a_0$ (the fact that it has degree $n$ follows from the same argument as follows). Then $\beta^n + a_{n-1}\beta^{n-1} + \dots + a_1\beta + a_0 = 0$ in $B$. Since $w(a_i\beta^i) \equiv i \pmod{n}$, and the two terms with smallest $w$ have to have the same valuation, we conclude that $w(a_0) = w(\beta^n) = n$, so $v(a_0) = 1$ and $v(a_i) \ge 1$ for $i = 1,\dots,n-1$. Also, $A[x]/(f(x))$ is a DVR that injects into $B$, so $A[x]/(f(x)) = B$.
\end{proof}

We now consider the following ``AKLBG'' setup: in addition to having the original AKLB, we require $L/K$ to be a finite \emph{Galois} extension with $G = \Gal(L/K)$.

\begin{prop}
Fix a nonzero prime $\fp\subset A$. Then the $G$-action on $L$ induces a transitive $G$-action on $\{\fq \subset B: \fq\mid\fp\}$.
\end{prop}

\begin{proof}
Fix on $\fq$ above $\fp$. If $\fq'$ above $\fp$ is not in the orbit of $\fq$, then by prime avoidance, we may find $b\in \fq'$, such that $b\notin g\fq$ for all $g\in G$. This means that $gb \notin \fq$ for all $g\in G$. Consider the norm $\N_{L/K}(b) = \prod_{g\in G} gb \in A$, then $\N_{L/K}(b)\in \fq'$ but $\N_{L/K}(b) \notin \fq$. This is a contradiction to $\fq\cap A = \fq'\cap A$.
\end{proof}

Because of this, the $e_{\fq}$ and $f_{\fq}$ are the same for all $\fq\mid \fp$, for any fixed $\fp$, so we can just call them $e_{\fp}$ and $f_{\fp}$. Also, let $g_{\fp}$ denote the number of primes above $\fp$. Then:

\begin{prop}
$e_{\fp}f_{\fp}g_{\fp} = n$.
\end{prop}

Fix $\fq$ a prime upstairs. Define the \emph{decomposition group} $D = D_{\fq} \le G$ as the stabilizer of $\fq$ in $G$. Then $(G:D) = g_{\fq}$ by the orbit-stabilizer theorem, so $|D| = e_{\fp}f_{\fp}$.

The reason we define $D$ is that while $G$ preserves $B$ and permutes the primes $\{\fq: \fq\mid \fp\}$, $D$ preserves both $B$ and $\fq$, which means that it acts on $B/\fq$. 

\begin{prop}
\label{GaloisDecomp}
Suppose $B/\fq$ is separable over $A/\fp$. Then:
\begin{itemize}
    \item $B/\fq$ is Galois over $A/\fp$;
    \item The natural map $D\to \Gal(\FF_{\fq}/\FF_{\fp})$ is surjective. (Here, $\FF_{\fq} = B/\fq$, $\FF_{\fp} = A/\fp$.)
\end{itemize}
\end{prop}

We defer the proof to the next lecture.

\section{Lecture, Oct. 7}

(Inertia group, Frobenius class)

We first give the proof of Proposition \ref{GaloisDecomp}. For the first bullet point, it suffices to show that $B/\fq$ is normal over $A/\fp$. Given $\bar{b}\in B/\fq$, represented by $b\in B$, we let $P(x) = \prod_{g\in G} (x-gb)$. This polynomial is $G$-invariant, hence is in $K[x]$, hence in $A[x]$. Reducing modulo $\fq$, we get $\bar{P}(x) = \prod_{g\in G} (x - g\bar{b}) \in (A/\fp)[x]$. This shows that $\bar{b}$ is the root of a polynomial in $(A/\fp)[x]$ that splits completely, so the extension is indeed normal.

For the second bullet point, by primitive element theorem, $\FF_{\fq} = \FF_{\fp}(\bar{b})$ for some nonzero $\bar{b}\in \FF_{\fq}$. Strong approximation gives us $b\in B$ such that $b = \bar{b}$ mod $\fq$ and $b\in \fq'$ for all other $\fq'\mid \fp$. Then $gb\in \fq$ for all $g\in G\backslash D$. Let $P(x) = \prod_{g\in G}(x-gb) \in A[x]$, then reducing mod $\fq$, we get $\bar{P}(x) = \prod_{g\in G}(x - g\bar{b}) \in \FF_{\fp}[x]$. But since $g\bar{b} = 0$ in $\FF_{\fp}$, $\bar{P}(x) = \prod_{g\in D} (x - g\bar{b})x^{|G|-|D|}$. Since $\bar{P}(b) = 0$, every conjugate of $\bar{b}$ is a nonzero root of $\bar{P}(x)$, hence equals $g\bar{b}$ for some $g\in D$. This shows that $D\to \Gal(\FF_{\fq}/\FF_{\fp})$ is surjective.

\begin{defn}
The \emph{inertia group} $I_{\fq}$ satisfies the short exact sequence
\[1\to I_{\fq} \to D_{\fq} \to \Gal(\FF_{\fq}/\FF_{\fp}) \to 1.\]
\end{defn}

In other words, $I_{\fq}$ consists of the elements of $G$ that preserve $B$ and $\fq$, and act as the identity on $B/\fq = \FF_{\fq}$.

Because $|D| = ef$, $|\Gal(\FF_{\fq}/\FF_{\fp})| = [\FF_{\fq}:\FF_{\fp}] = f$, we see that $|I| = e$. So the inertia group ``detects'' ramification in some sense.

By Galois theory, the sequence of subgroups $1 \le I \le D \le G$ corresponds to a tower of fields $L \supseteq L^I \supseteq L^D \supset K$, where $L^I$ is the \emph{inertia field} and $L^D$ is the \emph{decomposition field}. Computing the group indices, we get $[L:L^I] = e$, $[L^I:L^D] = f$, $[L^D:K] = g$.

In addition, $I$ and $D$ behave well under sub- and quotient groups, as follows: fix AKLBG, and let $H$ be a subgroup of $G$. Let $L^H\subset L$ be the fixed field of $H$. The corresponding $B^H\subset L^H$ is the integral closure of $A$ in $L^H$, then $B$ is the integral closure of $L^H$ in $K$ by transitivity of integrality. Fixing $\fq\subset B$, it pulls back to $\fq^H\subset B^H$ and $\fp \subset A$, and similarly we have a tower of fields $\FF_{\fq} \supset \FF_{\fq^H} \supset \FF_{\fp}$. For the Galois extension $L/L^H$, we can similarly define the inertia and composition groups $I_H\le D_H \le H$.

\begin{prop}
$D_H = D\cap H$, $I_H = I\cap H$. \qed
\end{prop}

If, in addition, $H$ is a \emph{normal} subgroup, then $L^H/K$ is Galois as well, with Galois group $G/H$. Then:

\begin{prop}
The following diagram commutes and has exact rows and columns:
\[
\begin{tikzcd}
 & 1 \arrow[d] & 1 \arrow[d] & 1\arrow[d] & \\
1 \arrow[r] & I_H \arrow[d] \arrow[r] & D_H \arrow[d] \arrow[r] & \Gal(\FF_{\fq}/\FF_{\fq^H}) \arrow[d] \arrow[r] & 1 \\
1 \arrow[r] & I \arrow[d] \arrow[r] & D \arrow[d] \arrow[r] & \Gal(\FF_{\fq}/\FF_{\fp}) \arrow[d] \arrow[r] & 1 \\
1 \arrow[r] & I_{G/H} \arrow[d] \arrow[r] & D_{G/H} \arrow[d] \arrow[r] & \Gal(\FF_{\fq^H}/\FF_{\fp}) \arrow[d] \arrow[r] & 1 \\
& 1 & 1 & 1 &
\end{tikzcd}
\]
\end{prop}

Now, we consider the case where $\FF_{\fp}$ is a \emph{finite field}. Then it is a well-known result that finite extensions of finite fields are always cyclic and generated by the Frobenius element
\[\Frob_{\fq}: x\mapsto x^{|\FF_{\fp}|}.\]
Suppose $L/K$ is unramified at $\fq$, then $D \cong \Gal(\FF_{\fq}/\FF_{\fp})$ is a cyclic group, and we can view $\Frob_{\fq}$ as an element in $D$ with order $f$.

For $\fq'\mid \fp$, $\fq' = \sigma\fq$ for some $\sigma\in G$, so $D_{\fq'} = \sigma D \sigma^{-1}$ are conjugate subgroups of $G$, and $\Frob_{\fq'} = \sigma \Frob_{\fq} \sigma^{-1}$. Therefore, $\fp$ determines a conjugacy class in $G$, called the \emph{Frobenius class}. (So if $G$ is abelian, the Frobenius class is actually an element in $G$.)


\begin{defn}
    Assume AKLBG with finite residue fields. For $\fq$ unramified, define the Artin symbol
    \[\pr{\frac{L/K}{\fq}} := \Frob_{\fq}.\]
    When $G$ is abelian, this only depends on $\fp$, so we may instead write $\pr{\frac{L/K}{\fp}}$.
\end{defn}

\begin{defn}[Artin map]
    Let $A$ be Dedekind, $K = \Frac A$, $L/K$ abelian extension. There is a homomorphism from the subgroup of the ideal group $\cI_A$ generated by unramified primes to $G$, given by
    \[\prod \fp_i^{e_i} \mapsto \prod\pr{\frac{L/K}{\fp_i}}^{e_i}.\]
\end{defn}

\begin{Rem}
Here's how to determine the splitting type of a prime in a separable but not necessarily Galois field extension. Assume AKLB, and let $M$ be the Galois closure of $L/K$. (So $M$ is the splitting field of the minimal polynomial of $\alpha$, where $L = K(\alpha)$). 

Let $G = \Gal(M/K)$, then $G$ naturally embeds into $S_n$ by permuting the $n$ maps $\Hom_K(L, M)$. The subgroup of $G$ corresponding to $L$ is $H = G\cap S_{n-1}$, where $S_{n-1}$ is the subgroup of all permutations fixing the identity embedding $L\into M$. Because the $G$-action on $\Hom_K(L, M)$ is transitive, this action is exactly the $G$-action on $H\backslash G$, the right cosets of $H$.

Fix a prime $\fp\subset A$ that we want to study. Suppose $C$ is the integral closure of $B$ in $M$, and fix an arbitrary prime $\fP \subset C$ above $\fp$. Let $I \subseteq D \subseteq G$ be the inertia and decomposition groups of $\fP$. Then the transitive $G$-action on $H\backslash G$ induces a $D$-action on $H\backslash G$.

The main claim here is that the orbits of this $D$-action corresponds precisely to the primes $\fq\subset B$ above $\fp$, and the size of the orbit corresponding to $\fq$ is $e_{\fq}f_{\fq}$. Proof of this claim: given some orbit $[Hg]$ of $H\backslash G$ under $D$, we map this to $g\fP\cap L$. 
\begin{itemize}
    \item Injectivity: suppose $g_1\fP\cap L = g_2\fP\cap L = \fq$, then $g_1g_2^{-1}$ maps $g_2\fP$ to some prime that is also above $\fq$. Because $L/M$ is Galois, there is an element $h\in H\subset G$ mapping $g_1\fP$ back to $g_2\fP$. Then $hg_1g_2^{-1} \in D$, so $[Hg_1] = [Hg_1(g_2^{-1}g_2)] = [H(hg_1g_2^{-1})g_2] = [Hg_2]$.
    \item Surjectivity: follows because $G$ is transitive on the primes in $C$ above $\fp$.
    \item Size of the orbit: by orbit-stabilizer theorem, this is equal to $\frac{|D_{\fP/\fp}|}{|D_{\fP/\fq}|} = \frac{e_{\fP/\fp}f_{\fP/\fp}}{e_{\fP/\fq}f_{\fP/\fq}} = e_{\fq/\fp} f_{\fq/\fp}$.
\end{itemize}
Even better, we have that $I\trianglelefteq D$ is normal, so every $I$-orbit in a $D$-orbit corresponding to $\fq$ (of size $e_{\fq}f_{\fq}$) has the same size. By orbit-stabilizer theorem, this size is $\frac{|I_{\fP/\fp}|}{|I_{\fP/\fq}|} = e_{\fq/\fp}$. Notice that:
\begin{itemize}
    \item When $L/K$ is already Galois, $H = \{1\}$, and every orbit of the $D$-action on $G$ (i.e. the $D$-cosets) have the same size.
    \item When $\fp$ is unramified and residue fields are finite (e.g. $K,L$ are local fiels), $D$ is generated by the Frobenius element, so $D$-orbits are the same as the orbits of Frobenius.
\end{itemize}

Reference: \href{https://people.math.harvard.edu/~mmwood/Splitting.pdf}{Melanie Wood}.

\end{Rem}


\section{Lecture, Oct. 12}

(Local fields, Hensel's lemma)

Let $K$ be a field with a discrete valuation $v: K\to \ZZ\cup\{\infty\}$. This induces a nonarchimedean absolute value, which induces a metric, which induces a topology on $K$. Let $(A,\fm)$ be the valuation ring, $\fm = (\pi)$.

\begin{prop}
$K$ is locally compact, iff $K$ is complete and the residue field is finite.
\end{prop}

\begin{proof}
($\implies$) It is clear that $K$ is Hausdorff. If $K$ is locally compact, then each point of $K$ has a local base of closed compact neighborhoods. Given any Cauchy sequence, we can find a descending, nested sequence of closed compact sets, so by Cantor intersection theorem there is a unique point inside all of them whence the sequence converges. Also, since some $\pi^n A$ is compact, multiplying by $\pi^{-n}$ shows that $A$ is compact, so $A/\pi A$ is compact and discrete, hence finite.

($\given$) If $A/\pi A$ is finite, then $A/\pi^n A$ is also finite. Then $\wh{A} = \varprojlim A/\pi^n A$ is a closed subset of $\prod_{n\ge 0}A/\pi^n A$, which is compact by Tychonoff. So $\wh{A} = A$ is compact, so $\pi^n A$ is compact, and they form a basis of compact open neighborhoods of $K$.
\end{proof}

Such a field $K$ is called a \emph{local field}. Usually the term also includes $\RR$ and $\CC$, the only two archimedean local fields. 

\begin{lem}[Hensel's lemma]
Let $A$ be a complete DVR with residue field $k$, $F\in A[x]$, and $f \in k[x]$ be the image of $F$. Suppose $\alpha\in k$ is a simple root of $f$, then there exists a unique $a\in A$ lifting $\alpha$, such that $F(a) = 0$.
\end{lem}

\begin{lem}[Hensel's lemma, stronger]
Let $A,k,F,f$ as before. If $f(x) = g(x)h(x)$, where $g,h$ are coprime monic polynomials in $k[x]$, then $F(x) = G(x)H(x)$, with $G,H\in A[x]$ lifting $g,h$.
\end{lem}


\section{Lecture, Oct. 14}

(Extensions of complete DVRs)

\begin{thm}
In the AKLB setup, assume $A$ is a \emph{complete DVR} with prime ideal $\fp$. Then $B$ is a DVR, i.e. there is only 1 prime above $\fp$.
\end{thm}

(In fact, this holds even when $L/K$ is finite and not necessarily separable --- see Serre's book.)

\begin{proof}[First proof]
Suppose there are at least two primes $\fq_1,\fq_2$ above $\fp$. Pick $b\in \fq_1, b\notin \fq_2$, then $\fq_1\cap A[b]$ and $\fq_2\cap A[b]$ are distinct primes in $A[b]$, both containing $\fp$. So $A[b]/\fp A[b]$ has at least 2 primes as well. Now, let $F(x)\in A[x]$ be the minimal polynomial of $b$ in $K$, so that
\[\frac{A[b]}{\fp A[b]} \cong \frac{A[x]}{(F(x),\fp)} \cong \frac{k[x]}{(f(x))}\]
where $f$ is the reduction of $F$ mod $\fp$. Because $k[x]/(f(x))$ has at least 2 primes, $f$ factors into coprime monic $g, h\in k[x]$, which we lift into a factorization of $F$ by Hensel's lemma. But this contradicts the irreducibility of $F$.
\end{proof}

\begin{lem}
If $(K, |\cdot|)$ is complete and $V$ is a f.d. vector space over $K$, then any two norms are equivalent.
\end{lem}

\begin{proof}[Second proof]
Each prime $\fq\mid \fp$ defines an norm on $L$ (as a f.d. $K$-vector space) extending the absolute value on $K$. It suffices then to find a way to characterize $\fq$ in terms of the topology it induces. In fact, for $x\in L$, $x$ is in the valuation ring of $\fq$ iff the sequence $x^{-1}, x^{-2}, \dots$ does not converge to 0, so the topology uniquely characterizes the valuation ring of $\fq$, which uniquely characterizes $\fq$ as its maximal ideal.
\end{proof}

Some corollaries of the above theorem:
\begin{itemize}
    \item $B$ is a DVR and a free $A$-module of rank $n$.
    \item There exists a unique discrete valuation $w$ on $L$ extending $v$ on $K$, with index $e$.
    \item $B$ and $L$ are complete with respect to $w$. (since it is equivalent to the sup norm, which is complete)
    \item If $x,y\in L$ are conjugate over $K$, then $w(x) = w(y)$. (suppose $y=\sigma x$, then $w$ and $w\circ\sigma$ are two discrete valuations extending $v$, so they are the same)
    \item For $x\in L$, $w(x) = \frac{1}{f} v(\N_{L/K}(x))$. (use the ideal norm interpretation)
\end{itemize}


\begin{cor}
The valuation $v: K\to \ZZ\cup\{\infty\}$ is the restriction of a unique valuation $\bar{K} \onto \QQ\cup\{\infty\}$.
\end{cor}

\begin{proof}
For each finite algebraic extension $L/K$, $v$ can be uniquely extended to $L$. The map $\bar{K} \to \QQ\cup\{\infty\}$ is surjective because $\bar{K}$ contains all $n$th roots.
\end{proof}

However, by taking the algebraic closure, $\bar{K}$ is no longer complete! For example, $\bar{\QQ_p}$ has a valuation with value group $\QQ$ and residue field $\bar{\FF_p}$, but it is not complete anymore. So we can define $\CC_p = \wh{\bar{\QQ_p}}$, which is complete, but it is not obvious that it is still algebraically closed. Fortunately:

\begin{thm}
Let $K$ be a field complete with respect to a nontrivial non-archimedean absolute value. Then the completion of $\bar{K}$ is algebraically closed.
\end{thm}

\begin{proof}
See Brian Conrad's handout \href{http://math.stanford.edu/~conrad/248APage/handouts/algclosurecomp.pdf}{here}.
\end{proof}

\section{Lecture, Oct. 17}

(Newton polygons, $p$-adic analysis, and an example)

Let $K$ be a field with a valuation $v: K\to \RR\cup\{\infty\}$ (not necessarily surjective). For a polynomial $f(x) = a_nx^n + \dots + a_0 \in K[x]$, we may construct its \emph{Newton polygon} as the lower convex hull of the points $(i, v(a_i))$. The main theorem is the follows:

\begin{thm}
The width of the slope $s$ segment of the Newton polygon is at least the number of zeros of $f$ with valuation $-s$, with equality when $f$ splits completely into linear factors.
\end{thm}

Note that this provides additional motivation for Eisenstein's criterion.

\begin{proof}
WLOG pass to the case $K = \bar{K}$. First, notice that changing $f(x)$ to $f(ax)$ or $af(x)$ by any constant $a\in K^{\times}$ does not alter the content of the theorem. As such we can reduce to the case $s = 0$ and suppose $f$ factors as
\[f(x) = \prod_{i=1}^{a} (x-r_i) \prod_{j=1}^{b} (x-t_j) \prod_{k=1}^{c} (1-x/u_k) \in A[x]\]
where $v(r_i) > 0$, $v(t_j) = 0$, and $v(u_k) < 0$. Reducing modulo the maximal ideal of $A$, we get
\[\bar{f}(x) = x^a\prod_{j=1}^b(x-\bar{t_j}).\]
This means that the Newton polygon of $f$ has a segment from $(a,0)$ to $(a+b,0)$, which has width $b$ equal to the number of zeros of $f$ with valuation 0.
\end{proof}


Let $K$ be complete with respect to a nonarchimedean aboslute value, i.e. coming from some valuation. Because we have a notion of size, we can do ``$p$-adic analysis'' much like how we do real or complex analysis. But here, lots of small errors cannot add up to a big error because of the nonarchimedean triangle inequality, so very nice things hold. 

For example, for a sequence $a_0,a_1,\dots\in K$, the series $\sum a_n$ converges if and only if $a_n\to 0$. 

For another example, we have the Cauchy-Hadamard formula for the radius of convergence: given $f(x) = \sum a_nx^n \in K[[x]]$, its radius of convergence
\[R = \frac{1}{\limsup_{n\to\infty} |a_n|^{1/n}}.\]

\begin{thm}[Strassmann's theorem]
Let $A$ be the valuation ring of $K$, $f(x) = \sum a_nx^n \in A[[x]]$ a nonzero formal power series such that $a_n\to 0$. Then the number of zeros of $f(x)$ in $A$ is at most $N$, where $N$ is the largest such that $|a_N| = \max |a_n|$. 
\end{thm}

We now specialize to the case $K = \CC_p$. Here, we have the $p$-adic exponential function
\[\exp(x) = \sum_{n\ge 0} \frac{x^n}{n} \in \QQ_p[[x]].\]
Its radius of convergence is $R = p^{-\frac{1}{p-1}}$. Using the Newton polygon, we see that the truncated $\exp$ has no roots with valuation at least $\frac{1}{p-1}$.

Conversely, we may wish to find a $p$-adic logarithm. There is a natural one, called the Iwasawa logarithm.

\begin{prop}
There exists a unique homomorphism
\[\log: \CC_p^{\times} \to (\CC_p, +)\]
satisfying:
\begin{enumerate}
    \item For $|x| < 1$, $\log(1+x) = x - \frac{x^2}{2} + \frac{x^3}{3} -\dots$;
    \item $\log p = 0$.
\end{enumerate}
\end{prop}

\begin{proof}
Let $\fm$ be the maximal ideal of the valuation ring of $\CC_p$. Construct the logarithm in stages:
\begin{itemize}
    \item First, for $x\in \fm$, define $\log(1+x)$ according to the infinite series. Then 
    \[\log(1+x) + \log(1+y) = \log((1+x)(1+y))\]
    holds as an identity on power series, so it holds as numbers in $\CC_p$.
    \item Second, for $x\in G = p^{\ZZ}(1+\fm)$, define $\log(p^n(1+x)) = \log(1+x)$.
    \item Third, we claim that $\CC_p^{\times}/G$ is in fact torsion. This would allow us to uniquely extend $\log$ to the entire $\CC_p^{\times}$. To show this is torsion, let $\cO = \{x\in\CC_p: v(x) = 0\}$ be the group of units in the valuation ring, and notice that
    \[\cO^{\times}/(1+\fm) \to \CC_p^{\times}/G \xto{v_p} \QQ/\ZZ \]
    is exact. The left side is isomorphic to $\bar{\FF_p}^{\times}$, which is torsion; the right side is also torsion. So the middle term must be torsion as well, which finishes the proof.
\end{itemize}
\end{proof}

Finally, we compute field extensions of $\QQ_p$ such as $\QQ(i)\otimes_{\QQ} \QQ_p$. This is clearly an \'etale algebra over $\QQ_p$, and depending on how $x^2+1$ factors in $\QQ_p[x]$ (read: in $\FF_p[x]$, because of Hensel's lemma), it is either
\begin{itemize}
    \item $\QQ_p\times \QQ_p$, in the case that $x^2+1$ factors into two distinct factors (e.g. $p=5$). There are two primes above $p\ZZ_p$.
    \item a totally ramified extension over $\QQ_p$, in the case that $x^2+1$ factors into the same factors (e.g. $p=2$). There is one prime above $p\ZZ_p$ with $e=2$, $f=1$.
    \item an unramified extension over $\QQ_p$, in the case that $x^2+1$ does not factor (e.g. $p=7$). There is one prime above $p\ZZ_p$ with $e=1$, $f=2$.
\end{itemize}


\section{Lecture, Oct. 19}

(Completing a separable extension of Dedekind domains)

The following theorem generalizes the observation in the end of the previous lecture.

\begin{thm}
Assume AKLB, and fix a prime $\fp\subset A$ and the valuation $v = v_{\fp}$ on $K$. Let $w_i$ be the distinct discrete valuations on $L$ extending $v$, which are in bijection with primes $\fq\mid\fp$. Let $\wh{K}$ be the completion of $K$ wrt $v$, and let $\wh{L}_i$ be the completions of $L$ wrt $w_i$. Then:
\begin{enumerate}
    \item $\wh{L}_i/\wh{K}$ is a field extension;
    \item The induced $\wh{w_i}$ on $\wh{L}_i$ is the unique extension of $\wh{v}$ on $\wh{K}$.
    \item $e(\wh{w_i}/\wh{v}) = e_i$, and $f(\wh{w_i}/\wh{v}) = f_i$.
    \item $[\wh{L}_i:\wh{K}] = e_if_i$.
    \item $L\otimes_K \wh{K} \to \prod_i \wh{L}_i$ is an isomorphism.
\end{enumerate}
\end{thm}

\begin{proof}
(1) through (4) are easy. For (5), there is a natural $K$-bilinear $L\times \wh{K} \to \prod_i\wh{L}_i$ given by $(\ell, \alpha) \mapsto \ell\alpha$, which induces a linear map $L\otimes_K \wh{K} \to \prod_i \wh{L}_i$. To show this is an isomorphism, it suffices to show this is surjective, since both sides have the same $\wh{K}$-dimension ($n = \sum_i e_if_i$). 

Choose a $\wh{K}$-basis $\alpha_i$ ($i=1,2,\dots,n$) for $\prod_i \wh{L}_i$. For each $\alpha_i$, using weak approximation, we could find $\ell_i\in L$ such that its diagonal embedding into $\prod_i L_i$ is close to $\alpha_i$. Then these $\ell_i$ still forms a basis (because the change-of-basis matrix is close enough to id). This shows surjection, as desired.
\end{proof}

\begin{prop}
If, in addition, $L/K$ is Galois, then each $\wh{L}_i/\wh{K}$ is Galois as well, with Galois group $D_i$.
\end{prop}

\begin{proof}
Each $\sigma\in D_i$ acts on $L$ respecting $w_i$, so it acts on $\wh{L}_i$ fixing $\wh{K}$. This gives a homomorphism $\phi: D_i\to \Aut(\wh{L}_i/\wh{K})$. Conversely, there is a map $\psi: \Aut(\wh{L}_i/\wh{K}) \to D_i$ by restricting to $L$. Since $\psi\circ\phi = \id$, $\phi$ is injective. But
\[e_if_i = |D_i| \le |\Aut(\wh{L}_i/\wh{K})| \le [\wh{L}_i:\wh{K}] = e_if_i,\]
so all inequalities must be equal, and $\wh{L}_i/\wh{K}$ is Galois.
\end{proof}

\begin{prop}
Let $B_i$ be the valuation of $v_i$ on $L$. Then $B\otimes_A \wh{A} \cong \prod_i \wh{B_i}$.
\end{prop}

\begin{proof}
Both sides are free $\wh{A}$-modules of rank $n$. So it suffices to check isomorphism after reducing mod $\wh{p}$. The LHS reduces to $B/\fp B$, and the RHS reduces to $\prod_i B/\fq_i^{e_i} B$, and the two are equal by CRT.
\end{proof}


\section{Lecture, Oct. 21}

(The different and the discriminant)

Setup: AKLB. Recall that an \emph{$A$-lattice} $M$ is a finitely generated $A$-submodule of $L$, such that $MK = L$. Then we can define its \emph{dual} as
\[M^* = \{x\in L: \Tr(xm) \in A, \forall m\in M\}.\]
If $M$ is free, then so is $M^*$ (with the dual basis). If $M$ is a $B$-module (i.e. a fractional $B$-ideal), then so is $M^*$.

\begin{defn}
The \emph{different ideal} $\cD_{B/A}$ is defined as the inverse of the dual of $B$ as an $A$-lattice: 
\[\cD_{B/A} := (B^*)^{-1}.\]
\end{defn}

This is in fact an actual ideal inside $B$, since $B\subseteq B^* \implies (B^*)^{-1}\subseteq B$.

\begin{prop}
For any prime $\fp\subset A$, $(\cD_{B/A})_{\fp} = \cD_{B_{\fp}/A_{\fp}}$.
\end{prop}

\begin{prop}
For primes $\fq\mid \fp$, $\cD_{B/A}\cdot \wh{B_{\fq}} = \cD_{\wh{B_{\fq}}/\wh{A_{\fp}}}$. (Both sides are ideals in $\wh{B_{\fq}}$.)
\end{prop}

\begin{proof}
Assume WLOG $A$ is a DVR with maximal ideal $\fp$, by localizing. Let $\wh{L} = L\otimes_K \wh{K} = \prod_{\fq\mid\fp} \wh{L_{\fq}}$, and $\wh{B} = B\otimes_A \wh{A} = \prod_{\fq\mid\fp} \wh{B_{\fq}}$ (cf. previous lecture). Even though $\wh{L}$ may not be a field, it is still an \'etale $\wh{K}$-algebra, so the trace pairing is still nondegenerate. Consequently, we can form $\wh{B}^* = B^* \otimes_A \wh{A} =  \prod_{\fq\mid \fp} \wh{B_{\fq}}^*$. This shows that $B^*$ generates each $\wh{B_{\fq}}^*$ over $\wh{A}$, so $\cD_{B/A}$ generates $\cD_{\wh{B_{\fq}}/\wh{A_{\fp}}}$ as desired.
\end{proof}

The different $\cD_{B/A}$ is an ideal in $B$. We will define another ideal, the \emph{discriminant} $D_{B/A}$, which is an ideal in $A$. 

\begin{defn}
Given elements $e_1,\dots,e_n\in L$, their \emph{discriminant}
\[\disc(e_1,\dots,e_n) = \det (\Tr(e_ie_j))_{i,j}.\]
\end{defn}

This has the following properties:
\begin{itemize}
    \item If $e_1,\dots,e_n\in B$, $\disc(e_1,\dots,e_n)\in A$.
    \item Suppose $\phi\in \End_K(L)$ mapping $e_1,\dots,e_n$ to $e_1',\dots,e_n'$, then 
    \[\disc(e_1',\dots,e_n') = (\det\phi)^2\disc(e_1,\dots,e_n).\]
    \item Let $M$ be a free $A$-lattice. For two bases of $M$, their discriminants must differ by the square of a unit in $A$ (which must be 1 when $A=\ZZ$!)
\end{itemize}

\begin{defn}
Assuming AKLB and given an $A$-lattice $M$:
\begin{itemize}
    \item When $A=\ZZ$, $M$ is necessarily free, and $\disc M \in \ZZ$ is an integer (given by the discriminant of any set of $A$-basis of $M$).
    \item When $A$ is general and $M$ is a free $A$-module, the discriminant $D(M)$ is the principal (fractional) ideal generated by the discriminant of any basis of $M$.
    \item When $A,M$ are both general: the discriminant $D(M)$ is the $A$-module generated by $\disc(x_1,\dots,x_n)$ for any $n$ elements $x_1,\dots,x_n\in M$.
\end{itemize}
\end{defn}

\begin{prop}
The discriminant $D(M)$ is finitely generated over $A$, and therefore it is a fractional $A$-ideal.
\end{prop}

\begin{proof}
Choose independent elements $e_1,\dots,e_n\in M$ generating $L/K$, and let $N$ be the free $A$-lattice generated by them. Then $M\subseteq a^{-1}N$ for some $a\in A$, so $D(M)\subseteq D(a^{-1}N)$. The latter is generated by 1 element, so it is a Noetherian $A$-module, so $D(M)$ is finitely generated.
\end{proof}

\begin{prop}
For any prime $\fp\subset A$, $(D_{B/A})_{\fp} = D_{B_{\fp}/A_{\fp}}$.
\end{prop}

\begin{prop}
Let $L/K$ be a finite separable extension with degree $n$, and suppose $\sigma_i: L\to \Omega$ are $n$ distinct elements in $\Hom_K(L, \Omega)$. Then given $e_1,\dots,e_n\in L$, 
\[\disc(e_1,\dots,e_n) = \det(\sigma_i(e_j))_{i,j}^2.\]
\end{prop}


\begin{proof}
$\Tr(e_ie_j)_{ij} = (\sum_k \sigma_k(e_i) \sigma_k(e_j))_{ij} = (\sigma_k(e_i))_{ik}(\sigma_j(e_k))_{jk}$.
\end{proof}

\begin{prop}
For $x\in L$, 
\[\disc(1,x,x^2,\dots,x^{n-1}) = \prod_{i<j} (\sigma_i(x) - 
\sigma_j(x))^2.\]
\end{prop}

\begin{proof}
This is the Vandermonde determinant.
\end{proof}

\begin{defn}
If $f = \prod (x-\alpha_i)$, then the \emph{discriminant} of this polynomial 
\[\disc f = \prod_{i<j} (\alpha_i - \alpha_j)^2.\]
\end{defn}

\begin{prop}
If $A$ is a Dedekind domain, $f\in A[x]$ a monic separable polynomial, then $\disc(f) = \disc(1,x,x^2,\dots,x^{n-1})$.
\end{prop}

\begin{defn}
The \emph{discriminant} ideal $D_{B/A} = D(B)\subseteq A$, which is an actual ideal in $A$.
\end{defn}

\begin{exm}
$D_{\ZZ[i]/\ZZ} = (-4) = (4)$.
\end{exm}


\section{Lecture, Oct. 24}

(Detecting ramification, computing the different)

\begin{thm}
Assume AKLB, then
$D_{B/A} = N(\cD_{B/A})$, where $N$ is the ideal norm.
\end{thm}

\begin{proof}
Since everything is compatible with localization, WLOG $A$ is a DVR, so $B$ is free, say with basis $e_1,\dots,e_n$. Then $B^*$ is free also, with the dual basis $e_1',\dots,e_n'$. 

In general, if $m_1,\dots,m_n$ is an $A$-basis for another free lattice $M$, then $(\Tr(m_ie_j))$ is the change-of-basis matrix sending $e_1',\dots,e_n'$ to $m_1,\dots,m_n$. Setting $m_i = e_i$, we see that $(\Tr(e_ie_j))$ is the change-of-basis matrix sending $e_1',\dots,e_n'$ to $e_1,\dots,e_n$. Taking the ideal generated by the determinant on both sides, we see that $D_{B/A}$ is equal to the index $(B^*: B)_A = (B: (B^*)^{-1})_A = N(\cD_{B/A})$.
\end{proof}


\begin{thm}
Assume AKLB, $\fp\in A$, $\fq\mid \fp$. Then $L/K$ is unramified at $\fq$ iff $\fq\nmid \cD_{B/A}$.
\end{thm}

\begin{proof}
In the general case, first localize, then complete with respect to the unique discrete valuation to reduce to the case where $A$ is a \emph{complete} DVR. Then $B$ is a DVR as well, with $\fp B = \fq^e$. The different is a power of $\fq$, $\cD_{B/A} = \fq^m$, for some $m\ge 0$. Then $D_{B/A} = N(\cD_{B/A}) = \fp^{fm}$. Pick an $A$-basis $b_1,\dots,b_n$ of $B$, and let $\bar{b_1},\dots,\bar{b_n}$ be their images in $B/\fp B$. Then $L/K$ is unramified at $\fq$ if and only if $B/\fq^e = B/\fp B$ is a separable field extension of $A/\fp$, iff $\det(\Tr(\bar{b_i}\bar{b_j}))_{i,j}\neq 0$, iff $\bar{\det(\Tr(b_ib_j)_{i,j})} \neq 0$ mod $\fp$, iff $\fp\nmid D_{B/A}$, iff $\fq\nmid \cD_{B/A}$.
\end{proof}

\begin{cor}
Assume AKLB, $\fp\in A$, then $L/K$ is unramified at $\fp$ (i.e. unramified at all primes above $\fp$) iff $\fp\nmid D_{B/A}$.
\end{cor}

\begin{cor}
Only finitely many pimes of $B$ ramify.
\end{cor}

\begin{exm}
Take $A=\ZZ$, $K=\QQ$, $L=\QQ(\alpha)$ where $\alpha$ is a root of $x^3-x-1$. We wish to compute the ring of integers $\cO_K$. Clearly, $\ZZ[\alpha]\subseteq \cO_K$. Suppose $m$ is the index of $\ZZ[\alpha]$ in $\cO_K$. The discriminant $D(\ZZ[\alpha]) = \disc(1,\alpha,\alpha^2) = \disc(x^3-x-1) = -23$. But $\disc \cO_K = -23/m^2$ is necessarily an integer, so $m=1$ and $\cO_K = \ZZ[\alpha]$. Moreover, Dedekind-Kummer theorem tells us that the factorization of a prime $(p)$ in $\ZZ[\alpha]$ corresponds to factorization of $x^3-x-1$ modulo $p$. In the case $p=23$, the fact that $(23)\mid D_{L/K}$ corresponds to the fact that $x^3-x-1 = (x-10)^2(x-3)$ is ramified.
\end{exm}

There is a neat formula for the different in the case where $B$ is monogenic:

\begin{prop}
\label{different_f'}
If $B = A[\alpha]$, and $f$ is the minimal polynomial of $\alpha$, then $\cD_{B/A} = (f'(\alpha))$.
\end{prop}


\begin{lem}
Under the hypotheses above,
\[\Tr(\alpha^i/f'(\alpha)) = \begin{cases}
0, & \text{for } i=0,1,\dots,n-2\\
1, & \text{for } i=n-1
\end{cases}\]
and for all $i$, $\Tr(\alpha^i/f'(\alpha))\in A$.
\end{lem}

\begin{proof}
Expand both sides of
\[\frac{1}{f(x)} = \sum_{f(\beta) = 0} \frac{1}{(x-\beta)f'(\beta)}\]
at infinity, and compare the coefficients.
\end{proof}

\begin{proof}[Proof of \ref{different_f'}]
Let $I = (1/f'(\alpha)) \subseteq B^*$ be the fractional $B$-ideal, i.e. the $A$-span of $\alpha^i/f'(\alpha)$ for $i=0,\dots,n-1$. We compute
\[(B^*: I) = (\det(\Tr(\alpha^{i+j}/f'(\alpha))_{i,j})) = (1)\]
by the lemma, so $B^* = I$, and $\cD_{A/B} = (B^*)^{-1} = (f'(\alpha))$.
\end{proof}

\section{Lecture, Oct. 26}

(Geometric meaning of the different, unramified extensions of a complete DVR)

\begin{lem}
Assume AKLB. Let $\fa$ be a fractional ideal of $A$, $\fb$ a fractional ideal of $B$. Then $\Tr(\fb) \subseteq \fa$ iff $\fb \subseteq \fa B^*$.
\end{lem}

\begin{proof}
Assume WLOG $\fa\neq 0$. Then $\Tr(\fb) \subseteq \fa \iff \fa^{-1}\Tr(\fb) \subseteq (1) \iff \Tr(\fa^{-1}\fb) \subseteq (1) \iff \fa^{-1}\fb \subseteq B^* \iff \fb\subseteq \fa B^*$.
\end{proof}

\begin{prop}
For a tower AKBLCM, we have that
\[\cD_{C/A} = \cD_{C/B}\cD_{B/A}\]
as ideals of $C$, and
\[D_{C/A} = N_{L/K}(D_{C/B}) \cdot D_{B/A}^{[M:L]}\]
as ideals of $A$.
\end{prop}

\begin{proof}
For a fractional ideal $\fc$ of $C$, we have the following equivalence:
\begin{align*}
    \fc \subseteq \cD_{C/B}^{-1}
    &\iff \Tr_{M/L}(\fc) \subseteq B \\
    &\iff \cD_{B/A}^{-1}\Tr_{M/L}(\fc) \subseteq \cD_{B/A}^{-1} \\
    &\iff \Tr_{L/K}(\cD_{B/A}^{-1}\Tr_{M/L}(\fc)) \subseteq A \\
    &\iff \Tr_{L/K}(\Tr_{M/L}(\cD_{B/A}^{-1}\fc)) \subseteq A \\
    &\iff \Tr_{M/K}(\cD_{B/A}^{-1}\fc)) \subseteq A \\
    &\iff \cD_{B/A}^{-1}\fc \subseteq \cD_{C/A}^{-1} \\
    &\iff \fc \subseteq \cD_{B/A}\cD_{C/A}^{-1}.
\end{align*}
This implies $\cD_{C/B}^{-1} = \cD_{B/A}\cD_{C/A}^{-1}$, i.e. $\cD_{C/A} = \cD_{C/B}\cD_{B/A}$. Taking the ideal norm $N_{M/K}$ of both sides, we get the formula for the discriminant.
\end{proof}

Geometrically, the different ideal corresponds to the \emph{ramification divisor}. Fix an algebraically closed $k$, and let $K$ be a finite type $k$-algebra of transcendental degree 1. Then $K$ is a finite extension of $k(t)$, and there is an unique regular projective curve $X$ over $k$ whose function field $K(X) = K$. Here, $X$ serves as the analog of Dedekind rings --- the stalk at each non-generic point is a DVR. Moreover, any nonempty proper open subset of $X$ is $\Spec A$ for some Dedekind $A$. 

Now, suppose $L/K$ is a finite separable extension of degree $n$, and $L$ is the function field of another curve $Y$. Then there is a dominant morphism $\pi: Y\to X$, and for any nonempty proper open $\Spec A\subset X$, its preimage is $\Spec B\subset Y$. In this case, we return to our familiar AKLB setup, where an ideal of $B$ corresponds to an effective divisor on $\Spec B$. In the case of the different, because the different is compatible with localization, the corresponding divisors on $\Spec B$'s glue together to give a divisor on $Y$. This is called the ramification divisor $R$ if $\pi: Y\to X$, and the points that appear are exactly primes that ramify.

The ramification divisor appears in the Riemann-Hurwitz formula: $2g_Y - 2 = n(2g_X - 2) + \deg R$.


The following theorem is a continuation of our discussion about extensions of (complete) DVRs in Lectures 11, 14:

\begin{thm}
Let $A$ be a complete DVR with residue field $k$. Let $K = \Frac A$. Then there is an equivalence of categories between the category of finite unramified extensions $L/K$ and the category of finite separable extensions $k'/k$, given by the functor $F$ mapping $L$ to its residue field $k'$.
\end{thm}

\begin{proof}
It suffices to show the functor $F$ is essentially surjective and fully faithful.

Essentially surjective: consider a finite separable $k'/k$, say $k' = k[x]/(\bar{f}(x))$ with $\bar{f}(x)$ monic irreducible separable of degree $n$. Lift $\bar{f}$ to $f(x)\in K[x]$ (monic, irreducible and separable), and let $L = K[x]/(f(x))$. This is a finite separable extension of $K$, and suppose its Dedekind ring is $B$ with maximal ideal $\fq$. Then because $f$ is irreducible mod $\fq$, $L/K$ is unramified, with residue field $B \cong A[x]/(f(x))$, so that $B/\fq = A[x]/(f(x),\fq) = k[x]/(\bar{f}(x)) = k'$.
    
Fully faithful: The map of Homs is given by
\[\Hom_K(L_1,L_2) \to \Hom_A(B_1,B_2) \to \Hom_k(k_1',k_2').\]
The first map is bijective, with inverse given by tensoring a map $B_1\to B_2$ with $K$. So we focus on the second map. Write $k_1' = k[x]/(\bar{g}(x)) = k(\bar{\alpha})$, and lift $\bar{\alpha}$ to $\alpha\in B$.  Then $L_1 = K(\alpha)$, because $[L_1:K] = [k_1':k] = \deg \bar{g}(x)$ is at most the degree of the (monic) minimal polynomial $g(x)\in A[x]$ of $\alpha$. Then $B_1 = A[\alpha]$ as well, and $\Hom_A(B_1,B_2)$ then corresponds bijectively to the roots of $g$ in $B_2$. Similarly, $\Hom_k(k_1',k_2')$ corresponds to the roots of $\bar{g}$ in $k_2'$. But every root of $\bar{g}$ in $k_2'$ lifts uniquely to a root of $g$ in $B_2$, by Hensel's lemma. This finishes the proof. (In fact, here only the fact that $L_1/K$ is unramified is used, so $L_2/K$ does not need to be unramified for this to hold.)
\end{proof}


\section{Lecture, Oct. 28}

(Totally ramified extensions of a complete DVR, Krasner's lemma)

Suppose $K$ is a local field, and fix a separable closure $K^{\sep}/K$. The maximal unramified extension of $K$ can be defined as
\[K^{\unr} = \bigcup_{\substack{K'\subseteq K^{\sep} \\ K'/K \text{ f. unram.}}} K'.\]

\begin{exm}
Consider the case $K = \QQ_p$. Because $k = \FF_p$, the only finite separable extensions of $k$ are $\FF_{p^n}$, one for each $n$. As such, there is one unramified extension of $\QQ_p$ of degree $n$ for each $n$. Therefore, $\QQ_p^{\unr}/\QQ_p$ is an infinite Galois extension, with Galois group the profinite integers
\[\Gal(\QQ_p^{\unr}/\QQ_p) = \Gal(\bar{\FF_p}/\FF_p) = \varprojlim \Gal(\FF_{p^n}/\FF_p) = \varprojlim \ZZ/n\ZZ = \wh{\ZZ} = \prod_{\ell \text{ prime}} \ZZ_{\ell}.\]
Note that $\QQ_p^{\unr}$ has value group $\ZZ$ and residue field $\bar{\FF_p}$.
\end{exm}

Now, we show that any finite extension can be broken down into an unramified part and a totally ramified part. Let $A$ be a complete DVR, $K = \Frac A$ with residue field $k$, $L/K$ f. sep. and with residue field $\ell$. Assuming that $\ell/k$ is separable (which is true e.g. for number fields), each unramified subextension of $L/K$ corresponds to a separable subextension of $\ell/k$, which is contained in $\ell$. So the unramified subextension $K'/K$ corresponding to $\ell/k$ contains all unramified subextensions of $L/K$. We have $[K':K] = [\ell:k] = f$, so $[L:K'] = e$. Also, $f = 1$ for the extension $L/K'$, so in fact it is totally ramified. Furthermore, if $L/K$ is Galois, then $\Gal(L/K') = I_{L/K'} = I_{L/K}$ since everything has size $e$.

Next, we study totally ramified extensions. Assume AKLB with $A,B$ complete DVRs, $L/K$ totally ramified with residue field $k$, and let $p = \characteristic k$. 

\begin{defn}
Say $L/K$ is \emph{tamely ramified} if $p\nmid e$ (which is automatically true when $k$ has characteristic 0). Otherwise, say $L/K$ is \emph{wildly ramified}.
\end{defn}

For example, $L = K(\pi^{1/e}) = K[x]/(x^e - \pi)$ is a totally ramified extension of degree $e$ (here $\pi$ is a uniformizer in $K$). It turns out that all \emph{tamely} ramified extensions must be of this form:

\begin{thm}
\label{TotallyTamelyRam}
Assume AKLB as above, $L/K$ totally tamely ramified of degree $e$. Then $L = K(\pi^{1/e})$ for some uniformizer $\pi$.
\end{thm}

\begin{proof}
Choose uniformizers $\pi_K$ of $K$, $\pi_L$ of $L$. Then $[L:K]\ge [K(\pi_L):K] \ge e = [L:K]$, so $L = K(\pi_L)$. We have $\pi_L^e = u\cdot \pi_K$ for some unit $u$ of $B$. We wish to get rid of that unit to conclude $L = K(\pi_K^{1/e})$. This requires us to use the tamely ramified condition.

Because $A$ and $B$ have the same residue field, we may assume WLOG $u\equiv 1\pmod{\fq}$ by adjusting $\pi_K$ by a unit in $A$. Now, the polynomial $x^e - u = 0$ has a simple root of $1$ in $k$ (since $e\neq 0$), so by Hensel's lemma it has a root in $B$. In other words, $u$ has an $e$-th root in $B$, so we're done.
\end{proof}


\begin{lem}[Krasner's lemma]
Let $K$ be a field complete with respect to a nontrivial non-archimedean absolute value, and $\bar{K}$ a separable closure of $K$. Given an element $\alpha\in \bar{K}$, let its Galois conjugates be $\alpha_i$. If an element $\beta\in \bar{K}$ is such that $|\alpha-\beta| < |\alpha-\alpha_i|$ for all $i$, then $K(\alpha)\subseteq K(\beta)$.
\end{lem}

\begin{proof}
Suppose for contradiction that $\alpha\notin K(\beta)$. Then there exists $\sigma \in \Aut(\bar{K}/K(\beta))$ sending $\alpha$ to $\sigma\alpha\neq\alpha$. Then $|\alpha - \beta| = |\sigma(\alpha-\beta)| = |\sigma\alpha - \beta| > |\alpha - \beta|$, which is a contradiction.
\end{proof}


\section{Lecture, Oct. 31}

(Continuity of roots, lattices in $\RR^n$)

We use Krasner's lemma to derive a result known as ``continuity of roots''.

\begin{prop}[Continuity of roots]
Let $K$ be a field complete wrt a nontrivial nonarchimedean absolute value. Then we can uniquely extend the absolute value to $\bar{K}$. Let $f\in K[x]$ be a separable monic irreducible degree $n$ polynomial. If $g\in K[x]$ of degree $n$ has all coefficients sufficiently close to $f$'s, then the following holds:
\begin{itemize}
    \item Each root $\beta$ of $g$ belongs to a root $\alpha$ of $f$;
    \item $K(\beta) = K(\alpha)$;
    \item $g$ is separable and irreducible.
\end{itemize}
\end{prop}

\begin{proof}
To start with, it is clear that when $f$ and $g$ are close enough, the roots of $g$ have absolutely bounded size. This is because if $g(\beta) = 0$ where $g(x) = b_nx^n + \dots + b_0$, then
\[|b_n||\beta|^n = |b_{n-1}\beta^{n-1}+\dots+b_0|\le \max(|b_{n-1}||\beta|^{n-1},\dots,|b_0|).\]
Now, since $|\beta|$ is bounded by an absolute constant, we have for $f,g$ close enough, if $g(\beta) = 0$,
\[\prod_{i=1}^n (\beta - \alpha_i) = f(\beta) \approx g(\beta) = 0.\]
So one of the factors $|\beta - \alpha_i| $ must be small. When $f,g$ are sufficiently close, we can force it to be smaller than all $|\alpha_i - \alpha_j|$ for $i\neq j$, so $\beta$ belongs to some $\alpha_i$. Then Krasner's lemma implies $K(\beta) \supseteq K(\alpha_i)$, but the former is of degree at most $n$ over $K$ and the latter is of degree $n$, so $K(\beta) = K(\alpha_i)$ and $g$ is irreducible and separable. 
\end{proof}



\begin{cor}
Let $K$ be a degree $n$ extension of $\QQ_p$. Then there exists a degree $n$ number field $F$ contained in $K$, such that $F\QQ_p = K$.
\end{cor}

\begin{proof}
Let $K = \QQ_p(\alpha) = \QQ_p[x]/(f(x))$ where $f$ is the min. poly of $\alpha$. Since $\QQ$ is dense in $\QQ_p$, we may approximate $f$ arbitrarily well by some $g\in \QQ[x]$. By the continuity of roots, $g$ is separable, irreducible, and has a root $\beta\in \bar{K}$ such that $\QQ_p(\beta) = \QQ_p(\alpha) = K$. Let $F = \QQ(\beta)$, then $F$ is a degree $n$ number field such that $F\QQ_p = \QQ_p(\beta) = K$.
\end{proof}

\begin{cor}
Choose an algebraic closure $\bar{\QQ_p}$ of $\QQ_p$. Let $\bar{\QQ}$ be the algebraic closure of $\QQ$ inside $\bar{\QQ_p}$. Then $\bar{\QQ}\QQ_p = \bar{\QQ_p}$.
\end{cor}

\begin{cor}
The map $\Gal(\bar{\QQ_p}/\QQ_p) \to \Gal(\bar{\QQ}/\QQ)$ given by $\sigma \mapsto \sigma|_{\bar{\QQ}}$ is injective. (The image is called the decomposition subgroup.)
\end{cor}

Remark: $\Gal(\bar{\QQ_p}/\QQ_p)$ is a pro-solvable group, while $\Gal(\bar{\QQ}/\QQ)$ is very poorly understood.



We move on to lattice methods in studying number fields (finite extensions of $\QQ$).

\begin{defn}
Let $V$ be a $n$-dimensinoal $\RR$-vector space. A \emph{lattice} in $V$ is a subgroup
\[\Lambda = \ZZ e_1 + \dots + \ZZ e_m\]
for some linearly independent $e_1,\dots,e_m$. It is \emph{full} if $m=n$.
\end{defn}

\begin{prop}
Let $\Lambda \subset V$ be a subgroup, then $\Lambda$ is discrete iff $\Lambda$ is a lattice.
\end{prop}

Equip $V$ with the dot product in $\RR^n$ so that we pin down the unit length. Then we get a unique Haar measure on $V$, such that $V$ together with the measure is isomorphic to $\RR^n$.

\begin{defn}
For a set $X$ and a $\sigma$-algebra $\Sigma$ on $X$, a map $\mu: \Sigma \to \RR\cup\{\pm\infty\}$ is a \emph{measure} if:
\begin{itemize}
    \item $\mu(\varnothing) = 0$;
    \item $\mu(E) \ge 0$ for all $E\in \Sigma$;
    \item For a countable family of pairwise disjoint sets $E_i\in \Sigma$, $\mu(\bigcup_i E_i) = \sum_i\mu(E_i)$.
\end{itemize}
\end{defn}

\begin{thm}[Haar's theorem]
\label{Haar}
Let $G$ be a locally compact Hausdorff topological group. A \emph{Borel set} is an element in the \emph{Borel algebra}, i.e. the $\sigma$-algebra generated by open sets of $G$. There is a unique (up to scaling) nontrivial measure $\mu$ on the Borel algebra such that:
\begin{itemize}
    \item $\mu(gS) = \mu(S)$ (left translation-invariant);
    \item $\mu(K) < \infty$ for $K$ compact;
    \item $\mu(S) = \inf\{\mu(U): S\subseteq U, U\text{ open}\}$;
    \item $\mu(U) = \sup\{\mu(K): K\subseteq U, K\text{ compact}\}$ for $U$ open.
\end{itemize}
\end{thm}

For a full lattice $\Lambda = \ZZ e_1 + \dots + \ZZ e_n$, let 
\[F = \{a_1e_1 + \dots + a_ne_n : 0\le a_i < 1\}.\]
Then $\RR^n = \coprod_{\lambda\in \Lambda} (F+\lambda)$. Also, $\vol(F) = |\det(e_1,\dots,e_n)| = \sqrt{\det (\langle e_i,e_j\rangle)_{i,j}}$.

More generally:

\begin{defn}
A \emph{fundamental domain} for $\Lambda\subset V$ is a measurable $F\subset V$ such that $V = \coprod_{\lambda\in \Lambda} (F+\lambda)$.
\end{defn}

\begin{prop}
If $F,G$ are two fundamental domains then they have the same volume.
\end{prop}

\begin{proof}
For each $\lambda\in \Lambda$, $(F+\lambda)\cap G$ is a translate of $F\cap(G-\lambda)$, so they have the same volume. Taking the sum over $\lambda\in \Lambda$, we get $\vol(G) = \vol(F)$.
\end{proof}

\begin{defn}
The \emph{covolume} $\covol(\Lambda)$ of a full lattice $\Lambda$ is defined to be the volume of any fundamental domain of $\Lambda$.
\end{defn}

\begin{prop}
Suppose $\Lambda \supseteq \Lambda'$ are full lattices, then
\[\covol(\Lambda') = (\Lambda:\Lambda')\covol(\Lambda).\]
\end{prop}

\section{Lecture, Nov. 2}

(Minkowski's lattice point theorem, Places)

\begin{lem}
Let $S\subset \RR^n$, $\vol(S) > 1$. Then there exist distinct $s,s'\in S$, such that $s-s'\in \ZZ^n$.
\end{lem}

\begin{proof}
Cut up $\RR^n$ into unit cubes, and translate pieces of $S$ into $[0,1)^n$. They must overlap.
\end{proof}

\begin{thm}[Minkowski's lattice point theorem for $\ZZ^n$]
Let $S\subset \RR^n$ be a symmetric convex region such that $\vol(S) > 2^n$. Then $S$ contains a nonzero lattice point.
\end{thm}

\begin{proof}
The dilation $\frac12 S$ must contain two distinct points $\frac12 s,\frac12 s'$ where $\frac12(s-s')\in \ZZ^n$, which is the point we want.
\end{proof}

\begin{thm}[Minkowski's lattice point theorem, full version]
Let $V$ be a finite dimensional $\RR$-vector space, $\Lambda$ a full lattice, $S\subset V$ a symmetric convex region with $\vol(S) > 2^n\covol(\Lambda)$, then it contains a nonzero lattice point.
\end{thm}

As an application, we prove the following classical result:

\begin{thm}
If $p\equiv 1\pmod{4}$ is a prime, then $p=x^2 + y^2$ for $x,y\in \ZZ$.
\end{thm}

\begin{proof}
Because $(\frac{-1}{p})=1$, there exists $i\in \FF_p$ with $i^2+1\equiv 0\pmod{p}$. Let $\Lambda\subset \ZZ^2$ be the lattice consisting of points $\lambda \pmod{p}$ that is a multiple of $(1,i)$ mod $p$. Clearly, $\Lambda$ has index $p$ in $\ZZ^2$, so $\covol(\Lambda) = p$. Let $S = \{x\in \RR^2: |x| < \sqrt{2p}\}$. Then $|S| = 2p\pi > 4p = 2^2\covol(\Lambda)$, so $S$ contains a lattice point in $\Lambda$, which is necessarily a solution to $x^2+y^2=p$.
\end{proof}


We transition to a discussion of places, which are like primes but generalizes to the archimedean case as well.

\begin{defn}
A \emph{global field} is a finite extension of $\QQ$ or $\FF_q(t)$.
\end{defn}

\begin{thm}
    The category of global function fields with field inclusions is equivalent to the category of smooth projective curves with dominant rational maps, via $X\mapsto K(X)$.
\end{thm}


Let $K$ be a number field.

\begin{defn}
A \emph{place} of $K$ is an equivalence class of nontrivial absolute values on $K$. The set of all places is commonly denoted by $M_K$.
\end{defn}


By Ostrowski's theorem, $M_{\QQ}$ corresponds set-theoretically with $\Spec \ZZ$. Every place $v\in M_K$ is an extension of $|\thinspace|_p$ for some $p\le \infty$ (we write $v\mid p$ for this). We already know that places $v\mid p$ for finite $p$ correspond bijectively to primes $\fq\mid (p)$.

\begin{prop}
$v$ is archimedean if $v\mid \infty$, and nonarchimedean otherwise.
\end{prop}


\begin{proof}
Complete wrt $v$ to get an extension $K_v/\QQ_p$, and use theorem \ref{extend_abs_val}.
\end{proof}


\begin{lem}
Suppose $K = \QQ(\alpha)$. If $v\mid p$ for $p\le \infty$, then $K_v = \QQ_p(\alpha)$.
\end{lem}

\begin{proof}
Consider $\QQ_p(\alpha)$, which must be contained in $K_v$. The absolute value on $\QQ_p$ then extends uniquely to an absolute value on $\QQ_p(\alpha)$, under which $\QQ_p(\alpha)$ is complete. Since this absolute value concides with that of $K_v$ and $K\subset \QQ_p(\alpha)$, we have $K_v = \QQ_p(\alpha)$.
\end{proof}


The minimal polynomial of $\alpha$ in $K_v$ is then an irreducible factor of the min. poly of $\alpha$ in $K$. Conversely, any irreducible factor gives a finite extension $F/\QQ_p$, which is equipped with a complete absolute value, and there is a unique extension $K\into F$, which is the completion of $K$ wrt that absolute value. Therefore we have

\begin{thm}
$K\otimes_{\QQ} \QQ_p \cong \prod_{v\mid p} K_v$, for $p\le \infty$.
\end{thm}

\begin{exm}
If $v\mid\infty$, then $K_v$ is a finite extension of $\RR$, so either $\RR$ or $\CC$. Suppose $K = \QQ[x]/f(x)$, then $f(x)$ in $\RR[x]$ factors as the product of $r_1$ linear factors and $r_2$ quadratic factors. The linear factors $(x-a)$ correspond to embeddings $K\into \RR$ mapping $x\mapsto a$ (these are the ``real places''), and the quadratic factors $(x-z)(x-\bar{z})$ correspond to pairs of embeddings $K\into \CC$ mapping $x\mapsto z$ or $\bar{z}$ (these are the ``complex places''). Then $r_1+2r_2 = [K:\QQ]$ by counting degrees.
\end{exm}

\begin{cor}
The places $v\mid p$ correspond bijectively to $\Hom_{\QQ}(K, \bar{\QQ_p}) / \Gal(\bar{\QQ_p}/\QQ_p)$.
\end{cor}


\section{Lecture, Nov. 4}

(Product formula, orders, embedding $\cO$ as a lattice)

\begin{defn}
If $v\mid p$, the \emph{normalized absolute value} on $K_v$ is $|x|_v = |\N_{K_v/\QQ_p} (x)|_p$.
\end{defn}

\begin{prop}
Suppose $p$ is finite, $v\mid p$, and let $\cO_v$ be the DVR in $K_v$. If $x\in \cO_v$, then
\[|x|_v := (\# \cO_v/x\cO_v)^{-1}.\]
\end{prop}

\begin{proof}
We have $(\N_{K_v/\QQ_p}(x)) = N(x\cO_v) = (\cO_v: x\cO_v)_{\ZZ_p} = \chi(\cO_v/x\cO_v) = (\# \cO_v/x\cO_v)$. Taking $|\thinspace|_p$ on both sides gives us the formula.
\end{proof}

\begin{exm}
If $v$ is complex, then $|x|_v = |x|^2$, which is actually not an absolute value! In general, $|x|_v = |x|_p^{[K_v:\QQ_p]}$ for $x\in \QQ$. This normalization is ``intrinsic'', because given $x\in \cO_K$, multiplication by $x$ scales the Haar measure on $K_v$ by a factor of $|x|_v$.
\end{exm}

\begin{thm}[Product formula]
If $x\in K^{\times}$, then $\prod_{v\in M_K} |x|_v$ makes sense and is equal to 1.
\end{thm}

\begin{proof}
$\N_{K/\QQ}(x) = \N_{K\otimes_{\QQ} \QQ_p/\QQ_p}(x) = \prod_{v\mid p} \N_{K_v/\QQ_p}(x)$, so taking $|\thinspace|_p$ on both sides gives us
\[|\N_{K/\QQ}(x)|_p = \prod_{v\mid p} |x|_v.\]
Taking the product over all $p$ an using the product formula for $\QQ$, we get the desired formula.
\end{proof}

We are on our way to apply Minkowski's lattice point formula to say something nontrivial about the ideal class group. 


\begin{defn}
An \emph{order} in a number field $K$ is a subring $\cO$ of finite index in $\cO_K$.
\end{defn}

Equivalently, $\cO$ is a $\ZZ$-lattice in $K$ that is also a ring.

For an order $\cO$, we have the following inclusions: 
\[\cO \into K \into K_{\RR} := K\otimes_{\QQ}\RR \into K_{\CC}:= K_{\RR}\otimes_{\RR} \CC.\]
Thus, $\cO$ is a lattice in the $\RR$-vector space $K_{\RR}$. The canonical Hermitian inner product on $\CC^n$ restricts to an inner product on $K_{\RR}\cong \RR^n$ (note that this inner product is not equal to the canonical one on $\RR^n$: for example, $(x,y) = x+yi\in \CC$ is embedded as $(x+yi, x-yi)\in \CC^2$, so $(x+yi, x-yi)\cdot (z+wi, z-wi) = 2(xy+zw) = 2(x,y)\cdot (z,w)$. Consequently, the volume under this inner product is scaled by a factor of $2^{r_2}$). For $x,y\in K$, we then get an inner product
\[\langle x, y \rangle = \sum_{\sigma: K\into \CC} \sigma x \cdot \bar{\sigma y}.\]

\begin{prop}
$\covol(\cO) = \sqrt{|\disc \cO|}$.
\end{prop}

\begin{proof}
Let $e_1,\dots,e_n$ be a $\ZZ$-basis of $\cO$. Let $A = (\sigma(e_j))_{\sigma,j}\in M_{n\times n}(\CC)$. Then $|\disc \cO| = (\det A)^2$. But $\covol(\cO)^2 = \det \langle e_i, e_j\rangle = \det(\sum_{\sigma} \sigma e_i \cdot \sigma e_j) = |\det A|^2$. So $\covol(\cO) = \sqrt{|\disc \cO|}$.
\end{proof}

\begin{cor}
Suppose $I$ is an invertible fractional $\cO$-ideal, then $\covol(I) = \sqrt{\disc \cO}\cdot N(I)$.
\end{cor}



\section{Lecture, Nov. 7}


(Finiteness of class number, and other applications)

Now we are ready to apply Minkowski's lattice point theorem to show that every fractional ideal contains a relatively short vector. Use $(r,s)$ to denote $(r_1,r_2)$.


\begin{thm}
Let $K$ be a number field, $\cO$ an order. Let $m = \frac{n!}{n^n} (\frac{4}{\pi})^{s} \sqrt{|\disc \cO|}$, then for any invertible fractional $\cO$-ideal $I$, there exists a nonzero $a\in I$, such that $|N(a)|\le m\cdot |N(I)|$.
\end{thm}

\begin{proof}
Let $S = \{z = (z_{\sigma})_{\sigma\in \Hom_{\QQ}(K,\CC)} \in K_{\RR}: \sum |z_{\sigma}| < t\}$, where $t$ is a constant we fix later. Then it is not hard to show that $\vol(S) = 2^r \pi^s \frac{t^n}{n!}$. Choose $t$ such that $\vol(S) > 2^n\covol(I)$. By Minkowski's lattice point theorem, there exists nonzero $a\in I$ lying in $S$, such that $t > \sum_{\sigma} |\sigma a|\ge n\sqrt[n]{\prod_{\sigma} |\sigma a|} = n\sqrt[n]{|\N_{K,\QQ}(a)|}$. We know that $t$ can be chosen to be arbitrarily close to $\sqrt[n]{(\frac{4}{\pi})^s n! \sqrt{\disc \cO} \cdot |N(I)|}$ from above, so we see that
\[|N(a)| = |\N_{K,\QQ}(a)| \le \frac{n!}{n^n} \pr{\frac{4}{\pi}}^{s} \sqrt{|\disc \cO|} \cdot |N(I)| = m\cdot |N(I)|,\]
as desired.
\end{proof}

\begin{cor}
Every ideal class contains an integral ideal of norm at most $m$.
\end{cor}

\begin{proof}
Let $[I]$ be the inverse of the target ideal class, then there exists $a\in I$, such that $|N(a)|\le m\cdot |N(I)|$. This means that $(a)I^{-1}$ is an integral ideal in the target ideal class, whose norm is at most $m$.
\end{proof}

\begin{lem}
There are finitely many ideals of norm at most $m$.
\end{lem}

\begin{proof}
It suffices to show that $\ZZ^n$ has finitely many subgroups of a given index. This is because any subgroup of index $q$ contains $(q\ZZ)^n$, so there can only be finitely many.
\end{proof}

\begin{thm}
The class group of a number field is finite.
\end{thm}

\begin{prop}
\label{disc_lower_bound}
$\sqrt{\disc \cO_K} \ge \frac{n^n}{n!} (\frac{\pi}{4})^{s} \ge \frac{n^n}{n!} (\frac{\pi}{4})^{n/2}$.
\end{prop}

\begin{proof}
Take $I$ to be the unit ideal, so that its norm is 1. Because the norm of any nonzero element is at least $1$, $m\ge 1$.
\end{proof}

\begin{cor}
If $K\neq \QQ$, then $|\disc \cO_K|>1$. In other words, there are no everywhere unramified nontrivial extensions of $\QQ$.
\end{cor}


\begin{prop}
There are finitely many number fields $K$ with $|\disc \cO_K| < B$, for any real $B$.
\end{prop}

\begin{proof}

By proposition \ref{disc_lower_bound}, it suffices to show that there are finitely many such number fields of any fixed degree $n$. 


Case 1: $K$ is totally real. Let $S := \{(x_1,\dots,x_n)\in \RR^n: |x_1| \le 2B^{1/2}, |x_i| < 1 \text{ for all } i\neq 1\}$. Then $\vol(S) \approx 2^{n+1}B^{1/2} > 2^n |\disc \cO_K|^{1/2} = 2^n\covol(\cO_K)$. By Minkowski, there exists a nonzero $\alpha\in \cO_K \subset \RR^n$ in $S$. Then $\prod |\alpha_i| = |N(\alpha)| \ge 1$ while $|\alpha_2|,\dots,|\alpha_n| < 1$, which forces $|\alpha_1| > 1$. If $\QQ(\alpha)\neq K$, then each $\alpha_i$ will be repeated $[K:\QQ(\alpha)]$ times (by the norm formula), which is not the case because $\alpha_1$ is the only one with absolute value larger than 1. The minimal polynomial of $\alpha$, which is in $\ZZ[x]$, has finitely many possibilities, since its coefficients, as symmetric functions in its roots which have bounded sizes, have bounded sizes. So there are only finitely many possibilities for $K$ also.

Case 2: The signature of $K$ is $(r,s)$, then $K_{\RR}\cong \RR^r \times \CC^s$. Let $S := \{(x_1,\dots,x_r, z_1,\dots,z_s)\in \RR^r\times \CC^s: |z_1|^2 \le cB^{1/2}, |x_i|, |z_j| < 1 \text{ for all } i \text{ and for all } j\neq 1\}$, where $c$ is large enough that $\vol(S) > 2^n\covol(\cO_K)$. The argument in Case 1 continues verbatim.
\end{proof}

\begin{lem}
Let $K$ be a number field of degree $n$, then for any prime $p$, $v_p(D_K) \le n\floor{\log_p n} + n - 1$. 
\end{lem}

\begin{proof}
We have $v_p(D_K) = v_p(N(\cD_K)) = \sum_{\fq\mid p} f_{\fq}v_{\fq}(\cD_K) \le \sum_{\fq\mid p} f_{\fq}(e_{\fq}-1+v_p(e_{\fq}))\le n-1+n\floor{\log_p n}$ by trivial bounding.
\end{proof}


\begin{thm}[Hermite]
Let $S$ be a finite set of places of $\QQ$, and let $n$ be an integer. Then there are finitely many number fields $K$ of degree $n$ that are unramified outside $S$.
\end{thm}

\begin{proof}
Each valuation $v_p(D_K)$ is bounded, so $D_K$ is bounded, so there are finitely many $K$'s.
\end{proof}


\section{Lecture, Nov. 9}

(Ad\`ele ring)


Let $K$ be a global field, $v$ a place, $\cO_v$ the valuation ring of $K_v$ (defined to be equal to $K_v$ when $v$ is archimedean). The normalized absolute value induces a topology on $K_v$, under which it is locally compact. Furthermore, if $v$ is nonarchimedean, $\cO_v$ is compact.

We now define the \emph{ad\`ele ring} of $K$, which will be a topological ring:

\begin{defn}
The ad\`ele ring $\AA_K$ of a global field $K$ is the \emph{restricted product}
\[\prod\nolimits_v ' (K_v, \cO_v),\]
which as a set is equal to
\[\{(a_v)\in \prod K_v: \text{all but finitely many } a_v\in \cO_v\}.\]
It is easy to verify that this forms a ring. The topology on this is \emph{finer than} the subset topology of the product topology; instead, a base is given by open sets of the form $\prod_v U_v$, where $U_v\subseteq K_v$ are open and all but finitely many $U_v = \cO_v$. (In particular, $\prod_v \cO_v$ is a locally compact open.)
\end{defn}

\begin{prop}
$\AA = \AA_K$ is locally compact.
\end{prop}

\begin{proof}
$\prod_v \cO_v$ is a locally compact neighborhood of $0$.
\end{proof}


Because any element of $K$ has only finitely many absolute values where it is 1, $K$ embeds into $\AA_K$ naturally.

\begin{prop}
\label{extension_adele}
If $L/K$ is a finite separable extension of global fields, $\AA_L \cong L\otimes_K \AA_K$ as topological rings.
\end{prop}

In fact, $K\into \AA$ is very much like the embedding $\ZZ\into \RR$:

\begin{thm}
$K$ is a discrete subgroup of $\AA$, and $\AA/K$ is compact.
\end{thm}


\begin{proof}
We only prove this for $K = \QQ$, and the number field case then follows from proposition \ref{extension_adele}. The function field case follows from a similar argument, so we focus on $\QQ$ form here.

Discreteness of $\QQ$: $U = (-1,1)\times \prod_p \ZZ_p$ is an open neighborhood of 0 that contains no points of $\QQ$.

Compactness of $\AA/\QQ$: we claim that $\AA = \QQ + [0,1]\times \prod_p \ZZ_p$. Given $x = (x_p)_{p\le \infty}$, expand $x_p$ in powers of $p$, and let $y_p$ be the decimal part of $x_p$ (i.e. $x_p-y_p\in \ZZ_p$ and the denominator of $y_p$ is a power of $p$). Almost all $y_p$ are zero, so it makes sense to talk about $x - \sum_p y_p$, which belongs to every $\ZZ_p$. Now adjust by an integer to get in $[0,1]$.
\end{proof}


\section{Lecture, Nov. 14}

(Id\`ele group)

\begin{defn}
The \emph{id\`ele group} is $\AA^{\times} = \prod_v' (K_v^{\times}, \cO_v^{\times})$, with the restricted product topology.
\end{defn}


Remark:
This is finer than the topology inherited as a subspace of $\AA$! For example, $\prod \cO_v^{\times}$ is open in $\AA^{\times}$ but not in $\AA$. But the topology is induced from $\AA$ via the map $\AA^{\times} \into \AA\times \AA$, $x\mapsto (x, x^{-1})$.

\begin{prop}
$K^{\times}$ is discrete in $\AA^{\times}$.
\end{prop}

\begin{proof}
$K^{\times} = \AA^{\times} \cap (K\times K)$ inside $\AA\times \AA$, so it is discrete.
\end{proof}


\begin{defn}
For an idele $a = (a_v)_v \in \AA^{\times}$, define $|a| = \prod_v |a_v|_v$.
\end{defn}

This is also the correct notion of ``size'' in terms of scaling the Haar measure.

\begin{defn}
The group $\AA_1^{\times} = \ker(\AA^{\times} \xto{|\thinspace|} \RR_{>0}^{\times})$.
\end{defn}

\begin{prop}
$K^{\times}$ embeds into $\AA^{\times}_1$.
\end{prop}


Let $K$ be a number field for now. We also have a natural map $\AA^{\times} \to \cI$ assembled from each $K_{\fp}^{\times} \xto{v_p} \ZZ$. This is surjective (in contrast to the case where $\AA^{\times}$ is replaced with the group of principal fractional ideals, when there is a problem with the class group). Also, it is clear that $\ker(\AA^{\times} \to \cI) = \prod_{v} \cO_v^{\times}$.


The ideal class group $\Cl(\cO_K)$ can be recast in adelic language:
\[\Cl(\cO_K) = \cI/K^{\times} = \frac{\AA^{\times}}{K^{\times}\cdot \prod \cO_v^{\times}}.\]

\begin{defn}
The \emph{id\`ele class group} is $\AA^{\times}/K^{\times}$.
\end{defn}

\begin{thm}
$\AA_1^{\times}/K^{\times}$ is compact.
\end{thm}

This is a hard theorem and directly implies finiteness of the class group.


\begin{defn}
For $d = (d_v)_v \in \AA^{\times}$, define the adelic parallelotope (box)
\[\mathbb{L}(d) := \{(x_v)\in \AA: |x_v|_v \le |d_v|_v \text{ for all } v\}, \]
and
\[L(d) = \mathbb{L}(d)\cap K\]
is the set of lattice points in the box.
\end{defn}

Clearly $\mathbb{L}(d)$ is a compact neighborhood of 0. In the function field case, $L(d)$ is like ``functions with prescribed orders of poles and zeroes'', cf. Riemann-Roch. Since $K$ is discrete in $\AA$, $L(d)$ is discrete and compact, hence finite.


\begin{thm}[Adelic Minkowski]
\label{adelic_minkowski}
There exists a constant $c$ (depending on $K$), such that if $|d| > c$, then $L(d)$ contains a nonzero element.
\end{thm}


\begin{proof}
We only prove this for number fields. Then $d$ maps to some ideal $I$. For $x\in K$, unwrapping the condition $|x|_v \le |d_v|_v$ for archimedean and nonarchimedean $v$, we need $x\in I$ (which is a lattice in $K_{\RR}$) and $x$ belongs to a product of intervals and disks, a symmetric convex set in $K_{\RR}$. When $|d|$ is big enough, Minkowski's theorem applies and we get a nonzero point in $L(d)$.
\end{proof}



\section{Lecture, Nov. 16}

(Strong approximation, compactness of $\AA^{\times}_1/K^{\times}$)



Let $K$ be a global field. For any finite set of places $S$ containing all archimedean places, the \emph{$S$-integral ad\`eles} are elements of the ring
\[\AA_S := \prod_{v\in S} K_v \times \prod_{v\notin S} \cO_v.\]
This is equipped with the actual product topology. For $S\subseteq T$, there is a natrual $\AA_S\into \AA_T$. Then
\[\AA = \varinjlim_{S} \AA_S.\]

Here is a corollary of the adelic Minkowski theorem, about scaling a box by elements of $K^{\times}$ to fit in another box.


\begin{cor}
If $a,b\in \AA^{\times}$ such that $|b| > c|a|$ (where) $c$ is as in theorem \ref{adelic_minkowski}. Then there exists $u\in K\cross$, such that $u\LL(a)\subseteq \LL(b)$.
\end{cor}


\begin{proof}
By adelic Minkowski, there exists $u\in K\cross$ such that $u\in \LL(b/a)$. This is the same as $u\LL(a)\subseteq \LL(b)$.
\end{proof}



\begin{lem}

There exists $a\in \AA\cross$, such that $\AA = K + \LL(a)$.

\end{lem}



\begin{proof}
Just for this problem, let $\LL(d)'$ be the same as $\LL(d)$, except it is open for archimedean $v$. These are open neighborhoods of 0 that cover $\AA$, so their images in $\AA/K$ cover $\AA/K$. The image is open because its preimage is the union of all translations of $\LL(d)'$ by elements of $K$. Since $\AA/K$ is compact, we need only finitely many $\LL(d)'$s whose images cover $\AA/K$. So we can choose $a$ big enough such that $\LL(a)$ contains each of the finitely many $\LL(d)'$s.
\end{proof}

\begin{lem}
If $b\in \AA^{\times}$, $|b|$ sufficiently large, then $\AA = K + \LL(b)$. 
\end{lem}

\begin{proof}
Combine the previous two claims.
\end{proof}


\begin{thm}[Strong approximation]
Let $K$ be a global field, and suppose that the set of places of $K$ are partitioned into $S\sqcup T \sqcup \{w\}$, where $S$ is finite. For each $v\in S$, fix $a_v \in K_v$ and a real $\eps_v > 0$. Then there exists $x\in K$ such that:
\begin{itemize}
    \item $|x - a_v|_v \le \eps_v$ for all $v\in S$;
    \item $|x|_v \le 1$ for all $v\in T$.
\end{itemize}
Note that $|x|_w$ may behave wildly.
\end{thm}


\begin{proof}
Define $a_v = 0$ for all $v\notin S$ to make an ad\`ele $a = (a_v)_v$. For $v\in S$: we may shrink $\eps_v$ so that $\eps_v = |b_v|_v$ for some $b_v\in K_v$. For $v\in T$: let $b_v = 1$. Let $b_w \in K_w$ be large enough that $|b| = \prod_v |b_v|_v$ is large enough, as in the previous lemma. Then there exists $x\in K$, $x-a\in \LL(b)$, which is what we wanted.
\end{proof}


\begin{lem}
$\AA^{\times}_1$ is a closed subset of $\AA$ and of $\AA^{\times}$, and the two subspace topologies coincide.
\end{lem}

\begin{proof}

First of all, we remark that $\AA_1^{\times}$ is closed because it is cut out by an equation. So it suffices to show that the two subspace topologies coincide.

We claim that for any id\`ele $a = (a_v)_v$, $\AA_1^{\times} \cap \LL(a) \subseteq \AA_S^{\times}$ for some finite $S$. To show this claim, let $S$ contain the places $v$ such that $|a_v|_v \neq 1$ and the nonarchimedean places whose residue field has size at most $|a|$, as well as the archimedean ones. If $(x_v)_v\in \AA^{\times}_1\cap \LL(a)$, then at all $w\notin S$, $|x_w|_w \le |a_w|_w = 1$, so $x_w \in \cO_w$. If $|x_w|_w < 1$, then $|x_w|_w \le \frac{1}{q}$ where $q$ is the size of the residue field at $w$. Then $|x| \le |a|/q < 1$, which is a contradiction to $x\in \AA^{\times}_1$. So $x\in \AA_S^{\times}$ and the lemma is proved.

Now, because the topology of $\AA_S^{\times}$ is just the product topology, it is the same in both $\AA$ and $\AA^{\times}$. Because $\AA^{\times}_1$ is covered by $\AA_1^{\times} \cap \LL(a)$'s, we are done.
\end{proof}


\begin{thm}
$\AA^{\times}_1/K^{\times}$ is compact.
\end{thm}

\begin{proof}
Choose $d\in \AA^{\times}$ large enough for the adelic Minkowski. By the above lemma, $\AA_1^{\times}\cap \LL(d)$ is closed inside $\LL(d)$, which is compact. So $\AA_1^{\times}\cap \LL(d)$ is compact.

It remains to show that $\AA_1^{\times}\cap \LL(d)$ surjects onto $\AA_1^{\times}/K^{\times}$. Given any $u\in \AA^{\times}_1$, we have $|d/u| = |d|$, so there exists a nonzero element $x\in K^{\times}$ in $\LL(d/u)$ by adelic Minkowski. This is equivalent to $ux \in \LL(d)$, but $ux\in \AA_1^{\times}$ also. So the above map is indeed a surjection, which tells us that $\AA_1^{\times}/K^{\times}$ is compact. 
\end{proof}


\section{Lecture, Nov. 18}

(Finiteness of class group, Dirichlet's unit theorem)

We will use the compactness of $\AA_1^\times/K^\times$ to show:

\begin{thm}[finiteness of class group]
We have:
    \begin{enumerate}
        \item If $K$ is a number field, then $\Cl(\cO_K)$ is finite.
        \item If $K$ is a global function field, and $X$ is the associated smooth projective curve, then $\Pic^0(X) = \Div^0(X)/\im(K^{\times})$ is finite.
    \end{enumerate}
\end{thm}


\begin{proof}
    (1) Consider the natural surjective map $\AA^{\times}\onto\cI$. The induced $\AA^{\times}_1\to \cI$ is still surjective because we can always normalize at archimedean places. So we get a surjection $\AA_1^{\times}/K^{\times} \onto \cI/\im(K^{\times}) = \Cl(\cO_K)$. The kernel of this map is open, so the LHS quotient the kernel is compact and discrete, hence finite.

    (2) Consider the natural surjective map $\AA^{\times}\onto\Div(X)$, which induces $\AA_1^{\times}\onto \Div^0(X)$. So we get a surjection $\AA_1^{\times}/K^{\times} \onto \Pic^0(X)$, and we can argue as in (1).
\end{proof}

\begin{defn}
    Let $S$ be a set of places containing all archimedean ones. Let 
    \[\cO_S = \{x\in K: |x_v|_v\le 1 \text{ for all } v\notin S\}.\] 
    Then $\cO_S = \AA_S\cap K$, and $\cO_S^{\times} = \AA_S^{\times}\cap K^{\times}$. Let $\mu = (\cO_S^{\times})_{\tors} = (K^{\times})_{\tors}$ be the group of roots of unities.
\end{defn}


Define a continuous homomorphism
\begin{align*}
    \Log: \AA_S^{\times} &\to \RR^S \\
    (a_v) &\mapsto (\log |a_v|_v)_{v\in S}.
\end{align*}

\begin{lem}
    Let $S$ be the set of archimedean places. Then the induced map $\Log: \AA^{\times}_{S,1}\to \RR^S_0$ is surjective. \qed
\end{lem}

\begin{lem}
    The induced $\Log: \cO_S^{\times}\to \RR^S$ has finite kernel and discrete image.
\end{lem}

\begin{proof}
    Let $B$ be a compact neighborhood of $0$. Then $\Log^{-1}(B)$ is contained in some $\LL(d)$, so $\Log^{-1}(B)\cap K^{\times}$ is finite. In particular, $\Log$ has finite kernel, and $0$ is an isolated point in the image, i.e. the image is discrete.
\end{proof}

\begin{cor}
    $\ker(\Log: \cO_S^{\times}\to \RR^S) = \mu$, and $\Log(\cO_S^{\times})$ is a free abelian group of finite rank.
\end{cor}

\begin{proof}
    Clearly, $\mu$ is in the kernel. Because the kernel is finite, it must be torsion.
\end{proof}


\begin{thm}[Dirichlet's $S$-unit theorem]
    Let $K$ be a number field, then $\cO_S^{\times}$ is finitely genrated with rank $|S|-1 = r_1 + r_2 - 1$.
\end{thm}


\begin{proof}
    We only prove this in the case where $S$ is the set of archimedean places. By the previous corollary, $\cO_S^{\times}$ is finitely generated ($\mu$ is contained in some adelic parallelotope intersect $K^{\times}$, hence finite). 

    Consider the open and closed inclusion $\AA^{\times}_{S, q}\into \AA_1^{\times}$, which induces a map $\AA_{S,1}^{\times}/\cO_S^{\times}\to \AA_1^{\times}/K^{\times}$. This is open and closed, and the RHS is compact, so the LHS is compact also. Under the log map, $\AA^{\times}_S/\cO_S^{\times}\to \RR^S_0/\log(\cO_S^{\times})$ is surjective, so the RHS is compact as well. This means that the lattice is a full lattice, i.e. is of full rank $|S|-1$.
\end{proof}

We transition to the next topic, cyclotomic fields. Let $n$ be a natural number, $K$ be a field whose characteristic does not divide $n$, and let $L$ be the splitting field of the separable polynomial $x^n - 1$ in $K$, i.e. $L = K(\zeta_n)$. We get an injection $\Gal(L/K)\into (\ZZ/n\ZZ)^{\times}$, which is not always surjective. However, this is surjective when $K = \QQ$. This amounts to showing that $\Phi_n(x)$ is irreducible in $\QQ[x]$. Consider the discriminant $\disc(x^n-1) = \pm n^n$. Let $f(x)$ be a factor of $x^n-1$, and $\zeta$ a root of $f$. Let $p$ be a prime coprime to $n$. Suppose $\zeta^p$ is not a root of $f$, then $f(\zeta^p)\neq 0$ is a product of differences of roots of unity, hence an algebraic integer dividing $n^n$. But $f(\zeta^p)\equiv f(\zeta)^p = 0$ mod $p$, so $p\mid f(\zeta^p)$, so $p\mid n^n$, a contradiction. So $\zeta^p$ is a root of $f$. By induction, $\zeta^m$ is a root of $f$ for any $m$ coprime to $n$, as desired.


\section{Lecture, Nov. 21}

(Cyclotomic fields, etc.)

Another way to write the proof is as follows:

\begin{prop}
    If a prime $p$ is coprime to $n$, then $\QQ(\zeta_n)/\QQ$ is unramified above $p$, and $\Frob_p$ acts by $\zeta_n\mapsto \zeta_n^p$.
\end{prop}

So all primes coprime to $n$ lie in the image of $\Gal(\QQ(\zeta_n)/\QQ)\to (\ZZ/n\ZZ)^{\times}$, so it must be surjective.

\begin{cor}
    If $p\nmid n$, then $f_p = [\FF_{\fq}:\FF_p]$ is equal to the order of $\Frob_p$ in $G$, which is equal to the order of $p$ in $(\ZZ/n\ZZ)^{\times}$.
\end{cor}


\begin{prop}
    The ring of integers in $\QQ(\zeta_n)$ is $\ZZ[\zeta_n]$.
\end{prop}

\begin{proof}
    Induct on the number of primes dividing $n$. Suppose $n = mp^r$, $p\nmid m$. We have a tower of extensions $K = \QQ(\zeta_m)/\QQ$, and $K(\zeta_{p^r})/K$. By induction we know that $\cO_K = \ZZ[\zeta_m]$.

    We claim that $\cO_K[\zeta_{p^r}]$ is integrally closed. This can be checked after localizing at each prime $\fp$ in $K$, i.e. $(\cO_K)_{\fp}[\zeta_{p^r}]$ is integrally closed. Consider
    \[\Phi_{p^r}(x) = \frac{x^{p^r}-1}{x^{p^{r-1}}-1} = 1 + x^{p^{r-1}} + x^{2p^{r-1}} + \dots + x^{(p-1)p^{r-1}}.\]
    If $\fp$ lies above $p$, then $\Phi_{p^r}(x+1)$ is Eisenstein at $\fp$ (this uses that $p\nmid m$, which implies $\fp$ is unramified), so $(\cO_K)_{\fp}[\zeta_{p^r}]$ is a DVR (see $\S$\ref{dvrextensions}). But there can be no nontrivial rings between a DVR and its field of fractions, so $(\cO_K)_{\fp}[\zeta_{p^r}]$ is integrally closed.

    If $\fp\mid\ell\neq p$, then $x^{p^r}-1$ is separable mod $\ell$, and so is $\Phi_{p^r}(x)$ mod $\fp$. So $(\cO_k)_{\fp}[\zeta_{p^r}]$ is a DVR (?), and therefore integrally closed.
\end{proof}

We transition to yet another topic: analytic number theory. A good reference is Davenport's \emph{Multiplicative Number Theory}.


\begin{defn}[Riemann zeta function]
    For $\Re(s) > 1$, 
    \[\zeta(s) := \prod_{p} \frac{1}{1-p^{-s}} = \sum_{n\ge 1} n^{-s}.\]
\end{defn}

\begin{defn}[Dedekind zeta function]
    Let $K$ be a number field,
    \[\zeta_K(s) := \prod_{\text{nonzero }\fp} \frac{1}{1 - \N(\fp)^{-s}} = \prod_{\text{nonzero }\fa} \N(\fa)^{-s}.\]
    where $\N(\fp)$ is the absolute norm, i.e. $\N(\fp) = p^{[\FF_{\fp}:\FF_p]} = |\cO_K/\fp|$. 
\end{defn}



\section{Lecture, Nov. 23}

(Zeta functions, character theory)

\begin{prop}
    There are infinitely many primes. Even better, $\sum \frac1p$ diverges.
\end{prop}

\begin{proof}
    Clearly, $\lim_{s\to 1^+} \zeta(s) = \infty$, so $\log\zeta(s) = \sum_p -\log(1-p^{-s})$ also tends to $\infty$. Expanding as a Taylor series, the main part is $\sum \frac{1}{p^s}$ and the rest is obviously bounded.
\end{proof}

\begin{prop}
    $\zeta(s) = \frac{1}{s-1} + \phi(s)$, where $\phi(s)$ extends to a holomorphic function on $\Re(s) > 0$.
\end{prop}

\begin{proof}
    For $\Re(s) > 1$, 
    \begin{align*}
        \zeta(s) &= \sum_{n\ge 1} n^{-s} \\
        &= \sum_{n\ge 1} n(n^{-s} - (n+1)^{-s}) \\
        &= \sum_{n\ge 1} \pr{n\int_n^{n+1} sx^{-s-1}dx} \\
        &= s \int_1^{\infty} \floor{x}x^{-s-1}dx \\
        &= \frac{1}{s-1} + \pr{1 - s\int_1^{\infty} \{x\}x^{-s-1}dx},
    \end{align*}
    where the latter term (which we call $\phi(s)$) converges absolutely for $\Re(s) > 0$, and uniformly so on $\Re(s)\ge \eps$ for any $\eps>0$.
\end{proof}

\begin{prop}
    The following are true about $\zeta(s)$:

    \begin{enumerate}
        \item memoromorphic on $\CC$; has a simple pole at 1, and no other poles

        \item functional equation 
        \item trivial zeros at negative even numbers
        \item (infinitely many) all other zeros lie in the critical strip $0 < \Re(s) < 1$, conjectured to all lie on $\Re(s) = 1/2$.
    \end{enumerate}
\end{prop}


\begin{thm}[Dirichlet]
\label{dirichlet}
    If $\gcd(a,m) = 1$, then there exist infinitely many primes congruent to $a$ mod $m$.
\end{thm}


\begin{defn}
    A mod $m$ \emph{Dirichlet character} is a character on $(\ZZ/m\ZZ)^{\times}$, i.e. a homomorphism
    \[\chi: (\ZZ/m\ZZ)^{\times} \to \CC^{\times}.\]
    Extend to $\chi: \ZZ_{\ge 0}\to \CC$ by mapping to zero the numbers $a$ not coprime to $n$.
\end{defn}


We review some character theory of finite abelian groups.

\begin{prop}
    For a character $\chi\in \wh{G}$,
    \[\sum_{g\in G} \chi(g) = \begin{cases}
    |G| & \text{if } \chi \text{ is trivial,}\\
    0 & \text{otherwise}.
    \end{cases}\]
\end{prop}

\begin{proof}
    When $\chi$ is nontrivial, there exists $a\in G$, with $\chi(a)\neq 1$. Let $s$ be the sum. Then
    \[\chi(a)s = \sum_g \chi(ag) = \sum_g \chi(g) = s,\]
    so $s = 0$.
\end{proof}

\begin{prop}
    For an element $g\in G$,
    \[\sum_{\chi\in \wh{G}} \chi(g) = \begin{cases}
    |\wh{G}| = |G| & \text{if } g=1,\\
    0 & \text{otherwise}.
    \end{cases}\]
\end{prop}

The proof is similar.

\begin{thm}[Fourier transform on finite abelian groups]
    Any function $f:G\to \CC^{\times}$ is a linear combination of characters:
    \[f = \sum_{\chi} \wh{f}(\chi) \chi\]
    where
    \[\wh{f}(\chi) = \frac{1}{|G|} \sum_g \chi(g^{-1})f(g).\]
\end{thm}

\begin{proof}
    By linearity, it suffices to prove this for a basis of functions $G\to \CC^\times$. Take $f$ to be the indicator function for $a\in G$. Then
    \[\wh{f}(\chi) = \frac{1}{|G|}\chi(a^{-1}).\]
    So $\sum_\chi \wh{f}(\chi)\chi(g) = \frac{1}{|G|}\chi(a^{-1}{g})$, which is 1 when $a^{-1}g = 1_G$ and $0$ otherwise, i.e. the same as $f$.
\end{proof}


\section{Lecture, Nov. 28}

(Proof of Dirichlet's theorem, modulo two theorems)


\begin{defn}
    Let $\chi$ be a Dirichlet character mod $m$. Define the \emph{Dirichlet $L$-series}
    \[L(s,\chi) = \prod_p \frac{1}{1-\chi(p)p^{-s}} = \sum_{n\ge 1} \chi(n)n^{-s}.\]
    This \textit{a priori} converges absolutely for $\Re(s)>1$.
\end{defn}


\begin{prop}
    If $\chi\neq \mathbf{1}$ (the  trivial character), then $L(s,\chi)$ extends to a holomorphic function for $\Re(s) > 0$.
\end{prop}

\begin{proof}
    Let $T(x) := \sum_{1\le n<x} \chi(n)$ for $x\in \RR$. This is periodic with period $m$, hence bounded. So
    \begin{align*}
    L(s,\chi) &= \sum_{n\ge 1}\chi(n)n^{-s} \\
    &= \int_1^\infty x^{-s}dT(x) \\
    &= x^{-s}T(x)|_1^\infty - \int_1^\infty -T(x)sx^{-s-1}dx \\
    &= s\int_1^\infty T(x)x^{-s-1}dx 
    \end{align*}
    where we've used the Riemann-Stieltjes integral. (Here it is just a fancy way to justify summation by parts.) This integral converges as long as $\Re(s)>0$. Furthermore, it converges uniformly on $\Re(s) \ge \eps$ for every $\eps$, so $L$ can be extended holomorphically to $\Re(s) > 0$.
\end{proof}


\begin{proof}[Proof of theorem \ref{dirichlet}, Dirichlet's theorem on arithmetic progressions]
Writing the indicator function as the sum of characters,
\begin{align*}
    \sum_{p\equiv a} p^{-s}
    &= \sum_{p} p^{-s} \pr{\frac{1}{\phi(m)} \sum_{\chi} \chi(a^{-1})\chi(p)} \\
    &= \frac{1}{\phi(m)} \sum_\chi \chi(a^{-1}) \pr{\sum_p \chi(p)p^{-s}} \\
    &= \frac{1}{\phi(m)} \sum_\chi \chi(a^{-1})\pr{\log L(s,\chi) + O(1)}  \\
    &= \frac{1}{\phi(m)}\log L(s, \mathbf{1}) + \frac{1}{\phi(m)}\sum_{\chi\neq \mathbf{1}} \chi(a^{-1})\log L(s,\chi) + O(1).
\end{align*}
We have
\[\log L(s,\mathbf{1}) = \log \zeta(1) + O(1) \to \infty\]
as $s\to 1^+$. The goal now is to show that the other terms are in fact $O(1)$ as $s\to 1^+$. It is then sufficient to show that if $\chi\neq \mathbf{1}$, $L(1,\chi) \neq 0$. This would follow from the following two theorems, by analyzing the order of vanishing at $s=1$.
\end{proof}


\begin{thm}
\label{zeta_cyclotomic}
    Up to Euler factors at primes dividing $m$,
    \[\zeta_{\QQ(\zeta_m)}(s) = \prod_{\chi: (\ZZ/m\ZZ)^\times\to \CC^\times} L(s,\chi).\]
\end{thm}

\begin{thm}
    For any number field $K$, $\zeta_K(s)$ has a simple pole at $s=1$.
\end{thm}

We will prove these in the next lecture.


\section{Lecture, Nov. 30}

(Measure theory)

We first remark that the proof above in fact shows that $\delta(\{p\equiv a\pmod{m}\}) = \frac{1}{\phi(m)}$.

\begin{proof}[Proof of theorem \ref{zeta_cyclotomic}]
    We compare the two sides. Consider a prime $p\nmid m$ (so unramified), and consider primes $\fp\mid \fp$. So $e_p = 1$, $f_p$ is the order of $p$ in $(\ZZ/m\ZZ)^\times$, and $g_p = \phi(m)/f_p$. The corresponding term on the LHS is
    \[\prod_{\fp\mid p} (1 - \N(\fp)^{-s})^{-1} = (1-(p^{-s})^{f_p})^{-g_p}.\]
    So it suffices to show that
    \[\prod_{\chi} (1 - \chi(p)p^{-s}) = (1 - (p^{-s})^{f_p})^{g_p}.\]
    Among the characters $\chi$ of $(\ZZ/m\ZZ)^\times$, the values of $\chi(p)$ are $1,\mu_f,\mu_f^2,\dots,\mu^{f_p-1}$, where $\mu_f$ is a primitive $f$-th root of unity, each with multiplicity $g_p$. This completes the proof.
\end{proof}

\begin{thm}[analytic class number formula]
    Let $K$ be a number field, then $\zeta_K(s)$ extends to a meromorphic function in a neighborhood of $s=1$ with a simple pole at 1. Moreover,
    \[\lim_{s\to 1} (s-1)\zeta_K(s) = \frac{\vol(\AA_1^\times/K^\times)}{\vol(\AA/K)} = \frac{2^{r_1}(2\pi)^{r_2}h_KR_K/w_k}{\sqrt{|D_K|}},\]
    where $h_K = \#\Cl_K$, $w_K = \# \mu_K$.
\end{thm}

We will prove the second equality and, in particular, define the volumes, so we will do a bit of review of analysis. The first equality will be proven next semester using methods in Tate's thesis.


\begin{defn}
    Let $X$ be a set, $\sM$ a collection of subsets of $X$. If $\sM$ is closed under countable unions and complements, call $\sM$ a \emph{$\sigma$-algebra}.
\end{defn}

\begin{exm}
    Let $X$ be a topological space. The set of \emph{Borel sets} $\sB$ is the $\sigma$-algebra generated by the open sets.
\end{exm}

The sets in $\sM$ are called \emph{measurable sets}.

\begin{defn}
    A function $f:X\to \CC$ is called \emph{measurable} if the inverse image of measurable subsets are measurable. (It suffices to check the inverse images of open disks.)
\end{defn}

\begin{defn}
    A \emph{measure} on $(X,\sM)$ is a function $\mu:\sM\to [0,\infty]$ such that $\mu(\bigcup A_i) = \sum \mu(A_i)$ for any countable collections of disjoint measurable sets. Call $\mu$ the \emph{Borel measure} if $\sM = \sB$. A null set is a subset $N\subseteq X$ contained in a measure-0 set. It is easy to enlarge $\sM$ so that all null sets are measurable. A function $f:X\to \CC$ is a \emph{null function} if $\{x\in X: f(x)=0\}$ is a null set.
\end{defn}

We now define in stages a notion of integrals. Fix $(X,\sM, \mu)$.
\begin{itemize}
    \item Given $S\in \sM$ with $\mu(S) < \infty$, let $1_S$ be the function that is 1 on $S$ and 0 on $X-S$. Then define $\int 1_S = \mu(S)$. 
    \item A \emph{step function} $f$ is a finite $\CC$-linear combination of $1_S$'s. Define $\int f$ linearly.
    \item Define the $L^1$ norm of $f$, $\norm{f}_1 := \int |f| \in \RR_{\ge 0}$. Call a function $f:X\to \CC$ \emph{integrable} if outside a null set, it is equal to the pointwise limit of some $L^1$-Cauchy sequence $(f_i)$ of step functions. Then define $\int_X fd\mu = \int f = \lim_i \int f_i \in \CC$. (The pointwise limit of measurable functions is measurable, so in particular integrable functions are measurable.)
\end{itemize}


\section{Lecture, Dec. 2}

(Radon measures and integrals, Haar measures)

There is an alternative definition of integration for all measurable functions $f:X\to [0, \infty]$, which agrees with the previous definition if $f$ is integrable:
\[\textstyle \int f := \sup \{\int g: g  \text{ is a step function and } 0\le g\le f\} \in [0,\infty].\]
Also, for a measurable function $f:X\to \CC$, $f$ is integrable iff $|f|$ is integrable, in which case we have $|\int f| \le \int |f|$.

\begin{thm}[Monotone convergence theorem]
    Suppose $(f_n)$ is a sequence of measurable functions $X\to [0,\infty]$ such that $0\le f_1\le f_2\le\dots$, then the pointwise limit $f$ satisfies $\int f = \lim \int f_n$.
\end{thm}

\begin{thm}[Dominated convergence theorem]
    Suppose measurable functions $f_1,f_2,\dots : X\to \CC$ converge pointwise to $f:X\to \CC$. If there is an integrable $g:X\to \CC$ such that $|f_n| \le |g|$ for all $n$, then $f$ and $f_n$ are all integrable and $\int f = \lim \int f_n$.
\end{thm}



\begin{defn}
    Let $X$ be a Hausdorff topological space. $X$ is \emph{locally compact} if every $x\in X$ has a compact neighborhood (i.e. $x\in U\subseteq K$ where $U$ open and $K$ compact).
\end{defn}

\begin{defn}
    An \emph{outer Radon measure} is a Borel measure (a measure on $\sB$) such that:
    \begin{itemize}
        \item (locally finite) Every $x\in X$ has an open neighborhood $U$ such that $\mu(U)<\infty$;
        \item (outer regular) Every $S\in \sB$ satisfies $\mu(S) = \inf \{\mu(U): U\supseteq S \text{ open}\}$;
        \item (inner regular) Every open $U$ satisfies $\mu(U) = \sup \{\mu(K): K\subseteq U \text{ compact}\}$;
    \end{itemize}
\end{defn}

Let $C(X)$ be the $\CC$-vector space of continuous functions $f:X\to \CC$, and let $C_c(X)$ be the $\CC$-vector space of continuous functions with compact support.


\begin{defn}
    A \emph{Radon integral} on $X$ is a $\CC$-linear map $I: C_c(X)\to \CC$ such that $I(f) \ge 0$ if $f\ge 0$. (It is assumed that $f$ is real-valued.)
\end{defn}

Given an outer Radon measure $\mu$, we can define an integral $I_\mu: f\mapsto \int_X f d\mu$. The converse is:

\begin{thm}[Riesz--Markov--Kakutani representation theorem]
    Let $X$ be a LCH space, then the map
    \[\{\text{outer Radon measures } \mu\} \to \{\text{Radon integrals on } X\}\]
    by $\mu\mapsto I_\mu$, is a bijection.
\end{thm}

\begin{exm}
    Let $X = \RR^n$, the Riemann integral corresponds to the Lebesgue measure.
\end{exm}

\begin{exm}
    Examples of LCH topological groups:
    \begin{itemize}
        \item $\RR, \CC, \ZZ_p, \QQ_p, \AA$;
        \item The unit groups $A^\times$ of any of the above topological rings;
        \item $\GL_n(A)$ of any of the above;
        \item Any group equipped with the discrete topology.
    \end{itemize}
\end{exm}

\begin{defn}
    Let $G$ be a LCH topological group. A \emph{left Haar measure} on $G$ is a nonzero left-invariant outer Radon measure.
\end{defn}

Theorem \ref{Haar} says that such a measure always exists and is unique up to multiplication by a positive constant.

\begin{prop}
    $G$ is compact iff $\mu(G) < \infty$. In this case, the \emph{normalized Haar measure} is the unique Haar measure with $\mu(G) = 1$.
\end{prop}

\begin{exm}
Examples of Haar measures:
\begin{itemize}
    \item On $\RR^n$, the Lebesgue measure is a Haar measure;
    \item On a discrete group, the counting measure is a Haar measure.
\end{itemize}
\end{exm}

\begin{defn}
    An \emph{LCA group} is a locally compact abelian Hausdorff topological group. This forms a category, with morphisms being continuous homomorphisms.
\end{defn}

For example, $\TT \cong \RR/\ZZ$ is the unit circle in the complex plane; it is an LCA group.


\section{Lecture, Dec. 5}

(Duality of locally compact abelian groups)


\section{Lecture, Dec. 7}



\section{Lecture, Dec. 9}



\section{Lecture, Dec. 12}



\section{Lecture, Dec. 14}



\section{Lecture, Feb. 6}

(Kronecker--Weber theorem)

\begin{thm}[global KW]
    Any finite abelian extension of $\QQ$ is contained in a cyclotomic extension $\QQ(\zeta_m)$.
\end{thm}

\begin{thm}[local KW]
    Any finite abelian extension of $\QQ_p$ is contained in a cyclotomic extension $\QQ_p(\zeta_m)$.
\end{thm}

\begin{lem}[Galois group of compositum]
    Let $L_1,L_2/K$ be finite Galois extensions that lie in some bigger extension $\Omega/K$. Then $L_1L_2$ is Galois over $K$, with
    \[\Gal(L_1L_2/K) \cong \{(\sigma_1,\sigma_2)\in \Gal(L_1/K)\times \Gal(L_2/K): \sigma_1|_{L_1\cap L_2} = \sigma_2|_{L_1\cap L_2}\}.\]
\end{lem}


\begin{prop}
    Local KW implies global KW.
\end{prop}

\begin{proof}
    Consider each prime $p\in \ZZ$ where a finite abelian extension $K/\QQ$ is ramified. Fix $\fp\mid p$ to be a prime in $K$ above $p$, and consider the extension $K_{\fp}/\QQ_p$, which is finite abelian with $\Gal(K_{\fp}/\QQ_p) = D_{\fp}$. Assuming local KW, suppose $K_{\fp}\subseteq \QQ_p(\zeta_{m_p})$. Let $n_p = v_p(m_p)$ and $m = \prod p^{n_p}$, among all (finitely many) $p$ that ramify. Let $L = K(\zeta_m)$. It suffices to show $L = \QQ(\zeta_m)$.

    Because $L = K\cdot\QQ(\zeta_{m})$, $L/\QQ$ is abelian as well. Pick a prime $\fq\mid \fp$ in $L/K$, then $L_{\fq}$ is also finite abelian over $\QQ_p$. Let $F_{\fq}$ be the maximal unramified extension of $\QQ_p$ in $L_{\fq}$. Then $L_{\fq}/F_{\fq}$ is totally ramified with Galois group $I_{\fq} =: I_p$, which only depends on $p$ (since the Galois group is abelian).

    We claim that $I_p \cong (\ZZ/p^{n_p}\ZZ)^{\times}$. To show this, notice that $\QQ_p(\zeta_{m_p/p^{n_p}})$ is unramified over $\QQ_p$, so $K_{\fp} \subset F_{\fq}(\zeta_{p^{n_p}})$. Now, since $L_{\fq}\supseteq K_{\fp}(\zeta_{m})$ and $[L_{\fq}:K_{\fp}] = [L:K]$, $L_{\fq} = K_{\fp}(\zeta_{m}) \subseteq F_{\fq}(\zeta_{p^{n_p}})$, so in fact $L_{\fq} = F_{\fq}(\zeta_{p^{n_p}})$. So we have the following field inclusions
    \[
    \begin{tikzcd}
        & L_{\fq} \arrow[ld, dash]  \arrow[rd, dash] & \\
    F_{\fq} \arrow[rd, dash] & & \QQ_p(\zeta_{p^{n_p}}) \arrow[ld, dash] \\
     & \QQ_p, &
    \end{tikzcd}
    \]
    where $\QQ_p = F_{\fq}\cap \QQ_p(\zeta_{p^{n_p}})$ since one is unramified and the other is totally ramified. So
    \[I_p = \Gal(L_{\fq}/F_{\fq}) = \Gal(\QQ_p(\zeta_{p^{n_p}})/\QQ_p) \cong (\ZZ/p^{n_p}\ZZ)^{\times}.\]
    Now, let $I$ be the subgroup of $\Gal(L/\QQ)$ generated by $I_p$'s. Then
    \[|I| \le \prod |I_p| = \prod \phi(p^{n_p}) = \phi(m) = [\QQ(\zeta_m):\QQ].\]
    Let $L^{I}$ be the fixed field of $I$. Then $L^I/\QQ$ is unramified, so $L^I = \QQ$. This means 
    \[[L:\QQ] = [L:L^I] = |I|\le [\QQ(\zeta_m):\QQ] \le [L:\QQ],\] so $L = \QQ(\zeta_m)$ as desired.
\end{proof}


\begin{prop}
    It suffices to show local KW for cyclic extensions with Galois group $\ZZ/\ell^r\ZZ$.
\end{prop}

\begin{proof}
    For an arbitrary abelian extension $K/\QQ_p$, decompose its Galois group into the product of prime-power cyclic groups $H_i$, and let $K_i = K^{H_i}$. Then $K = \bigvee K_i$ (compositum), from which the proposition is clear.
\end{proof}

Now we begin the proof of local KW, with $\Gal(K/\QQ_p)\cong \ZZ/\ell^r\ZZ$. There are three cases:
\begin{itemize}
    \item tamely ramified case, $\ell\neq p$;
    \item wildly ramified case with odd degree, $\ell = p \ge 3$;
    \item wildly ramified case with even degree, $\ell = p = 2$.
\end{itemize}

\begin{proof}[Proof of case 1]

Let $F$ be the maximal unramified extension of $\QQ_p$ in $K$. Then $F/\QQ_p$ is already equal to some cyclotomic extension (to see this, consider the corresponding finite separable extension of residue fields; the Galois group of finite field extensions is cyclic). Furthermore, $K = F(\pi^{1/e})$ for some uniformizer $\pi$ in $F$ (cf. \ref{TotallyTamelyRam}). Assume that $\pi = -pu$, where $u \in \cO_K^\times$. Then $K$ lies in the compositum $F((-p)^{1/e})\cdot F(u^{1/e})$, and it suffices to show both are included in some cyclotomic extension of $F$.

For $F(u^{1/e})/F$, it is unramified since the discriminant is $\disc(x^e-u)$, which is a unit in $F$. This implies that it is also equal to some cyclotomic extension.

Consider $K(u^{1/e})/\QQ_p$, which is the compositum of $K$ and $F(u^{1/e})$, so it is also an abelian extension. Therefore, since $F((-p)^{1/e})\subseteq K(u^{1/e})$, $\QQ_p((-p)^{1/e})/\QQ_p$ is Galois as well, which implies $\zeta_e\in \QQ_p((-p)^{1/e})$  because $\QQ_p((-p)^{1/e})$ then must contain all $e$-th roots of $-p$. And it is totally ramified since the minimal polynomial of $(-p)^{1/e}$, $x^e + p$, is Eisenstein. Since $\QQ_p(\zeta_e)\subset \QQ_p((-p)^{1/e})$ is unramified over $\QQ_p$, we conclude $\QQ_p(\zeta_e) = \QQ_p$. Because the residue field of $\QQ_p$ contains only $(p-1)$-th roots of unity, $e\mid (p-1)$. Then
\[\QQ_p((-p)^{1/e}) \subseteq \QQ_p((-p)^{1/(p-1)}) = \QQ_p(\zeta_p),\]
by the lemma that follows. But from this we conclude that $F((-p)^{1/e})$ is also in some cyclotomic extension, so we are done.
\end{proof}

\begin{lem}
    $\QQ_p((-p)^{1/(p-1)}) = \QQ_p(\zeta_p)$.
\end{lem}

\begin{proof}
    Let $\alpha = (-p)^{1/(p-1)}$. Then $\alpha^{p-1} + p = 0$, which is an Eisenstein polynomial of degree $p-1$, so $\alpha$ is a uniformizer for $\QQ_p(\alpha)$. Let $\pi = \zeta_p-1$, whose minimal polynomial is also Eisenstein of degree $p-1$, so $\pi$ is a uniformizer for $\QQ_p(\zeta_p)$. The goal now is to show that $\alpha \in \QQ_p(\zeta_p)$, from which the lemma will follow by a degree argument.

    Let $u = -\pi^{p-1}/p \equiv 1\pmod{\pi}$, so $u$ is an unit in the valuation ring of $\QQ_p(\zeta_p)$. Consider $g(x) = x^{p-1} - u$, which, mod $\pi$, has $1$ as a simple root, so by Hensel's lemma we obtain a root $\beta$ of $g(x)$. Then 
    \[(\pi/\beta)^{p-1} + p = \frac{\pi^{p-1} + p\beta^{p-1}}{\beta^{p-1}} = 0,\]
    so $\alpha\mapsto \pi/\beta$ gives an injection.
\end{proof}


\begin{proof}[Proof of case 2]
    Suppose $K/\QQ_p$ cyclic of degree $p^r$, $p\ge 3$. There are two obvious cyclotomic extensions of degree $p^r$; in the unramified case we have $\QQ_p(\zeta_{p^{p^r}-1})$, and in the totally ramified case we have the index-$(p-1)$ subfield of $\QQ_p(\zeta_{p^{r+1}})$. Suppose for contradiction $K$ does not lie in $\QQ_p(\zeta_{p^{r+1}(p^{p^r}-1)})$. Then 
    \[\Gal(K(\zeta_{p^{r+1}(p^{p^r}-1)})/\QQ_p) \subseteq \Gal(K/\QQ_p) \times (\ZZ/p^r\ZZ)^2 \times \ZZ/(p-1)\ZZ\]
    surjecting onto the last two factors, and nontrivial in the first. So the Galois group has a quotient group that is $(\ZZ/p\ZZ)^3$, i.e. there exists an extension of $\QQ_p$ with Galois group $(\ZZ/p\ZZ)^3$. We are going to show that no such extensions exist in the next lecture.
\end{proof}


\section{Lecture, Feb. 8}

(Kronecker-Weber theorem continued, Artin map)

\begin{defn}[semidirect product]
    Let $G$ be a group, $N\lhd G$ a normal subgroup, and $H\le G$ a subgroup. If $H\to G\to G/N$ is an isomorphism, then we say $G = N\rtimes H$. 

    More generally, let $H,N$ be groups, with a homomorphism $\phi: H\to \Aut(N)$. Then $N\rtimes H$, as a set, is equal to $N\times H$, but the group operation is given by
    \[(n_1,h_1)(n_2,h_2) = (n_1\phi_{h_1}(n_2), h_1h_2).\]
    This is the (outer) semidirect product.
\end{defn}

\begin{prop}[Schur-Zassenhaus lemma]
\label{schur-zass}
    Let $N \lhd G$ with $|N|$ and $|G/N|$ coprime, then there exists a section $G/N\to G$. Consequently $G = N \rtimes G/N$.
\end{prop}

\begin{prop}
    Let $p$ be an odd prime, then any totally wildly ramified Galois extension of $\QQ_p$ is cyclic.
\end{prop}

\begin{proof}
    See 18.726 pset 1.
\end{proof}

\begin{thm}
    Let $p$ be an odd prime, then no $(\ZZ/p\ZZ)^3$-extension $K/\QQ_p$ exists.
\end{thm}

\begin{proof}
    We first only assume $K/\QQ_p$ is Galois. Let $G = \Gal(K/\QQ_p)$, and let $\fp\subset \cO_K$ be the unique prime above $p$. Since $\cO_K$ is a DVR, $G = D_{\fp}$. Let the \emph{ramification groups} $G_i = \{\sigma\in G: \sigma(x) \equiv x \pmod{\fp^{i+1}} \text{ for any } x\in \cO_K\}$, and let $\pi_{\fp}: D_{\fp}\to \Gal(\FF_{\fp}/\FF_p)$ be the natural map whose kernel is $I_{\fp} = G_0$. 
    
    Let $U_i = 1 + \fp^i$ be subgroups of $\cO_K^{\times}$ for $i\ge 1$, and set $U_0 = \cO_K^{\times}$. Then $U_0/U_1 \cong \FF_{\fp}^{\times}$ and $U_i/U_{i+1} \cong \FF_{\fp}$ as abelian groups. For each $i\ge 0$, there is an injection $G_i/G_{i+1} \into U_i/U_{i+1}$ given by $\sigma \mapsto \sigma(\pi)/\pi$ where $\fp = (\pi)$. Therefore, $G_0/G_1$ is cyclic with order coprime to $p$, and $G_1$ is a $p$-group. Consider the normal subgroups $G_1\lhd G_0\lhd G$ (which implies that $G$ is solvable), then the corresponding subfield $K^{G_0} = K^I$ is the maximal unramified extension of $\QQ_p$ in $K$, $K^{G_1}/\QQ_p$ is the maximal tamely ramified extension, and $K/K^{G_1}$ is totally wildly ramified.
    
    By Proposition \ref{schur-zass}, $G_0 \cong G_1 \ltimes G_0/G_1$. 
    
    In the case $G = (\ZZ/p\ZZ)^3$, since all nontrivial proper subgroups are $\ZZ/p\ZZ$ or $\ZZ/p^2\ZZ$, so $G\cong I\times H$, where $H:=\Gal(\FF_\fp/\FF_p)$ is cyclic. Since $K^H/\QQ_p$ is totally wildly ramified ($I = \Gal(K^H/\QQ_p)$ is a $p$-group), it is cyclic. But $G$ is not the product of two cyclic groups.
\end{proof}

\begin{Rem}
    If $p$ is odd, then there are exactly $p$ ramified extensions with degree $p$, namely
    \[\QQ_p[x]/(x^p + px^{p-1} + p(1+ap))\]
    for $0\le a\le p-1$. 
\end{Rem}


\begin{proof}[Proof of case 3]
    In this case, $\QQ_2(\zeta_{24})/\QQ_2$ has Galois group $(\ZZ/2\ZZ)^3$. But we can still follow a similar argument. Suppose $K/\QQ_2$ is cyclic with order $2^r$. As usual, the suspects are $\Gal(\QQ_2(\zeta_{2^{2^r}-1})/\QQ_2) \cong \ZZ/2^r\ZZ$ and $\Gal(\QQ_2(\zeta_{2^{r+2}})/\QQ_2) \cong (\ZZ/2^{r+2}\ZZ)^{\times} \cong \ZZ/2\ZZ\times \ZZ/2^r\ZZ$. We claim that $K\subseteq \QQ(\zeta_{2^{r+2}(2^{2^r}-1)})$. Suppose otherwise, then either
    \[\Gal(K(\zeta_{2^{r+2}(2^{2^r}-1)})/\QQ_2) \cong (\ZZ/2^r\ZZ)^2\times \ZZ/2\ZZ \times \ZZ/2^s\ZZ\]
    for $s\ge 1$, or
    \[\Gal(K(\zeta_{2^{r+2}(2^{2^r}-1)})/\QQ_2) \cong (\ZZ/2^r\ZZ)^2\times \ZZ/2^s\ZZ\]
    for $s\ge 2$. So it has a quotient equal to either $(\ZZ/2\ZZ)^4$ or $(\ZZ/4\ZZ)^3$. In the first case, we can show that there are 7 quadratic extensions of $\QQ_2$, but $(\ZZ/2\ZZ)^4$ has 15 subgroups of index 2; in the second case, there are 12 cyclic quartic extensions of $\QQ_2$, but $(\ZZ/4\ZZ)^3$ has 28 subgroups whose quotient is $\ZZ/4\ZZ$ (see LMFDB).
\end{proof}

This finishes the proof of Kronecker-Weber theorem.

Now fix $L/K$ an abelian extension of global fields, so that we have the Artin symbol
\[\pr{\frac{L/K}{\fp}} = \Frob_{\fp} =: \sigma_{\fp}\]
for unramified $\fp$.
Let $\fm$ be an ideal divisible by all ramified primes. Then we have the Artin map
\[\psi_{L/K}^{\fm}: \cI_K^{\fm} \to \Gal(L/K).\]
The first major step in proving class field theory is the following:

\begin{prop}
    Let $K$ be a number field, $L/K$ abelian. Then the Artin map $\psi_{L/K}^{\fm}$ is surjective.
\end{prop}

\begin{prop}
    The primes in $\ker\psi_{L/K}$ are the primes in $K$ that split completely in $L$. \qed
\end{prop}

\begin{prop}
    Let $K\subseteq L\subseteq M$ be a tower of abelian extensions of global fields. Then the Artin maps commute with the restriction map $\Gal(M/K)\to \Gal(L/K)$.
\end{prop}



\section{Lecture, Feb. 13}

(Ray class groups, polar density)


\begin{prop}
    The Artin map is surjective for abelian extensions $L/\QQ$.
\end{prop}


\begin{proof}
    By KW it suffices to show this for $L = \QQ(\zeta_m)$. In this case, $(p)$ hits the residue class of $p$ in $(\ZZ/m\ZZ)^{\times}$, so the Artin map is clearly surjective.
\end{proof}

For global field $K$, let $M_K$ be the set of places of $K$. Finite places $v$ are ones corresponding to prime ideals, and the rest are infinite places. (Infinite places can be nonarchimedean; for example, since function fields have characteristic $p$, all nontrivial places are nonarchimedean. Places of a function field correspond 1-to-1 with closed points of its associated smooth projective curve. For number fields, however, infinite places are all archimedean, and are either real or complex.)


\begin{defn}
    Let $K$ be a number field. A \emph{modulus} for $K$ is a function $\fm: M_K \to \ZZ_{\ge 0}$ with finite support, such that $\fm(v)\le 1$ for infinite places, and $\fm(v) = 1$ only when $v$ is real.
\end{defn}

This should be thought of as a product of prime ideals and some set of real places.

\begin{defn}
    A fractional ideal $I$ of $\cI_K$ is \emph{coprime} to $\fm$ if $v_{\fp}(I) = 0$ for finite primes $\fp\mid \fm$. The subgroup of fractional ideals coprime to $\fm$ is denoted by $\cI_{K}^{\fm}$. The subgroup of elements $a\in K^{\times}$ such that $(a)\in \cI_K^{\fm}$ is denoted by $K^{\fm}$. Finally, $K^{\fm, 1}$ is the subgroup of elements $a$ where $v_{\fp}(a-1)\ge v_{\fp}(\fm)$ for finite $\fp\mid\fm$, and $a_v>0$ for all infinite $v\mid \fm$ where $a_v$ is the image of the embedding $K\into K_v = \RR$.
\end{defn}

\begin{defn}
    The \emph{ray group} $\cR_K^{\fm}\subseteq \cI_K^{\fm}$ is the image of $K^{\fm,1}$ in $\cI_K^{\fm}$. The \emph{ray class group} for $\fm$ is $\Cl_K^\fm = \cI_K^\fm / \cR_K^\fm$.
\end{defn}

\begin{defn}
    A finite abelian extension $L/K$ unramified at all primes that do not divide $\fm$, for which $\ker \psi_{L/K}^{\fm} = \cR_K^\fm$ is called a \emph{ray class field} for $\fm$. When $\fm$ is trivial, it is the \emph{Hilbert class field}, i.e. the maximal unramified abelian extension (which we will show). 
\end{defn}


When $\fm$ has only all the real places, this is called the \emph{narrow class group}.


\begin{lem}
    Let $A$ be a Dedekind domain, $\fa$ an $A$-ideal. Then every ideal class in $\Cl(A)$ contains an $A$-ideal coprime to $\fa$.
\end{lem}

\begin{proof}
    Let $I$ be a nonzero fractional ideal. For each $\fp\mid \fa$, pick $\pi_\fp\in \fp$ such that $v_\fp(\pi_\fp) = 1$ and $v_\fq(\pi_\fp) = 0$ for all other $\fq\mid \fa$ by strong approximation. Then $I' = (\prod_{\fp\mid \fa} \pi_\fp^{-v_\fp(\fa)})I$ is in the class of $I$ and satisfies $I'$ coprime to $\fa$. Then make it integral by multiplying by the appropriate elements again found by strong approximation.
\end{proof}


\begin{prop}
    Let $\fm = \fm_0\fm_\infty$ be a modulus for $K$. We have an exact sequence
    \[0\to \cO_K^\times \cap K^{\fm, 1} \to \cO_K^\times \to K^\fm/K^{\fm,1} \to \Cl_K^\fm \to \Cl_K \to 0. \tag{$\ast$}\]
    and $K^{\fm}/K^{\fm,1} \cong \{\pm 1\}^{\#\fm_{\infty}} \times(\cO_K/\fm_0)^\times$ canonically. \
\end{prop}

\begin{proof}
    Consider the composition $K^{\fm,1}\xto{f}K^\fm \xto{g} \cI_K^\fm$. Then $f$ is injective, $\ker(g) = \cO_K^\times$, $\ker(g\circ f) = \cO_K^\times\cap K^{\fm,1}$, $\coker(g) = \cI_K^\fm/\im(K^\fm) = \Cl_K$ by the previous lemma, and $\coker(g\circ f) = \Cl_K^\fm$. The kernel-cokernel exact sequence yields
    \[0\to \ker(f) \to \ker(g\circ f) \to \ker(g) \to \coker(f) \to \coker(g\circ f) \to \coker(g) \to 0,\]
    which becomes ($\ast$).

    For the second statement, given $\alpha\in K^\fm$, write $\alpha = a/b\in K^\fm$ where $a,b\in \cO_K$ are both coprime to $\fm$. Send
    \[\phi:K^\fm \to \{\pm 1\}^{\#\fm_{\infty}} \times(\cO_K/\fm_0)^\times\]
    by $\alpha$ mapping to $(\sgn(\alpha_v), \bar{\alpha} = \bar{a}\bar{b}^{-1})$. This is surjective by strong approximation, and the kernel is precisely $K^{\fm,1}$. This is canonical because $\bar{\alpha}$ does not depend on $a,b$.
\end{proof}

\begin{cor}
    Let $h_K^\fm = |\Cl_K^\fm|$ be the ray class number. Then 
    \[h_K^\fm = \frac{\phi(\fm)h_K}{[\cO_K^\times : \cO_K^\times\cap K^{\fm, 1}]}.\]
    Here $\phi(\fm) = \phi(\fm_0)\phi(\fm_\infty) = |K^\fm/K^{\fm,1}|$, where 
    \[\phi(\fm_\infty) = 2^{\#\fm_\infty}, \quad \phi(\fm_0) = |(\cO_K/\fm_0)^\times| = \prod_{\fp\mid \fm_0} |(\cO_K/\fp^{\fm(\fp)})^\times| = \N(\fm_0)\prod_{\fp\mid \fm_0}(1-\N(\fp)^{-1}).\]
\end{cor}


\begin{defn}
    Let $S$ be a set of primes of a global field $K$. The \emph{partial Dedekind zeta function}
    \[\zeta_{K, S} := \prod_{\fp\in S} \frac{1}{1- \N(\fp)^{-s}}.\]
    This converges on $\Re(s)>1$.
\end{defn}

If $S$ is finite then this is just holomorphic on a neighborhood of $s=1$. If $S$ is cofinite then this is $\zeta_K$ over a holomorphic function, hence meromorphic on a neighborhood of $1$ with a simple pole at 1.

\begin{defn}
    If $\zeta_{K,S}^n$ extends to a meromorphic function on a neighborhood of 1, the \emph{polar density}
    \[\rho(S) := \frac{m}{n},\]
    where $m$ is the order of the pole.
\end{defn}

The \emph{Dirichlet density} is
\[d(S) = \lim_{s\to 1^+}\frac{\sum_{\fp\in S} \N(\fp)^{-s}}{\sum_{\fp} \N(\fp)^{-s}} = \lim_{s\to 1^+}\frac{\sum_{\fp\in S} \N(\fp)^{-s}}{\log\frac{1}{s-1}},\]
and the \emph{natural density} is
\[\delta(S) = \lim_{n\to\infty}\frac{\#\{\fp\in S: \N(\fp) \le n\}}{\#\{\fp: \N(\fp) \le n\}}.\]

\begin{prop}
    If $S$ has a natural density, then it has a Dirichlet density, and the two densities agree.
\end{prop}

\begin{proof}
    18.786 problem set 2.
\end{proof}

\begin{prop}
    If $S$ has a polar density, then it has a Dirichlet density, and the two densities agree.
\end{prop}

\begin{proof}
    Suppose $\rho(S) = m/n$, then the Laurent series for $\zeta_{K,S}^n$ is
    \[a(s-1)^{-m} + \sum_{r > -m} a_{r}(s-1)^{r}.\]
    Since $\zeta_{K,S}(s)>0$ for real $s>1$, $a>0$. Taking logarithms on both sides,
    \[n \sum_{\fp\in S} \N(\fp)^{-s} \sim m\log \frac{1}{s-1}\]
    as $s\to 1^+$. This shows that $d(S) = m/n = \rho(S)$.
\end{proof}


\begin{prop}
    Let $S,T$ be sets of primes in a number field $K$. Let $\cP$ be the set of all primes, and $\cP_1$ the set of primes with $f = 1$. Then:
    \begin{enumerate}[(a)]
        \item If $S$ is finite, $\rho(S) = 0$. If $\cP\backslash S$ is finite, then $\rho(S) = 1$.
        \item If $S\subseteq T$ then $\rho(S)\le \rho(T)$ if both exist.
        \item If $S\cap T$ is finite, then $\rho(S\cup T) = \rho(S) + \rho(T)$ whenever two of the three exist.
        \item $\rho(\cP_1) = 1$, and $\rho(S\cap \cP_1) = \rho(S)$ whenever $S$ has polar density.
    \end{enumerate}
\end{prop}

\begin{proof}
    (d) Let $\cP_2$ be the other primes. The key fact here is that there are at most $n = [K:\QQ]$ primes above $p$ in $\cP_2$, each with norm at least $p^2$. So $\zeta_{K,\cP_2}(s) < \zeta^n(2s)$, so $\zeta_{K,\cP_2}$ is holomorphic and vanishing around 1.
\end{proof}


\section{Lecture, Feb. 15}

(Surjectivity of Artin map, conductors)

We begin by commenting that all this works for global functions as well, only the proofs will be sllightly different. Our goal in this lecture is to show surjectivity of the Artin map.

\begin{thm}
    Let $L/K$ be Galois extensions of number fields of degree $n$. Let $\Spl(L/K)$ be the set of primes in $K$ that split completely in $L$. Then $\rho(\Spl(L/K)) = 1/n$.
\end{thm}

\begin{proof}
    Let $S$ be the set of degree-1 primes that split completely, it suffices to show $\rho(S) = 1/n$. For these $\fp$, $e = f = 1$. Let $T = \{\fq\mid \fp: \fp\in S\}$, then $\N_{L/K}(\fq) = \fp$, and $\N(\fq) = \#(\cO_L/\fq) = \N(\fp)$, so $\fq$ is degree 1 as well. On the other hand, any unramified $\fq$ of degree 1 must lie above an unramified degree-1 prime $\fp$, which is in $S$; so all but finitely many (ramified) degree-1 primes $\fq\in T$. This means $\rho(T) = 1$.

    Each prime $\fp\in S$ has $n$ primes above it in $T$. So
    \[\zeta_{L, T} = \prod_{\fq\in T} \frac{1}{1 - \N(\fq)^{-s}} = \prod_{\fp\in S} \frac{1}{(1-\N(\fp)^{-s})^n} = \zeta_{K, S}^n.\]
    This shows $\rho(S) = \frac1n\rho(T) = \frac1n$.
\end{proof}

\begin{cor}
    Let $L/K$ be a finite extension with Galois closure $M/K$ of degree $n$. Then $\rho(\Spl(L/K)) = \rho(\Spl(M/K)) = \frac1n$.
\end{cor}

\begin{proof}
    A prime $\fp\subset K$ splits completely in $L$ iff it splits completely in every conjugate of $L$ in $M$, iff it splits completely in $M$.
\end{proof}

\begin{cor}
    Let $L/K$ be finite Galois with Galois group $G$, and $H\lhd G$. Then $S = \{\fp\in K: \Frob_\fp \subseteq H\}$ has polar density $\rho(S) = \#H/\#G$. 
\end{cor}

\begin{proof}
    We have $\Gal(L^H/K) \cong G/H$, and $\Frob_\fp \subseteq H$ iff every $\Frob_\fq$ fixes $L^H$ for $\fq\mid \fp$ in $L$, iff $\fp$ splits completely in $L^H$.
\end{proof}

Write $S\sim T$ if $S\triangle T$ is finite; $S\lesssim T$ if $S-T$ is finite.

\begin{lem}
    Let $L/K, M/K$ be finite Galois extensions, and $LM$ be their compositum. Then a prime in $K$ splits completely (resp. is unramified) in $LM$ iff it splits completely (resp. is unramified) in both $L$ and $M$.
\end{lem}

\begin{proof}
    Use the fact that for a tower of Galois extensions $M/N/K$, if $\fp\subset K$ and $\fq\subset M$ lies above $\fp$, then $D(\fq)$ fixes $N$ iff $e_\fp(N/K) = f_\fp(N/K) = 1$. Then since $\fp$ splits completely in both $L$ and $M$, for any $\fq$ in $LM$ above $\fp$, both $L,M\subseteq (LM)^{D_\fq}$, hence $LM\subseteq (LM)^{D_\fq}$, hence $|D_\fq| = 1$.
\end{proof}

\begin{thm}
\label{two_galois_spl_equal}
    If $L/K$, $M/K$ are finite Galois, then 
    \[L\subseteq M \iff \Spl(M)\subseteq \Spl(L) \iff \Spl(M) \lesssim \Spl(L).\]
\end{thm}

\begin{proof}
    The nontrivial direction is showing that $\Spl(M)\lesssim \Spl(L) \implies L\subseteq M$. Consider the compositum $LM$, then a prime $\fp$ in $K$ splits completely in $LM$ if and only if it splits completely in both $L$ and $M$. So $\Spl(LM)\sim \Spl(M)$. This implies $\frac{1}{[M:K]} = \frac{1}{[LM:K]}$, so $LM = M$, so $L\subseteq M$.
\end{proof}

\begin{thm}[the Artin map is surjective]

Let $L/K$ be finite abelian, $\fm$ a modulus divisible by all primes in $K$ that ramify and all real places in $K$ that ramify (that extend to a complex place). Then
\[\psi_{L/K}^m: \cI_{L/K}^\fm \to \Gal(L/K)\]
is surjective.
\end{thm}

\begin{proof}
    Let $H$ be the image, and $F:= L^H$; we will show $F = K$.

    For any $\fp\in \cI_{L/K}^\fm$, $\psi_{L/K}^\fm(\fp) \in H$, so $\Frob_\fp$ acts trivially on $F$, so $\fp$ splits completely in $F$. But $\cI_{L/K}^\fm$ contains all but finitely many primes, so $\rho(\Spl(F/K)) = 1$. But $\rho(\Spl(F/K)) = \frac{1}{[F:K]}$, so $F=K$ as desired.
\end{proof}



\begin{thm}
    Let $\fm$ be a modulus for $K$, and $L/K,M/K$ finite abelian extensions unramified away from $\fm$. If $\ker\psi_{L/K}^\fm = \ker\psi_{M/K}^\fm$, then $L = M$. In particular, the ray class field is unique (only depends on $\fm$).
\end{thm}


\begin{proof}
    Consider the set $S$ of primes not dividing $\fm$. Then $\fp\in S$ splits completely in $L$ iff it is in $\ker\psi_{L/K}^\fm$. So $\Spl(L/K) \sim S\cap \ker\psi_{L/K} = S\cap \ker\psi_{M/K} \sim \Spl(M/K)$, so $L = M$ by applying Theorem \ref{two_galois_spl_equal} twice.
\end{proof}


By surjectivity of the Artin map, if the ray group $\cR_K^\fm\subseteq \ker\psi_{L/K}^\fm$, then $\Gal(L/K)$ is a quotient of $\Cl_K^\fm$, with equality iff $L$ is the ray class field, which we denote by $K(\fm)$. In general, the intermediate fields between $K$ and $K(\fm)$ correspond 1-to-1 to subgroups between $\cR_K^\fm$ and $\cI_K^\fm$, by $L\mapsto \cC = \ker\psi_{L/K}^\fm$ and $\cI_K^\fm/\cC \cong \Gal(L/K)$.


Given a finite abelian $L/K$, there may be many choices of $\fm$, and as we make $\fm$ smaller, the ray group $\cR_K^\fm$ gets bigger so that it might not be contained inside $\ker\psi_{L/K}^\fm$. Fortunately there is a minimal modulus that works, called the \emph{conductor}, for which $\cR_K^\fm\subseteq \ker\psi_{L/K}^\fm$, which implies $\Spl(K(\fm)) \subseteq \Spl(L)$, which implies $L\subseteq K(\fm)$.


\begin{defn}
    A \emph{congruence subgroup} for a modulus $\fm$ in a global field $K$ is a subgroup $\cC\subseteq \cI_K^\fm$ that contains the ray group $R_K^\fm$.
\end{defn}


\begin{defn}
    For two congruence subgroups $\cC_1$ for $\fm_1$ and $\cC_2$ for $\fm_2$, say that
    \[(\cC_1,\fm_1)\sim (\cC_2,\fm_2)\]
    iff $\cI_K^{\fm_1}\cap \cC_2 = \cI_K^{\fm_2}\cap \cC_1$. This defines an equivalence relation, and if $\fm_1=\fm_2$ then $\cC_1=\cC_2$. 
\end{defn}

The reason we are interested in this equivalence relation, is that if $(\cC_1,\fm_1)\sim (\cC_2, \fm_2)$, then $\cI_K^{\fm_1}/\cC_1\cong \cI_K^{\fm_2}/\cC_2$ canonically, and the isomorphism preserves cosets of ideals coprime to $\fm_1\fm_2$. And these quotients are what we really care about.

If $\cC$ is a congruence subgroup for two moduli $\fm_1$ and $\fm_2$, then $(\cC, \fm_1)\sim (\cC, \fm_2)$. So each subgroup $\cC\subseteq \cI_K$ lies in at most one equivalence class. So we can just write $\cC_1\sim\cC_2$ without specifying the moduli. Also, within one equivalence class, there can be at most one congruence subgroup with a specified modulus.

\begin{lem}
    Let $(\cC_1,\fm_1)$ be a congruence subgroup, and $\fm_2\mid \fm_1$. There exists $(\cC_2,\fm_2)$ in the same equivalence class iff
    \[\cI_K^{\fm_1} \cap \cR_K^{\fm_2} \subseteq \cC_1,\]
    in which case $\cC_2 = \cC_1\cR_K^{\fm_2}$.
\end{lem}

\begin{prop}
    If $(\cC_1,\fm_1)\sim (\cC_2,\fm_2)$, then there exists a congruence subgroup $\cC$ in the same equivalence class, with modulus $\fm = \gcd(\fm_1,\fm_2)$.
\end{prop}


\begin{cor}
    If $(\cC, \fm)$ is a congruence subgroup, then there exists a unique $\cC'\sim \cC$ whose modulus divides that of any $\cC''\sim \cC$. 
\end{cor}

\begin{defn}
    The unique modulus $\fc = \fc(\cC)$ given by the above corollary is called the \emph{conductor} of $\cC$. We say $\cC$ is primitive if $\cR_K^\fc \subseteq \cC$.
\end{defn}

\begin{prop}
    If $\cC$ is a primitive congruence subgroup of modulus $\fm$, then $\fm$ is the conductor of all $\cC'\subset \cC$ with modulus $\fm$. In particular, $\fm$ is the conductor of $\cR_K^\fm$.
\end{prop}

\begin{proof}
    Suppose $\cC'\subseteq \cC$ with modulus $\fm$, and let $(\cC_0,\fc)$ be its conductor. Obviously $\fc\mid\fm$. On the other hand, 
    \[\cI_K^\fm\cap \cR_K^\fc \subseteq \cI_K^\fm\cap \cC_0 = \cI_K^\fc \cap \cC' \subseteq \cC'\subseteq \cC,\]
    so if we let $\cC'' = \cC\cR_K^\fc$, then $\cC''$ has modulus $\fc$ and 
    \[\cI_K^\fc\cap \cC = \cC = \cC(\cI_K^\fm\cap \cR_K^\fc) = \cI_K^\fm \cap \cC\cR_K^\fc = \cI_K^\fm\cap \cC'',\]
    so $\cC\sim\cC''$. Because $\cC$ is primitive, $\fm\mid \fc$. So $\fc=\fm$.
\end{proof}


\begin{exm}
    Let $K= \QQ$, $\fm = (2)$. Then $\cR_\QQ^{(2)} = \cI_\QQ^{(2)}$ has conductor $(1)$, since it is equivalent to $\cI_\QQ^{(1)}$. So $(2)$ is not the conductor of any congruence subgroup of $\QQ$.
\end{exm}

\begin{exm}
    Let $K = \QQ$, $L = K[x]/(x^3-3x-1)$, $G =\Gal(L/K) = \ZZ/3\ZZ$. This is unramified away from $(3)$, since it has discriminant 81. So the Artin map makes sense for any modulus divisible by 3. The ray class field for $(3)$ is $\QQ(\zeta_3)^+ = \QQ$, and the ray class field for $(3)\infty$ is $\QQ(\zeta_3)$. These both have degree at most 2, so cannot contain $L$; equivalently, $\cR_K^\fm$ is not contained in $\ker\psi_{L/K}^\fm$. The correct modulus to use is $\fm = (9)$, and indeed $L = \QQ(\zeta_9)^+$ is the ray class field for $(9)$.
    
    In general, the ray class field for $(n)$ is $\QQ(\zeta_n)^+$, and the ray class field for $(n)\infty$ is $\QQ(\zeta_n)$.
\end{exm}


\section{Lecture, Feb. 21}

(Ray class characters, Weber $L$-functions)

\begin{defn}
    A totally multiplicative function $\chi: \cI_K\to \CC$ with finite image for which $\cR_K^\fm \subseteq \ker(\chi) := \chi^{-1}(1)$ and $\cI_K^\fm = \chi^{-1}(U_1)$ (unit circle) is a \emph{ray class character} of $\fm$. Equivalently, $\chi$ is the extension by zero of a character of the finite abelian group $\Cl_K^\fm$.
\end{defn}


\begin{exm}
    When $K= \QQ$, a ray class character of modulus $(m)\infty$ is just a Dirichlet character of modulus $m$, and its conductor divides $(m)$ iff the character is \emph{even}, i.e. $\chi(-1) = 1$.
\end{exm}

\begin{defn}
    Suppose $\chi_1,\chi_2$ are ray class characters of moduli $\fm_1\mid \fm_2$. If $\chi_2(I) = \chi_1(I)$ for all ideals $I\in \cI_K^{\fm_2}$, then we say $\chi_2$ is \emph{induced} by $\chi_1$. A ray class character is \emph{primitive} if it is not induced by any character other than itself.
\end{defn}

\begin{defn}
    The \emph{conductor} of a ray class character is the conductor of its kernel (which is a congruence subgroup).
\end{defn}

\begin{prop}
    A ray class character is primitive iff its kernel is primitive, and every ray class character is induced by a primitive one.
\end{prop}

\begin{proof}
    Let $\chi$ be a ray class character with (some) modulus $\fm$. Let $\kappa$ be the corresponding group character on $\cI_K^\fm/\ker\chi$. Let $\cC$ be the primitive congruence subgroup equivalent to $\ker\chi$, with modulus $\fc$, the conductor, dividing $\fm$. We have a canonical isomorphism $\phi: \cI_K^\fc/\cC \to \cI_K^\fm/\ker\chi$. Let $\wt{\chi}$ be the ray class character of $\fc$ that is the extension by zero of $\kappa\circ \phi$. By definition of $\phi$, $\wt{\chi}(I) = \chi(I)$ for $I\in \cI_K^\fm$, so $\chi$ is induced by $\wt{\chi}$ (whose kernel is primitive).

    In general, if $(\chi_2,\fm_2)$ is induced by $(\chi_1,\fm_1)$, then $\ker\chi_1\cap \cI_K^{\fm_2} = \ker\chi_2 = \ker\chi_2\cap \cI_K^{\fm_1}$, so $\ker\chi_1$, $\ker\chi_2$ are equivalent. If, furthermore, $\chi_1\neq\chi_2$, then $\cI_K^{\fm_1}\neq \cI_K^{\fm_2} \implies \fm_1\neq \fm_2$. Applying this to the above situation of $\chi$ and $\wt{\chi}$: if $\wt{\chi}$ is induced by some other character with modulus $\fc'$, then $\fc$ cannot divide $\fc'$, a contradiction; so $\wt{\chi}$ is primitive. Moreover, $\chi$ is primitive iff $\chi = \wt{\chi}$ iff $\ker\chi = \ker\wt{\chi}$ is primitive.
\end{proof}

For a modulus $\fm$, ley $X(\fm)$ denote the set of primitive ray class characters of conductor
dividing $\fm$, which is in bijection with the character group of $\Cl_K^\fm$. For a congruence subgroup $\cC$ of modulus $\fm$, let $X(\cC)$ denote the set of primitive ray class characters
whose kernels contain $\cC$, and $X(\cC)$ is in bijection with the character group of $\cI_K^\fm/\cC$, a subgroup of $X(\fm)$. (Why?)


\begin{defn}
    A ray class character is \emph{principal} if $\ker\chi = \chi^{-1}(U_1)$. We use $\mathbf{1}$ to denote the unique primitive principal ray class group. (It is not the unique primitive character of conductor $(1)$; when $\Cl_K$ is nontrivial, any character on $\Cl_K$ induces a primitive character of conductor $(1)$, but only one is principal.)
\end{defn}


\begin{defn}[Weber $L$-function]
The \emph{Weber $L$-function} $L(s,\chi)$ of ray class character $\chi$ is
\[L(s,\chi) = \prod_{\fp\in K} \frac{1}{1 - \chi(\fp)\N(\fp)^{-s}} = \sum_{\fa} \chi(\fa)\N(\fa)^{-s},\]
which converges absolutely to a nonvanishing holomorphic function for $\Re(s) > 1$.
\end{defn}

This generalizes Dirichlet $L$-functions ($K = \QQ$) and Dedekind zeta functions ($\chi = \mathbf{1}$), both of which generalize the Riemann  zeta function.


\begin{prop}
    Let $\chi$ be a ray class character for a global field $K$. Then $L(s,\chi)$ extends to a meromorphic function on a neighborhood of $s=1$, with a simple pole at $s=1$ if $\chi = \mathbf{1}$ and holomorphic otherwise.
\end{prop}

\begin{proof}
    Wait for Tate's thesis.
\end{proof}


\begin{prop}
    Let $\cC$ be a congruence subgroup of modulus $\fm$ for $K$. Let $n = [\cI_K^\fm:\cC]$, then $S = \{\fp\in \cC\}$ has Dirichlet density
    \[d(S) = 
    \begin{cases}
    1/n, & \text{if } L(1, \chi) \neq 0 \text{ for all } \chi\neq \mathbf{1} \text{ in } X(\cC); \\
    0, & \text{otherwise.}
    \end{cases}\]
\end{prop}
(Actually the second case never happens, but that will be shown later.)

\begin{proof}
    By character theory,
    \[\frac{1}{n} \sum_{\chi\in X(\cC)} \chi(\fp) = \begin{cases}
    1, & \text{if }
    \fp \in \cC; \\
    0, & \text{otherwise.}
    \end{cases}\]
    Because as $s\to 1^+$,
    \[
    \log L(s,\chi) \sim \sum_{\fp} \chi(\fp)\N(\fp)^{-s},
    \]
    we have
    \begin{align*}
        \sum_{\chi\in X(\cC)} \log L(s,\chi) 
        &\sim \sum_{\chi\in X(\cC)} \sum_\fp \chi(\fp)\N(\fp)^{-s} \\
        &= \sum_\fp \N(\fp)^{-s} \sum_{\chi\in X(\cC)} \chi(\fp) \\
        &= n\sum_{\fp\in \cC} \N(\fp)^{-s}.
    \end{align*}
    By the above proposition, near $s=1$, $L(s,\chi) = (s-1)^{e(\chi)}g(s)$ where $g$ is holomorphic and nonvanishing, and $e(\chi) = -1$ if $\chi = \mathbf{1}$ and $e(\chi) \ge 0$ otherwise. So
    \[n\sum_{\fp\in \cC}\N(\fp)^{-s} \sim \log  \frac{1}{s-1} - \sum_{\chi\neq \mathbf{1}} e(\chi)\log\frac{1}{s-1}\]
    as $s\to 1^+$. This is equivalent to saying as $s\to 1^+$,
    \[0\le d(S) = \frac{\sum_{p\in \cC} \N(\fp)^{-s}}{\log \frac{1}{s-1}} = \frac{1 - \sum_{\chi\neq \mathbf{1}} e(\chi)}{n},\]
    which is either 0 or $1/n$ depending on whether one of the $e(\chi) = 1$.
\end{proof}


\begin{prop}
    Let $\cC$ be a congruent subgroup of modulus $\fm$, $n = [\cI_K^\fm: \cC]$. Then for any $I\in \cI_K^\fm$, the coset $\{\fp\in IC\}$ has Dirichlet density the same as the trivial coset.
\end{prop}

\begin{proof}
    Same proof, just change the indicator function.
\end{proof}

\begin{cor}
    The coset $\{p\in IC\}$ has Dirichlet density $1/n$ (so the second possibility never occurs), and every non-primitive $\chi \in X(\cC)$ is nonvanishing at $s=1$.
\end{cor}

\begin{proof}
    Summing over all cosets, the Dirichlet densities should add up to 1.
\end{proof}

\begin{cor}
\label{main_ineq}
    Let $L/K$ be a finite abelian extension, $\cC$ a congruence subgroup of modulus $\fm$. If $\Spl(L/K) \lesssim \{\fp\in \cC\}$, then $[I_K^\fm : \cC] \le [L:K]$.
\end{cor}

\begin{proof}
    We know $\Spl(L/K)$ has polar density (hence also Dirichlet density) $1/[L:K]$, and $\{p\in \cC\}$ has Dirichlet density $1/[I_K^\fm: \cC]$.
\end{proof}


\section{Lecture, Feb. 22}

(Second main inequality of CFT, simple pole of $\zeta_K$ at $s=1$)

\begin{defn}
    Let $L/K$ be a finite abelian extension of \emph{local} fields, then the \emph{conductor} 
    \[
    \fc(L/K) := \begin{cases}
        1, & \text{if } L=\CC, K = \RR \\
        0, & \text{if } L = K \text{ archimedean} \\
        \min\{n: 1 + \fp^n \subseteq \N_{L/K}(L^\times)\}, & \text{otherwise.}
    \end{cases}
    \]
    For a finite abelian extension of \emph{global} fields, $\fc(L/K)$ is a map from $M_K$ (the set of places of $K$) to $\ZZ$, given by mapping $v \mapsto \fc(L_w/K_v)$, where $w$ is any place above $v$. (Since $L/K$ is Galois, this does not depend on the choice of $w$.) 
\end{defn}

\begin{prop}
    Let $L/K$ be a finite abelian extension of local or global fields. For each prime $\fp$ of $K$, 
    \[v_\fp(L/K) = \begin{cases}
    0 & \text{if } \fp \text{ is unramified} \\
    1 & \text{if } \fp \text{ is tamely ramified} \\
    \ge 2 & \text{if } \fp \text{ is wildly ramified.}
    \end{cases}\]
\end{prop}

\begin{proof}
    Pset 2.
\end{proof}


\begin{Rem}
    The conductor and the discriminant are supported on the same primes (but the valuations can be very different).
\end{Rem}

\begin{lem}
    Let $L_1,L_2$ be finite abelian extensions of local or global fields. Suppose $L_1\subseteq L_2 \implies \fc(L_1/K) \mid \fc(L_2/K)$.
\end{lem}

\begin{proof}
    In the local nonarchimedean case, $\N_{L_2/K}(L_2^\times) = \N_{L_1/K}(\N_{L_2/L_1}(L_2^\times)) \subseteq \N_{L_1/K}(L_1^\times)$. In the local archimedean case this is obvious. So this also holds for global fields.
\end{proof}


\begin{defn}
    Let $L/K$ be a finite abelian extension of global fields, $\fm$ a modulus divisible by $\fc(L/K)$. The \emph{norm group} (also \emph{Takagi group}) for $\fm$ is
    \[T_{L/K}^\fm = \cR_K^\fm \N_{L/K}(\cI_L^\fm),\]
    where $\cI_L^\fm$ are the fractional $\cO_L$-ideals coprime to $\fm_0\cO_L$. 
\end{defn}

\begin{prop}
    Let $L/K$ be a finite abelian extension of global fields, $\fm$ a modulus divisible by $\fc(L/K)$, then $\Spl(L/K)\lesssim \{\fp\in T_{L/K}^\fm\}$.
\end{prop}

\begin{proof}
    Suppose $\fp$ is coprime to $\fm$, and splits completely in $L$, so $e_\fp = f_\fp = 1$. Pick $\fq\mid\fp$, then $\fq \in \cI_K^\fm$ and $\N_{L/K}(\fq) = \fp$, so $\fp$ is in $T_{L/K}^\fm$.
\end{proof}

\begin{thm}[second main inequality]
\label{2ndMainIneq}
    Let $L/K$ be a finite abelian extension of global fields, $\fm$ a modulus divisible by $\fc(L/K)$. Then
    \[[\cI_K^\fm: T_{L/K}^\fm] \le [L:K].\]
\end{thm}

\begin{proof}
    Follows from corollary \ref{main_ineq}.
\end{proof}

The goal now is to show that this is actually an equality. What we are working towards is the following:

\begin{thm}[global CFT, via ideals]
\label{idealCFT}
    The main theorems of ideal-theoretic CFT:
    \begin{itemize}
        \item The ray class field $K(\fm)$ exists;
        \item For $L/K$ finite abelian extension, $L\subseteq K(\fm)$ iff $\fc(L/K)\mid\fm$.
        \item Artin reciprocity: If $L\subseteq K(\fm)$, then $\ker\psi_{L/K}^\fm = T_{L/K}^\fm$, its conductor is $\fc(L/K)\mid\fm$, and $\cI_K^\fm/T_{L/K}^\fm \cong \Gal(L/K)$ canonically.
    \end{itemize}
\end{thm}

Artin reciprocity gives the following commutative diagram of canonical bijections:

\[
\begin{tikzcd}
    \{\text{finite abelian } L/K \text{ with } \fc(L/K)\mid \fm\} \arrow[r, "{L\mapsto T_{L/K}^\fm}"] \arrow[d, "{L\mapsto \Gal(L/K)}"'] & \{\text{congruence subgroups of modulus } \fm\} \arrow[d, "{\cC\mapsto I_K^\fm/\cC}"] \\
    \{\text{quotients of } \Gal(K(\fm)/K)\} & \{\text{quotients of } \Cl_K^\fm\} \arrow[l, "{\psi_{L/K}^\fm}"]
\end{tikzcd}
\]

\begin{defn}
    The \emph{Hilbert class field} of a global field $K$ is the maximal unramified abelian extension of $K$ (in some fixed algebraic closure).
\end{defn}

From class field theory, taking the trivial modulus, we see in particular that this is a finite extension, which is already not obvious.

In the rest of the lecture, we show that $\zeta_K(s)$ can be meromorphically continued to have a simple pole at $s = 1$. We use the following fact without proof:


\begin{prop}
    Let $a_1,a_2,\dots\in \CC$ be a sequence of complex numbers, $\rho$ a nonzero real, and $\sigma\in [0,1)$, such that $\sum_{k=1}^t a_k = \rho t + O(t^\sigma)$, then $\sum a_nn^{-s}$ has a meromorphic continuation to $\Re(s) > \sigma$ with a simple pole at $s=1$ with residue $\rho$. \qed
\end{prop}

So to show analytic continuation of $\zeta_K(s)$, it suffices to show that $\#(\fa: \N(\fa)\le t) = \rho t + O(t^\sigma)$ for $\sigma\in [0,1)$. The strategy is to first count the principal ideals, then count the ideals by partitioning into ideal classes: note that if we fix ideal class representatives $\fa\in \cI_K$. Then
\begin{align*}
\{\text{integral ideals } \fb\in [\fa^{-1}]: \N(\fb) \le t\} &\xto{\cong} \{\text{nonzero principal integral } (\alpha)\subseteq \fa: \N(\alpha)\le t\N(\fa)\} \\
&\xto{\cong} \{\text{nonzero integral } \alpha\in \fa: \N(\alpha)\le t\N(\fa)\}/\cO_K^\times.
\end{align*}
by multiplying by $\fa$.


Recall that for a number field $K$, $K_\RR := K\otimes_\QQ \RR = \prod_{v\mid\infty} K_v = \RR^{r_1}\times \CC^{r_2}$. We have an injection $K^\times\into K_\RR^\times$ by embedding diagonally, and a map
\[\Log: K_\RR^\times \to \RR^{r_1+r_2}\]
sending $(x_v)\mapsto \log \norm{x_v}_v$, where $\norm{}_v$ is the usual norm in $\RR$ and the square of the absolute value in $\CC$. By Dirichlet's unit theorem, $\cO_K^\times = \mu_K\times U$, where $\Log$ maps $\cO_K^\times$ into a full lattice $\Lambda_K$ in $\RR^{r_1+r_2}_0$, with kernel $\mu_K$.

Define $\nu: K_\RR^\times \to K_{\RR,1}^\times$ by $x \N(x)^{-1/n}$, where $n = r_1+2r_2$. Then $\Log(\nu(K_\RR^\times)) = \RR^{r_1+r_2}_0$. Let us fix a fundamental domain $F$ for the lattice $\Lambda_K$ (whose covolume is $R_K$, the \emph{regulator}), and let $S := \nu^{-1}(\Log^{-1}(F))$. Then $S$ is a set of coset representatives for $K_\RR^\times/U$. Let $S_{\le t} = \{x\in S: \N(x)\le t\} \subseteq K_\RR \cong \RR^n$. It then suffices to estimate $\#(S_{\le t}\cap \cO_K)$: the method only uses the fact that $\cO_K$ is a lattice, so the same method will work for counting $\#(S_{\le t}\cap \fa)$.

Since $t^{1/n}S_{\le 1} = S_{\le t}$ (where we work in $\RR^n$), what we want is:

\begin{prop}
    Let $\Lambda$ be a lattice in $V\cong \RR^n$, let $S$ be a ``nice'' (Lebesgue) measurable set, then $\#(tS\cap\Lambda) = \frac{\mu(S)}{\covol(\Lambda)}t^n + O(t^{n-1})$.
\end{prop}

This would imply that $\#(S_{\le t}\cap \cO_K) = \rho t + O(t^{1-\frac{1}{n}})$, which is the bound we want. We now need to say what it means to be ``nice''.

\begin{defn}
    Let $X,Y$ be metric spaces. A map $f:X\to Y$ is \emph{Lipschitz continuous} if there exists $c>0$, such that $d(f(u),f(v))\le cd(u,v)$ for all $u,v\in X$.
\end{defn}

This is a stronger condition than uniform continuity.


\begin{defn}
    A set $B$ in a metric space $X$ is \emph{$d$-Lipschitz parametrizable} if it is the union of finitely many images for Lipschitz-continuous functions $f:[0,1]^d\to X$.
\end{defn}

\begin{lem}
    Let $S\subseteq \RR^n$ be measurable with boundary $(n-1)$-Lipschitz parametrizable. Then $\#(tS\cap \ZZ^n) = \mu(S)t^n + O(t^{n-1})$. \qed
\end{lem}

So what we need to show is that $\partial S_{\le 1}$ is $(n-1)$-Lipschitz parametrizable. The kernel of $\Log$ is $(\pm 1)^{r_1}\times U(1)^{r_2}$. We thus have a continuous isomorphism of locally compact groups
\[K_{\RR}^\times \to \RR^{r_1+r_2} \times \{\pm 1\}^{r_1} \times [0,2\pi)^{r_2}\]
mapping $(x_1,\dots,x_{r_1},z_1,\dots,z_{r_2}) \mapsto (\Log x) \times (\sgn x_1,\dots, \sgn x_{r_1})\times (\arg z_1,\dots,\arg z_{r_2})$.

Analyzing $S_{\le 1}$, it has $2^{r_1}$ connected components, each parametrized by $n$ parameters: 
\begin{itemize}
    \item $r_1+r_2-1$ parameters in $[0,1)$ encoding a point in $F$ as an $\RR$-linear combination of $\Log$ applied to a basis of $U$;
    \item $r_2$ parameters in $[0,1)$ encoding an element of $U(1)$;
    \item one parameter in $(0,1]$ encoding the $n$-th root of the norm.
\end{itemize}
This gives a continuously differentiable bijection from $[0,1)^{n-1}\times (0,1]$ to a connected component of $S_{\le 1}$. So its boundary is clearly $(n-1)$-Lipschitz parametrizable, proving the theorem.

\begin{Rem}
    If we keep track of the coefficient of the linear term, we actually get the analytic class number formula.
\end{Rem}


\section{Lecture, Feb. 27}

(Group cohomology I)


\begin{defn}
    Let $G$ be \emph{any} group, a \emph{left $G$-module} is an abelian group $A$ with a compatible $G$-action: $g(a+b) = ga + gb$. Equivalently, $A$ is a left $\ZZ[G]$-module. A \emph{morphism} of $G$-modules is a morphism of $\ZZ[G]$-modules. The category of $G$-modules is denoted $\Mod_G$. Since it is just the category of modules over a ring $\ZZ[G]$, it is an abelian category.
\end{defn}

\begin{Rem}
    When $G$ is a topological group, we need to require the $G$-action to be continuous.
\end{Rem}

\begin{exm}
    Examples of $G$-modules:
    \begin{itemize}
        \item If $A$ is any abelian group, $A$ can be made into a \emph{trivial $G$-module}, i.e. $G$ acts trivially.
        \item For $L/K$ Galois extension, the abelian groups $L, L^\times, \cO_L, \cO_L^\times$ are all $\Gal(L/K)$-modules.
        \item For $A,B\in \Mod_G$, the abelian group $\Hom_{\Ab}(A,B)$ has a natural $G$-module structure: $(g\phi)(a) = g\phi(g^{-1}a)$.
    \end{itemize}
\end{exm}

\begin{defn}
    For $A\in\Mod_G$, the subgroup $A^G = \{a\in A: ga = a \text{ for all } g\in G\}$ is the subgroup of \emph{$G$-invariants}.
\end{defn}

\begin{exm}
    $\Hom_G(A,B) \cong \Hom_{\Ab}(A,B)^G$. In particular, $\Hom_G(\ZZ, A)\cong A^G$. 
\end{exm}

Any morphism of $G$-modules $A\to B$ restricts to a morphism $A^G\to B^G$. We thus have a functor $\bullet^G: \Mod_G \to\Mod_G$ (in fact the subcategory of trivial $G$-modules, which is just $\Ab$), which is left exact because it is $\Hom_G(\ZZ,\bullet)$. (Recall that this is exact iff $\ZZ$ is a projective $\ZZ[G]$-module, which is not true when $G$ is nontrivial.) 

The category $\Mod_G$ is in fact a Grothendieck category (in particular, has enough injectives). So we can define $H^n(G,A)$ to be the $n$-th right derived functors of the left exact $\bullet^G : \Mod_G\to \Ab$. In particular $H^0(G, A) = A^G$.

Now, we give another definition of group cohomology using cochains.

\begin{defn}
    Let $A$ be a left $G$-module, $n\ge 0$. The group $C^n(G,A)$ of \emph{$n$-cochains} is the abelian group of maps of sets $f:G^n\to A$, under pointwise addition. The $n$-th \emph{coboundary map} is a homomorphism $d^n: C^n(G,A) \to C^{n+1}(G,A)$ given by
    \[d^nf(g_0,\dots,g_n) := g_0f(g_1,\dots,g_n) + \sum_{i=1}^n (-1)^if(\dots, g_{i-2},g_{i-1}g_i, g_{i+1},\dots) + (-1)^{n+1}f(g_0,\dots,g_{n-1}).\]
    Define the $n$-cocycles and $n$-coboundaries $Z^n(G,A) = \ker d^n$ and $B^n(G,A) = \im d^{n-1}$. Since $d^{n+1}d^n = 0$, $B^n(G,A)\subseteq Z^n(G,A)$. In other words, we get a cochain complex
    \[0\to C^0(G,A) \to C^1(G,A) \to C^2(G,A) \to\dots,\]
    and the \emph{$n$-th cohomology group of $G$ with coefficients in $A$} is
    \[H^n(G,A) = \frac{Z^n(G,A)}{B^n(G,A)}.\]
\end{defn}


\begin{exm}
    Low-degree cohomologies:
    \begin{itemize}
        \item $C^0(G,A)\cong A$;
        \item $d^0: C^0(G,A)\to C^1(G,A)$ sends $a\mapsto (g\mapsto ga-a)$;
        \item $H^0(G,A) = \ker d^0 = A^G$;
        \item $B^1(G,A)$ is the group of \emph{principal crossed homomorphisms};
        \item $d^1: C^1(G,A)\to C^2(G,A)$ sends $f \mapsto ((g,h) \mapsto gf(h) - f(gh) + f(g))$.
        \item $Z^1(G,A) = \ker d^1$ consists of $f:G\to A$ such that $f(gh) = f(g) + gf(h)$. This is the group of \emph{crossed homomorphisms}.
        \item $H^1(G,A) = Z^1(G,A)/B^1(G,A)$ are the crossed homomorphisms modulo the principal ones.
        \item If $A = A^G$, then $H^1(G,A) = \Hom_{\Grp}(G,A) = \Hom_{\Ab}(G^{\text{ab}}, A)$.
    \end{itemize}
\end{exm}


\section{Lecture, Mar. 1}

(Group cohomology II)


We give a useful interpretation of $H^2(G,A)$.

\begin{defn}
    Let $A\in \Mod_G$, a group extension $E$ of $G$ by $A$ is a short exact sequence of groups:
    \[0\to A\to E\to G \to 0,\]
    such that for any set-theoretic section $s: G\to E$, we have $s(g)as(g)^{-1} = ga$.
\end{defn}

In other words, $A$ has a $G$-action because it is a $G$-module, and $G\cong E/A$ also acts on $A$ by conjugation, and we require these two actions to be the same.

Two extensions $E,E'$ are \emph{isomorphic} if there is an isomorphism $\theta: E\to E'$ such that
\[
\begin{tikzcd}
    0 \arrow[r] & A \arrow[r] \arrow[d, equal] & E\arrow[r] \arrow[d, "\theta"] & G\arrow[r]\arrow[d, equal] & 0 \\
    0\arrow[r] & A\arrow[r] & E'\arrow[r] & G\arrow[r] & 0
\end{tikzcd}
\]
commutes.

\begin{prop}
    $H^2(G,A)$ is canonically the abelian group of isomorphism classes of extensions of $G$ by $A$, which sends $f:G^2\to A$ to $E_f = A\times G$ (as a set) with the group law
    \[(a,g)\cdot (b,h) = (a + gb + f(g,h), gh).\]
    By definition, the image of $0\in H^2(G,A)$ is $A\rtimes G$.
\end{prop}


\begin{lem}
    Given a map of $G$-modules $\alpha:A\to B$, there is an induced map of cochain complexes $C^\bullet(G,A)\to C^\bullet(G,B)$  (which in turn induces maps $\alpha^n: H^n(G,A)\to H^n(G,B)$).
\end{lem}

\begin{proof}
    It suffices to show that $\alpha^n: C^n(G,A)\to C^n(G,B)$ commutes with $d^n$. For $f\in C^n(G,A)$,
    \begin{align*}
    \alpha^{n+1}d^nf(g_0,\dots, g_n) &= \alpha(g_0f(g_1,\dots,g_n) + \sum_{i=1}^n (-1)^n f(\dots, g_{i-1}g_i, \dots) + f(g_0,\dots,g_{n-1})) \\
    &= g_0\alpha f(g_0,\dots, g_n) + \sum_{i=1}^n (-1)^n \alpha f(\dots, g_{i-1}g_i, \dots) + \alpha f(g_0,\dots,g_{n-1}) \\
    &= d^n\alpha^nf(g_0,\dots,g_n).
    \end{align*}
    That a map of cochain complexes induces a map of cohomologies is clear.
\end{proof}

\begin{lem}
    If $0\to A\xto{\alpha} B\xto{\beta} C\to 0$ is a exact sequence of $G$-modules, then $0\to C^i(G,A)\to C^i(G,B)\to C^i(G,C)\to 0$ is exact for all $i\ge 0$, hence an exact sequence $0\to C^\bullet(G,A) \to C^\bullet(G,B) \to C^\bullet(G,C)\to 0$. \qed
\end{lem}

\begin{thm}
    Every short exact sequence $0\to A\to B\to C\to 0$ induces a long exact sequence
    \begin{align*}
        0 &\to H^0(G,A) \to H^0(G,B) \to H^0(G,C) \\
        &\to H^1(G,A) \to H^1(G,B) \to H^1(G,C) \\
        &\to H^2(G,A) \to \dots
    \end{align*}
    and this is functorial.
\end{thm}

\begin{proof}
    Apply the snake lemma to
    \[
    \begin{tikzcd}
        & \coker d_A^{n-1} \arrow[r] \arrow[d, "d_A^n"] & \coker d_B^{n-1} \arrow[r] \arrow[d, "d_B^n"] & \coker d_C^{n-1} \arrow[r] \arrow[d, "d_C^n"] & 0 \\
        0 \arrow[r] & \ker d_A^{n+1} \arrow[r] & \ker d_B^{n+1} \arrow[r] & \ker d_C^{n+1} & 
    \end{tikzcd}
    \]
    where the resulting connecting homomorphism $\delta: H^i(G,C)\to H^{i+1}(G,A)$ is explicitly given by sending $[f]$ to $[\alpha^{-1}\circ d^n_B(\bar{f})]$, where we lift $f$ along $\beta$ to $\bar{f}\in H^i(G,B)$.
\end{proof}

\begin{defn}[cohomological $\partial$-functors]
    Let $\sC$ be abelian, $\sC'$ additive. A (covariant) cohomological $\partial$-functor $\sC\to \sC'$ is:
    \begin{itemize}
        \item a system of additive functors $T^i: \sC\to \sC'$ ($i\ge 0$), and
        \item connecting morphisms $\delta: T^i(A'') \to T^{i+1}(A')$, for every $i\ge 0$ and each short exact $0\to A'\to A\to A''\to 0$ in $\sC$,
    \end{itemize}
    satisfying:
    \begin{itemize}
        \item Given a map of short exact sequences
        \[
        \begin{tikzcd}
            0 \arrow[r] & A' \arrow[r] \arrow[d] & A \arrow[r] \arrow[d] & A'' \arrow[r] \arrow[d] & 0 \\
            0 \arrow[r] & B' \arrow[r] & B \arrow[r] & B'' \arrow[r] & 0,
        \end{tikzcd}
        \]
        the diagram
        \[
        \begin{tikzcd}
            T^i(A'') \arrow[r, "\delta"] \arrow[d] & T^{i+1}(A') \arrow[d] \\
            T^i(B'') \arrow[r, "\delta"] & T^{i+1}(B')
        \end{tikzcd}
        \]
        commutes;
        \item Given an exact sequence $0\to A'\to A \to A''\to 0$, the sequence
        \[0 \to T^0(A') \to T^0(A) \to T^0(A'') \xto{\delta} T^1(A') \to\dots\]
        is a chain complex.
    \end{itemize}
    When $\sC'$ is abelian as well, the $\partial$-functor is called \emph{exact} if the above chain complex is exact.
\end{defn}

In this context, $H^i(G,\bullet)$ is the unique universal exact cohomological $\delta$-functor extending $\bullet^G$.



\section{Lecture, Mar. 6}

(Group cohomology III)

We will give yet another equivalent definition of group cohomology.

\begin{defn}
    The \emph{standard resolution} of $\ZZ$ by $G$-modules is
    \[\dots\to \ZZ[G^{n+1}] \xto{d_n} \ZZ[G^n] \xto{d_{n-1}} \dots \xto{d_1} \ZZ[G] \xto{d_0} \ZZ \to 0,\]
    where $\ZZ[G^n]$ is the free $\ZZ$-algebra generated by the direct product $G^n$, with left diagonal action $g\cdot(g_1,\dots,g_n) = (gg_1,\dots,gg_n)$, and
    \[d_n(g_0,\dots,g_n) := \sum_{i=0}^n (-1)^i (g_0,\dots,g_{i-1},g_{i+1},\dots,g_n).\]
    Note that $d_0: \ZZ[G]\to \ZZ$ is the augmentation map $\sum n_g g \mapsto \sum n_g \in \ZZ$.
\end{defn}

\begin{lem}
    The standard resolution is exact, so that it is a free resolution of $\ZZ$ as a (trivial) $\ZZ[G]$-module.
\end{lem}


\begin{defn}[Ext groups]
    Let $A,B$ be $R$-modules. Take $P_\bullet\to B$ to be a projective resolution of $B$. Applying the contravariant left exact functor $\Hom_R(\bullet, A)$ to $P_\bullet \to 0$ and deleting the $\Hom(B,A)$-term, we get a cochain complex
    \[0\to \Hom(P_0,A) \to \Hom(P_1,A) \to\dots,\]
    then $\Ext_R^n(B,A)$ is defined as its $n$-th cohomology.
\end{defn}

\begin{lem}
    The groups $\Ext_R^n(B,A)$ do not depend on the projective resolution.
\end{lem}

Applying this for $B = \ZZ$, $R=\ZZ[G]$, we can use the standard resolution to compute $\Ext_{\ZZ[G]}^n(\ZZ, A)$, as the $n$-th cohomology of
\[0\to \Hom_{\ZZ[G]}(\ZZ[G],A) \xto{d_1^\ast} \Hom_{\ZZ[G]}(\ZZ[G^2],A) \xto{d_2^\ast} \dots.\]

\begin{prop}
    We have isomorphisms of abelian groups ($n\ge 0$):
    \[\Phi^n: \Hom_{\ZZ[G]}(\ZZ[G^{n+1}], A) \to C^n(G,A)\]
    by
    \[\phi \mapsto [(g_1,\dots,g_n) \mapsto \phi(1, g_1, g_1g_2, \dots, g_1g_2\cdots g_n)].\]
    Furthermore, this commutes with the coboundary maps, so that it defines a chain isomorphism.
\end{prop}

\begin{proof}
    $\Phi^n$ is clearly a homomorphism. 
    
    Injectivity: let $\phi\in \ker \Phi^n$. For any $g_0,\dots,g_n\in G$, define $h_i = g_{i-1}^{-1}g_i$. Then
    \[\phi(g_0,\dots,g_n) = g_0\phi(1, h_1, h_1h_2,\dots,h_1h_2\cdots h_n) = 0,\]
    so $\phi = 0$.

    Surjectivity: for $f\in C^n(G,A)$, define $\phi\in \Hom_{\ZZ[G]}(\ZZ[G^{n+1}], A)$ by
    \[(g_0,\dots,g_n) \mapsto g_0f(g_0^{-1}g_1,\dots, g_{n-1}^{-1}g_n).\]
    This gets sent to $f$ by $\Phi^n$.

    Finally, we show that $\Phi^n$ commutes with coboundary maps, i.e. $\Phi^{n+1}d_{n+1}^\ast = d^n\Phi^n$. We compute:
    \begin{align*}
        \Phi^{n+1}(d_{n+1}^\ast(\phi))(g_1,\dots,g_{n+1})
        &= d_{n+1}^\ast(\phi)(1,g_1,\dots,g_1\cdots g_{n+1}) \\
        &= \phi(d_{n+1}(1,g_1,\dots,g_1\cdots g_{n+1}))\\
        &= \phi(g_1,\dots,g_1\cdots g_{n+1}) - \sum_{i=1}^n (-1)^i \phi(\dots,g_1\cdots g_{i-1}, g_1\cdots g_{i+1},\dots) \\
        &\ \ \ \ + (-1)^{n+1}\phi(1,g_1,\dots, g_1\cdots g_n) \\
        &= g_1\Phi^n(\phi)(g_2,\dots,g_{n+1}) - \sum_{i=1}^n (-1)^i \Phi^n(\phi)(\dots,g_{i-1},g_ig_{i+1},\dots) \\
        &\ \ \ \ + (-1)^{n+1}\Phi^n(\phi)(g_1,\dots, g_n) \\
        &= d^n\Phi^n(\phi)(g_1,\dots,g_{n+1}),
    \end{align*}
    as desired.
\end{proof}

\begin{cor}
    $H^n(G,A) = \Ext_{\ZZ[G]}^n(\ZZ,A)$.
\end{cor}

We remark that $\Ext^n$ are also the right derived functors of $\Hom_{\ZZ[G]}(\ZZ,\bullet) = \bullet^G$, and right derived functors of any left-exact functor $F$ is the unique universal exact cohomological $\partial$-functor extending $F$. This shows the equivalence of the four definitions of group cohomology we gave:
\begin{itemize}
    \item via injective resolutions, i.e. as right derived functors of $\bullet^G$;
    \item as the unique universal exact cohomological $\partial$-functor extending $\bullet^G$;
    \item via cochains;
    \item via the standard resolution.
\end{itemize}

\begin{cor}
    $H^n(G,A\oplus B) = H^n(G,A) \oplus H^n(G,B)$.
\end{cor}

\begin{proof}
    This is because in general, 
    \[\Ext_R^i(\bigoplus M_\alpha, N) = \prod \Ext_R^i(M_\alpha, N) \quad \text{and} \quad \Ext_R^i(M, \prod N_\alpha) = \bigoplus \Ext_R^i(M, N_\alpha)\]
    for any $R$-modules $M$ and $N$.
\end{proof}


\begin{defn}
    Let $H\le G$ be a subgroup, $A$ and $H$-module. The \emph{induced} $G$-module
    \[\Ind_H^G(A) := \ZZ[G]\otimes_{\ZZ[H]} A,\]
    and the \emph{coinduced} $G$-module
    \[\CoInd_H^G(A) := \Hom_{\ZZ[H]}(\ZZ[G], A).\]
\end{defn}

\begin{thm}
    If $H$ has finite index in $G$, then $\Ind_H^G(A) \cong \CoInd_H^G(A)$.
\end{thm}

When $H = \{1\}$ we just write $\Ind^G$ and $\CoInd^G$.

\begin{lem}
    Group cohomology of coinduced modules from the trivial group:
    \[H^n(G,\CoInd^G(A)) = 
    \begin{cases}
    A, & \text{if } n = 0; \\
    0, & \text{otherwise.}
    \end{cases}\]
\end{lem}

\begin{proof}
    For $n\ge 1$ we have isomorphisms of abelian groups
    \[\Hom_{\ZZ[G]}(\ZZ[G^n], \CoInd^G(A)) \to \Hom_\ZZ(\ZZ[G^n], A),\]
    given by
    \[\phi \mapsto [z\mapsto \phi(z)(1)]\]
    \[[z\mapsto [y\mapsto \phi(yz)]]\mapsfrom \phi.\]
    so $H^n(G,\CoInd^G(A)) = H^n(\{1\}, A)$ for $n\ge 0$. Just use the stupid resolution $0\to \ZZ\to \ZZ\to 0$.
\end{proof}


\section{Lecture, Mar. 8}

(Group homology, Tate cohomology)

A minor point concerning tensor products over noncommutative rings: for $M\otimes_R N$, it only makes sense when $M$ is a right $R$-module and $N$ is a left $R$-module, and the resulting $M\otimes_R N$ is a priori only an abelian group. So, in the definition of $\Ind_H^G(A)$, we really think of $\ZZ[G]$ as a \emph{right} $\ZZ[H]$-module, and then ``manually'' define the extra structure of $\Ind_H^G(A)$ as a $\ZZ[G]$-module, by $g(\alpha\otimes a) = (g\alpha) \otimes a$. Similarly, the $\ZZ[G]$-module structure on $\CoInd_H^G(A)$ is given by $(g\phi)(\alpha) = \phi(\alpha g)$.

\begin{lem}
    When $G$ is finite, there is a canonical isomorphism $\CoInd^G(A) \cong \Ind^G(A)$ given by
    \begin{align*}
        \phi &\mapsto \sum_{g\in G} g^{-1}\otimes \phi(g) \\
        (g^{-1}\mapsto a) &\mapsfrom g\otimes a.
    \end{align*}
    where $(g^{-1}\mapsto \alpha)$ maps $g'$ to 0 for $g'\neq g^{-1}$.
\end{lem}

\begin{defn}[group homology]
    The $n$-th group homology with coefficients in $A$ is 
    \[H_n(G,A) = \Tor_n^{\ZZ[G]}(\ZZ, A) = L_n(\ZZ\otimes_{\ZZ[G]} \bullet)(A) = L_n(\bullet\otimes_{\ZZ[G]} A)(\ZZ).\]
    In practice, we use the last expression, with the standard resolution of $\ZZ$ by right $\ZZ[G]$-modules (the same as the standard resolution by left $\ZZ[G]$-modules, except $G$ acts diagonally on the right). This is the (unique) universal exact homological $\partial$-functor extending $\bullet\otimes_{\ZZ[G]} A$.
\end{defn}

\begin{lem}
    $H_n(G,A\oplus B) \cong H_n(G,A) \oplus H_n(G,B)$.
\end{lem}

\begin{proof}
    This is just because $\Tor$ commutes with arbitrary direct sums and filtered colimits in each variable.
\end{proof}

\begin{defn}[coinvariants]
    Let $A$ be a left $G$-module. The $G$-\emph{coinvariants} $A_G$ of $A$ is the $G$-module $A/I_GA$, where $I_G$ is the \emph{augmentation ideal}
    \[I_G = \ker(\eps: \ZZ[G] \to \ZZ) = \ZZ[g-1: g\in G].\]
    In other words, $A_G$ is the largest quotient of $A$ which is a trivial $G$-module. Observe that naturally $\ZZ\otimes_{\ZZ[G]}A \cong A_G$, so that $H_0(G,A) = A_G$ (just like $H^0(G,A) = A^G$).
\end{defn}

Similar to group cohomology, we have:

\begin{lem}
    Group homology of induced modules from the trivial group:
    \[H_n(G,\Ind^G(A)) = 
    \begin{cases}
    A, & \text{if } n = 0; \\
    0, & \text{otherwise.}
    \end{cases}\]
\end{lem}

\begin{lem}
    Let $G$ be finite, and let $N_G = \sum_{g\in G} g$ be the \emph{norm element}. Let $N_G: A\to A$ be the multiplication-by-$N_G$ map. Then $I_G A \subseteq \ker N_G$ and $\im N_G\subseteq A^G$. Consequently, we get an induced map $\hat{N}_G: A_G\to A^G$. \qed
\end{lem}

\begin{defn}[Tate (co)homology]
    Define $\hat{H}^n(G,A) = H^n(G,A)$ for $n>0$, and $\hat{H}^0(G,A) = \coker \hat{N}_G$. Define $\hat{H}_n(G,A) = H_n(G,A)$ for $n>0$, and $\hat{H}_0(G,A) = \ker \hat{N}_G$. Define $\hat{H}^{-n}(G,A) = \hat{H}_{n-1}(G,A)$ and $\hat{H}_{-n}(G,A) = \hat{H}^{n-1}(G,A)$ for $n>0$.
\end{defn}

Then, it is easy to check that a morphism of $G$-modules induces natural morphisms of Tate (co)homology groups in all degrees. The key theorem is the following:

\begin{thm}
    Let $0\to A\to B\to C\to 0$ be a short exact sequence of $G$-modules. Then we get a long exact sequence of abelian groups
    \begin{align*}
        \dots &\to \hat{H}_1(G,A) \to \hat{H}_1(G,B) \to \hat{H}_1(G,C) \\
        &\to \hat{H}_0(G,A) \to \hat{H}_0(G,B) \to \hat{H}_0(G,C) \\
        &\to \hat{H}^0(G,A) \to \hat{H}^0(G,B) \to \hat{H}^0(G,C) \\
        &\to \hat{H}^1(G,A) \to \hat{H}^1(G,B) \to \hat{H}^1(G,C) \to\dots
    \end{align*}
    Furthermore, this is functorial.
\end{thm}


\begin{proof}
    Apply the snake lemma to the commutative diagram
    \[
    \begin{tikzcd}
        H_1(G,C) \arrow[r] & A_G \arrow[r] \arrow[d, "\hat{H}_G"] & B_G \arrow[r] \arrow[d, "\hat{H}_G"] & C_G \arrow[r] \arrow[d, "\hat{H}_G"] & 0 \\
        0 \arrow[r] & A^G \arrow[r] & B^G \arrow[r] & C^G \arrow[r] & H^1(G,A).
    \end{tikzcd}
    \]
    Furthermore, by diagram chasing, the image of $H_1(G,C)$ lies in $\hat{H}_0(G,A)$, and $C^G\to H^1(G,A)$ factors through $\hat{H}^0(G,A)$. Finally, it is not hard to verify exactness at these two terms, and to check that a commutative diagram of short exact sequences induces a commutative diagram of long exact sequences.
\end{proof}

\begin{lem}
    $\hat{H}^n(G,A\oplus B) \cong \hat{H}^n(G,A)\oplus \hat{H}^n(G,B)$, and $\hat{H}_n(G,A\oplus B) \cong \hat{H}_n(G,A)\oplus \hat{H}_n(G,B)$.
\end{lem}

\begin{thm}
    Let $G$ be finite, and $B = \Ind^G(A) \cong \CoInd^G(A)$. Then the Tate (co)homology groups of $G$ with coefficients in $B$ all vanish.
\end{thm}

\begin{proof}
    It suffices to show that for $B = \ZZ[G]\otimes_\ZZ A$, $\ker(N_G: B\to B) = I_G B$ and $\im(N_G) = B^G$. Since $G$ acts on $B$ only on its $\ZZ[G]$-component, it suffices to show this for $\ZZ[G]$, in which case it is easily verified.
\end{proof}


\begin{cor}
    Let $A$ be a free $\ZZ[G]$-module, then it has trivial Tate (co)homology.
\end{cor}

\begin{proof}
    Let $B$ be the free $\ZZ$-module generated by a $\ZZ[G]$-basis of $A$. Then $A \cong \Ind^G(B)$.
\end{proof}

Finally, we specialize to the case where $G = \langle g\rangle$ is a finite cyclic group. Then, instead of using the standard resolution, we can use instead
\[\dots \to \ZZ[G] \xto{N_G} \ZZ[G] \xto{g-1} \ZZ[G] \xto{N_G} \ZZ[G] \xto{g-1} \ZZ[G] \xto{\eps} \ZZ\to 0. \tag{$\ast$}\]
Since $G$ is abelian, we may view $\Hom_{\ZZ[G]}(\ZZ[G], A)$ as a $\ZZ[G]$-module by $(g\phi)(h) := \phi(gh)$, and $\ZZ[G]\otimes_{\ZZ[G]} A$ as a $\ZZ[G]$-module by $g(h\otimes a) := (gh)\otimes a = h\otimes (ga)$. Of course, both of these are canonically isomorphic to $A$ as $G$-modules.


\begin{thm}
    Let $G = \langle g\rangle$ be a finite cyclic group, then the even-indexed Tate cohomologies (i.e. odd-indexed Tate homologies) of any $G$-module $A$ are all equal to $\hat{H}^0(G,A)$, and the odd-indexed Tate cohomologies (i.e. even-indexed Tate homologies) are all equal to $\hat{H}_0(G,A)$.
\end{thm}

\begin{proof}
    Apply the tensor and hom functors on ($\ast$).
\end{proof}


\section{Lecture, Mar. 13}

(Herbrand quotient, Herbrand unit theorem)


\begin{defn}
    Let $G$ be a finite cyclic group, $A$ a $G$-module. Let $h^0(A) = h^0(G,A) = \#\hat{H}^0(G,A)$, and $h_0(A) = h_0(G,A) = \#\hat{H}_0(G,A)$. When both of these are finite, the \emph{Herbrand quotient} is defined as
    \[h(A) = h^0(A)/h_0(A)\in \QQ.\]
\end{defn}


\begin{prop}
    Let $G$ be a finite cyclic group, $0\to A\to B\to C\to 0$ be a short exact sequence of $G$-modules. Then there is an exact hexagon
\[\begin{tikzcd}
	& {\hat{H}^0(A)} & {\hat{H}^0(B)} \\
	{\hat{H}_0(C)} &&& {\hat{H}^0(C)} \\
	& {\hat{H}_0(B)} & {\hat{H}_0(A)}
	\arrow[from=1-2, to=1-3, "\alpha^0"]
	\arrow[from=1-3, to=2-4, "\beta^0"]
	\arrow[from=2-4, to=3-3, "\delta^0"]
	\arrow[from=3-3, to=3-2, "\alpha_0"]
	\arrow[from=3-2, to=2-1, "\beta_0"]
	\arrow[from=2-1, to=1-2, "\delta_0"]
\end{tikzcd}\]
where the map $\delta^0$ is given by $\hat{H}^0(C) \cong \hat{H}^{-2}(C) = \hat{H}_{1}(C) \to \hat{H}_0(A)$. \qed
\end{prop}

\begin{cor}
    In $0\to A\to B\to C\to 0$, if two of $h(A),h(B),h(C)$ are defined then so is the third, and $h(B) = h(A)h(C)$.
\end{cor}

\begin{proof}
    We have $h^0(A) = \# \hat{H}^0(A) = \# \ker\alpha^0 \# \im \alpha^0 = \#\ker\alpha^0\#\ker\beta^0$. Similarly, we obtain
    \[h^0(A) h^0(C) h_0(B) = \#\ker\alpha^0\#\ker\beta^0\#\ker\delta^0\#\ker\alpha_0\#\ker\beta_0\#\ker\delta_0 = h_0(A) h_0(C) h^0(B),\]
    as desired.
\end{proof}


\begin{cor}
    If $A$ is either (a) induced or coinduced, or (b) finite, then $h(A) = 1$.
\end{cor}

\begin{proof}
    If $A$ is induced or coinduced, then both $h^0(A)$ and $h_0(A)$ are 1.

    If $A$ is finite: consider the exact sequence $0\to A^G\to A \xto{g-1} A \to A_G \to 0$, which implies
    \[\# A^G = \#\ker(g-1) = \#\coker(g-1) = \# A_G,\]
    so $h_0(A) = \#\ker(\hat{N_G}) = \#\coker(\hat{N_G}) = h^0(A)$.
\end{proof}


\begin{cor}
    Let $A$ be a finitely generated abelian group, then $h(A) = h(A/A_{\tors})$. Moreover, if $A$ is a trivial $G$-module, then $h(A) = \#(G)^{\rk A}$. \qed
\end{cor}


\begin{lem}
    Let $\alpha: A\to B$ have finite kernel and cokernel. Then $h(A) = h(B)$.
\end{lem}


\begin{proof}
    Use the exact sequences $0\to \ker\alpha \to A\to \im\alpha\to 0$ and $0\to \im\alpha \to B\to \coker\alpha\to 0$
\end{proof}

\begin{cor}
    Let $A\subseteq B$ be a submodule with finite index, then $h(A) = h(B)$.
\end{cor}


We now apply all this to the class field theory setting.
Let $L/K$ be a finite Galois extension of local or global fields. Then the abelian groups $L$, $L^\times$, $\cO_L$, $\cO_L^\times$, $\cI_L$, $\cP_L$ (principal) are all (nontrivial) $G$-modules, where $G = \Gal(L/K)$.

In the case $G = \langle \sigma\rangle$ is cyclic, we can compute the Herbrand quotient for all of the above (recall, again, that $\hat{H}_0(A) = \ker \hat{N_G} = \ker(N_G)/\im(\sigma-1)$ and $\hat{H}^0(A) = \coker \hat{N_G} = \ker(\sigma-1)/\im(N_G)$). Also, in the case for $L^\times, \cO_L^\times, \cI_L, \cP_L$, the norm map corresponds to the element norm and the ideal norm; in the case $L$ and $\cO_L$, the norm map corresponds to the trace.


\begin{lem}[linear independence of automorphisms]
    Let $L/K$ be finite Galois, then the set $\Aut_K(L)$ is linearly independent in the $L$-vector space $f:L\to L$.
\end{lem}

\begin{proof}
    Suppose otherwise, then suppose $n$ is smallest such that there exists distinct $f_1,\dots,f_n \in \Aut_K(L)$ and $a_1,\dots,a_n\in L^\times$ with $\sum a_if_i \equiv 0$. Since $f_1\neq f_2$, there exists $x_0\in L$ such that $f_1(x_0)\neq f_2(x_0)$. Then $\sum a_if_i(x_0x) = \sum a_if_i(x_0)f_i(x) = 0$ for all $x\in L$. Canceling out the two equations gives us a linear dependence among $n-1$ automorphisms, a contradiction. 
\end{proof}


\begin{lem}
    Let $L/K$ finite Galois, $G = \Gal(L/K)$. Then:
    \begin{enumerate}[(i)]
        \item $\hat{H}^0(G,L) = \hat{H}^1(G,L) = 0$;
        \item $\hat{H}^0(G,L^\times) \cong K^\times/\N(L^\times)$, and $\hat{H}^1(G,L^\times)$ is trivial.
    \end{enumerate}
\end{lem}

\begin{proof}
    For (i): first, since $\ker(\sigma-1: L\to L) = L^G = K$, and $\im(N_G) = \im(\Tr_{L/K}) = K$ ($L/K$ Galois hence separable hence trace form nondegerate), we have $\hat{H}^0(G,L) = 0$. To find $H^1(G,L)$, we use its description as the crossed homomorphisms $f:G\to L$ modulo the principal ones. Let $f:G\to L$ be any crossed homomorphism, then let $\beta = \sum_{\tau\in G}f(\tau)\tau(\alpha)\in L$, where $\alpha\in L$ is a fixed element with trace 1. Then for any $\sigma\in G$,
    \[\sigma(\beta) = \sum_{\tau\in G}\sigma(f(\tau))\sigma(\tau(\alpha)) = \sum_{\tau\in G}(f(\sigma\tau) - f(\sigma))(\sigma\tau(a)) = \beta - f(\sigma),\]
    so $f(\sigma) = \beta - \sigma(\beta)$, so $f$ is in fact principal.

    For (ii), $(L^\times)^G = K^\times$, and $\im(N_G) = \N(L^\times)$. To find $H^1(G,L^\times)$, let $f:G\to L^\times$ be any crossed homomorphism. Let $\beta = \sum_{\tau\in G} f(\tau)\tau(\alpha)$ where $\alpha$ is chosen so that $\beta\in L^\times$ (by linear independence of automorphisms). Then
    \[\tau(\beta) = \sum_{\tau\in G} \sigma(f(\tau))\sigma(\tau(\alpha)) = \sum_{\tau\in G} f(\sigma\tau)f(\sigma)^{-1} (\sigma\tau(\alpha)) = f(\sigma)^{-1}\beta,\]
    so $f(\sigma) = \beta/\tau(\beta)$ is principal.
\end{proof}

Let $L/K$ be a Galois extension of global fields. Then $\Gal(L/K)$ acts on the set of places $M_L$, via $\norm{\alpha}_{\sigma w} = \norm{\sigma\alpha}_w$. Also, for a fixed place $v$ of $K$, it permutes the places $w\mid v$.


\begin{defn}
    The \emph{decomposition group} $D_w$ of a place $w\in M_L$ is the stabilizer
    \[D_w := \{\sigma\in \Gal(L/K): \sigma(w) = w\}.\]
\end{defn}

We know $\Gal(L/K)$ acts transitively on $\{w\mid v\}$, so the $D_w$'s are conjugate.

\begin{Rem}
    For archimedean places for number fields, $w\mid v$, $D_w$ is trivial unless $w$ is complex and $v$ is real, in which case $\# D_w = 2$. Also, in the archimedean case, we define $I_w = D_w$. So $f_w=1$ always, and $e_w=2$ iff $w$ is a complex place that extends a real place.
\end{Rem}

With these definitions, $[L:K] = e_vf_vg_v$ for \emph{all} places $v\in M_K$.


\begin{defn}
    Let $L/K$ be an extension of number field. Let $e_0 = \prod_{v\nmid \infty} e_v$ and $e_\infty = \prod_{v\mid \infty} e_v$, $e(L/K) = e_0e_\infty$.
\end{defn}

\begin{thm}[Herbrand unit theorem]
    Let $L/K$ be a Galois extension of number fields, and let $w_1,\dots,w_{r+s}$ be the archimedean places of $L$. Then there exist $\eps_1,\dots,\eps_{r+s}\in \cO_L^\times$, such that:
    \begin{itemize}
        \item $\sigma(\eps_i) = \eps_j \iff \sigma(w_i) = w_j$, for $\sigma\in G$;
        \item $\eps_1,\dots,\eps_{r+s}$ generate a finite index subgroup of $\cO_L^\times$;
        \item $\eps_1\eps_2\dots\eps_{r+s} = 1$, and all other multiplicative relations are multiples of this.
    \end{itemize}
\end{thm}

\begin{proof}
    Pick $v_1,\dots,v_{r+s}\in \cO_L^\times$ (i.e. 1 at all finite places) such that $\abs{v_i}_{w_j} < 1$ when $i\neq j$ and $\abs{v_i}_{w_i} > 1$ (which is then automatic). These can be picked as follows (say $i=1$): we use the adelic Minkowski theorem. Choose the id\`ele $d$ as follows: $\abs{d_w}_w = 1$ for nonarchimedean $w$, $\abs{d_{w_i}}_{w_i} = \frac{1}{M}$ for $i\neq 1$ ($M$ a large number chosen afterwards), and $\abs{d_{w_1}}_{w_1}$ large enough such that $|d| = c$ (the bound in adelic Minkowski), so that $L(d)$ contains a nonzero point $x \in L$. In fact, by construction, $x\in \cO_L$, and $\N(x) = \prod_i |x|_{w_i} = c$. To modify $x$ so that it lies in $\cO_L^\times$, choose a generator $\gamma$ for all (finitely many) principal ideals of norm at most $c$. Then dividing $x$ by the previously fixed generator of $(x)$ gives a number in $\cO_L^\times$. To control its absolute value under $w_i$ ($i\neq 1$), let 
    \[M = \max_{\substack{\gamma \\ i\neq 1}} \frac{1}{\abs{\gamma}_i},\]
    so that $\abs{x/\gamma}_i < 1$ for $i\neq 1$. This concludes the process of choosing $v_1 = x/\gamma$.

    Let $\alpha_i = \prod_{\sigma\in D_{w_i}} \sigma(v_i) \in \cO_L^\times$. Then it is easy to compute $\abs{\alpha_i}_{w_i} > 1$ and $\abs{\alpha_i}_{w_j} < 1$ for $j\neq i$, and furthermore the stabilizer of $\alpha_i$ in $G$ is $D_{w_i}$.

    Now, the Galois group partitions $w_i$ into $m$ orbits, where $m$ is the number of archimedean places of $K$. Reindex $w_i$ and $a_i$ such that $w_1,\dots,w_m$ lie in distinct orbits. For $i=1,\dots,r+s$, let $r(i) = \min\{j : \sigma(w_j) = w_i \text{ for some }\sigma\}$, and call the corresponding $\sigma_i$ (which is unique up to $D_{w_{r(i)}}$). 
    
    Now, let $\beta_i = \sigma_i(\alpha_{r(i)})$, which does not depend on the choice of $\sigma_i$ since $\alpha_{r(i)}$ is fixed by $D_{w_{r(i)}}$. Then it is not hard to verify that $\beta_i$ satisfy the first bullet point. Furthermore,
    \[\abs{\beta_i}_{w_j} = \abs{\sigma_i(\alpha_{r(i)})}_{w_j} = \abs{\alpha_{r(i)}}_{\sigma_i(w_j)},\]
    so $|\beta_i|_{\sigma_i^{-1}w_{r(i)}} > 1$ and for all other places of $L$, $|\beta_i|<1$. Furthermore, it is clear that $\sigma_i^{-1}w_{r(i)}$ are simply a permutation of $w_i$: if $\sigma_{i_1}^{-1}w_{r(i_1)} = \sigma_{i_2}^{-1}w_{r(i_2)}$, then $r(i_1)=r(i_2)$ and $\sigma_{i_1}\sigma_{i_2}^{-1}\in D_{w_{r(i_1)}}$, so $\beta_{i_1} = \beta_{i_2}$, which implies $w_{i_1} = w_{i_2}$, so $i_1 = i_2$.
    Thus, to show that $\beta_i$'s generate a finite index subgroup of $\cO_L^\times$, we observe that in fact any $r+s$ units satisfying the condition $\eps_i$ has this property (essentially because a $(r+s-1)\times (r+s-1)$ matrix with positive row sums, where only diagonal elements are positive and the rest are negative, is necessarily invertible).

    Finally, because $\beta_i$'s must have one relation, suppose $\prod_i \beta_i^{n_i}$ is one with coprime exponents. By a rank argument, these cannot have other relations. Then, we claim that taking $\eps_i = \beta_i^{n_i}$ finishes the problem. Indeed, (iii) and (ii) are easy to verify. To show (i), we need $n_i = n_j$ whenever $w_i$ and $w_j$ are in the same $G$-orbit. But this is true, since applying $\sigma\in G$ should not give any additional relations between $\beta_i$.
\end{proof}


\section{Lecture, Mar. 15}

(Ambiguous class number formula)

\begin{lem}[Noether]
    For $L/K$ finite cyclic with $G = \Gal(L/K)$, $\hat{H}_0(G, L) = \hat{H}_0(G,L^\times) = 0$.
\end{lem}

\begin{proof}
    Let $\sigma$ be a generator of $G$. By normal basis theorem (theorem \ref{normal_basis}), there exists $\beta\in L^\times$ such that $\{\sigma^i \beta\}$ is a basis of $L/K$. Under this basis, $\sigma$ acts by translating the coordinates. So for $\alpha\in \ker(N_G)\subseteq L$, $\alpha = \sum_i \alpha_i (\sigma^i\beta)$, let us define $\gamma = \sum_i \gamma_i(\sigma^i\beta)$ where $\gamma_i = - \sum_{j=1}^i \alpha_i$. Since $\sum_i \alpha_i = 0$, we have $\alpha = \sigma\gamma - \gamma$, i.e. $\alpha\in \im(\sigma-1)$. This shows $\hat{H}_0(G,L) = 0$. A similar proof works for $\hat{H}_0(G,L^\times)$.
\end{proof}

\begin{Rem}
    This also follows from the vanishing of $\hat{H}^1(G,L)$ and $\hat{H}^1(G,L^\times)$ in general, and that for $G$ cyclic, $\hat{H}^1 = \hat{H}_0$.
\end{Rem}


\begin{cor}[Hilbert 90, original form]
    Let $L/K$ be a finite cyclic extension, with $\Gal(L/K)$ generated by $\sigma$. Then for $\alpha\in L^\times$, $\N(\alpha) = 1$ iff $\alpha = \beta/\sigma(\beta)$ for some $\beta\in L^\times$.
\end{cor}

\begin{thm}
    Let $L/K$ finite cyclic, then 
    \[h(\cO_L^\times) = \frac{e_\infty(L/K)}{[L:K]}.\]
\end{thm}

\begin{proof}
    Let $\eps_1,\dots,\eps_{r+s}$ be as in the Herbrand unit theorem, and let $A$ be the finite-index subgroup of $\cO_L^\times$ they generate. Then $A$ is also a $G$-module. For an embedding $\phi: K\into \CC$, let $E_\phi$ be the free $Z$-module with basis $\varphi: L\into \CC$ extending $\phi$. Then $E_\phi$ are also $G$-modules; in fact, $G$ acts on $\{\varphi\mid \phi\}$ freely and transitively, so $E_\phi \cong \ZZ[G] \cong \Ind^G(\ZZ)$. Let $v_\phi$ be the place of $K$ corresponding to $\phi$. Let $A_v$ be the free $G$-module with basis $w$ (places above $v$). Consider the $G$-module morphism $\pi: E_\phi \to A_v$, sending $\varphi\mapsto w_{\varphi}$. We have an exact sequence
    \[0 \to \ker\pi \to E_\phi \xto{\pi} A_v \to 0,\]
    where $\ker \pi = (\sigma^{m}-1) E_\phi$, where $\sigma$ is a generator for $G$ and $m = \#\{w\mid v\}$. If $\phi$ is unramified, then $\ker\pi = 0$ and  $h(A_v) = h(E_\phi) = 1$. If $G$ is ramified, then a more careful analysis gives $h(\ker \phi) = 1/2$, so $h(A_v) = 2$. In any case, $h(A_v) = e_v$.

    Now, consider the exact sequence of $G$-modules
    \[0 \to \ZZ \to \bigoplus_{v\mid\infty} A_v \xto{\phi} A \to 0,\]
    where $\psi$ sends $w_i \mapsto \eps_i$. We are done because $h(\ZZ) = \#G = [L:K]$.
\end{proof}

\begin{lem}
    Let $L/K$ be a cyclic extension of global fields. Then $h_0(\cI_L) = 1$ and $h(\cI_L) = h^0(\cI_L) = e_0(L/K)[\cI_K: N(\cI_L)]$.
\end{lem}

\begin{proof}
    Suppose $I\in \ker N_G$, i.e. $I\in \cI_L$ satisfies $N(I) = \cO_K$. By using the explicit description $N(\fq) = p^{f_\fq}$, we can conclude that for each $\fp$ in $K$, $\sum_{\fq\mid \fp} v_\fq(I) = 0$. Since $G = \Gal(L/K)$ is cyclic, we can order $\{\fq\mid \fp\} = \{\fq_1,\dots,\fq_g\}$ such that $\sigma \fq_i = \fq_{i+1}$ where $\sigma$ is a fixed generator of $G$ (of course, $\sigma \fq_g = \fq_1$). Let $n_i = v_{\fq_i}(I)$ and $m_i = -\sum_{j=1}^i n_j$, and let $J_\fp = \sum \fq_i^{\fm_i}$. Then $\sigma(J_\fp)/J_{\fp} = \prod_{\fq\mid \fp} \fq^{v_q(I)}$. The conclusion is that $I = \sigma(J)/J$, i.e. $I\in \im(\sigma-1)$. This shows $h_0(I_L) = 1$.

    Now, we compute $h^0(\cI_L)$. Suppose $I\in \ker(\sigma-1) = \cI_L^G$, then this is equivalent to $v_\fq(I)$ being constant for $\fq$ over a fixed $\fp$. Then $I$ is a product of ideals of form $(\fp \cO_L)^{1/e_\fp}$. So $[\cI_L^G: \cI_K] = e_0(L/K)$, so $h^0(\cI_L) = [\cI_L^G: N(\cI_L)] = [\cI_L^G: \cI_K][\cI_K:N(\cI_L)] = e_0(L/K)[\cI_K: N(\cI_L)]$.
\end{proof}

\begin{thm}[ambiguous class number formula]
    Let $L/K$ be a finite cyclic extension of number fields. Then
    \[\#\Cl_L^G = \frac{e(L/K) \#\Cl_K}{n(L/K)[L:K]},\]
    where $n(L/K) = [\cO_K^\times : \N(L^\times)\cap \cO_K^\times] \in \ZZ_{\ge 1}$.
\end{thm}


\begin{proof}
    Consider the long exact sequence in cohomology
    \[0 \to \cP_L^G \to \cI_L^G \to \Cl_L^G \to H^1(\cP_L) \to 0,\]
    since $H^1(\cI_L) \cong \hat{H}_0(\cI_L) = 0$. Therefore, $\#\Cl_L^G = h_0(\cP_L)\cdot [\cI_L^G : \cP_L^G]$. 

    Consider the inclusions $\cP_K \subseteq \cP_L^G \subseteq \cP_L$, so 
    \[[\cI_L^G:\cP_L^G] = \frac{[\cI_L^G : \cP_K]}{[\cP_L^G: \cP_K]} = \frac{[\cI_L^G : \cI_K][\cI_K:\cP_K]}{[\cP_L^G: \cP_K]} = \frac{e_0(L/K)\#\Cl_K}{[\cP_L^G: \cP_K]}.\]
    Now, consider another long exact sequence in cohomology
    \[0 \to (\cO_L^\times)^G \to (L^\times)^G \to \cP_L \to H^1(\cO_L^\times) \to H^1(L^\times) \to H^1(\cP_L^G) \to H^2(\cO_L^\times) \to H^2(L^\times),\]
    which can be simplified into
    \[0 \to \cO_K^\times \to K^\times \to \cP_L^G \to \hat{H}_0(\cO_L^\times) \to 0 \to \hat{H}_0(\cP_L) \to \hat{H}^0(\cO_L^\times) \xto{f} K^\times/\N(L^\times).\]
    Since $K^\times/\cO_K^\times \cong \cP_K$, we get
    \[[\cP_L^G: \cP_K] = h_0(\cO_L^\times) = \frac{h^0(\cO_L^\times)[L:K]}{e_\infty(L/K)}.\]
    The last three terms of the above long exact sequence also gives
    \[\frac{h^0(\cO_L^\times)}{h_0(\cP_L)} = \#\im f = [\cO_K^\times : \N(L^\times)\cap \cO_K^\times].\]
    Therefore,
    \[\#\Cl_L^G = \frac{h_0(\cP_L)e(L/K)\#\Cl_K}{h_0(\cO_L^\times)[L:K]} = \frac{e(L/K) \#\Cl_K}{n(L/K)[L:K]},\]
    as desired.
\end{proof}

\section{Lecture, Mar. 20}

(Norm index equality, local class field theory)

We begin with some remarks on the ambiguous class number formula. First, if $L/K$ is quadratic, then $G = \{1,\sigma\}$ has order 2. In this case, for any $I\in \cI_L$, $N(I) = I\cdot \sigma I$, so passing to $\Cl_L$ gives $[1] = [I][\sigma I]$. This means that $[I]$ is a 2-torsion element in $\Cl_L$ iff $[I]$ is $G$-invariant. In particular, when $L/K$ is an imaginary quadratic extension with discriminant $D$, $e_\infty(L/K) = [L:K] = 2$ and $n(L/K) = 2$, so the ambiguous class number formula gives $\#\Cl_L[2] = \frac{e_0(L/K)}{2}$, i.e. its $\ZZ/2\ZZ$-rank is $\#\{p\mid D\}-1$. This has applications in factoring integers.

\begin{lem}
    Let $f:A\to C$ be a map of abelian groups, such that $\ker f\subseteq B\subseteq A$, then $A/B\cong f(A)/f(B)$.
\end{lem}

\begin{proof}
    Use snake lemma.
\end{proof}

And now the payoff:

\begin{thm}[first main inequality]
\label{1stMainIneq}
    Let $L/K$ be a totally unramified cyclic extension of number fields (i.e. $e(L/K) = 1$). Then
    \[[\cI_K : T_{L/K}] \ge [L:K],\]
    where $T_{L/K} = \cP_K N(\cI_L)$ is the norm group (Takagi group) for the trivial modulus.
\end{thm}

\begin{proof}
    Let us rewrite
    \begin{align*}
    [\cI_K:\cP_K N(\cI_L)]
    &= \frac{[\cI_K : \cP_K]}{[\cP_K N(\cI_L) : \cP_K]} \\
    &= \frac{\#\Cl_K}{[N(\cI_L) : N(\cI_L)\cap \cP_K]} \\
    &= \frac{\#\Cl_K}{[\cI_L : N^{-1}(\cP_K)]} \\
    &= \frac{\#\Cl_K}{[\cI_L/\cP_L : N^{-1}(\cP_K)/\cP_L]} \\
    &= \frac{\#\Cl_K}{[\Cl_L : \Cl_L[N_G]]} \\
    &= \frac{\#\Cl_K}{\# N_G(\Cl_L)}.
    \end{align*}
    Now, $h^0(\Cl_L) = [\Cl_L^G : N_G(\Cl_L)]$, so by the ambiguous class number formula:
    \[[\cI_K : T_{L/K}] = \frac{\#\Cl_K h^0(\Cl_L)}{\#\Cl_L^G} = \frac{h^0(\Cl_L)n(L/K)[L:K]}{e(L:K)} = h^0(\Cl_L)n(L/K)[L:K] \ge [L:K],\]
    as desired.
\end{proof}

\begin{cor}[norm index equality, etc.]
    Let $L/K$ be a totally unramified cyclic extension of number fields, then:
    \begin{itemize}
        \item $[\cI_K:T_{L/K}] = [L:K]$;
        \item $\# \Cl_L^G = \# \Cl_K/[L:K]$;
        \item the Tate cohomologies of $\Cl_L$ all vanish;
        \item every unit in $\cO_K^\times$ is the norm of an element in $L$.
    \end{itemize}
\end{cor}


\begin{proof}
Equality follows from theorems \ref{2ndMainIneq}, \ref{1stMainIneq}. In fact, because equality holds, the proof of the first main inequality tells us more things: $\hat{H}^0(\Cl_L) = 0$ and $\cO_K^\times \subset \N(L^\times)$ (every unit is a norm). The ambiguous class number formula then says $\# \Cl_L^G = \# \Cl_K/[L:K]$. In addition, $h(\Cl_L) = 1$ since $\Cl_L$ is finite, and since we know $h^0(\Cl_L) = 1$, $h_0(\Cl_L) = 1$ as well.
\end{proof}

In the homework, it will be shown that this implies $\ker\psi_{L/K} = T_{L/K}$, and a similar equality holds in the ramified case where there is a nontrivial modulus. This then immediately implies that $\cI_K^\fm/T_{L/K} \cong \Gal(L/K)$ is an isomorphism, i.e. Artin reciprocity.

In the rest of the lecture, we will focus on local class field theory. Since what we've shown points to the importance of the images of norm maps, and norms can be computed locally, it makes sense for us to start there.

Let $K$ be a local field, with a fixed separable closure $K^{\sep}$, and let
\[K^{\ab} = \bigcup_{\substack{L\subseteq K^{\sep} \\ L/K\text{ finite abelian}}} L\]
\[K^{\unr} = \bigcup_{\substack{L\subseteq K^{\sep} \\ L/K\text{ finite unramified}}} L\]
be the maximal abelian and unramified extensions of $K$ (inside $K^{\sep}$), so $K \subseteq K^{\unr} \subseteq K^{\ab} \subseteq K^{\sep}$. The middle inclusion is true because any finite unramified extension of $K$ is cyclic. Infinite Galois theory tells us that there is a one-to-one correspondence
\[\{\text{extensions } L/K \text{ in } K^{\ab}\} \longleftrightarrow \{\text{closed subgroups of } \Gal(K^{\ab}/K)\}\]
\[\{\text{Galois extensions}\} \longleftrightarrow \{\text{closed normal subgroups}\}\]
\[\{\text{finite extensions}\} \longleftrightarrow \{\text{open subgroups}\}.\]
The archimedean case is not very interesting, so let us assume $K$ is nonarchimedean. Then the discrete valuation ring $\cO_K$ is a DVR, with prime $\fp$, and let $\FF_\fp$ be the residue field. 

Let $L/K$ be unramified, then the Galois group $\Gal(L/K)$ is generated by the Frobenius element $\Frob_{L/K}$. The Artin map $\psi_{L/K}: \cI_K\to \Gal(L/K)$ sends $\fp \mapsto \Frob_{L/K}$. Since $\cO_K$ is a PID, we can extend $\psi_{L/K}$ multiplicatively to a map $\psi_{L/K}: K^\times \to \Gal(L/K)$.

\begin{thm}[local Artin reciprocity]
Let $K$ be a local field. There is a unique continuous homomorphism $\theta_K: K^\times \to \Gal(K^{\ab}/K)$, such that for any finite abelian $L/K$ in $K^{\ab}$, we have an induced map $\theta_{L/K}: K^\times \to \Gal(K^{\ab}/K) \onto \Gal(L/K)$, which satisfies:
\begin{itemize}
\item If $K$ is nonarchimedean and $L$ is unramified, then $\theta_{L/K}(\pi) = \Frob_{L/K}$, where $\pi$ is any uniformizer of $K$;
\item $\theta_{L/K}$ is surjective with kernel $\N_{L/K}(L^\times)$, hence induces an isomorphism $K^\times/\N_{L/K}(L^\times) \cong \Gal(L/K)$.
\end{itemize}
\end{thm}

\begin{Rem}
    Mentally compare this to the more complicated global CFT: there is no modulus since $K^{\ab}$ covers everything, and $\cI_K^\fm/T_{L/K}$ is replaced with $K^\times/\N(L^\times)$. The analogue in global CFT is by considering the \emph{idele} class group, which contains everything and hides the moduli.
\end{Rem}

\begin{defn}
    A \emph{norm group} of a local field $K$ is any subgroup of $K^\times$ of the form $\N_{L/K}(L^\times)$, $L$ finite ab. extension.
\end{defn}

\begin{Rem}
    The word ``abelian'' can be removed without changing anything. If $L/K$ is any finite extension, not even necessarily Galois, then the \emph{norm limitation theorem} implies that $\N(L^\times) = \N(M^\times)$, where $M$ is the maximal abelian extension of $K$ in $L$.
\end{Rem}

\begin{cor}
    The map $L\mapsto \N(L^\times)$ induces an inclusion-reversing bijection between finite abelian extensions $L/K$ and norm groups of $K$, satisfying:
    \begin{itemize}
        \item $\N((L_1L_2)^\times) = \N(L_1^\times)\cap \N(L_2^\times)$;
        \item $\N((L_1\cap L_2)^\times) = \N(L_1^\times)\N(L_2^\times)$.
    \end{itemize}
\end{cor}

\begin{proof}
    The inclusion-reversal follows from transitivity of norms. We use Artin reciprocity to prove the two bullet points.

    To show $\N(L_1^\times) \cap \N(L_2^\times) \subseteq \N((L_1L_2)^\times)$: because $\Gal(L_1L_2/K) \to \Gal(L_1/K)\times \Gal(L_2/K)$ is injective, we can conclude by Artin reciprocity. The other direction is clear.

    To show the map $L\mapsto \N(L^\times)$ is a bijection: surjectivity follows by definition. Suppose $L_1,L_2$ give rise to the same norm group, then $L_1L_2$ also gives rise to the same norm group. By Artin reciprocity, $\Gal(L_1L_2/K) = \Gal(L_1/K) = \Gal(L_2/K)$, so $L_1=L_2$. This shows injectivity.

    Finally, to show the second bullet point, note that $\N(L_1^\times)\N(L_2^\times)$ is the smallest subgroup of $K^\times$ containing both norm groups, and $L_1\cap L_2$ is the largest extension of $K$ contained in both $L_1$ and $L_2$. So $\N((L_1\cap L_2)^\times) = \N(L_1^\times)\N(L_2^\times)$ by the bijection described above.
\end{proof}

\begin{cor}
    Every norm group has finite index in $K^\times$, and every group that contains a norm group is a norm group.
\end{cor}

\begin{proof}
    By Artin reciprocity, $K^\times/\N(L^\times) \cong \Gal(L/K)$ is a finite group, so every norm group has finite index. 

    Suppose $\N(L^\times) \le H \le K^\times$. Consider $F = L^{H/\N(L^\times)}$, where $H/\N(L^\times)$ is viewed as a subgroup of $K^\times/\N(L^\times) \cong \Gal(L/K)$. Then Artin reciprocity shows that $\N(F^\times) \cong H$. 
\end{proof}

\begin{lem}
\label{finite_index_open}
    Let $L/K$ be any extension of local fields. If $\N(L^\times)$ has finite index in $K^\times$, then it is open.
\end{lem}

\begin{proof}
    The archimedean case is not interesting, so WLOG $K$ is nonarchimedean. Since $\cO_L^\times$ is compact, its image $\N(\cO_L^\times)$ must also be compact, hence closed ($K^\times$ is Hausdorff). Because for $\alpha\in L^\times$,
    \[\alpha\in \cO_L^\times \iff \abs{\alpha} = 1 \iff \abs{\N_{L/K}(\alpha)} = 1 \iff \N_{L/K}(\alpha)\in\cO_K^\times,\]
    we have $\N(\cO_L^\times) = \N(L^\times)\cap \cO_K^\times$, so it is the kernel of the map $\cO_K^\times\into K^\times\onto K^\times/\N(L^\times)$. This shows $\cO_K^\times/\N(\cO_L^\times)$ is finite, and thus $\N(\cO_L^\times)$ is closed and of finite index in $\cO_K^\times$, hence open. But $\cO_K^\times$ is open in $K^\times$, so $\N(\cO_L^\times)$ is open in $K^\times$, so $\N(L^\times)$ is open as well, being the union of cosets of $\N(\cO_L^\times)$.
\end{proof}



\section{Lecture, Mar. 22}

(Local CFT continued, global CFT via ideles)

The two other main statements of local CFT are the following:

\begin{itemize}
    \item Existence: for any $H\subseteq K^\times$ of finite index, there exists a unique $L/K$ in $K^{\ab}$ such that $H = \N(L^\times)$. By virtue of Lemma \ref{finite_index_open}, this means that for subgroups of $K^\times$, finite index open $\iff$ is a norm group.
    \item Main Theorem: $\theta_K$ induces a canonical homeomorphism of profinite groups
    \[\wh{\theta}_K: \wh{K^\times} \xto{\cong} \Gal(K^{\ab}/K).\]
\end{itemize}

\begin{proof}[Proof of the Main Theorem]
    By Artin reciprocity and the existence theorem,
    \[\Gal(K^{\ab}/K) \cong \varprojlim_{L/K \text{ f. ab.}} \Gal(L/K) \cong \varprojlim_{H \text{ norm group}} \frac{K^\times}{H} = \varprojlim_{H \text{ finite index open}} \frac{K^\times}{H} \cong \wh{K^\times},\]
    as desired.
\end{proof}

When $K$ is archimedean, $\wh{K^\times}$ is either trivial ($K = \CC$) or has order 2 ($K = \RR$). So we focus on the nonarchimedean case. By picking a uniformizer $\pi$, we get a non-canonical isomorphism $K^\times \cong \cO_K^\times\times \ZZ$. So $\wh{K^\times} \cong \wh{\cO_K^\times}\times \wh{\ZZ} = \cO_K\times \wh{\ZZ}$, where $\cO_K^\times$ is already profinite because it is compact, Hausdorff, and totally disconnected. More canonically, we have the commutative diagram of split exact sequences
\[\begin{tikzcd}
	1 & {\mathcal{O}_K^\times} & {K^\times} & \ZZ & 1 \\
	1 & {\Gal(K^{\ab}/K^{\unr})} & {\Gal(K^{\ab}/K)} & {\Gal(K^{\unr}/K)} & 1
	\arrow[from=1-1, to=1-2]
	\arrow[from=1-2, to=1-3]
	\arrow["v", from=1-3, to=1-4]
	\arrow[from=1-4, to=1-5]
	\arrow[from=2-1, to=2-2]
	\arrow[from=2-2, to=2-3]
	\arrow[from=2-3, to=2-4]
	\arrow[from=2-4, to=2-5]
	\arrow["\cong", from=1-2, to=2-2]
	\arrow["{\theta_K}", from=1-3, to=2-3]
	\arrow["\phi", hook, from=1-4, to=2-4]
\end{tikzcd}\]
where $\phi$ becomes the inclusion $\phi: \ZZ\into \wh{\ZZ}$ under the identification $\Gal(K^{\unr}/K)\cong \Gal(\bar{\FF_\fp}/\FF_\fp) \cong \wh{\ZZ}$, and sends $1$ to the element $(\Frob_{L/K})_L$, called the \emph{arithmetic Frobenius}. (Aside: $\phi(-1)$ is called the \emph{geometric Frobenius}.) Taking the profinite completion of the top row yields the bottom row. The arithmetic/geometric Frobenius is a topological generator (generates a dense subgroup) of $\Gal(K^{\unr}/K)$. 

Now consider $\Gal(K^{\ab}/K)$. Because the top sequence splits, the bottom does as well (also non-canonically): $\Gal(K^{\ab}/K) \cong \cO_K^\times\times \wh{\ZZ}$. The fixed field of $\cO_K\cong \Gal(K^{\ab}/K^{\unr})$ is $K^{\unr}$, and let $K_\pi$ be the fixed field of $\theta_K(\pi)$. Then $K^{\ab} = K^{\unr}K_\pi$. The fact that $K_\pi$ is not canonical reflects the fact that one cannot say the ``maximal totally ramified extension''. But what we can say is that $K_\pi$ is the compositum of all finite, totally ramified $L/K$ in $K^{\ab}$ such that $\pi\in \N(L^\times)$.

\begin{exm}
    Let $K = \QQ_p$, and pick $\pi = p$ (of course, we could have picked any valuation-1 element). Then the picture looks like this:
    \[\begin{tikzcd}
	& {\QQ_p^{\ab}} \\
	{\QQ_p^{\unr} \cong \bigcup_n \QQ_p(\zeta_{p^n})} && {(\QQ_p)_p = \bigcup_{\gcd(m,p)=1} \QQ_p(\zeta_m)} \\
	& {\QQ_p}
	\arrow["{\wh{\ZZ}}", no head, from=3-2, to=2-1]
	\arrow["{\ZZ_p^\times}"', no head, from=3-2, to=2-3]
	\arrow["{\ZZ_p^\times}", no head, from=2-1, to=1-2]
	\arrow["{\wh{\ZZ}}"', no head, from=2-3, to=1-2]
\end{tikzcd}\]
\end{exm}

Because of spring break, we move the focus of the rest of this class to global CFT.

Let $K$ be a global field. Recall the group of ideles
\[\II_K = \AA_K^\times := \prod\nolimits_v ' (K_v^\times, \cO_v^\times).\]
Standard caveat is that in the first equality, the topology of $\II_K$ is finer than the one inherited as a subset of $\AA_K$. We have a natural map
\begin{align*}
    \varphi: \II_K &\to \cI_K \\
    a &\mapsto \prod_\fp \fp^{v_\fp(a)}.
\end{align*}
This ignores the infinite places. There is a natural commutative diagram
\[\begin{tikzcd}
	1 & {K^\times} & {\II_K} & {C_K} & 1 \\
	1 & {\cP_K} & {\cI_K} & {\Cl_K} & 1
	\arrow[from=2-3, to=2-4]
	\arrow[from=1-1, to=1-2]
	\arrow[from=1-2, to=1-3]
	\arrow[from=1-3, to=1-4]
	\arrow[from=1-4, to=1-5]
	\arrow[from=2-1, to=2-2]
	\arrow[from=2-2, to=2-3]
	\arrow[from=2-3, to=2-4]
	\arrow[from=2-4, to=2-5]
	\arrow["{x\mapsto (x)}"', two heads, from=1-2, to=2-2]
	\arrow["\varphi", two heads, from=1-3, to=2-3]
	\arrow[two heads, from=1-4, to=2-4]
\end{tikzcd}\]
where $C_K$ is the id\`ele class group.

\begin{defn}
    Given finite separable $L/K$, define the norm map
    \[\N_{L/K}:\II_L \to \II_K\]
    mapping
    \[(a_w)_w \mapsto \pr{\prod_{w\mid v} \N_{L_v/K_w}(a_w)}_v.\]
\end{defn}


This behaves well with the other norm maps:
\[\begin{tikzcd}
	{L^\times} & {\II_L} & {\cI_L} \\
	{K^\times} & {\II_K} & {\cI_K},
	\arrow["{\N_{L/K}}", from=1-1, to=2-1]
	\arrow[from=1-1, to=1-2]
	\arrow["{\N_{L/K}}", from=1-2, to=2-2]
	\arrow[from=2-1, to=2-2]
	\arrow[from=1-2, to=1-3]
	\arrow["N", from=1-3, to=2-3]
	\arrow[from=2-2, to=2-3]
\end{tikzcd}\]
so this induces a map
\[\begin{tikzcd}
	{C_L} & {\Cl_L} \\
	{C_K} & {\Cl_K}.
	\arrow["{\N_{L/K}}"', from=1-1, to=2-1]
	\arrow[two heads, from=1-1, to=1-2]
	\arrow[two heads, from=2-1, to=2-2]
	\arrow["{\N_{L/K}}", from=1-2, to=2-2]
\end{tikzcd}\]

We wish to glue together the local Artin homomorphisms to get a global Artin homomorphism.

Define $\varphi_w: \Gal(L_w/K_v)\into \Gal(L/K)$ by restricting $\sigma \mapsto \sigma|_L$. Then the image of $\varphi_w$ is just $D_w$. Because $L/K$ is abelian, $D_w$ only depends on $v$. Furthermore, $\varphi_w\circ \theta_{L_w/K_v}: K^\times \to \Gal(L/K)$ does not depend on $w$. This is easy to see in the unramified nonarchimedean case.


Define $i_v: K_v^\times \into \II_K$ sending $\alpha \mapsto (1,\dots,\alpha, \dots, 1)$ at the entry corresponding to $v$. The image intersects the principal ideles trivially. In addition, $i_v$ commutes with the norm maps $L_w\to K_v$ and $\II_L\to \II_K$.

Now, for a finite abelian extension $L/K$, define a map
\[\theta_{L/K}: \II_K \to \Gal(L/K)\]
mapping
\[(a_v)_v \mapsto \prod_{v} \phi_w(\theta_{L_w/K_v}(a_v))\]
where we fix a place $w\mid v$ for each $v$; this does not depend on which $w$ we pick. This product is well-defined, because for unramified (all but finitely many) $v$, $\phi_w(\theta_{L_w/K_v}(a_v)) = \Frob_v^{v(a_v)}$, which is 1 for all but finitely many $a_v$.

It is clear that $\theta_{L/K}$ is a group homomorphism. It is also continuous, because its kernel is the union of open sets. In addition, if $L_1\subseteq L_2$ are two finite abelian extensions of $K$, then $\theta_{L_1/K}$ is the same as $\theta_{L_2/K}$ composed with $\Gal(L_2/K)\onto \Gal(L_1/K)$. So we get a unique induced continuous homomorphism
\[\theta_{K}: \II_K \to \Gal(K^{\ab}/K).\]

\begin{defn}
    This is called the \emph{global Artin homomorphism}.
\end{defn}

\begin{prop}
    The global Artin homomorphism is the unique continuous homomorphism characterized by the property that for any finite abelian $L/K$, and any place $w$ of $L$ extending $v$ of $K$, the diagram
    \[
    \begin{tikzcd}
        K_w^\times \arrow[r, "\theta_{L_w/K_v}"] \arrow[d, "i_v"] & \Gal(L_w/K_v) \arrow[d, "\phi_w"] \\
        \II_K \arrow[r, "\theta_{L/K}"] & \Gal(L/K)
    \end{tikzcd}
    \]
    commutes. \qed
\end{prop}


Now we are ready to state the main theorems of the idele-theoretic formulation of global CFT.

\begin{thm}[global CFT, via ideles]
\label{ideleCFT}
    The global Artin homomorphism $\theta_K$ satisfies:
    \begin{itemize}
    \item (Artin reciprocity) $\ker \theta_K$ contains $K^\times$, and the induced map $\theta_K: C_K\to \Gal(K^{\ab}/K)$ satisfies that for any $L/K$ finite abelian, the induced $\theta_{L/K}: C_K\to\Gal(L/K)$ is surjective, with kernel $\N_{L/K}(C_L)$.
    \item (Existence theorem) For any finite index open $H\le C_K$, there exists a unique finite abelian $L/K$ in $K^{\ab}$ such that $\N_{L/K}(C_L) = H$.
    \item (Main theorem) $\theta_K$ induces an isomorphism
    \[\wh{\theta}_K: \wh{C_K} \to \Gal(K^{\ab}/K).\]
    \item (Functoriality) For any finite separable $L/K$, the diagram
    \[
    \begin{tikzcd}
        C_L \arrow[r, "\theta_L"] \arrow[d, "\N_{L/K}"] & \Gal(L^{\ab}/L) \arrow[d, "\res"] \\
        C_K \arrow[r, "\theta_K"] & \Gal(K^{\ab}/K)
    \end{tikzcd}
    \]
    commutes.
    \end{itemize}
\end{thm}

\begin{Rem}
    There is then an inclusion-reversing bijection
    \[\{\text{finite index open subgroups } H\le C_K\} \longleftrightarrow \{\text{finite abelian extensions } L/K \text{ in } K^{\ab}\}\]
    \begin{align*}
        H &\mapsto (K^{\ab})^{\theta_K(H)} \\
        \N_{L/K}(C_L) &\mapsfrom L.
    \end{align*}
\end{Rem}

\begin{Rem}
    When $K$ is a number field, $\theta_K$ is surjective with kernel he connected component of the identity in $\II_K$. When $K$ is a global function field, $\theta_K$ is injective with dense image.
\end{Rem}


Finally, we state the connection to ideal-theoretic CFT (Theorem \ref{idealCFT}). Let $\fm = \prod_v v^{e_v}$ be a modulus for $K$. Define the group
\[U_K^\fm(v) :=
\begin{cases}
\cO_v^\times, & \text{for } v\nmid \fm \\
\RR_{>0}, & \text{for } v \text{ real, } v\mid \fm \\
1 + \fp^{e_v}, & \text{for } v \text{ finite, } v\mid \fm, \text{ where } \fp = \{x\in \cO_v: |x|_v < 1\}.
\end{cases}\]

Let $U_K^\fm = \prod_v U_K^\fm(v)$, then this is an open subgroup of $\II_K$. Its image $\bar{U}_K^\fm$ in $C_K$ is a finite index open subgroup. Define
\[C_K^\fm = \II_K/(K^\times U_K^\fm) = C_K/\bar{U}_K^\fm,\]
then it turns out that
\[C_K^\fm  \cong \Cl_K^\fm \cong \Gal(K(\fm)/K).\]
The existence of ray class fields $K(\fm)$ is then the reincarnation of the existence of a field $L$ such that $\N(C_{L}) = \bar{U}_K^\fm$. 

Finally, for a finite abelian $L/K$, $\N(C_L)$ contains $\bar{U}_K^\fm$ for some $\fm$; in fact, the $\bar{U}_K^\fm$ forms a neighborhood basis of 1 in $C_K$, and the smallest $\fm$ for which $\bar{U}_K^\fm\subseteq \N(C_L)$ is true is the conductor $\fc(L/K)$. This then shows that $L$ is contained in some ray class field.

\section{Lecture, Apr. 3}

(Dimension shifting and more cohomological tools)


In the next few lectures we develop more cohomological tools to prove local CFT.

To see the connection with cohomology: $\hat{H}^0(G,A) = A^G/N_G(A)$, so taking $A = L^\times$ and $G = \Gal(L/K)$ gives precisely that $\hat{H}^0(\Gal(L/K), L^\times) = K^\times/\N(L^\times)$ for any Galois $L/K$. We will use a theorem of Tate to construct an explicit isomorphism $\Gal(L/K)\cong \hat{H}^0(\Gal(L/K), L^\times)$.

\begin{defn}
    Let $A$ be a $G$-module. Define another $G$-action on $\Ind^G(A)$ and $\CoInd^G(A)$:
    \[g(z\otimes a) = gz\otimes ga\]
    \[g\varphi = [z\mapsto g\varphi(g^{-1}z)].\]
    This only makes sense when $A$ is a $G$-module (while the usual Ind and CoInd make sense for any abelian group $A$).
\end{defn}

\begin{lem}
    Let $A$ be a $G$-module, $A^\circ$ the corresponding abelian group by forgetting its $G$-module structure. Then the maps
    \begin{align*}
        \Phi: \Ind^G(A) &\to \Ind^G(A^\circ) \\
        g\otimes a &\mapsto g\otimes g^{-1}a
    \end{align*}
    and
    \begin{align*}
        \Psi: \CoInd^G(A) &\to \CoInd^G(A^\circ) \\
        \phi &\mapsto [g\mapsto g\phi(g^{-1})]
    \end{align*}
    are $G$-module isomorphisms.
\end{lem}

\begin{proof}
    It is straightforward to check these are $G$-module homomorphisms. The inverse of the first one is $g\otimes a \mapsto g\otimes ga$, and the second one is its own inverse.
\end{proof}


Recall the augmentation ideal $I_G$ satisfies an exact sequence of $G$-modules
\[0\to I_G\to \ZZ[G]\xto{\eps} \ZZ\to 0\]
where $\eps: \sum n_g g \mapsto \sum n_g$. As $\ZZ$-modules, this sequence obviously splits. But the splitting is not a map of $G$-modules: $\ZZ\cong \ZZ 1_G$ is not a $G$-submodule of $\ZZ[G]$.

\begin{lem}
    Let $A$ be a $G$-module, then the map 
    \begin{align*}
        \pi: \Ind^G(A)&\to A \\
        z\otimes a &\mapsto \eps(z)a
    \end{align*}
    is surjective with kernel $I_G\otimes_{\ZZ} A$, and the map
    \begin{align*}
        \iota: A&\to \CoInd^G(A) \\
        a &\mapsto [z\mapsto \eps(z)a]
    \end{align*}
    is injective with cokernel $\Hom_\ZZ(I_G,A)$. \qed
\end{lem}

So we get two short exact sequences of $G$-modules
\[0\to I_G\otimes_{\ZZ} A \to \Ind^G(A) \xto{\pi} A\to 0\]
and
\[0\to A \xto{\iota} \CoInd^G(A) \to \Hom_\ZZ(I_G, A) \to 0.\]
Recall that $\Ind^G$ and $\CoInd^G$ have trivial (co)homology at $n>0$, and when $G$ is finite, their Tate cohomologies all vanish (even as $H$-modules where $H\le G$ finite index). So we have:

\begin{thm}[dimension shifting]
    Let $A$ be a $G$-module, $H\le G$ a subgroup of finite index. If $G$ is finite, then for any $n\in \ZZ$,
    \[\hat{H}^{n+1}(H, A) = \hat{H}^n(H, \Hom_\ZZ(I_G, A))\]
    and
    \[\hat{H}^{n-1}(H,A) = \hat{H}^n(H, I_G\otimes_{\ZZ} A).\]
    When $G$ is any (not necessarily finite) group, this holds for $H^n$ and $H_n$ for $n>0$.
\end{thm}

Using this theorem, one could alternatively \emph{define} Tate (co)homology using only the zeroth Tate cohomology. Dimension shifting gives us theorems about all cohomologies provided we have proven it \emph{in general} for the zeroth.

\begin{prop}
    When $G$ is finite, $A$ any $G$-module, then $\hat{H}^n(G,A)$ is torsion with exponent dividing $\#G$.
\end{prop}

\begin{proof}
    By dimension shifting, it suffices to show this for $n=0$, where $\hat{H}^0(G,A) = A^G/N_G(A)$. But for $a\in A^G$, $N_G a = (\#G)a$, so $\#G$ kills $\hat{H}^0$.
\end{proof}

\begin{cor}
    Let $G$ be finite, $A$ any $G$-module. If multiplication by $\#G$ is an isomorphism $A\to A$, then $A$ has trivial Tate cohomology.
\end{cor}

In particular, this holds when $A$ is the additive group of a ring and $\#G$ is a unit in it.

\begin{proof}
    $[\#G]$ then induces isomorphisms on all $\hat{H}^n(G,A)$, but they are all killed by $\#G$, hence trivial.
\end{proof}

\begin{cor}
    Let $G$ be finite, $A$ any finitely generated $G$-module. Then $\hat{H}^n(G,A)$ is finite for all $n\in \ZZ$. In particular, the Herbrand quuotient will be defined.
\end{cor}

\begin{proof}
    It is a finitely generated torsion abelian group, hence finite.
\end{proof}

Recall the functoriality of group (co)homology: a map of $G$-modules $\phi: A\to B$ induces maps
\[\phi_n: H_n(G,A) \to H_n(G,B), \quad \phi^n: H^n(G,A)\to H^n(G,B).\]
In the other input, if $\varphi: H\to G$ is a group homomorphism, we get a homomorphism from the standard resolution of $\ZZ$ by $H$-modules to the standard resolution of $\ZZ$ by $G$-modules. This induces maps
\[\varphi_n: H_n(H, \Res^G_H(A)) \to H_n(G,A), \quad \varphi^n: H^n(G, A) \to H^n(H, \Res^G_H(A)).\]

\begin{defn}
    Let $\varphi:H\to G$ be a group homomorphism, $A$ an $H$-module, and $B$ a $G$-module. Suppose $\phi: A\to B$ or $\phi: B\to A$ is a map of $H$-modules, then we say $\phi$ is \emph{compatible} with $\varphi$.
\end{defn}

If $\phi:A\to B$ is compatible with $\varphi:H\to G$, we get homomorphisms
\[H_n(H,A) \xto{\phi_n} H_n(H,B) \xto{\varphi_n} H_n(G, B)\]
and if $\phi: B\to A$ then we get
\[H^n(G,B) \xto{\varphi^n} H^n(H,B) \xto{\phi^n} H^n(H,A).\]

\begin{defn}
    Let $A$ be a $G$-module, $H\le G$. The morphisms
    \[\Res: H^n(G,A) \to H^n(H,A)\]
    \[\CoRes: H_n(H,A) \to H_n(G,A)\]
    are the above maps induced by $\varphi: H\to G$ and $\phi: A\xto{\id} A$.
\end{defn}

\begin{exm}
    When $n=0$, $\Res: A^G \to A^H$ is the natural inclusion, and $\CoRes: A_H\to A_G$ is the natural quotient.
\end{exm}


\section{Lecture, Apr. 5}

\begin{defn}
Let $A$ be a $G$-module, $H\le G$ of finite index. Fix $S\subseteq G$ a set of left coset representatives for $H$. Define 
\[N_{G/H} := \sum_{s\in S} s\in \ZZ[G], \quad N_{G/H}^{-1} := \sum_{s\in S} s^{-1} \in \ZZ[G].\]
Define a restriction map on homology by
\begin{align*}
    \Res: H_0(G,A) &\to H_0(H,A)\\
    a + I_GA &\mapsto N_{G/H}^{-1}a + I_H A
\end{align*}
\end{defn}

It is easy to check that this does not depend on the set of representatives we choose, and for $\alpha: A\to B$ a map of $G$-modules, the diagram
\[
\begin{tikzcd}
    H_0(G, A) \arrow[d, "{\Res}"] \arrow[r, "\alpha_0"] & H_0(G,B) \arrow[d, "{\Res}"] \\
    H_0(H, A) \arrow[r, "\alpha_0"] & H_0(H,B)
\end{tikzcd}
\]
commutes.

If $G$ is finite, then $\Res(\ker \hat{N}_G) \subseteq \ker \hat{N}_H$, so we have an induced map
\[\hat{H}_0(G,A) \to \hat{H}_0(H,A).\]
Similarly, define the corestriction for cohomology
\begin{align*}
    \CoRes: H^0(H,A) &\to H^0(G,A)\\
    a &\mapsto N_{G/H}a
\end{align*}
and it is also functorial and does not depend on the coset representatives $S$.

Now, we extend $\Res$ to higher homologies. From the long exact sequence for $0\to I_G\otimes_{\ZZ} A \to \Ind^G(A) \to A \to 0$, we can uniquely extend
\[
\begin{tikzcd}
0 \arrow[r] & H_1(G,A) \arrow[r] \arrow[d, "\exists!", dashed] & H_0(G, I_G\otimes_\ZZ A) \arrow[r] \arrow[d, "{\Res}"] & H_0(G, \Ind^G(A)) \arrow[r] \arrow[d, "{\Res}"] & 0 \\
0 \arrow[r] & H_1(H,A) \arrow[r] & H_0(H, I_G\otimes_\ZZ A) \arrow[r] & H_0(H, \Ind^G(A)) \arrow[r] & 0
\end{tikzcd}
\]
and similarly dimension shifting gives maps $\Res: H_n(G,A)\to H_n(H,A)$. 

Similarly, we get $\CoRes: H^n(H,A)\to H^n(G,A)$. Restriction and corestriction are transitive and $\partial$-functorial.


\begin{prop}
    Let $A$ be a $G$-module, $H\le G$ fintie index, then $\CoRes\circ \Res$ is multiplication by $[G:H]$ on $H_n(G,A)$ and $H^n(G,A)$ (and all $\hat{H}^n(G,A)$ when $G$ is finite).
\end{prop}

\begin{proof}
    Prove this for $n = 0$, and use dimension shifting.
\end{proof}

\begin{defn}
    Let $A$ be a $G$-module, $H\lhd G$. Then $A^H, A_H$ are trivial $H$-modules, hence $G/H$-modules. Then the map induced by $\varphi: G\to G/H$ and $\phi:A^H\to A$ is the \emph{inflation}
    \[\Inf: H^n(G/H, A^H) \to H^n(G,A)\]
    and the map induced by $\varphi: G\to G/H$ and $\phi: A\to A_H$ is the \emph{coinflation}
    \[\CoInf: H_n(G, A) \to H_n(G/H,A_H).\]
    These are also $\partial$-functorial.
\end{defn}

\begin{exm}
    In degree $n=0$, Inf and CoInf are just the identity maps on $A_G$ and $A^G$.
\end{exm}

\begin{exm}
    Let $f:G^n\to A$ be a $n$-cochain representing $\gamma\in H^n(G,A)$. Then $\Res(\gamma)\in H^n(H,A)$ is represented by the restriction of $f$ to $H^n$.

    Let $f:(G/H)^n\to A$ be a $n$-cochain representing $\gamma\in H^n(G/H,A)$. Then $\Inf(\gamma)\in H^n(G,A)$ is given by composing $f$ with the projection $G^n\to (G/H)^n$.
\end{exm}

\begin{thm}[inflation-restriction theorem]
    Let $A$ be a $G$-module, $H\lhd G$, $n\ge 1$. If $H^i(H,A) = 0$ for $1\le i\le n-1$, then
    \[0\to H^n(G/H, A^H) \xto{\Inf} H^n(G,A) \xto{\Res} H^n(H,A)\]
    is exact.
\end{thm}

\begin{proof}
    Use induction on $n$. 
    
    In the base case $n=1$, everything can be written down explicitly. Let $f: G/H\to A^H$ be a 1-cochain representing $\gamma\in \ker\Inf$. Since $f$ composed with $G\to G/H$ must be of form $[g\mapsto ga-a]$ for some $a\in A^H$, $f$ itself must be given by $[\bar{g}\mapsto \bar{g}a-a]$, so it is a coboundary, so $\gamma = 0$. Next, since $H\to G\to G/H$ is trivial, $\im \Inf \subseteq \ker \Res$. To show equality, let $f: G\to A$ be a $1$-cochain representing $\gamma\in \ker\Res$. Then on $H$, $f$ must act as $[h\mapsto ha-a]$ for some $a\in A$. Define $\bar{f}: G\to A$ by $g\mapsto f(g)-ga+a$, then $\bar{f}$ vanishes on $H$, so
    \[\bar{f}(gh) = g\bar{f}(h) + \bar{f}(g) = \bar{f}(g)\]
    and
    \[\bar{f}(hg) = h\bar{f}(g) + \bar{f}(h) = h\bar{f}(g).\]
    The first equation tells us that $\bar{f}$ factors through $G/H$, and the second tells us that the image of $\bar{f}$ is $H$-invariant. So $\bar{f}$ gives an element in $H^1(G/H, A^H)$ whose inflation is $f$. This shows the case $n=1$.

    Now the induction step. Assume this holds for $n$ (for all $G,H,A$), and we show this for $n+1$. By dimension shifting, if $A$ satisfies the hypothesis for $n+1$, then $\Hom_{\ZZ}(I_G, A)$ satisfies the hypothesis for $n$. By inductive hypothesis, 
    \[0\to H^n(G/H, \Hom_{\ZZ}(I_G,A)^H) \xto{\Inf} H^n(G,\Hom_{\ZZ}(I_G,A)) \xto{\Res} H^n(H,\Hom_{\ZZ}(I_G,A))\]
    is exact. By dimension shifting again,
    \[0\to H^{n+1}(G/H, A^H) \xto{\Inf} H^{n+1}(G,A) \xto{\Res} H^{n+1}(H,A)\]
    is exact.
\end{proof}

\begin{Rem}
    There is an analogous theorem for $\CoRes$ and $\CoInf$:
    \[H_n(H,A) \xto{\CoRes} H_n(G,A) \xto{\CoInf} H_n(G/H, A_H) \to 0\]
    is exact, if $H_i(H,A) = 0$ for $1\le i\le n-1$.
\end{Rem}

\begin{thm}
    Let $A$ be a $G$-module, where $G$ is finite. Suppose for all $H\le G$, we have $H^1(H,A) = H^2(H,A) = 0$. Then $\hat{H}^n(G,A) = 0$ for all $n\in \ZZ$.
\end{thm}

\begin{proof}
    For $G$ cyclic, this is clear since Tate cohomology is periodic with period 2.

    For $G$ solvable, let $1 = H_0\lhd H_1\lhd\dots \lhd H_m = G$ be the shortest possible subnormal series, such that all consecutive quotients are cyclic. Proceed by induction on $m$, with the base case clear. Let $H\neq G$ be a normal subgroup of $G$ such that $G/H$ is cyclic, then by induction hypothesis, $\hat{H}^n(H,A) = 0$ for all $n\in \ZZ$. 
    
    By the inflation-restriction theorem, we have $H^n(G/H, A^H) \cong H^n(G,A)$ for $n\ge 1$ (since $H^n(H,A) = \hat{H}^n(H,A) = 0$). So $H^1(G/H,A^H) = H^2(G/H,A^H) = 0$, and consequently for all $n\in\ZZ$, $\hat{H}^n(G/H, A^H) = 0$. This implies that $H^n(G,A) = 0$ for all $n\ge 1$, and also 
    \[0 = \hat{H}^0(G/H, A^H) = (A^H)^{G/H}/N_{G/H}(A^H).\]
    Combine this with
    $0 = \hat{H}^0(H,A) = A^H/N_H(A)$, we have
    \[A^G = (A^H)^{G/H} = N_{G/H}(A^H) = N_{G/H}(N_H(A)) = N_G(A),\]
    so $\hat{H}^0(G,A) = 0$. Since this holds for general $A$, we may use dimension shifting to address $n<0$: since $\hat{H}^{n-1}(H,A) = \hat{H}^n(H, I_G\otimes_\ZZ A)$, the hypothesis $H^1(H, I_G\otimes_\ZZ A) = H^2(H, I_G\otimes_\ZZ A) = 0$ holds, so $\hat{H}^{-1}(G,A) = \hat{H}^0(G, I_G\otimes_\ZZ A) = 0$, and repeating this proves that $\hat{H}^n(G,A) = 0$ for all $n\in\ZZ$.

    In general, suppose $G$ is not necessarily solvable. Let $H$ be a Sylow $p$-subgroup of $G$, then $H$ is solvable. Consider the composition
    \[H^n(G,A) \xto{\Res} H^n(H,A) \xto{\CoRes} H^n(G,A)\]
    which is multiplication by $(G:H)$, a number coprime to $p$. But for $n\ge 1$, this is also the zero map since the middle group is zero. So $H^n(G,A)$ has no elements of order $p$. Since this is for any $p$, we conclude $\hat{H}^n(G,A) = 0$ for $n\ge 1$. For $n=0$, since $\hat{H}^0(H,A) = 0$, the map $N_H: A\to A^H$ is surjective, so for any $a\in A^G\subset A^H$, there exists $a'\in A$ such that $a = \sum_{h\in H} ha'$, so $N_G(a') = [G:H]a$. This shows that multiplication by $[G:H]$ kills $\hat{H}^0(G,A)$ as well, so it has no elements of order $p$, and since this is for any $p$ we conclude $\hat{H}^0(G,A) = 0$. Finally, for $n < 0$, again use the same dimension shifting argument as in the solvable case.
\end{proof}


\begin{thm}[Tate's theorem]
    Let $A$ be a $G$-module where $G$ finite, and suppose for every $H\le G$, $H^1(H,A) = 0$ and $H^2(H,A)$ is cyclic with order equal to $\#H$. For any generator $\gamma$ of $H^2(G,A)$ and all $n\in\ZZ$, there is a uniquely determined isomorphism
    \[\Phi_\gamma: \hat{H}^n(G,\ZZ) \to \hat{H}^{n+2}(G,A)\]
    compatible with $\Res$ and $\CoRes$.
\end{thm}


%%%%%%%%%%%%%%%%%%%%%%%%%%%
%%%%%%%%%APPENDIX%%%%%%%%%%
%%%%%%%%%%%%%%%%%%%%%%%%%%%

\newpage


\section{Appendix: Extensions of absolute values}

This is material not covered in the lecture (but important nonetheless).

\begin{prop}[Strong Hensel's lemma]

Let $K$ be complete wrt a nontrivial, nonarchimedean absolute value $|\thinspace|$. Let $\cO_K$, $\fm_K$ be the corresponding valuation ring and maximal ideal. Let $f(x)\in \cO_K[x]$ such that its image $\bar{f}$ in $\frac{\cO_K}{\fm_K}[x]$ is nonzero. Suppose $\bar{f}(x) = \bar{g}(x)\bar{h}(x)$ in $\frac{\cO_K}{\fm_K}[x]$ where $\bar{g}$ is monic and $\bar{g},\bar{h}$ are relatively prime. Then we have lifts $g,h\in \cO_K[x]$ such that $f(x) = g(x)h(x)$, and $g(x)$ is monic with degree equal to $\deg \bar{g}$.
\end{prop}

\begin{cor}
\label{max_coeff}
Let $f(x)$ be irreducible in $K[x]$, with degree $n$. Then
\[|f| := \max(|a_0|,\dots,|a_n|) = \max(|a_0|, |a_n|).\]
\end{cor}

\begin{prop}[Complete archimedean fields]
\label{complete_arch}
Let $K$ be complete with respect to a nontrivial, archimedean absolute value. Then $(K, |\thinspace|)$ is isometrically isomorphic to either $(\RR, |\thinspace|_{\infty}^{r})$ or $(\CC, |\thinspace|_{\infty}^{r})$ for some $0<r\le 1$.
\end{prop}

\begin{thm}
\label{extend_abs_val}
Let $K$ be complete wrt a nontrivial absolute value $|\thinspace|$, and $L/K$ a finite extension of degree $n$. Then
\[\norm{\beta} := |\N_{L/K}(\beta)|^{1/n}\]
is the unique absolute value on $L$ extending that on $K$, and $L$ is complete with respect to $\norm{\thinspace}$.
\end{thm}

\begin{proof}
If $|\thinspace|$ is archimedean, then there is not much to show because of proposition \ref{complete_arch}. Assume for the rest that $|\thinspace|$ is nonarchimedean. We will show that so is $\norm{\thinspace}$.

\begin{Lem}
For $\beta\in L$, if $\norm{\beta}\le 1$, then $\norm{1+\beta} \le 1$.
\end{Lem} 

\begin{proof}[Proof of lemma]
Let $\beta\in L$, $\norm{\beta} = 1$. Let $f_{\beta}(x)\in K[x]$ be its minimal polynomial. Then 
\[\N_{L/K}(\beta) = ((-1)^{\deg f_{\beta}} f_{\beta}(0))^{[L: K(\beta)]},\]
which implies $|f_{\beta}(0)| = \norm{\beta}^{\deg f_{\beta}} \le 1$. Then by corollary \ref{max_coeff}, $f_{\beta}(x) \in \cO_K[x]$. 

Since the minimal polynomial of $1+\beta$ is $f_{\beta}(x-1)$, 
\[\norm{1+\beta}^{n} = |\N_{L/K}(1+\beta)| = |((-1)^{\deg f_{\beta}} f_{\beta}(-1))^{[L:K(\beta)]}| \le 1,\]
which proves the lemma.
\end{proof}

By the lemma, if $\norm{\alpha} \le \norm{\beta}$, we then have $\norm{\alpha+\beta} = \norm{\beta}\norm{1+\alpha\beta^{-1}} \le \norm{\beta}$, which is the nonarchimedean triangle inequality. Uniqueness follows because any two absolute values on $L$ are norms on $L$ (as $K$-vector spaces), which must induce the same topology on $L$, so they must be equivalent absolute values, so one must be a power of another, so they must be equal since they agree on $K$. Completeness is also clear.
\end{proof}

Even better, it is easy to see that these extensions are compatible with each other, i.e. this gives us a unique extension of an absolute value on $\bar{K}$.



\section{Appendix: Cyclotomic fields}


Let $n$ be a positive integer, $\zeta_n$ a primitive root of unity. The goal in this section is to show:

\begin{thm}
\label{cyclotomic_integers}
The ring of integers in the cyclotomic extension $\QQ(\zeta_n)/\QQ$ is $\ZZ[\zeta_n]$.
\end{thm}

We will in fact prove a bit more about the discriminant of cyclotomic extensions along the way.

Our strategy is to first show Theorem \ref{cyclotomic_integers} in the case where $n = p^r$ is a prime-power, then use that to deduce the general case. 

For simplicity, let $\zeta = \zeta_{p^r}$ be a primitive $p^r$-th root of unity. Let $\cO$ be the ring of integer in $\QQ(\zeta)$.

\begin{prop}
\label{prop1}
$\ZZ[\zeta]\cap p\cO = p\ZZ[\zeta]$.
\end{prop}

\begin{prop}
\label{prop2}
$\disc \ZZ[\zeta]$ is a power of $p$.
\end{prop}

We first see how the above two propositions imply that $\cO = \ZZ[\zeta]$. Clearly $\ZZ[\zeta] \subseteq \cO$. If $p\mid (\cO:\ZZ[\zeta])$, then $\cO/\ZZ[\zeta]$ has a subgroup of order $p$. Then there exists $a\in \cO, a\notin \ZZ[\zeta]$, such that $pa\in \ZZ[\zeta]$, so $pa\in \ZZ[\zeta]\cap p\cO = p\ZZ[\zeta]$, which implies $a\in \ZZ[\zeta]$, a contradiction. Thus, $p\nmid (\cO:\ZZ[\zeta])$. But $(\cO:\ZZ[\zeta])^2\cdot \disc \cO = \disc \ZZ[\zeta]$ is a power of $p$, so $\cO = \ZZ[\zeta]$.

\begin{proof}[Proof of proposition \ref{prop1}]
It is clear that $\ZZ[\zeta] = \ZZ[1-\zeta]$, and $(1-\zeta)^i$ ($0\le i\le p^{r-1}(p-1)-1$) forms a $\ZZ$-basis for $\ZZ[1-\zeta]$.

\begin{Lem}
$\N_{\QQ(\zeta)/\QQ}(1-\zeta) = p$.
\end{Lem}
\begin{proof}[Proof of lemma]
The conjugates of $1-\zeta$ are $1-\alpha$, where $\alpha$ are the roots of 
\[P(X) = X^{p^{r-1}(p-1)} + X^{p^{r-1}(p-2)} + \dots + X^{p^{r-1}} + 1.\]
The product of these $(1-\alpha)$ is precisely $P(1) = p$.
\end{proof}


Let $\sum_i c_i(1-\zeta)^i \in \ZZ[1-\zeta]\cap p\cO$, where $c_i\in \ZZ$. We will prove via induction on $i$ that $p\mid c_i$. Because $\N(1-\zeta) = p$, $p\in (1-\zeta)$, so $(1-\zeta)\cap \ZZ = (p)$. So $c_0 \in (1-\zeta)\cap \ZZ$ implies $p\mid c_0$. For the induction step, suppose we have shown $p\mid c_0,\dots,c_{i-1}$. It suffices to show that $(1-\zeta)^{p^{r-1}(p-1)} \in p\cO$, since then we can cancel out factors of $(1-\zeta)$ and repeat the same argument to show $p\mid c_i$. We know that $p$ is the product of all $p^{r-1}(p-1)$ conjugates of $1-\zeta$, so it suffices to show $\frac{1-\zeta^i}{1-\zeta}$ is a unit in $\cO$ for all $i$, which is easy to see.
\end{proof}

\begin{proof}[Proof of proposition \ref{prop2}]
$\disc \ZZ[\zeta] = \disc(1,\zeta,\dots,\zeta^{p^{r-1}(p-1)-1})$, which is equal to $\N_{\QQ(\zeta)/\QQ}(P'(\zeta))$ up to sign. After a easy computation (using the lemma above), we in fact have $\disc \ZZ[\zeta] = \pm p^{p^{r-1}(r(p-1)-1)}$.
\end{proof}

This finishes our proof of theorem \ref{cyclotomic_integers} in the case $n = p^r$. In general, use induction on the number of distinct prime divisors of $n$, with the additional claim that $\disc \cO_n$ divides $n^{\phi(n)}$. The base case is handled above. Say $n = p^{r}m$, where $p\nmid m$. It is clear that then $\QQ(\zeta_{n}) = \QQ(\zeta_{p^r})\QQ(\zeta_{m})$ and $[\QQ(\zeta_n):\QQ] = \phi(n) = \phi(p^r)\phi(m) = [\QQ(\zeta_{p^r}):\QQ][\QQ(\zeta_m):\QQ]$. It suffices to show that $\cO_n$, the ring of integers in $\QQ(\zeta_n)$, is included in $\cO_{p^r}\cdot \cO_m$, which by induction hypothesis is $\ZZ[\zeta_{p^r}]\ZZ[\zeta_{m}] = \ZZ[\zeta_n]$.

Given an element $\alpha\in \cO_n$, it must be of the form
\[\alpha = \frac{1}{d}\sum_{i,j} c_{i,j}\zeta_{p^r}^i \zeta_{m}^j\]
where $d,c_{i,j}\in \ZZ$, since $\zeta_{p^r}^i \zeta_{m}^j$ forms a $\QQ$-basis of $\QQ(\zeta_n)$. Because the discriminants $\disc \ZZ[\zeta_{p^r}]$ and $\disc \ZZ[\zeta_{m}]$ are coprime, it suffices to show $d$ divides each of these determinants.

Let $\sigma$ be the automorphism on $\QQ(\zeta_{n})$ sending $\zeta_{p^r} \mapsto \zeta_{p^r}^a$ and $\zeta_{m}\mapsto \zeta_m$. Then
\[\sigma\alpha = \frac{1}{d}\sum_{i,j} c_{i,j}\zeta_{p^r}^{ai} \zeta_{m}^j = \sum_{i} \zeta_{p^r}^{ai} x_i\]
where $x_i := \sum_j c_{i,j}\zeta_m^j/d$. Varying $a$ and solving for $x_i$ by Cramer's rule, we see that $x_i\cdot \disc \QQ(\zeta_{p^r})$ is integral over $\ZZ$. So $d\mid \disc \QQ(\zeta_{p^r})$, and similar for $\disc \QQ(\zeta_m)$. Finally, it is easy to show $\disc \cO_n$ divides $n^{\phi(n)}$ through a direct computation. This completes the proof.


\section{Appendix: Kummer theory}


\begin{defn}
    Let $G$ be a group which acts upon an abelian group $(M,+)$. Then $H^1(G,M)$ is the group of functions $f: G\to M$ such that $f(gh) = f(g) + gf(h)$, modulo functions of the form $f: g\mapsto gx - x$ ($x\in M$).
\end{defn}

\begin{thm}[Hilbert's theorem 90]
    Let $L/K$ be a finite Galois extension, $G = \Gal(L/K)$, then $H^1(G, L^{\times}) = 0$.
\end{thm}

In the case where $G$ is cyclic and generated by $\sigma$, suppose $a\in L^{\times}$ with norm 1. Then the function $f: G\to L^{\times}$ given by
\[\sigma^n \mapsto a\cdot \sigma(a)\cdot\dots \cdot \sigma^{n-1}(a)\]
must be of form $\sigma^n \mapsto \sigma^n(b)/b$ for some $b\in L^{\times}$, so in particular $a = b/\sigma(b)$. 

\begin{thm}
    Let $K$ be a field that contains $\zeta_n$. Then every degree-$n$ cyclic extension $L/K$ is of form $K(\alpha^{1/n})$, where $\alpha^{1/d}\notin K$ for $1\neq d\mid n$.
\end{thm}

\begin{proof}
    Let $L/K$ be a degree-$n$ cyclic extension with $\sigma \in G$ generating the Galois group. By Hilbert 90, there exists $t\in L^{\times}$ with $\zeta_n^r = \sigma^r(t)/t$. So $t^n$ is fixed by $G$ and $t^n = \alpha \in K$, and $L = K(t) = K(\alpha^{1/n})$.

    Conversely, it is clear that there is an injective map $\Gal(K(\alpha^{1/n})/K)\to \ZZ/n\ZZ$. Surjectivity is clear in the case $n$ is prime, and in general, the image of this map cannot be contained in $p\ZZ/n\ZZ$ for any $p\mid n$, and therefore is the whole group $\ZZ/n\ZZ$.
\end{proof}

\begin{defn}
    Let $K$ be a field that contains $\zeta_n$. The \emph{Kummer pairing} 
    \[\Gal(\bar{K}/K) \times K^{\times} \to \{1,\zeta_n,\dots,\zeta_n^{n-1}\}\] 
    is defined by: given $\sigma\in\Gal(\bar{K}/K)$, $z\in K^{\times}$, choose $y\in \bar{K}$, with $y^n = z$, and define $\langle\sigma, z\rangle = \sigma(y)/y$.
\end{defn}

\begin{thm}
    The Kummer pairing induces an isomorphism
    \[K^{\times}/(K^{\times})^n \cong \Hom_{cts}(\Gal(\bar{K}/K), \ZZ/n\ZZ).\]
\end{thm}

\begin{prop}
    Let $n$ be an odd prime power, $K$ a field with $\characteristic K$ coprime to $n$. Let $L = K(\zeta_n)$ and $M = L(\alpha^{1/n})$ for some $\alpha\in L^{\times}$. Define $\omega: \Gal(L/K)\to (\ZZ/n\ZZ)^{\times}$ by $\zeta_n^{\omega(g)} = g(\zeta_n)$. Then $M/K$ is abelian iff $g(a)/a^{\omega(g)} \in (L^{\times})^n$ for all $g$.
\end{prop}



\newpage


\section*{18.785 Problem Sets}


(1.2) An absolute value $\abs{}$ on $k$ is nonarchimedean iff $|n|\le 1$ for all positive integers $n$.

\begin{proof}
($\implies$) is easy. ($\given$): say $x,y\in k$, WLOG $|x| \le |y|$. Then $|x+y|^n = |(x+y)^n| = |\sum_i \binom{n}{i} x^i y^{n-i}| \le |x|^n + |x|^{n-1}|y| + \dots + |y|^n \le n|y|^n$. Taking $n\to\infty$, we obtain $|x+y| \le |y| = \max(|x|, |y|)$.

(Corollary: in a field of positive characteristic, every absolute value is nonarchimedian.)
\end{proof}


(1.3, Weak approximation)
Let $k$ be a field, $\abs{}_1,\dots, \abs{}_n$ be pairwise inequivalent nontrivial absolute values on $k$. Let $a_1,\dots,a_n\in k$, and $\eps>0$. Then there exists $x\in k$ such that $|x-a_i|_i < \eps$ for each $i=1,\dots,n$.

\begin{proof}
First, we find $z$ such that $|z|_1>1$ and $|z|_2,\dots,|z|_n < 1$. The induction basis $n = 2$ follows from the fact that $||_1, ||_2$ are inequivalent. Suppose we have found $z$ such that $|z|_1>1$ and $|z|_2,\dots,|z|_n < 1$. If $|z|_{n+1}\le 1$ we are already done, so suppose $|z|_{n+1} > 1$. Then as $m\to\infty$, $\abs{\frac{z^m}{1+z^m}}_1, \abs{\frac{z^m}{1+z^m}}_{n+1} \to 1$, whereas $\abs{\frac{z^m}{1+z^m}}_2,\dots, \abs{\frac{z^m}{1+z^m}}_{n} \to 0$. Take $y$ such that $|y|_1>1$ and $|y|_{n+1}<1$, then $\frac{yz^m}{1+z^m}$ satisfies the induction step for sufficiently large $m$.

Next, we solve the case $a_1\neq 0$, $a_2,\dots,a_n = 0$. This amounts to finding $y$ such that $|y-1|_1, |y|_2,\dots,|y|_n$ are all arbitrarily small. Take $z$ as above, and consider $y = \frac{z^m}{1+z^m}$ once again.

Finally, we find $y_i$ replacing $a_1$ with each nonzero $a_i$, and add them all together. This element satisfies the desired approximation.
\end{proof}

(1.4, Ostrowski's theorem)
The only nontrivial absolute values on $\QQ$ are $\abs{}_p$ for some $p\le \infty$, up to equivalence.

\begin{proof}
Case 1: there exists a positive integer $b$ with $|b| > 1$. Say $|b| = b^{\alpha}$. For any positive integer $n$, write $n = a_k b^k + a_{k-1}b^{k-1} + \dots + a_0$, where $a_0,\dots,a_k \in \{0,\dots, b-1\}$. Let $C = \max_{1\le m \le b-1} |m|/m^{\alpha}$. Then $|n|\le |a_k|b^{\alpha k} + |a_{k-1}|b^{\alpha(k-1)} + \dots + |a_0|\le C (a_k^{\alpha} b^{\alpha k} + \dots + a_0^{\alpha}) \le C(a_kb^k + \dots + a_0)^{\alpha} = Cn^{\alpha}$. Then $|n|^m = |n^m| \le Cn^{m\alpha}$, so taking $m\to\infty$ we obtain $|n|\le n^{\alpha}$. On the other hand, for any positive integer $n$, take $k$ such that $b^k \le n\le b^{k+1}$. Then $|n|\ge b^{\alpha(k+1)} - (b^{k+1}-n)^{\alpha} \ge b^{\alpha k}(b^{\alpha} - (b-1)^{\alpha}) = Cn^{\alpha}$ for a fixed $C$ not depending on $n$, so similar to above we obtain $|n| \ge n^{\alpha}$. This means $|n| = n^{\alpha}$, so $|x| = x^{\alpha}$ for all $x\in \QQ^{\times}$, so the absolute value is equivalent to $||_{\infty}$.

Case 2: $|n|\le 1$ for all integers $n$. Then by problem 2, $||$ is nonarchimedean. Let $\fp = \{n\in\ZZ: |n|<1\}$. Then $x,y\in \fp \implies |x+y| \le \max(|x|,|y|) < 1$, so $\fp$ is an ideal. Furthermore it is a prime ideal, since $|xy| < 1 \implies$ either $|x|<1$ or $|y|<1$, and $1\notin \fp$. Therefore, there exists a prime $p$ such that $|n| = 1$ for any integer $n$ coprime to $p$. Then since $||$ is multiplicative, it has to be equivalent to $||_p$.
\end{proof}

(1.5)
Let $(A,\fm,k)$ be a DVR, $n\ge 0$. 
\begin{enumerate}[(a)]
    \item $\dim_k\fm^n/\fm^{n+1} = 1$;
    \item Let $U_n = 1 + \fm^n$, for $n\ge 1$. Then $U_n/U_{n+1}\cong k$.
\end{enumerate}

\begin{proof}
(a) $\fm^n/\fm^{n+1}$ is an $(A/\fm)$-module, i.e. a $k$-vector space. Since $\fm^n = (\pi^n)$ is a principal ideal, the image of $\pi^n$ in $\fm^n/\fm^{n+1}$ is nonzero and generates the vector space. So $\dim_k \fm^n/\fm^{n+1} = 1$.

(b) It is clear that $v(\frac{1}{1+a\pi^n} - 1) \ge n$, so inverses exist in $U_n$, i.e. is a subgroup of $A^\times$. Map $U_n/U_{n+1} \to A\to k$ by $1+a\pi^n \mapsto a \to \bar{a}$. It is easy to check that this is a group isomorphism.
\end{proof}


(2.1)
    Let $v:K\to \RR\cup\{\infty\}$ be a valuation, and let $A$ be its valuation ring.
    \begin{enumerate}[(a)]
        \item Suppose $x_1,\dots,x_n\in K$ and $v(x_1) < v(x_i)$ for all $i\ge 2$. Then $v(x_1+\dots+x_n) = v(x_1)$.
        \item $A$ is integrally closed.
    \end{enumerate}

\begin{proof}
    (a) $v(x_1 + \dots + x_n)\ge \min(v(x_1), \dots,v(x_n)) = v(x_1)$, and $v(x_1)\ge \min(v(x_1+\dots+x_n), v(x_2),\dots,v(x_n))$. Since $v(x_1)$ is strictly the smallest, this minimum must be equal to $v(x_1+\dots+x_n)$. So we conclude $v(x_1+\dots+x_n) = v(x_1)$.

    (b) Suppose $x\in K$ is integral over $A$, i.e. $x^n + a_{n-1}x^{n-1} + \dots + a_1x + a_0 = 0$. Suppose $v(x) < 0$, then $v(x^n) < v(a_ix^i)$ for all $1\le i\le n-1$. So
    \[0\le v(a_0) = v(a_1x + \dots + x^n) = v(x^n),\]
    a contradiction. Therefore, $v(x)\ge 0$ and $x\in A$.
\end{proof}

(2.3)
Let $d\neq 1$ be a squarefree integer, $K = \QQ(\sqrt{d})$ be a quadratic extension. Then
\[\cO_K = \begin{cases}
\ZZ[\frac{1+\sqrt{d}}{2}] & \text{if } d\equiv 1\pmod{4};\\
\ZZ[\sqrt{d}] & \text{otherwise}.
\end{cases}\]

\begin{proof}
    Omitted.
\end{proof}


(3.1) Let $A$ be Dedekind, $K$ its field of fractions.
\begin{enumerate}[(a)]
    \item Describe all nonzero $A$-submodules of $K$.
    \item Describe all subrings of $K$ containing $A$.
\end{enumerate}

\begin{proof}
(a) Let $M\subseteq K$ be a nonzero $A$-submodule, then $M_{\fp} \subseteq K$ is a nonzero $A_{\fp}$-submodule (for any nonzero prime $\fp\subset A$). These can only be of the form $\fp^n A_{\fp} = \{x\in K: v_{\fp}(x) \ge n\}$ for $n \in \ZZ\cup\{-\infty\}$. So $M = \bigcap_{\fp} M_{\fp} = \{x\in K: v_{\fp}(x) \ge n_{\fp}\}$ for some combination of $n_{\fp}\in \ZZ\cup\{-\infty\}$. But only finitely many $n_{\fp}$ could be strictly positive, since any $x$ is contained in only finitely many primes. Conversely, if only finitely many $n_{\fp}$ are strictly positive, $M$ is clearly nonempty. So we map the set of nonzero $A$-submodules of $K$ injectively to the set of tuples $(n_{\fp})_{\fp}, n_{\fp}\in\ZZ\cup\{\infty\}$ where all but finitely many $n_{\fp} \le 0$. To show surjectivity, it suffices to show that, given any collection of $\fp^{n_{\fp}} A_{\fp}$ where almost all $n_\fp \le 0$,
\[\pr{\bigcap \fp^{n_{\fp}} A_{\fp}}_{\fq} = \fq^{n_{\fq}}A_{\fq}\]
for any nonzero prime $\fq$. It is clear that the LHS is contained in the RHS. The reverse inclusion follows from strong approximation.
%We claim that this is also injective. Suppose $(n_{\fp}), (m_{\fp})$ are distinct, say $n_{\fq} > m_{\fq}$ for some nonzero prime $\fq$. Then by strong approximation, we may find $x\in K$ such that $v_{\fq}(x) = m_{\fq}$ and $v_{\fp}(x) \ge m_{\fp}$ for all $\fp$. This shows that the 

(b) Let $B\subset K$ be a subring containing $A$. Then it is automatically an $A$-module, so $B = \{x\in K: v_{\fp_i}(x) \ge n_i\}$ where finitely many $n_i$ are positive. Since $A\subseteq B$, every $n_i\le 0$. Since $B$ is a ring, if some $n_i<0$, then it is equal to negative infinity. So we have $B = \bigcap_{\fp_i} A_{\fp_i}$ for some collection of primes $\fp_i$. In particular, if $A$ is a DVR, then $B = A$ or $K$.
\end{proof}

(3.2, Finitely generated modules over Dedekind domains)

In what follows, $A$ is a Dedekind domain, and $K = \Frac(A)$.

(a) Let $M$ be a finitely generated torsion $A$-module. Let $I = \Ann(M) \subseteq A$, then we can uniquely factor $I = \prod_i \fp_i^{e_i}$. By Chinese remainder theorem, $A/I = \bigoplus_i A/\fp_i^{e_i}$. Since $M$ is an $A/I$-module, this induces a decomposition $M = \bigoplus_i M_i$, where each $M_i$ is a $A/\fp_i^{e_i}$-module. Since $A/\fp_i^{e_i} = A_{\fp}/\fp_i^{e_i}A_{\fp}$ is a PID, we can use the structure theorem to write $M_i$ as the direct sum of modules of the form $A/I$.

(b) Let $P$ be a fractional ideal of $A$. Let $\psi: N\to P$ be a surjective homomorphism of $A$-modules. Let $\fp$ be a prime ideal of $A$, and consider the localized homomorphism $\psi_{\fp}: N_{\fp} \to P_{\fp}$ of $A_{\fp}$-modules. Since $P_{\fp}$ is a free $A_{\fp}$-module, there is a local splitting $\phi_{\fp}: P_{\fp} \to N_{\fp}$. Since $P$ is finitely presented, $\Hom_A(P, N)_{\fp} \cong \Hom_{A_{\fp}}(P_{\fp}, N_{\fp})$, so there exists $\Phi^{\fp}: P\to N$ whose localization at $\fp$ is equal to $s_{\fp}\phi_{\fp}$, for some $s_{\fp}\in A - \fp$. Then $\psi\Phi^{\fp}$ is $s_{\fp}$ times the identity on $P$, since it is so on the localization at $\fp$. Since $s_{\fp}\in A-\fp$, they generate the unit ideal in $A$, so there exist $a_{\fp}$ such that $\sum a_{\fp}s_{\fp} = 1$ (this is a finite sum). Let $\phi: N\to P$ be defined by $\phi = \sum a_{\fp} \Phi^{\fp}$, then $\psi\phi$ is the identity on $P$.

(c) Let $M$ be a finitely generated torsion free $A$-module. Then $M$ embeds into $M\otimes_A K = K^n$. Compose this with the projection onto the first coordinate, and let $P$ be the image, which is a fractional ideal:
\[0\to N\to M \to P \to 0.\]
Since $P$ is projective, $M = N\oplus P$. Also, $N = \ker(M\to P)$ injects into $K^{n-1}$, so by induction we can write $M$ as the direct sum of fractional ideals.

(d) Let $M$ be a finitely generated $A$-module, and $M_{\tors}$ its torsion submodule. Then in
\[0 \to M_{\tors} \to M \to M/M_{\tors} \to 0,\]
since $M/M_{\tors}$ is torsion free and finitely generated, it is projective by the above, so $M = M_{\tors}\oplus M/M_{\tors}$.

(e) Because $M$ is finitely generated, so is $N$, so $N$ is a fractional ideal. Furthermore, for two $i_1, i_2: M\otimes_A K \to K^n$, there exists an isomorphism $f: K^n \to K^n$ such that $f\circ i_1 = i_2$, so the fractional ideal induced by $i_2$ is exactly $\det f$ times the fractional ideal induced by $i_1$. Therefore, $N$ is unique up to multiplication by a principal ideal.

(f) Embed $I_1\oplus\dots \oplus I_n \into K^n$ naturally. Then by definition the Steinitz class is generated by $i_1\dots i_n$, where $i_1\in I_1$, etc.

(g) If $I_1\oplus\dots\oplus I_n \cong J_1\oplus\dots\oplus J_m$, tensoring with $K$ gives $m = n$, and it is clear that the two Steinitz classes are the same. Conversely, it suffices to show $I_1\oplus\dots \oplus I_n \cong A^{n-1}\oplus I_1\dots I_n$. By induction, we show $I_1 \oplus I_2 \cong A \oplus I_1I_2$. Without loss of generality, we can scale $I_1,I_2$ such that they are coprime integral ideals. Then there is an exact sequence
\[0 \to I_1\cap I_2 \to I_1\oplus I_2 \to I_1+I_2 \to 0,\]
where $I_1\cap I_2 = I_1I_2$, $I_1+I_2 = A$. Since $A$ is free, $I_1\oplus I_2 \cong I_1I_2\oplus A$ as desired.

(h) We show that $I_1\oplus I_2\oplus \dots \cong A\oplus A \oplus \dots$. Write inductively $I_n = P_n\oplus P_{n+1}^{-1}$, where $P_0 = A$. Then 
\begin{align*}
    I_1\oplus I_2\oplus \dots &= (P_0 \oplus P_1^{-1})\oplus (P_1\oplus P_2^{-1}) \oplus \dots \\
    &= P_0 \oplus (P_1^{-1}\oplus P_1) \oplus (P_2^{-1}\oplus P_2) \oplus \dots \\
    &= A\oplus A \oplus \dots.
\end{align*}




\end{document}