\documentclass[11pt]{amsart}
\usepackage{scorpius}

\title{Algebraic Geometry}
\author{}

\begin{document}

\maketitle

\tableofcontents

\newpage

\section{Category theory}

\begin{lem}
Consider the following cochain complex:
\[\dots \to C^{i-1}\xto{f^{i-1}} C^i \xto{f^i} C^{i+1} \to \dots\]
Then we have two pairs of short exact sequences:
\begin{itemize}
    \item $0 \to \ker f^i \to C^i \to \im f^i \to 0$ and $0 \to \im f^{i-1} \to \ker f^i \to H^i(C^{\bul}) \to 0$;
    \item $0 \to \im f^{i-1} \to C^i \to \coker f^{i-1} \to 0$ and $0 \to H^i(C^{\bul}) \to \coker f^{i-1} \to \im f^i \to 0$.
\end{itemize}
\end{lem}

\begin{prop}[FHHF theorem]
Let $F: \sA \to \sB$ be a covariant functor between abelian categories, and let $C^{\bul}$ be a cochain complex in $\sA$.
\begin{enumerate}[(a)]
    \item If $F$ is right exact, there is a natural morphism $FH^{\bul}(C^{\bul}) \to H^{\bul}F(C^{\bul})$.
    \item If $F$ is left exact, there is a natural morphism $H^{\bul}F(C^{\bul}) \to FH^{\bul}(C^{\bul})$.
    \item If $F$ is exact, the two morphisms are inverses of each other.
\end{enumerate}
\end{prop}

\begin{proof}
(a) Applying $F$ on $C^i \to C^{i+1} \to \coker f^i \to 0$, we get a natural isomorphism $F\coker f^i \to \coker Ff^i$. Applying $F$ on $0 \to \im f^{i} \to C^{i+1} \to \coker f^i \to 0$, we get a natural epimorphism $F\im f^i \onto \im Ff^i$. Applying $F$ on $0 \to H^i(C^{\bul}) \to \coker f^{i-1} \to \im f^i \to 0$ and chasing diagrams, we get a natural map $FH^i(C^{\bul}) \to H^i F(C^{\bul})$.

(b) Applying $F$ on $0 \to \ker f^i \to C^i \to C^{i+1}$, we get a natural isomorphism $\ker Ff^i \to F\ker f^i$. Applying $F$ on $0 \to \ker f^i \to C^i \to \im f^i \to 0$, we get a natural monomorphism $\im F f^i \into F\im f^i$. Applying $F$ on $0 \to \im f^{i-1} \to \ker f^i \to H^i(C^{\bul}) \to 0$ and chasing diagrams, we get a natural map $H^iF(C^{\bul}) \to F H^i(C^{\bul})$.

(c) Carefully trace where each element goes.
\end{proof}


\begin{prop}[Exactness and (co)limits]
Limits commute with limits and right adjoints. In particular, right adjoints and limits are both left exact since they commute with $\ker$.

Colimits commute with colimits and left adjoints. In particular, left adjoints and colimits are both right exact since they commute with $\coker$.

In $\Mod_A$, colimits over filtered index categories are exact.
\end{prop}



\section{Sheaves}

\subsection{The espace \'etal\'e of a (pre)sheaf}

Let $\cF$ be a (pre)sheaf on $X$. We can construct a topological space $F$ and a continuous $\pi: F\to X$ as follows:
\begin{itemize}
    \item As a set, $F = \coprod_{p\in X} \cF_p$.
    \item Open sets of $F$ are generated by the following base: given an open $U\subseteq X$ and $f\in \cF(U)$, the set $\{(p, U, f): p\in U\}$ is open.
\end{itemize}
Then $\pi: F\to U$ is a \emph{local homeomorphism}.

\subsection{Stalks and sheafification}


\subsection{Sheaf on a base} 

Suppose $X$ is a topological space with $\{B_i\}$ as a base of the topology. Suppose we're given the following information:
\begin{itemize}
    \item To each $B_i$, we have an associated set/abelian group/ring/module $\cF(B_i)$;
    \item For each $B_i\subseteq B_j$, a restriction map $\text{res}_{B_i, B_j}: \cF(B_i) \to \cF(B_j)$; this should be the identity when $i = j$;
    \item If $B_i\subseteq B_j \subseteq B_k$, then $\text{res}_{B_i, B_k} = \text{res}_{B_j, B_k}\circ 
    \text{res}_{B_i, B_j}$.
    \item If $B = \bigcup B_i$, then if $f, g\in \cF(B)$ restricts to the same function on each $\cF(B_i)$, then $f = g$;
    \item If $B = \bigcup B_i$, and $f_i\in \cF(B_i)$ such that if for any $B_k\subseteq B_i\cap B_j$, $f_i$ and $f_j$ restrict to the same function on $B_k$, then there exists $f\in \cF(B)$ such that $f$ restricts on each $f_i$ on each patch.
\end{itemize}
This is called a \emph{sheaf on a base}. Given this information, the sheaf on any open set can be uniquely determined up to unique isomorphism.

\subsection{Inverse image sheaf}


\section{Affine schemes}

\subsection{Spectrum of a ring}

\subsection{Hilbert's Nullstellensatz}

\subsection{Topological properties of affine schemes}

\begin{defn}
A topological space $X$ is \emph{Noetherian} if it satisfies the d.c.c. on closed sets.
\end{defn}

\begin{prop}
Let $X$ be Noetherian. Then every nonempty closed set $Z$ can be uniquely expressed as a finite union $Z = Z_1\cup \dots \cup Z_n$ of irreducible closed sets, none contained in any other.
\end{prop}


\section{Schemes}

\subsection{Proj construction}
\label{proj_construction}

Given a (commutative) ring $A$, $\Spec$ produces from it a locally ringed space $\Spec A$. If we take $A = k[x_1,\dots,x_n]$, then $\Spec A$ is the affine $n$-space $\AA^n_k$. Similarly, the Proj construction takes a $\ZZ_{\ge 0}$-graded ring $S$ as input, and produces from this data a scheme (not necessarily affine!) $\Proj S$, and in the special case $S = k[x_0,\dots,x_n]$, $\Proj S$ is the projective $n$-space $\PP^n_k$.

\begin{defn}
\label{proj}
Let $S$ be a $\ZZ_{\ge 0}$-graded ring. The scheme $\Proj S$ is given by:
\begin{itemize}
    \item As a set, the points in $\Proj S$ are the homogeneous prime ideals $\fp$ such that $S_+ \notin \fp$;
    \item As a topological space, the closed sets are given by $V(I) = \{[\fp] \in \Proj S: I\subseteq \fp\}$, for homogeneous ideals $I\subseteq S_+$. Equivalently, the topology is given by the base of distinguished opens $D(f) = \{[\fp] \in \Proj S: f\notin \fp\}$, for homogeneous $f\in S_+$.
    \item As a locally ringed space, the structure sheaf is given on the base by $\cO_{\Proj S}(D(f)) = (S_f)_{\deg 0}$.
\end{itemize}
\end{defn}

\begin{defn}
Let $S$ be a finitely generated graded ring over $A$. Then a scheme of the form $\Proj S$ is called a \emph{projective scheme over} $A$. An quasicompact open subscheme of a projective $A$-scheme is called a \emph{quasiprojective} $A$-\emph{scheme}.
\end{defn}

\subsection{Properties of schemes}

\begin{prop}
Let $X$ be a scheme. Then the points of $X$ correspond bijectively to irreducible closed sets of $X$, via the map
\[x \mapsto \bar{\{x\}}.\]
\end{prop}

\begin{proof}
Because the closure of an irreducible set is irreducible, this is a well-defined map. Conversely, given an irreducible closed set $T\subseteq X$, consider an affine open $U$ such that $T\cap U \neq \varnothing$. Then $T\cap U$ is an irreducible closed set in $U$, so it corresponds to a unique generic point in $U$. For affine opens $U,V$ both intersecting $T$, $U\cap V$ must also intersect $T$ because $T$ is irreducible. Pick an affine open $W\subseteq U\cap V$ that is distinguished in both $U$ and $V$ and also intersects $T$. Then the unique generic point corresponding to $T\cap W$ must simultaneously be the unique generic points corresponding to $T\cap U$ and $T\cap V$. In other words, there is a unique point $x\in T$ that is the unique generic point corresponding to $T\cap U$ for all affine opens $U$ intersecting $T$. 

We claim that $T = \bar{\{x\}}$; indeed for any closed $K\subseteq X$ containing $x$, and for any point $t\in T$, there is an affine open $U$ containing $t$ (and by default containing $x$ too), $K\cap U$ contains $x$, so it must contain $T\cap U$ (the closure of $\{x\}$ in $T\cap U$). In particular, $t\in K$ as well. 
\end{proof}

\begin{prop}
Let $X$ be a quasicompact scheme, then any point has a closed point in its closure.
\end{prop}

\begin{defn}
A scheme $X$ is called \emph{reduced} if all stalks are reduced rings. Equivalently, for all open $U$, $\cO_X(U)$ is reduced.
\end{defn}


\begin{defn}
A property P for affine open subsets of a scheme $X$ is called \emph{affine-local} if it satisfies:
\begin{itemize}
    \item If an affine open $\Spec A$ satisfies P, then any $\Spec A_f$ satisfies P also.
    \item If $f_1,\dots,f_n\in A$ generate the unit ideal, and all $\Spec A_{f_i}$ satisfy P, then $\Spec A$ satisfies P as well.
\end{itemize}
\end{defn}

\begin{lem}[Affine Communication Lemma]
\label{ACL}
Suppose P is an affine-local property, and $X = \bigcup_{i\in I} \Spec A_i$ where each $\Spec A_i$ satisfies property P. Then any affine open in $X$ satisfies P.
\end{lem}

Properties defined in this way:
\begin{itemize}
    \item Locally Noetherian
    \item Noetherian
    \item Locally of finite type over $B$
    \item Finite type over $B$
\end{itemize}

\subsection{Varieties}

An affine scheme that is reduced and of finite type over $k$ is called an \emph{affine $k$-variety}. A reduced quasiprojective $k$-scheme called a \emph{projective $k$-variety}.


\subsection{Normality and factoriality}

A scheme $X$ is \emph{normal} if all of its local rings are integrally closed domains.

Because being integrally closed is a local property, $\Spec A$ for $A$ integrally closed is an affine normal scheme. For a quasicompact scheme, this can also be checked at closed points only.

A scheme $X$ is \emph{factorial} if all of its local rings are UFDs. Since UFDs are all integrally closed, factorial schemes are normal. Factoriality is not affine-local.


\subsection{Associated points}

In the affine case, the associated points of an $A$-module $M$ are primes $\fp\subset A$ of the form $\fp = \Ann(m)$ for some $m\in M$. (See \href{https://stacks.math.columbia.edu/tag/00L9}{here}; also, taking $M = A/I$, we recover the usual associated points of an ideal.) They have the following properties:

\begin{thm}
    Suppose $A$ is Noetherian and $M\neq 0$ is finitely generated. Then:
    \begin{enumerate}
        \item $\Ass(M)$ is nonempty and finite.
        \item The natural map $M\to \prod_{\fp\in \Ass(M)} M_{\fp}$ is injective.
        \item $\bigcup_{\fp\in \Ass(M)}$ is precisely the set of zerodivisors of $M$.
        \item Associated primes commute with localization:
        \[\Ass_{S^{-1}A}(S^{-1}M) = \Ass_A(M)\cap \Spec S^{-1}A.\]
    \end{enumerate}
\end{thm}

In general (see \href{https://stacks.math.columbia.edu/tag/02OI}{here}): 

\begin{defn}
    Let $X$ be a scheme, and $F$ a quasicoherent sheaf. A point $x\in X$ is \emph{associated to} $F$ if $\fm_x$ is an associated point of the $\cO_{X,x}$-module $F_x$.
\end{defn}

\begin{prop}
    Let $X$ be locally Noetherian, $F$ quasicoherent. Let $U = \Spec A$ be an affine open, $x \in U$ corresponds to $\fp\subset A$, $M = \Gamma(U, F)$, then $x\in \Ass(F) \iff \fp\in \Ass(M)$.
\end{prop}

\begin{defn}
    Let $X$ be a scheme, $F$ a quasicoherent sheaf. An \emph{embedded associated point} is an associated point that is not minimal. 
\end{defn}

\begin{prop}
    Let $X$ be locally Noetherian, and $F$ coherent (e.g. $\cO_X$). Then the generic points of irreducible components of $\Supp F$ are associated points, and the rest of the associated points are embedded.
\end{prop}


\subsection{Weakly associated points}




\section{Morphisms of schemes}


\subsection{Morphisms to affine schemes}

These have a nice characterization:

\begin{prop}
The following are equivalent:
\begin{itemize}
    \item There is a morphism of schemes $X\to \Spec A$;
    \item For every open $U\subseteq X$, $\cO_X(U)$ is an $A$-algebra;
    \item There is a ring map $A\to \cO_X(X)$.
\end{itemize}
\end{prop}

\subsection{Morphisms from affine schemes}


Given any point $p\in X$, there is a canonical morphism $\Spec \cO_{X,p} \to X$. Composing this with the map induced by $\cO_{X,p} \to \kappa(p)$, we get a canonical $\Spec \kappa(p) \to X$, often written just as $p\to X$.

More generally: for a local ring $(A,\fm)$, a scheme morphism $\pi:\Spec A \to X$ sending $\fm$ to $p$ corresponds bijectively to local homomorphisms $\cO_{X,p} \to A$.

\begin{defn}[functor of points]
Let $Z$ be a scheme, the \emph{$Z$-valued points} of $X$ (denoted $X(Z)$) are the maps $Z\to X$. (When $Z=\Spec A$ or $\Spec k$, they are the $A$- or $k$-valued points.)
\end{defn}

If we're working with schemes over a base scheme $B$, then this data should also include a $Z\to B$ making $Z\to X\to B$ commute.

\subsection{Functoriality of Proj}

Suppose $\phi: S \to R$ is a map of graded rings (i.e. there exists $\NN_+$ such that $S_n$ maps to $R_{dn}$ for all $n$). This induces a morphism of schemes
\[\phi^*: (\Proj R)\backslash V(\phi(S_+)) \to \Proj S,\]
as follows: given $f\in S_+$, there is a map of rings $S_f\to R_{\phi(f)}$, hence a map of rings $(S_f)_{\deg 0} \to (R_{\phi(f)})_{\deg 0}$, hence a morphism of affine schemes $\Spec (R_{\phi(f)})_{\deg 0} \to \Spec (S_f)_{\deg 0}$, i.e. $D(\phi(f)) \to D(f) \into \Proj S$. These glue together to form the desired morphism of schemes.

In particular, if $V(\phi(S_+))$ is empty, then we get an actual morphism $\Proj R\to \Proj S$. This is satisfied when $\rad(\phi(S_+)) = R_+$. (Recall from \S\ref{proj_construction} that the radical turns out to be equal to the intersection of all homogeneous primes containing the ideal.)

\subsection{Veronese subring}

\subsection{The relative point of view}
\label{rel_POV}

Instead of thinking of properties of objects, it might be better to understand them as properties of morphisms between objects. For example, given a property P about schemes, one often turns it into a property about morphisms of schemes as follows: say $\pi: X\to Y$ has P if and only if for every affine open $U\subset Y$, $\pi^{-1}(U)$ has P.

\subsection{Green flags to look for in a property of morphisms}

\begin{enumerate}
    \item It is \emph{local on the target}: for a morphism $\pi: X\to Y$ and a open cover $V_i$ of $Y$, $\pi$ satisfies P iff all $\pi|_{\pi^{-1}(V_i)}$ satisfy P.
    \item It is closed under composition.
    \item It is closed under base change, pullback, fibered products, etc.
    \item $\dots$
\end{enumerate}

\subsection{Finiteness conditions on morphisms}

Recall that a scheme is called \emph{quasicompact} if it is the union of finitely many affine schemes, and a scheme is called \emph{quasiseparated} if the intersection of any two quasicompact open subsets is quasicompact. We turn them into properties of schemes as discussed in \S\ref{rel_POV}. These are both affine-local on the target and closed under composition. Conversely, a scheme $X$ is quasicompact (resp. quasiseparated) if the canonical $X\to \Spec \ZZ$ is so. Note that many schemes we commonly encounter are qcqs: in particular, all affine schemes are qcqs, and all Noetherian schemes are qcqs.

\begin{defn}[affine morphisms]
A morphism $\pi:X\to Y$ is \emph{affine} if the preimage of any affine open in $Y$ is affine open in $X$. Affine morphisms are automatically qcqs.
\end{defn}

\begin{lem}[qcqs lemma]
If $X$ is qcqs, $s\in \cO_X(X)$, then the natural map $\cO_X(X)_s \to \cO_X(X_s)$ is an isomorphism.
\end{lem}

\begin{proof}
Use the qcqs property as a finite presentation.
\end{proof}

\begin{prop}
Affineness is affine-local on the target. In other words, affineness of a morphism can be checked on affine covers of the target.
\end{prop}

\begin{proof}

\end{proof}

\begin{defn}[finite morphisms]
An affine morphism $\pi: X\to Y$ is \emph{finite} if for any affine $\Spec A\subset Y$, $\pi^{-1}(\Spec A)$ is the spectrum of a ring that is a finitely generated module over $A$.
\end{defn}

Finiteness is also affine-local on the target.

\begin{exm}
Examples of finite morphisms:

\begin{itemize}
    \item Branched covers: consider the map $k[u] \to k[t]$ given by $u\mapsto p(t)$ for a polynomial $p$. Then $\Spec k[t] \to \Spec k[u]$ is a finite morphism.
    
    \item Closed embeddings: $A/I$ is a finite $A$-module (generated by 1), so $\Spec A/I \to \Spec A$ is a finite morphism.
    
    \item Normalization: $k[x,y]/(y^2-x^2-x^3) \mapsto k[t]$ by $x\mapsto t^2-1$, $y\mapsto t^3-t$ induces a morphism of schemes $\Spec k[t] \to \Spec k[x,y]/(y^2-x^2-x^3)$. This is a finite morphism, and it is in fact an isomorphism from $D(t^2-1)$ to $D(x)$.
\end{itemize}
\end{exm}

\begin{prop}[7.3.H]
If $X\to \Spec k$ is a finite morphism, then $X$ is a finite union of points with the discrete topology, each point with residue field a finite extension of $k$.
\end{prop}

\begin{proof}
We must have $X = \Spec A$, where $A$ is a $k$-algebra that is finitely generated as a module. Then $A$ is Noetherian and any prime $\fp\subset A$ is maximal, so the (finitely many) irreducible components of $A$, which correspond to minimal primes, are all closed points. Therefore $\Spec A$ is finite discrete, and the residue field at each point $[\fp]$ is a finite extension of $k$.
\end{proof}

\begin{cor}[7.3.K]
Finite morphisms have finite fibers.
\end{cor}

\begin{defn}[integral morphisms]
A morphism $\pi:X\to Y$ is \emph{integral} if it is affine, and for every affine open $\Spec B \subset Y$, $\Spec A = \pi^{-1}(\Spec B)$, $B\to A$ is an integral extension.
\end{defn}

Because integrality is an affine-local property, a morphism being integral is affine-local on the target. Also, finite morphisms are integral, and integral morphisms are closed (they map closed sets to closed sets).

\begin{defn}[finite type morphisms]
A morphism $\pi:X\to Y$ is \emph{locally of finite type} if for every affine open $\Spec B \subset Y$, and for every $\Spec A \subset \pi^{-1}(\Spec B)$, $B\to A$ expresses $A$ as a finitely generated $B$-algebra. We say $\pi$ is \emph{finite type} if it is quasicompact and locally of finite type.
\end{defn}

\begin{prop}[7.3.P]
A morphism is finite iff it is integral and of finite type.
\end{prop}

\begin{defn}[finitely presented morphisms]
A morphism $\pi: X\to Y$ is \emph{locally finitely presented} if for every affine open $\Spec B \subset Y$, $\pi^{-1}(\Spec B) = \bigcup_i \Spec A_i$ with each $B\to A_i$ finitely presented. We say $\pi$ is finitely presented if it is locally finitely presented and qc\emph{qs}.
\end{defn}

It is clear that if $Y$ is locally Noetherian, then locally of finite presentation is the same as locally of finite type, and finite presentation is the same as finite type.

\begin{prop}
Locally finitely presented-ness is affine-local on both the target and the source.
\end{prop}

\subsection{Elimination theory}

\begin{lem}[Generic freeness]
Let $B$ be a Noetherian integral domain, $A$ a finite type algebra over $B$, and $M$ a finitely generated $A$-module. Then there exists $f\in B$ such that $M_f$ is a free $B_f$-module.
\end{lem}

\begin{thm}[Chevalley's theorem]
Let $\pi:X\to Y$ be a finite type morphism between Noetherian schemes. Then the image of any constructible set is constructible.
\end{thm}

\begin{thm}[Fundamental theorem of elimination theory]
The map $\PP^n_A \to \Spec A$ is closed, for any ring $A$.
\end{thm}


\subsection{Closed subschemes, and related constructions}

\begin{defn}
    A \emph{closed embedding} $\pi:X\into Y$ is an affine morphism where for each $\Spec B\subseteq Y$ and $\Spec A = \pi^{-1}(\Spec B)$, the induced ring map $B\to A$ is surjective.
\end{defn}

\begin{defn}[equivalent to the above]
    A \emph{closed embedding} $\pi:X\to Y$ is a morphism such that $\pi$ induces a homeomorphism of the underlying topological space of $X$ onto a closed subset of the topological space of $Y$, and the induced map $\pi^{\sharp}:\cO_Y\to \pi_*\cO_X$ of sheaves on $Y$ is surjective.    
\end{defn}

Ideal sheaf, scheme-theoretic image, intersection and union of closed subschemes

\subsection{Effective Cartier divisors and regular sequences}

\begin{defn}
    A \emph{locally principal} closed subscheme $\pi: X\into Y$ is one for which there exists an open cover $U_i$ of $Y$, such that each $\pi^{-1}(U_i)\to U_i$ is isomorphic to a closed subscheme $V(s_i)\subset U_i$, where $s_i\in \cO_Y(U_i)$. Equivalently, we may as well take all $U_i$ to be affine.
\end{defn}

\begin{defn}
    An \emph{effective Cartier divisor} is a locally principal closed subscheme where the ideal sheaf is locally generated near every point by a non-zero divisor.
\end{defn}

\begin{exm}
    Consider $\Spec A$, where $A = k[w,x,y,z]/(wz-xy)$. Let $X$ be the open subscheme $D(y)\cup D(w)$. The closed subscheme defined by $V(z/y)$ on $D(y)$ and $V(x/w)$ on $D(w)$ is an effective Cartier divisor, but it is not generated by a single element of $\Frac A$.
\end{exm}

\begin{defn}
    Let $M$ be an $A$-module. A sequence $x_1,\dots,x_r$ of elements in $A$ is called an \emph{$M$-regular sequence} if:
    \begin{itemize}
        \item For each $i$, $x_i$ is not a zero divisor for $M/(x_1,\dots,x_{i-1})M$ (exists no $m\in M\backslash (x_1,\dots,x_{i-1})M$ such that $mx_i\in (x_1,\dots,x_{i-1})M$), and
        \item $(x_1,\dots,x_r)M\neq M$.
    \end{itemize}
    In particular, an $A$-regular sequence is just called a regular sequence.
\end{defn}


\begin{exm}
    For any $M$-regular sequence $x_1,\dots,x_n$, and positive integers $a_1,\dots,a_n$, the sequence $x_1^{a_1},\dots,x_n^{a_n}$ is a regular sequence too.
\end{exm}

\begin{exm}
    Let $A = k[x,y,z]/(x-1)z$. Then $x, (x-1)y$ is a regular sequence, while $(x-1)y,x$ is not.
\end{exm}

\begin{thm}
    Let $A$ be a Noetherian local ring, and $M$ a finitely generated $A$-module. Then any $M$-regular sequence remains regular when reordered.
\end{thm}

\begin{defn}[regular embedding]
    Let $\pi:X\to Y$ be a locally closed embedding. Say that $\pi$ is a \emph{regular embedding} of codimension $r$ at $x\in X$ if in $\cO_{Y, \pi(x)}$, the ideal of $X$ is generated by a regular sequence of length $r$. Say that $\pi$ is a \emph{regular embedding} if it is at all points.
\end{defn}

\subsection{Fiber products}


\subsection{An interlude on closed points}

\begin{prop}
Let $X$ be a scheme locally of finite type over a field $k$. If $x\in \Spec A \subset X$ corresponds to a maximal ideal in some affine open subscheme of $X$, then $x$ is a closed point in $X$.
\end{prop}

\begin{proof}
Suppose $x$ corresponds to $\fm\subset A$, then $\kappa(x) = A/\fm$. By the nullstellensatz, $A/\fm$ is a finite extension of $k$. Now, suppose $\Spec B\subset X$ is some other affine open containing $x$, and say $x$ corresponds to a prime $\fp\subset B$. Then $\kappa(x) = \Frac B/\fp$, so in particular $k\subseteq B/\fp \subseteq \kappa(x)$. So $B/\fp$ is an integral extension of $k$, so it is a field as well, i.e. $\fp$ is maximal. So $\{x\}$ is closed in $X$.
\end{proof}

\begin{prop}
Let $X$ be a scheme locally of finite type over $k$. Suppose we have a morphism $\pi:\Spec k \to X$, then its image is a closed point.
\end{prop}

\begin{proof}
Let $\Spec A \subset X$ be an affine open subscheme. The morphism $\pi$ factors through $\Spec A$, so we get $\phi: \Spec k \to \Spec A$. Suppose $\fm$ is the kernel of the corresponding map $A\to k$, and $\fp$ is the prime ideal corresponding to the image of $\pi$. Then we get a map of stalks $A_{\fp}\to k$ through which the map $A\to k$ factors. Suppose $a\notin \fp$, then $a$ is invertible in $A_{\fp}$, so it is not in the kernel of $A\to k$, so $\fm\subseteq \fp$. Since $\fm$ is maximal, $\fm = \fp$, so we conclude by the previous proposition.
\end{proof}

\begin{prop}
Let $X$ be a scheme locally of finite type over $k = \bar{k}$. Then closed points of $X$ are in bijection with $k$-points of $X$.
\end{prop}


\begin{proof}
The bijection is given by:
\begin{itemize}
    \item Given a $k$-point $\Spec k \to X$, this maps to its image, which is a closed point in $X$;
    \item Given a closed point $x\in X$, its field of fractions is $k$ by the nullstellensatz, so we get $\Spec k = \Spec \kappa(x) \to X$.
\end{itemize}
It suffices to verify that these two are inverses. Given a closed point $x\in X$, it is clear from definition that the image of $\Spec \kappa(x) \to X$ is $x$. On the other hand, given a $k$-point $\Spec k \to X$, it is given by $\Spec k \to \Spec A \subseteq X$, where $A/k$ is the maximal ideal corresponding to the image $x$ of the $k$-point. So $A/k = \kappa(x)$, which finishes the proof.
\end{proof}

\begin{cor}
Let $f:X\to Y$ be a morphism between schemes over $k$ locally of finite type. Then $f$ maps closed points to closed points. In particular, maps between $k$-schemes map closed points to closed points.
\end{cor}


\subsection{Separated morphisms}

\begin{defn}
A morphism of schemes $\pi:X\to Y$ is \emph{separated} if the diagonal map $\Delta_{\pi}: X\to X\times_Y X$ is a closed embedding.
\end{defn}

To see that this definition isn't too crazy, we notice the following.

\begin{prop}
Let $\pi:X\to Y$ be a morphism. The diagonal $\Delta_{\pi}: X\to X\times_Y X$ is a locally closed embedding (i.e. a closed subscheme of an open subscheme).
\end{prop}

\begin{proof}
Cover $Y$ by affine opens $V_i$, and $\pi^{-1}(V_i)$ by affine opens $U_{ij}$. Then $U_{ij}\times_{V_i} U_{ij}$ is an affine open subscheme of $X\times_Y X$ by definition, and these cover the image of $\Delta_{\pi}$. Further, it is clear that $\Delta_{\pi}^{-1}(U_{ij}\times_{V_i} U_{ij}) = U_{ij}$, and $\Delta|_{U_{ij}}$ is a closed embedding.
\end{proof}

\begin{defn}
A \emph{variety} over a field $k$ is a reduced, separated, finite-type $k$-scheme.
\end{defn}

Because a locally closed embedding whose image is closed is in fact a closed embedding, to check that $\pi:X\to Y$ is separated, it suffices to check that the image of $\Delta$ is closed.

Examples of separated morphisms:
\begin{itemize}
    \item Locally closed embeddings (also called \emph{immersions});
    \item Morphisms between affine schemes;
    \item All quasiprojective $A$-schemes (with morphism to $\Spec A$);
    \item Any morphism between varieties is automatically separated and finite type (this will follow from the cancellation theorem).
\end{itemize}

\begin{lem}[Magic diagram]
Let $X_1,X_2,Y,Z$ be objects in a category where fiber products exist. Suppose we are given maps $f_1:X_1\to Y$, $f_2:X_2\to Y$, and $g:Y\to Z$. Then the following diagram is a Cartesian square:
\[
\begin{tikzcd}
X_1\times_Y X_2 \arrow[r] \arrow[d] & X_1\times_Z X_2 \arrow[d, "f_1\times f_2"] \\
Y \arrow[r, "\Delta"] & Y\times_Z Y.
\end{tikzcd}
\]
\end{lem}

\begin{prop}
Let $X$ be separated over a ring $A$. Then for $U,V\subset X$ affine opens, $U\cap V$ is an affine open as well.
\end{prop}

\begin{proof}
Consider the following fiber product:
\[
\begin{tikzcd}
U\cap V \arrow[r] \arrow[d] & U\times_A V \arrow[d] \\
X \arrow[r, hook, "\Delta"] & X\times_A X
\end{tikzcd}
\]
Here, $U\cap V = U\times_X V$ is the fiber product because of the magic diagram. Now, because the bottom map is a closed embedding, so is the top map. Since $U\times_A V$ is an affine scheme, so is $U\cap V$.
\end{proof}


\begin{prop}
Separatedness is well-behaved:
\begin{enumerate}[(1)]
    \item affine-local on the target;
    \item stable under composition;
    \item stable under base change.
\end{enumerate}
\end{prop}

\begin{proof}
(1) This follows from the fact that $\pi: X\to Y$ is separated if and only if $\im(\Delta)$ is closed.

(2) Suppose $f:X\to Y, g:Y\to Z$ are separated. Consider the following commutative diagram
\[
\begin{tikzcd}
X \arrow[r, hook, "\Delta_f"] & X\times_Y X \arrow[r] \arrow[d] & X\times_Z X \arrow[d] \\
& Y \arrow[r, hook, "\Delta_g"] & Y\times_Z Y
\end{tikzcd}
\]
The square is Cartesian by the magic diagram, so the top map $X\times_Y X\to X\times_Z X$ is a closed embedding. So the composition $X\to X\times_Z X$, which can be verified to be the diagonal of $g\circ f$, is a closed embedding.

(3) Suppose
\[
\begin{tikzcd}
X \arrow[r] \arrow[d] & Y \arrow[d] \\
Z \arrow[r] & W
\end{tikzcd}
\]
is a pullback square, where $Z\to W$ is separated. It suffices to show that
\[
\begin{tikzcd}
X \arrow[r, "\Delta"] \arrow[d] & X\times_Y X \arrow[d] \\
Z \arrow[r, "\Delta"] & Z\times_W Z
\end{tikzcd}
\]
is also a pullback square, which is a straightforward diagram chase.
\end{proof}

\begin{prop}
Let $\pi: X\to Y$ be a morphism of $Z$-schemes, and $Y\to Z$ separated. Then its \emph{graph} $\Gamma_{\pi}: X\xto{(\id, \pi)} X\times_Z Y$ is a closed embedding.
\end{prop}


\begin{prop}[Cancellation theorem]
Let $X\xto{f} Y\xto{g} Z$, and suppose P is a property of morphisms, such that:
\begin{itemize}
    \item P is stable under composition;
    \item P is stable under base change;
    \item $g\circ f$ satisfies P;
    \item $\Delta_g: Y\to Y\times_z Y$ satisfies P.
\end{itemize}
Then $f$ satisfies P also.
\end{prop}

\begin{proof}
We have the following Cartesian squares:
\[
\begin{tikzcd}
X \times_Z Y \arrow[r, "\pi"] \arrow[d] & Y \arrow[d, "g"] \\
X \arrow[r, "g\circ f"] & Z.
\end{tikzcd}
\]
Here, because $g\circ f$ satisfies P, so does $\pi: X\times_Z Y\to Y$. Also, we have 
\[
\begin{tikzcd}
X = X\times_Y Y \arrow[r, "\Gamma"] \arrow[d, "f"] & X\times_Z Y \arrow[d] \\
Y \arrow[r, "\Delta_g"] & Y\times_Z Y,
\end{tikzcd}
\]
and because $\Delta_g$ satisfies P, so does $\Gamma$. But $\pi\circ \Gamma$ is easily verified to be simply $f$, so $f$ satisfies P also.
\end{proof}

\begin{thm}[Reduced to separated theorem]
Suppose $X,Y$ are schemes over $Z$, where $X$ is reduced, and $Y\to Z$ is separated. Let $\pi,\pi':X\to Y$ be morphisms over $Z$. Suppose $U\subseteq X$ is a dense open on which $\pi$ and $\pi'$ agree. Then $\pi = \pi'$.
\end{thm}

\begin{proof}
Let $V$ be the fiber product
\[
\begin{tikzcd}
V \arrow[r] \arrow[d] & Y \arrow[d, "\Delta"] \\
X \arrow[r, "(\pi;\pi')"] & Y\times_Z Y.
\end{tikzcd}
\]
Because $\Delta$ is a closed embedding, so is $V\into X$. Because $\pi|_U = \pi'|_U$, we get a map $U\to V$ through the universal property of $V$. But $U$ is an open subscheme of $X$. Because $U$ is dense, $V =X$ as sets. Because $X$ is reduced, $V=X$ as schemes. So $\pi = \pi'$ on all of $X$. 
\end{proof}

\subsection{Dominant rational maps between irreducible varieties}

\begin{defn}
A \emph{rational map} between schemes $X\dashrightarrow Y$ is a map $U\to Y$ where $U$ is a dense open in $X$. Two rational maps $X\dashrightarrow Y$ are \emph{equivalent} if $\alpha|_W = \beta|_W$ on some dense open $W\subseteq U\cap V$.
\end{defn}

\begin{defn}
A morphism of schemes is \emph{dominant} if its image is dense.
\end{defn}

Fix a field $k$ (algebraically closed when necessary), consider the category of irreducible varieties over $k$, with morphisms as dominant rational maps.

Given an irreducible variety $X$, because irreducible and reduced implies integral, it has a unique generic point $\eta$. The stalk at $\eta$ is the \emph{function field} $K(X)$, which is equal to the fraction field $\Frac A$ of any affine open $\Spec A\subseteq X$. Given a rational map $X\dashrightarrow Y$, this induces a field homomorphism at the stalks of the generic points.

\begin{thm}
The functor described above gives an equivalence of categories between irreducible varieties with dominant rational maps and finitely generated field $L/k$ with inclusions of fields.
\end{thm}

\subsection{Ax-Grothendieck theorem}

\begin{thm}[Ax-Grothendieck]
Let $X$ be a variety over $\CC$, $f:X\to X$ a morphism over $\CC$. Suppose that the map of $\CC$-points $X(\CC)\to X(\CC)$ is injective (as a set), then it must be surjective.
\end{thm}


We will define the \emph{spreading out} of $X$, which is a finite type scheme over $\Spec R$, for some finitely generated $\ZZ$-algebra $R\subset \CC$. 

Cover $X$ by (finitely many, since $X$ is quasicompact) affine schemes $U_i$, which are of the form $\Spec \CC[x_1,\dots,x_n]/(f_1,\dots,f_r)$ since $X$ is finite type and by Hilbert's basis theorem. Because $X$ is separated, $U_i\cap U_j$ is also affine of the above form. Even further, each $f^{-1}(U_i)$ is covered by finitely many affine opens $U_{ij}$, because morphisms between varieties are automatically quasicompact, and the $U_{ij}$'s are again of the above form. So we can take $R$ to be the $\ZZ$-algebra generated by all coefficients of $f_i$ appearing in $U_i$, $U_i\cap U_j$, and $U_{ij}$'s, and define $\cX$ by glueing together $\Spec R[x_1,\dots,x_n]/(f_1,\dots,f_r)$. The map $f:X\to X$ also spreads out to a map $F: \cX \to \cX$. By definition, this satisfies the following Cartesian squares:
\[
\begin{tikzcd}
X \arrow[r] \arrow[d, "f"'] & \cX \arrow[d, "F"] \\
X \arrow[r] \arrow[d] & \cX \arrow[d] \\
\Spec \CC \arrow[r] & \Spec R.
\end{tikzcd}
\]
Now, set $U = X\times_{\CC} X \backslash \Delta(X)$, an open subscheme of $X\times_{\CC} X$. Let $W$ be the fiber product
\[
\begin{tikzcd}
W \arrow[rr] \arrow[d] & & X \arrow[d] \\
U \arrow[r, hook] & X\times_{\CC} X \arrow[r, "f\times_{\CC} f"] & X\times_{\CC} X,
\end{tikzcd}
\]
and supposing $x\in X$ is a point, let $Z$ be the fiber
\[
\begin{tikzcd}
Z \arrow[r] \arrow[d] & \Spec \CC \arrow[d, "x"] \\
X \arrow[r, "f"] & X.
\end{tikzcd}
\]
Then $X(\CC)\to X(\CC)$ is injective implies that $W = \varnothing$, and we wish to show surjectivity at $x$, i.e. $Z \neq \varnothing$.

Because spreading out behaves well with fiber products, we can similarly define the spread-out of $x, U, W, Z$ as $\chi, \cU, \cW, \cZ$.

Sketch of proof:
\begin{itemize}
    \item Reduce the problem into showing that given $W = \varnothing$, show that $Z \neq \varnothing$.
    \item Spread out to get $\cX, F, \chi, \cU, \cW, \cZ$.
    \item $W = \varnothing$ implies $\cW\times_R K = \varnothing$, where $K = \Frac R$. This implies the image of $\pi_{\cW}: \cW \to \Spec R$, which is constructible by Chevalley, does not include the generic point $\eta_R$. So $\eta_R\notin \bar{\im \pi_{\cW}}$, so we may invert finitely many elements of $R$ so that $\im \pi_{\cW} = \varnothing$, i.e. $\cW = \varnothing$.
    \item Let $t$ be a closed point in $\Spec R$, $F_t: \cX_t\to \cX_t$ be the induced map. Then $\kappa(t)$ is a finite field $\FF_q$. Because $\cW_t(\FF_q)=\varnothing$, the map $\cX_t(\FF_q) \to \cX_t(\FF_q)$ is injective, hence surjective. So $\cZ_t\neq\varnothing$ for all closed points $t$.
    \item So $\pi_{\cZ}:\cZ\to \Spec R$ has image containing all closed points, which are dense in $\Spec R$. So the generic point $\eta_{R}$ is contained in the image, which is constructible by Chevalley. So $\cZ\times_R K\neq\varnothing$, which implies $Z\neq\varnothing$. This concludes the proof.
\end{itemize}

\begin{lem}
Let $S$ be a constructible set in $\Spec R$, $R$ Noetherian. If $\eta_R\notin S$, then $\eta_R\notin \bar{S}$.
\end{lem}

\begin{proof}
Write $S = \coprod_i (U_i\cap K_i)$ as the disjoint union of locally closed sets over a finite index set. Then $\bar{S} = \bigcup_i \bar{U_i\cap K_i}$. Suppose for contradiction $\eta_R\in \bar{U_i\cap K_i}$ for some $i$, then $\Spec R = \bar{U_i\cap K_i} \subseteq K_i$, so $K_i = \Spec R$ and $U_i$ is a dense open in $\Spec R$, so $\eta_R\in U_i$, which implies $\eta_R\in S$, a contradiction.
\end{proof}



\begin{lem}
Let $k\subseteq \CC$ be a subfield, $V$ a $k$-variety. Then the following are equivalent:
\begin{itemize}
    \item $V = \varnothing$;
    \item $V_{\CC} := V\times_k \CC = \varnothing$;
    \item $V_{\CC}(\CC) = \varnothing$.
\end{itemize}
\end{lem}


\subsection{Proper maps}

Just as separatedness captures the topological concept of a Hausdorff space, properness is meant to capture the concept of compactness. Of course, quasicompactness won't do the job. Recall the topological notion:

\begin{defn}
A map of topological spaces is \emph{proper} if the inverse image of any compact set is compact.
\end{defn}

\begin{defn}
A \emph{universally closed} map $f:M\to N$ of topological spaces is one such that for all $P\to N$, $f_P: P\times_N M \to P$ is a closed map.
\end{defn}

We remark that the map from $M$ to a point is universally closed iff $M$ is compact.

The same definition moves over to schemes:

\begin{defn}
A \emph{universally closed} morphism $f:X\to Y$ of schemes is one such that for all $Z\to Y$, $f_Z: Z\times_Y X \to Z$ is a closed morphism.
\end{defn}

\begin{defn}
A morphism of schemes $\pi: X\to Y$ is \emph{proper} if it is finite type, separated, and universally closed. 
\end{defn}

So, $X\to \Spec k$ being universally closed corresponds to $X$ being ``compact''.

\begin{exm}
Examples of proper morphisms:
\begin{itemize}
    \item Closed embeddings;
    \item Properness is stable under composition and base change;
    \item $\PP^n_A\to \Spec A$ is proper; as a consequence, any projective morphism $Z\into \PP^n_A\to \Spec A$ is proper.
    \item In contrast, $\AA^1_{\CC}$ is not proper (this fits your intuition that a line is not compact). This can be seen by the following square:
    \[
    \begin{tikzcd}
    \AA^2 \arrow[r] \arrow[d] & \AA^1 \arrow[d] \\
    \AA^1 \arrow[r] & \bullet
    \end{tikzcd}
    \]
    But the left map is not closed: $V(xy - 1)$ maps to $D(x)$, which is not closed.
\end{itemize}
\end{exm}

\subsection{Chow's lemma}


Chow's lemma says that ``a proper morphism is fairly close to being a projective morphism''. Note that by the fundamental theorem of elimination theory, projective morphisms are proper. 

\begin{thm}[Chow]
    Let $f: X\to S$ be a separated, finite type morphism of Noetherian schemes. Then for there exists a diagram
    \[
    \begin{tikzcd}
        X' \arrow[r, hook, "i"] & \PP_X^n \arrow[d, "\pi'"] \arrow[r, "f'"] & \PP_S^n \arrow[d, "\pi"] \\
        & X \arrow[r, "f"] & S
    \end{tikzcd}
    \]
    where the square is Cartesian, $i$ is a closed immersion, $f'\circ i$ is an immersion, and $\pi'\circ i$ is surjective and induces an isomorphism on a dense open set $U\subseteq X$.
\end{thm}

In the case $f$ is proper, $f'$ must then be closed, so $X'$ is a projective $S$-scheme that surjects onto $X$ and is an isomorphism over a dense open of $X$.


\subsection{Valuative criteria}


\begin{thm}[valuative criteria]
We have the following criteria:
   \begin{itemize}
       \item Let $f:X\to Y$ be quasiseparated, then $f$ is separated iff for every valuation ring $V$ with field of fractions $K$, $X_Y(V)\to X_Y(K)$ is injective.
       \item Let $f: X\to Y$ be quasicompact, then $f$ is universally closed iff for every valuation ring $V$ with field of fractions $K$, $X_Y(V)\to X_Y(K)$ is surjective.
       \item Let $f:X\to Y$ be quasiseparated and finite type, then $f$ is proper iff for every valuation ring $V$ with field of fractions $K$, $X_Y(V)\to X_Y(K)$ is bijective.
   \end{itemize} 
\end{thm}

(Aside: in fact, universally closed implies quasicompact. Also, a map of schemes is a closed immersion if and only if it is a proper monomorphism.)


\section{Dimension and smoothness}


\subsection{Definitions of dimension}

The Krull dimension of a scheme is a purely topological construction and does not depend on the sheaf structure. 

\begin{lem}
    Let $X$ be a topological space, $U\subseteq X$ open. Then there is a bijection between closed irreducible subsets of $U$ and closed irreducible subsets of $X$ that meet $U$, given by
    \begin{align*}
        K \subseteq U &\longmapsto \bar{K} \subseteq X \\
        L\cap U\subseteq U \thinspace & \text{\reflectbox{$\longmapsto$}}\ L\subseteq X.
    \end{align*}
\end{lem}

\begin{proof}
First, we show that given a closed irreducible set $K\subseteq X$ that meets $U$, $\bar{K\cap U} = K$. Because $K$ meets $U$, $K\cap U^{c} \neq K$, so because $K = \bar{K\cap U} \cup (K\cap U^c)$ is irreducible, $K = \bar{K\cap U}$.

Next, we show that given a closed subset $K\subset U$, $\bar{K}\cap U = K$. Clearly $K\subseteq \bar{K}\cap U$. Since $K$ is closed in $U$, $K = L\cap U$ for some closed $L\subseteq X$. Then $\bar{K}\cap U \subseteq L\cap U = K\subseteq \bar{K}\cap U$, so equality holds.

Now we are ready to show the bijection. It suffices to show both maps are well-defined, since the above two paragraphs shows that the two maps are inverses of each other. Given a closed irreducible $K\subset U$, it is clear that its closure $\bar{K}$ is closed in $X$ and meets $U$. To show it is irreducible, suppose $\bar{K} = C_1\cup C_2$ for closed $C_1,C_2$. Then $K = \bar{K}\cap U = (C_1\cap U)\cup (C_2\cap U)$, so WLOG $C_1\cap U = K$. Then $C_1\subseteq \bar{K} = \bar{C_1\cap U}\subseteq C_1$, so equality holds and $C_1 = \bar{K}$. 

Conversely, given a closed irreducible $L\subseteq X$ that meets $U$, $L\cap U$ is closed in $U$. To show it is irreducible, suppose $L\cap U = (C_1\cap U)\cup (C_2\cap U)$, where $C_1,C_2\subseteq X$ are closed. Then 
\[L = \bar{L\cap U} = \bar{(C_1\cap U)\cup (C_2\cap U)} = \bar{C_1\cap U} \cup \bar{C_2\cap U},\]
so WLOG $\bar{C_1\cap U} = L$. Then $L\cap U = \bar{C_1\cap U} \cap U = C_1\cap U$. This shows $L\cap U$ is irreducible, which completes the proof.
\end{proof}


\begin{cor}
Suppose $X = \bigcup_i U_i$ is an open cover of a topological space. Then
    \[\dim X = \sup_i \dim U_i.\]
In particular, the dimension of a scheme can be checked on any affine open cover.
\end{cor}


\begin{proof}
    Consider any sequence
    \[\varnothing \neq Z_0 \subsetneq Z_1 \subsetneq \dots \subsetneq Z_n \subseteq X,\]
    where $Z_i$ are irreducible and closed. Because $Z_0\neq \varnothing$, there exists $U_i$ such that $Z_0\cap U_i\neq \varnothing$. Then
    \[\varnothing \neq Z_0\cap U_i \subsetneq Z_1\cap U_i \subsetneq \dots \subsetneq Z_n\cap U_i \subseteq U_i\]
    is also a chain of irreducible closed sets by the above lemma. This shows $\dim X \le \sup_i \dim U_i$. Conversely, for any $i$ and a chain of irreducible closed subsets
    \[\varnothing \neq Z_0 \subsetneq Z_1 \subsetneq \dots \subsetneq Z_n \subseteq U_i\]
    of $U_i$,
    \[\varnothing \neq \bar{Z_0} \subsetneq \bar{Z_1} \subsetneq \dots \subsetneq \bar{Z_n} \subseteq X\]
    is a chain of irreducible closed sets in $X$, again by the above lemma. So $\dim X \ge \sup_i \dim U_i$, so equality holds.
\end{proof}


\begin{defn}
    The \emph{codimension} $\codim_X Y$ of an irreducible subset $Y\subseteq X$ is the supremum of lengths of increasing chains of irreducible closed subsets starting with $\bar{Y}$. The corresponding ring-theoretic notion is the \emph{height} $\height \fp$ of a prime ideal $\fp$.
\end{defn}

Warning: Noetherian rings can be infinite-dimensional. On the other hand, Noetherian local rings must have finite dimension. 

\begin{thm}[Krull's height theorem]
    Let $A$ be a Noetherian ring, $I$ a proper ideal generated by $r$ elements, then every minimal prime of $I$ has height at most $r$.
\end{thm}

\begin{thm}[Algebraic Hartogs's Lemma]
\label{hartogs}
    Let $A$ be a Noetherian integrally closed domain. Then
    \[A = \bigcap_{\height \fp = 1} A_{\fp}.\]
\end{thm}

Intuitively, this says that on a normal Noetherian scheme, a rational function that is regular outside a closed set of codimension at least 2 can be uniquely extended to a regular function on the whole scheme. Compare this with Hartogs's lemma in complex analysis.

\begin{proof}
    This is trivially true when $\dim A \le 1$. In general, suppose for contradiction $x\in \Frac A$ belongs to $A_{\fp}$ for every prime of height 1, and $x\notin A$. Let $I = \{a\in A: ax\in A\}$, then $1\notin I$, so there exists a minimal prime $\fq\supseteq I$. Because $I_{\fq} = \{a\in A_{\fq}: ax\in A_{\fq}\}$ is not equal to $A_{\fq}$, we see that $\fq$ has height at least 2.

    Localize at $\fq$ to assume WLOG that $(A,\fq)$ is a local ring and $\fq$ is the unique prime containing $I$. Then $\fq = \rad(I)$, and because $A$ is Noetherian, $\fq$ is finitely generated, so $I\supseteq \fq^n$ for some $n$. Take the smallest such $n$. Consider an element $t\in \fq^{n-1}\backslash I$, and let $z = xt$. Because $t\notin I$, $z = xt \notin A$, but $z\fq \subseteq x\fq^n \subseteq xI \subseteq A$. 

    Now, if $z\fq \not\subseteq \fq$, then $z\fq = A$, so $\fq = \frac{1}{z}A$ is a principal ideal, contradicting $\height \fq \ge 2$. So we conclude that $z\fq\subseteq \fq$, and we have a faithful $A[z]$-action on the finitely generated $A$-module $\fq$, so $z$ is integral over $A$. But $A$ is integrally closed, so $z\in A$, a contradiction.
\end{proof}

\subsection{Dimension of fibers}

The main theorem here is the following:

\begin{thm}
    Let $X,Y$ be irreducible varieties, $\pi:X\to Y$ a dominant map. Suppose $\dim X = a$, $\dim Y = b$. Then:
    \begin{itemize}
        \item For any $y\in \im\pi$, $\dim \pi^{-1}(y)\ge a - b$.
        \item There exists a dense open $U\subset Y$, such that for any $y\in U$, $\dim \pi^{-1}(y) = a - b$.
        \item Given a point $x\in X$, define $e(x)$ to be the maximal $\dim Z$, where $Z$ ranges among the irreducible components of $\pi^{-1}(\pi(x))$ containing $x$. Then $e(x)$ is an upper semi-continuous function: the sets $X_n = \{x\in X: e(x)\ge n\}$ are closed.
    \end{itemize}
\end{thm}

\subsection{Cotangent and tangent spaces}

\begin{prop}
    Let $X$ be a scheme, $f\in \cO_x(X)$, $p \in V(f)$ a closed point, and $\bar{f}$ the image of $f$ in $T_{X,p}\dual$. Then
    \[T_{V(f), p}\dual = T_{X,p}\dual / \langle\bar{f}\rangle\]
\end{prop}

\begin{prop}[Jacobian computes Zariski cotangent space]
    Let $X$ be a finite type $k$-scheme, so that locally it is $\Spec k[x_1,\dots,x_n]/(f_1,\dots,f_r)$. Then for any closed point $p$, $T_{X,p}\dual = \coker J$, where $J: k^r\to k^n$ is the linear map given by the Jacobian matrix
    \[J = \begin{bmatrix}
        \pard{f_1}{x_1}(p) & \cdots & \pard{f_r}{x_1}(p) \\
        \vdots & \ddots & \vdots \\
        \pard{f_1}{x_n}(p) & \cdots & \pard{f_r}{x_n}(p)
    \end{bmatrix}.\]
\end{prop}

\begin{proof}
    Translate $p$ to the origin, and use the previous proposition repeatedly.
\end{proof}

Given a morphism of schemes $f:X\to Y$, mapping $p\in X$ to $q\in Y$, there is a naturally induced ring map $T_{Y,q}\dual \to T_{X,p}\dual$. If $\kappa(p) = \kappa(q)$, the above is a linear map, and we also get a map $T_{X,p} \to T_{Y,q}$.


\subsection{Regularity and smoothness}

\begin{prop}
    For a Noetherian local ring $(A,\fm, k)$, $\dim A \le \dim_k \fm/\fm^2$.
\end{prop}


\begin{proof}
    By Nakayama, a set of generators of $\fm/\fm^2$ over $k$ lifts to a set of generators of $\fm$, which is at least $\height \fm = \dim A$.
\end{proof}


\begin{defn}[regular local ring]
    A \emph{regular local ring} is a Noetherian local ring $(A,\fm,k)$ such that $\dim A = \dim_k \fm/\fm^2$.
\end{defn}

\begin{defn}[regularity]
    A locally Noetherian scheme $X$ is \emph{regular} at $p\in X$ if $\cO_{X,p}$ is a regular local ring. The word \emph{nonsingular} is synonymous. Otherwise, we say $X$ is \emph{singular} at $p$.

    $X$ is \emph{regular} if it is regular at all points, and it is \emph{singular} otherwise.
\end{defn}

\begin{exm}
    Regular local rings of dimension 0 are fields, while regular local rings of dimension 1 are DVRs.
\end{exm}

\begin{prop}[Jacobian criterion]
    Suppose $X = \Spec[x_1,\dots,x_n]/(f_1,\dots,f_r)$ has pure dimension $d$. (As usual, $k = \bar{k}$). Then a $k$-point $p\in X$ is regular iff $\dim \coker J(p) = d$ at $p$. 
\end{prop}

\begin{proof}
    We know $\dim T_{X,p}\dual = \dim \coker J(p) = d$. So it suffices to show that $\dim \cO_{X,p} = d$. But this is clear since $p$ is a closed point and $X$ has pure dimension $d$.
\end{proof}

In fact, for finite-type $k$-schemes, it suffices to check regularity at closed points (this is a hard fact). So for such schemes, regular of pure dim $d$ is equivalent to every irreducible component having dim $d$ and $\dim \coker J(p) = d$ for all $k$-points $p$. But this still requires $k = \bar{k}$. For general $k$, we have an alternate notion of \emph{smoothness over} $\Spec k$:


\begin{defn}
    A scheme $X/k$ is \emph{smooth of dimension $d$ over $k$} if there exists an affine cover by $\Spec k[x_1,\dots,x_n]/(f_1,\dots,f_r)$, for which the Jacobian matrix has $\dim \coker = d$ at all points.
\end{defn}

Remark: $k$-smoothness is equivalent to the Jacobian being corank $d$ everywhere for every affine open cover (and by any choice of generators of the ring corresponding to such an open set).

Regularity/smoothness correspond to the notion of ``smoothness'' in the world of manifolds. So:

\begin{center}
\begin{tabular}{c|c}
schemes & manifolds \\
\hline
Separated & Hausdorff \\
Universally closed & Compact \\
Proper & Compact + Hausdorff \\
Krull dimension & Dimension \\
Zariski (co)tangent space & (Co)tangent space \\
Regular, smooth & Smooth \\
Singular & Singular
\end{tabular}
\end{center}

More or less by definition, for a finite type scheme $X/k$ of pure dim $d$, where $k = \bar{k}$, $X$ is regular at all closed points iff $X$ is smooth over $k$. 


\begin{thm}
Comparison between regularity and smoothness:
\begin{enumerate}[(a)]
    \item If $k$ is perfect, every regular finite type $k$-scheme is smooth over $k$.
    \item Every smooth $k$-scheme is regular (with no hypotheses on perfection).
\end{enumerate}
\end{thm}

\begin{exm}
    Let $k = \FF_p(t)$, $L$ a field extension given by $L = k[x]/(x^p-t)$. Let $X = \Spec L$, then it is regular since $L$ is a field. But it is not smooth of dimension 0 since the derivative of $x^p-t$ vanishes.
\end{exm}

\begin{thm}
    Regular local rings are domains, so regular implies reduced. (In fact, they are UFDs, but this is a much harder fact.)
\end{thm}


\subsection{Bertini's theorem}


\begin{thm}[Bertini]
    Suppose $X$ is a smooth subvariety of $\PP^n_k$. Then there is a dense open $U\subseteq {\PP^n_k}\dual$ such that for any closed point $H\in U$ (corresponding to a hyperplane in $\PP^n_k$), $H$ does not contain any irreducible component of $X$, and $H\cap X$ is $k$-smooth.
\end{thm}



\section{Quasicoherent sheaves}

\subsection{Basic definitions}

\begin{defn}
    Let $X$ be a scheme. A \emph{quasicoherent sheaf} $\cE$ on $X$ is an $\cO_X$-module where there exists an affine cover $\{U_i = \Spec A_i \subseteq X\}$, such that $\cE|_{U_i} \cong \wt{M_i}$ for $A_i$-modules $M_i$.
\end{defn}

\begin{prop}
    Let $X = \Spec A$, $\cE$ a quasicoherent sheaf on $X$, then $\cE\cong \wt{M}$ for $M = \Gamma(X, \cE)$.
\end{prop}

\begin{proof}
    Define $\phi:\wt{M}\to \cE$ on each $D(f)$ by the natural map $M_f\to \Gamma(D(f), \cE)$. Check that these are bijections using the sheaf axioms.
\end{proof}

\begin{defn}
    Let $X$ be Noetherian, then $\cE$ is a \emph{coherent} sheaf if there exists an affine cover $\{U_i = \Spec A_i \subseteq X\}$, such that $\cE|_{U_i} \cong \wt{M_i}$ for \emph{finitely generated} $A_i$-modules $M_i$.
\end{defn}

Warning: locally free of rank $r$ is not an affine local condition.

\begin{prop}
    There is an equivalence of categories $A$-$\Mod \longleftrightarrow \QCoh(\Spec A)$.
\end{prop}

\begin{cor}
    Exact sequences of qcoh sheaves implies exactness on \emph{affine} opens.
\end{cor}

\begin{exm}
    Tensor product of qcoh sheaves: on affine opens, $(\cE_1\otimes \cE_2)(U) \cong \cE_1(U)\otimes \cE_2(U)$. This is the same as the sheafification of the obvious presheaf tensor product.
\end{exm}

\begin{prop}
Let $\cF$ be a finite type qcoh sheaf on $X$, then its rank at a point is upper-semicontinous on $X$.
\end{prop}

\subsection{$f_\ast$ and $f^\ast$}

\begin{prop}
    Let $f:X\to Y$ be qcqs. If $\cE\in \QCoh(X)$, then $f_\ast \cE\in \QCoh(Y)$.
\end{prop}

\begin{defn}
    $f^\ast$ in the affine case: for $f:\Spec A\to \Spec B$, $\cF = \wt{N}$, then $f^\ast \cF = \wt{A\otimes_B N}$.

    In general, cover $f:X\to Y$ by $f|_U:U\to V$ between affine opens. Pull $\cF$ back on each of them, and glue together by universal property. Quasicoherence is obvious.
\end{defn}

\begin{prop}
    $f^\ast \dashv f_\ast$. \qed
\end{prop}

\begin{prop}
    The pullback $f^\ast$ sends coherent sheaves (resp. locally free of rank $r$) on $Y$ to coherent sheaves (resp. locally free of rank $r$) on $X$.
\end{prop}

\begin{prop}[base change map]
    
\end{prop}

\begin{prop}[projection formula]
    Let $\pi:X\to Y$ be qcqs, and $\cF, \cG$ QCoh sheaves on $X,Y$. Then there is a natural map $\pi_\ast \cF\otimes \cG \to \pi_\ast(\cF\otimes \pi^{\ast}\cG)$, which is an isomorphism when either (1) $\cG$ is locally free or (2) $\pi$ is affine.
\end{prop}


\subsection{Invertible sheaves}

\begin{defn}
    An \emph{invertible sheaf} on $X$ is an $\cO_X$-module locally free of rank 1.
\end{defn}

Why are invertible sheaves so important?
\begin{itemize}
    \item Use global sections of an invertible sheaf $\cL$ as replacement for $\Gamma(X,\cO_X)$.
    \item Invertible sheaves are ``dual'' to Weil divisors.
\end{itemize}

Invertible sheaves are preserved under $\otimes$.

\begin{defn}
    The \emph{dual} $\cL^\vee$ of a qcoh sheaf $\cL$ is defined on affine opens by
    \[\Gamma(U, \cL^\vee) := \Hom_{\Gamma(U, \cO_U)}(\Gamma(U, \cL), \Gamma(U, \cO_U)).\]
    This is also a qcoh sheaf. There is a natural pairing
    \[\cL\otimes \cL^\vee \to \cO_X\]
    which is an isomorphism when $\cL$ is invertible.
\end{defn}

\begin{defn}
    The invertible sheaves on $X$ forms an abelian group, called the Picard group $\Pic(X)$. Given $f:X\to Y$, $f^\ast:\Pic(Y)\to \Pic(X)$ is a group homomorphism.
\end{defn}

\begin{exm}
    Consider $X = \PP^1$, then there is a homomorphism $\ZZ\to \Pic(X)$ mapping $a\mapsto \cO(a)$. This is in fact an isomorphism.

    In general, for $X = \PP^n$, then we can similarly define $\cO(a)$, and $\ZZ\to \Pic(X)$ is again an isomorphism.
\end{exm}


\subsection{Weil divisors}

Let $X$ be a Noetherian irreducible regular scheme. (Regular local rings are UFDs, so $X$ will be factorial.)

In topology, for a smooth compact oriented manifold $M$ with dimension $d$, $H^k(M)\cong H_{d-k}(M)$. For schemes and $k=1$, the left side is $\Pic(X)$, and the right side should be ``codimension 1 subsets of $X$''.

Let $p\in X$ be a codimension-1 point. Then $\cO_{X,p}$ is a DVR. For $f\in K(X)$, we may define $v_p(f)$ by the discrete valuation. 

\begin{defn}
    A \emph{Weil divisor} on $X$ is a $\ZZ$-linear finite sum of irreducible codimension-1 subsets $\sum a_Y [Y]$. 

    For nonzero $f\in K(X)$, its principal Weil divisor
    \[\operatorname{div} f = \sum_Y v_Y(f) [Y].\]
    This is a finite sum.
\end{defn}

By Hartogs's lemma \ref{hartogs}, if $f\in K(X)^\times$ such that $v_Y(f)\ge 0$ for all $Y$, then $f\in \cO_X(X)$. If $(f) = 0$, then both $f,f^{-1}\in \cO_X(X)$, so $f\in \cO_X(X)^\times$.

It is not hard to see that the principal divisors on $\PP^1$ all have degree 0. In contrast, all Weil divisors of $\AA^1$ are principal. 

\begin{defn}
    The \emph{class group} of $X$ is $\Cl(X) = \operatorname{Weil}(X)/\operatorname{Prin}(X)$.
\end{defn}

\begin{exm}
    Let $X = \Spec \cO_K$, then $\Cl(X) = \Cl_K$.
\end{exm}

\begin{thm}
\label{Pic=Cl}
    There is a natural isomorphism $\Pic(X)\to \Cl(X)$.
\end{thm}

Given $\cL\in \Pic(X)$, and a nonzero section $s\in \Gamma(X, \cL)$, consider an irred codim 1 subset $Y$ and its generic point $p_Y$. Pick an open neighborhood $U$ of $p_Y$ (equivalently, $U\cap Y \neq \infty$), such that $\cL|_U \cong \cO_U$, so that we can talk about $v_Y(s) = v_Y(s|_U)$. This is easily checked to be well-defined. So we can define
\[\operatorname{div}(s) := \sum_Y v_Y(s)[Y]\in \operatorname{Weil}(X).\]

\begin{exm}
    Consider the line bundle $\cO(1)$ on $\PP^1 = U_0\cup U_1$, and the section $s$ given by $t\in k[t]$ on $U_0$, and by $1\in k[t^{-1}]$ on $U_1$. Then $\operatorname{div}(s) = [0]$ has degree 1.
\end{exm}

\begin{defn}
    A \emph{rational section} of $\cL$ is a section of $\cL$ over some dense open $V\subset X$, modulo equivalence; two rational sections are the same if they agree on some smaller open.
\end{defn}

Given any nonzero rational section $s$ of $\cL$, we may similarly define $\operatorname{div}(s)$: this time, $s$ only represents a section on $U\cap V$, hence a rational function on $U$, to which we may still associate $v_Y(s)$. The set of $(\cL,s)$ with $\otimes$ forms a group. What we will show is:
\[
\begin{tikzcd}
    0 \arrow[r] & K(X)^\times/\cO_X(X)^\times \arrow[r] \arrow[d, "\cong"] & (\cL, s) \arrow[r] \arrow[d, "\cong"] & \Pic(X) \arrow[r] \arrow[d, "\cong"] & 0 \\
    0 \arrow[r] & \operatorname{Prin}(X) \arrow[r] & \operatorname{Weil}(X) \arrow[r] & \Cl(X) \arrow[r] & 0.
\end{tikzcd}
\]

To show theorem \ref{Pic=Cl}, we need to show bijectivity of the middle vertical map. 

Injectivity: Suppose $\operatorname{div}(s) = 0$ is defined on dense open $V$. For any irreducible codimension-1 $D$ with generic point $p$, pick an affine open neighborhood $U = \Spec A$ of $p = [\fp]$, then there is an isomorphism $\cL|_U\xto{\phi} \cO_U$. Then the rational function that $s$ corresponds to belongs to $A_{\fp}$. Since this holds for all height-1 $\fp\subset A$, $s\in A = \cO_U(U)$. So these glue together to form a global section $s\in \cO_X(X)$. We will show that the map $\cO_X\to \cL$ defined by $s$ is an isomorphism. Indeed, locally, after composing with local trivializations $\phi: \cL|_U\to \cO_U$, $\phi s:\cO_U\to \cO_U$ still has no zeros and no poles, so it belongs to $\cO_U(U)^\times$, i.e. is an isomorphism. Since $\phi$ and $\phi s$ are both isomorphisms, so is $s$ locally, hence globally.

Surjectivity: suppose $D$ is a Weil divisor. Define the sheaf $\cO(D)$ as follows: on any $U\subset X$ dense open, define
\[\Gamma(U, \cO(D)) := \{x\in K(X)^\times: \operatorname{div}(x|_U) + D|_U \ge 0 \}.\]
Define a rational section $s_D$ of $\cO(D)$ to be $1\in \Gamma(U, \cO(D)) \subseteq K(X)^\times$, where $U$ is the complement of $\Supp D$. We claim that $(\cO(D), s_D)$ is the desired preimage.
\begin{itemize}
    \item To show that $\cO(D)$ is a line bundle: first, we find an open cover of $X$, where on each open set $U$, $D|_U$ is principal. Suppose $S = \Supp D$, then $X\backslash S$ is such an open. We then construct such an open neighborhood of each $p\in S$. Consider any irreducible divisor $Y$ where $p\in Y$. Since $X$ is factorial, every stalk $\cO_{X,p}$ is a UFD. Since any open neighborhood of $p$ contains the generic point $\eta_Y$ of $Y$, there is a natural injection $\cO_{X,p} \to \cO_{X,\eta_Y}$. For each affine neighborhood $U = \Spec A$ of $p = [\fp]$ and $\eta_Y = [\fq]$, this is the natural localization $A_\fp \to A_\fq$. The preimage of $\fq A_\fq$ under this map is a height 1 prime in $\cO_{X,p}$, a UFD, so it is principal, say generated by $f \in \cO_{X,p} \subseteq K(X)$. WLOG we may choose $f\in A$, then $f$ has no poles in $U$, and if it has a zero at a divisor $Y'$ containing $p$, say with generic point $\eta_{Y'}$, then the preimage of $\fm_{\eta_{Y'}}$ in $\cO_{X,p}\to \cO_{X, \eta_{Y'}}$ is another height 1 prime $\fr$ containing $f$. Then $\fq = (f) \subseteq \fr$, which implies $\fq = \fr$. This shows that $f$ only has a zero of order 1 at $\eta_Y$. 

Now, let
\[U' = U\cap (X\backslash \bigcup_{\substack{Z \text{ irred codim } 1 \\ p\notin Z}} Z )\]
which contains $p$, so it is a dense open. On $U'$, $\div(f) = [Y]$.

Now, suppose $p\in Y_1,\dots,Y_n$ where $D  = \sum n_i[Y_i]$. Choose $f_i$ so that on an open neighborhood of $p$, $\div(f_i) = [Y_i]$. Then on their intersections, which is an open neighborhood $U$ of $p$, 
\[\div|_U (\prod f_i^{n_i}) = \sum n_i[Y_i] = D|_U.\]
This shows that we can find an open cover of $X$ where $D$ is locally principal. Now, fix one open $U$ in the cover, where $D = \div|_U(s)$. For each affine open $V\subseteq U$, there is an isomorphism $\Gamma(V,\cO(D)) \cong \cO_U(V)$ by sending $t\mapsto st$. This is functorial, so they glue together to form $\cO(D)|_U \cong\cO_U$. This shows that $\cO(D)$ is locally free of rank 1.
\item To show that $\cO(\div(s)) \cong \cL$ for $(\cL, s)$: We claim that any open $U$ that trivializes $\cL$ satisfies $\cO(\operatorname{div}(s))|_U \cong \cO_U$. Suppose $\cL|_U \cong \cO_U$ takes $s$ to a rational function on $U$, which we also denote by $s$. Then for any affine open $V = \Spec A\subseteq U$,
    \begin{align*}
    \Gamma(V, \cO(\operatorname{div}(s))) &= \{t\in K^\times: \div_V(t) + \div_V(s) \ge 0\} \\
    &= \{t\in K^\times: \div_V(st) \ge 0\} \\
    &= \{t\in K^\times: st\in A\} \\
    &= s^{-1}A,
    \end{align*}
    which is isomorphic to $\cO_V(V) = A$ as $A$-modules by sending $t$ to $st$. Furthermore, this isomorphism is clearly functorial as $V$ ranges among affine open subsets of $U$, so this induces an isomorphism of sheaves $\cO(\div(s))|_U \cong \cO_U$. Composing this with $\cO_U \cong \cL|_U$, for sections over $U$, this is the bijection sending $t$ to $st$. Now, the set of $U$s (open sets trivializing $\cL$) forms a base of the Zariski topology on $X$, and the isomorphism $\Gamma(U, \cO(\operatorname{div}(s))) \to \Gamma(U, \cL)$ is clearly functorial, so this defines an isomorphism of sheaves $\cO(\operatorname{div}(s))\to \cL$.

    Suppose the canonical section ``$1$'' is a section of $\cO(\div(s))$ over $U$. Its image is a section which, on each $V$ in part (a) (i.e. affine opens that trivialize $\cL$), agrees with $s|_V$. So its image is $s$ by the sheaf axiom.
\end{itemize}


\begin{cor}
    $\Pic(\PP_k^n) \cong \ZZ$.
\end{cor}

\begin{proof}
    There is an exact sequence $0\to \ZZ\to \operatorname{Weil}(\PP^n) \to \operatorname{Weil}(\AA^n)\to 0$, where $\AA^n = U_0$ is the complement of a hyperplane, and $\ZZ$ is freely generated by that hyperplane. This induces an exact $0\to \ZZ\to \Cl(\PP^n)\to \Cl(\AA^n)\to 0$. But $\Cl(\AA^n) = 0$.
\end{proof}


\subsection{Quasicoherent sheaf of graded module}




\subsection{Sections of line bundles}

One theme we see here is that global sections of line bundles on $X$ serve a similar purpose as functions on $X$.

\begin{defn}
    Let $X$ be a scheme, $\cL\in \Pic(X)$, $s \in \Gamma(X, \cL)$, $p\in X$. The \emph{value} of $s$ at $p$, $s(p)$, is the image of $s$ in the \emph{fiber} $\cL|_p := \cL_p/\fm_p\cL_p = \cL_p\otimes_{\cO_{X,p}} \kappa(p)$, which is naturally a 1-dim $\kappa(p)$-vector space. (In general this makes sense for any quasicoherent sheaf.)

    For $s\in \Gamma(X,\cL)$, the locus of points where $s$ does not vanish is denoted by $D(s)$. This is open.
\end{defn}

A map $X\to \AA_k^n$ is equivalent to choosing $n$ global sections of $\cO_X$. The analogous fact is:

\begin{prop}
    Let $X$ be an $A$-scheme, for a ring $A$. The following data are equivalent:
    \begin{itemize}
        \item A map $f: X\to \PP_A^n$;
        \item A line bundle $\cL\in \Pic(X)$, and sections $s_0,\dots,s_n\in \Gamma(X,\cL)$, such that $X = \bigcup D(s_i)$.
    \end{itemize}
\end{prop}

When $A=k$, on $k$-points, this is the map $X(k)\to \PP^n(k)$ given by $p \mapsto [s_0(p),\dots,s_n(p)]$.

\begin{proof}
    $(\given)$: Recall that affine schemes $U_i = \Spec A[x_{0/i},\dots,x_{n/i}]$ cover $\PP_A^n$. Given $(\cL, s_0,\dots,s_n)$, we define maps $D(s_i)\to U_i$ by specifying a ring homomorphism $A[x_{0/i},\dots,x_{n/i}] \to \Gamma(D(s_i), \cO_X)$. Because $s_j\in \Gamma(D(s_i), \cL)$ and $s_i^{-1}\in \Gamma(D(s_i), \cL^\vee)$, there is an element $s_js_i^{-1}\in \Gamma(D(s_i), \cO_X)$, which we map $x_{j/i}$ to. To check that these glue together, it suffices to show that
    \[
    \begin{tikzcd}
    A[x_{0/i},\dots,x_{n/i}]_{x_{j/i}} \arrow[r] \arrow[d] & \Gamma(D(s_i)\cap D(s_j), \cO_X)\arrow[d, equal] \\
    A[x_{0/j},\dots,x_{n/j}]_{x_{i/j}} \arrow[r] & \Gamma(D(s_i)\cap D(s_j), \cO_X)
    \end{tikzcd}
    \]
    This is true because $x_{k/i}\mapsto x_{k/j}x_{i/j}^{-1}\mapsto s_ks_j^{-1}(s_is_j^{-1})^{-1} = s_ks_i^{-1}$.

    ($\implies$): Let $\cL = f^\ast \cO_{\PP^n_A}(1)$, and $s_i = f^\ast x_i$ where $x_i\in A[x_0,\dots,x_n]_{\deg 0} \cong \Gamma(\PP^n_A,\cO(1))$. Then $D(s_i) = D(f^\ast x_i) = f^{-1}(D(x_i))$, so $X = \bigcup D(s_i)$.
\end{proof}


\begin{defn}
    Let $\cF$ be a finite type quasicoherent sheaf on $X$. Say $\cF$ is \emph{globally generated} if for any point $p\in X$, there exists a set of $s_i\in \Gamma(X, \cF)$ such that $s_i(p)$ generate $\cL|_p$ over $\kappa(p)$. Equivalently (Nakayama), there is a surjection of sheaves $\cO^{\oplus I} \onto \cL$ where $I$ is an index set.
\end{defn}


\begin{defn}
    Let $X$ be a $k$-scheme. A finite dimensional $k$-subspace $W\subset \Gamma(X, \cL)$ is called a \emph{linear series}. It is a \emph{complete linear series} if $W\cong \Gamma(X,\cL)$ and is often written $|\cL|$. Given a linear series $W$, the \emph{base locus} is the set of points where all of $W$ vanish. Then if $W$ globally generates $\cL$, we get a map $X\to \PP^{\dim W - 1}_k$. We say $\cL$ is \emph{basepoint free} if is is globally generated.
\end{defn}

\begin{exm}
    The Veronese embedding $\PP^n\to \PP^{\binom{n+1}{d}-1}$ can be seen as the map corresponding to picking the degree-$d$ monomials in $\Gamma(\PP^n, \cO(d)) \cong k[x_0,\dots,x_n]_{\deg d}$, which globally generate $\cO(d)$.
\end{exm}

\begin{exm}
    All maps $\PP^m\to \PP^n$ are be characterized by choosing a $d$ and $n+1$ degree-$d$ homogeneous polynomials in $k[x_0,\dots,x_m]$ with no common zeros. 
\end{exm}

\begin{thm}[Serre's Theorem A]
    Suppose $S_\bullet$ is generated in degree 1, and finitely generated over $A = S_0$. Then for any finite type quasicoherent sheaf $\cF$ on $\Proj S$, there exists $n_0$ such that for all $n>n_0$, $\cF\otimes \cO(n) = \cF(n)$ is finitely globally generated. 
\end{thm}

\begin{thm}[Curve to projective extension]
    Let $C/k$ be a smooth curve (i.e. pure dimension one), $Y$ projective over $k$, $p\in C$ a closed point. Then any map $f:C-p \to Y$ (uniquely) extends to $C$.
\end{thm}

\begin{proof}
    Uniqueness follows from the reduced-to-separated theorem (regular local rings are reduced). To show existence, we make several reductions:
    \begin{itemize}
        \item Assume $C$ is affine. This is because we can choose an affine neighborhood of $p$, and if the function is extended to that neighborhood, then it glues with $f$ to form an extension on the whole of $C$.
        
        \item Assume $Y = \PP^n_k$. This is because: suppose we have proven the theorem for $Y = \PP^n_k$. Then we may extend $f:C-p\to Y \to \PP^n_k$ to a map $f:C\to \PP^n_k$. Take affine open neighborhood $\Spec A\subseteq C$ of $p$ such that its image lands in $\AA_k^n$. Then functions vanishing on $Y\cap \AA_k^n$ pull back to functions vanishing at generic points of the irreducible components of $C$, hence they vanish on the entire $C$ (by reducedness), so $\Spec A\to \AA_k^n$ factors through $Y\cap \AA_k^n$.
    \end{itemize}

    Now, because $C$ is regular and $p$ is a closed point, $\cO_{C, p}$ is a DVR, so we can pick a uniformizer $\pi$. Pick a neighborhood $V$ of $p$, such that $\pi\in \Gamma(V, \cO_C)$. Shrink $V$ so that $V=\Spec A$ is affine, $\pi$ is nonvanishing on $V-p$, and the line bundle $\cL$ induced by $f$ is trivialized on $V-p$. Suppose $f|_{V-p} = [f_0:f_1:\dots:f_n]$, $f_i\in A_\pi$ (where $V-p = \Spec A_\pi$). Let $m = \min v_\pi(f_i)$, then $t^{-m}g_0, \dots, t^{-m}g_n \in A$ are $(n+1)$ functions with no common zeros, which gives a map $V\to \PP^n_k$ extending $f$. This glues with $f$ to produce an extension on the whole $C$. 
\end{proof}


\subsection{Ampleness}

Ample line bundles are ``positive'' in certain senses, and ampleness roughly means ``having many sections''.

\begin{defn}
    Let $X$ be a proper $A$-scheme. An invertible sheaf $\cL$ on $X$ is \emph{very ample} if there exist $n+1$ sections that globally generate $\cL$ such that the induced map to $\PP^n_A$ is a closed embedding. 
    
    Equivalently, $X \cong \Proj S_\bullet$, where $S_0=A$ and $S$ is generated in degree 1. Then $\cL$ is \emph{very ample} if $\cL = \cO(1)$.
\end{defn}


\begin{prop}
    If $\cL$ is very ample, then so are $\cL^{\otimes k}$ $(k\ge 1)$.
\end{prop}

\begin{proof}
    Suppose $\cL = f^\ast \cO_{\PP^n}(1)$ for $f:X\to \PP^n$. Let $g: \PP^n\to \PP^{N}$, $N=\binom{n-1}{k}+1$ be the Veronese embedding, so that $g^\ast\cO_{\PP^N}(1) = \cO_{\PP^n}(k)$. Then $(g\circ f)^\ast \cO_{\PP^N}(1) = f^\ast \cO_{\PP^n}(k) = f^\ast \cO_{\PP^n}(1)^{\otimes k} = \cL^{\otimes_k}$, so $\cL^{\otimes_k}$ is also pulled back from $\cO(1)$ of a projective space, and $\cL\into \PP^n\into \PP^N$ is a closed embedding.
\end{proof}

\begin{lem}[extending sections]
\label{extendingSections}
    Let $X$ be qcqs, $\cL$ a invertible sheaf, $s\in \Gamma(X,\cL)$, $\cF$ a quasicoherent sheaf. Then for any $t\in \Gamma(D(s), F)$, there exists $k\ge 0$, such that
    \[t\otimes s^{\otimes k} \in \Gamma(D(s), F\otimes \cL^{\otimes k})\]
    lies in the image of $\Gamma(X, F\otimes \cL^{\otimes k})$. \qed
\end{lem}

\begin{defn}[ample line bundles]
    Let $X$ be a proper $A$-scheme. An invertible sheaf $\cL$ on $X$ is \emph{ample} if any of the following equivalent conditions hold:
    \begin{itemize}
        \item[(a)] $\cL^{\otimes k}$ is very ample for some $k\ge 1$.
        \item[(a')] $\cL^{\otimes k}$ is very ample for all $k \gg 0$.
        \item[(b)] For all finite type quasicoherent sheaves $\cF$, $\cF\otimes \cL^{\otimes k}$ is globally generated for some $k\ge 1$.
        \item[(b')] For all finite type quasicoherent sheaves $\cF$, $\cF\otimes \cL^{\otimes k}$ is globally generated for all $k\gg 0$.
        \item[(c)] As $f$ varies over global sections of $\cL^{\otimes k}$ (over all $k\ge 1$), the open sets $D(f)$ form a base of the topology on $X$.
        \item[(c')] In the above, the affine ones already form a base.
        \item[(c'')] In the above, the affine ones cover $X$.
    \end{itemize}
\end{defn}

\begin{proof}
    Clearly, (a') $\implies$ (a), (b') $\implies$ (b), and (c') $\implies$ (c), (c'').

    (c) $\implies$ (c'): Consider $p\in X$ and any open neighborhood $U$ of $p$. WLOG $U$ is affine and trivializes $\cL$. Then there exists $f\in \Gamma(\cL^{\otimes k})$ such that $D(f)\subseteq U$. This $D(f)$ is affine.

    (a) $\implies$ (c): Suppose $\cL^{\otimes k}$ is very ample. Then there is a closed immersion $i:X\into \PP^n$ and $i^\ast(\cO_{\PP^n}(1)) = \cL^{\otimes k}$. Let $Z$ be closed in $X$, and $p$ a point in the complement of $Z$. We wish to find a neighborhood $D(f)$ of $p$ disjoint from $Z$. We can make $Z$ into a closed subscheme. Then $Z\into X\into \PP^n$ is a closed subscheme, so $Z \cong \Proj S_\bullet$ where $S = A[x_0,\dots,x_n]/I$ for some homogeneous ideal $I$. Pick a homogeneous element $s\in I$, say of degree $d$, so that $s\in \Gamma(\PP^n, \cO(d))$. Then $f:= i^\ast s \in \Gamma(X, \cL^{\otimes kd})$ vanishes on $Z$, and does not vanish at $p$, which is what we want.

    (b) $\implies$ (c): Similar to above, we wish to find a neighborhood $D(f)$ of $p$ disjoint from $Z$. Pick $\cF = \cI_Z$ to be the ideal sheaf of $Z$. Then since $\cI_Z\otimes \cL^{\otimes k}$ is globally generated for some $k$, there exists $s\in \Gamma(X, \cI_Z\otimes \cL^{\otimes k})$ such that $s(p)\neq 0$. Since $0\to \cI_Z\to \cO_X$ is an injection, tensoring with the locally free $\cL^{\otimes k}$ gives an injection $0\to \cI_Z\otimes\cL^{\otimes k} \to \cL^{\otimes k}$. Let $f\in \Gamma(X, \cL^{\otimes k})$ denote the image of $s$, and we claim that this works. For any $U$ trivializing $\cL$, $f|_U$ is the image of $s|_U$ under $0\to \cI_Z|_U \to \cO_U$, hence vanishes on $Z\cap U$. So $s$ vanishes on $Z$. Since $p\notin Z$, there exists a neighborhood $U$ of $p$ trivializing $\cL$ where $\cI_Z|_U\cong \cO_U$. So since $s(p) \neq 0$, $f(p)\neq 0$ as well, as desired.

    (c'') $\implies$ (b'): Let $X = \bigcup D(f_i)$ be the union of finitely many affine opens, where $f_i\in \Gamma(X, \cL^{\otimes a})$. (By scaling, $a$ can be chosen not to depend on $i$.) On each $D(f_i) = \Spec A_i$, $\cF$ is just some finitely generated $A_i$-module, so it is globally generated by $s_{ij}\in \Gamma(D(f_i), \cF)$. Extend these to $\wt{s}_{ij} \in \Gamma(X, \cF\otimes \cL^{\otimes k})$ where $k$ can be chosen to not depend on $i,j$. Then $\wt{s}_{ij}$ generates $\cF\otimes \cL^{\otimes k}$ on each stalk, hence globally generates $\cF\otimes \cL^{\otimes k}$. In fact, this shows that $\cF\otimes \cL^{\otimes (k+na)}$ is globally generated for all $n\ge 0$. Arguing similarly for all residues mod $a$ implies the desired statement.

    (c'') $\implies$ (a): Let $X = \bigcup D(f_i)$ be the union of finitely many affine opens, where $f_i\in \Gamma(X, \cL^{\otimes a})$ and $D(f_i) = \Spec A_i = A[a_{ij}]/I$, where $a_{ij}\in \Gamma(D(f_i),\cO_X)$. Extend these to $\wt{a}_{ij} \in \Gamma(X, \cL^{\otimes r})$. We may choose $r$ so that $f_i, \wt{a}_{ij}$ are all global sections of $\cL^{\otimes r}$. We claim that these give a closed embedding to a projective space. Since the linear series generated by $f_i$ is already basepoint-free, this gives us a map $X\to \PP^N_A$. We index the coordinates of $\PP^N_A$ correspondingly with $i$ and $ij$. Then it is clear that the ring homomorphisms $A[x_i,x_{ij}]/(x_k-1) \to \Gamma(D(f_k),\cL^{\otimes k}) = A_k$ are surjective. This shows that $X\to \PP^N_A$ is a closed immersion.

    (a), (b) $\implies$ (a'): very ample tensor basepoint-free is very ample.
\end{proof}


There is another, more geometric, interpretation of ampleness.

\begin{prop}[separating points and tangent vectors]

Let $X$ be proper over $k = \bar{k}$, $\cL$ an invertible sheaf, and $V$ a basepoint-free linear series giving a map $f: X\to \PP^n$. If:
\begin{itemize}
    \item for any two distinct $k$-points $x,y \in X$, there exists $s\in V$ with $s(x) = 0$, $s(y)\neq 0$;
    \item for any $k$-point $x$ and nonzero tangent vector $\theta: \Spec \kappa(x)[\eps] \to X$, there exists a section $s\in V$ vanishing at $x$ such that the pullback of $s$ along $\theta$ is nonzero,
\end{itemize}
then $\cL$ is very ample and $f$ is a closed immersion.
\end{prop}



\subsection{Projective morphism}

Recall that a morphism $X = \Proj S_\bullet \to \Spec A$, where $S_0 = A$ and $S_\bullet$ is finitely generated in degree 1, is called \emph{projective}. We wish to define a notion of projectiveness over any base scheme.


\begin{lem}
    Given a scheme $Y$, and the following data:
    \begin{itemize}
        \item for each affine open $U\subset Y$, a scheme $Z_U\to U$;
        \item for $V\subseteq U$, a map $\rho_{UV}: Z_V\subseteq Z_U$ such that $Z_V \cong Z_U\times_U V$;
        \item for $W\subset V\subset U$, $\rho_{UW} = \rho_{UV}\circ \rho_{VW}$,
    \end{itemize}
    then there exists a scheme $\pi: Z\to Y$ such that $\pi^{-1}(U) = Z_U$.
\end{lem}

Given a scheme $Y$, and a \emph{graded quasicoherent sheaf of $\cO_Y$-algebras} $\sS_\bullet = \bigoplus_{n\ge 0} \sS_n$ such that
\begin{itemize}
    \item $\sS_0 = \cO_Y$;
    \item $\operatorname{Sym}^{\bullet} \sS_1 := \bigoplus \operatorname{Sym}^k \sS_1 \to \sS_\bullet$ is surjective,
\end{itemize}
we can define $\Proj \sS_\bullet\to Y$ using the above gluing lemma. Also, the line bundles on each affine open glue together over $\Proj \sS_\bullet$.


\begin{exm}
    Let $\cE$ be locally free of rank $r$, then define $\sS_\bullet = \operatorname{Sym}^\bullet \cE$. Then $\Proj \sS_\bullet$ is a \emph{projective bundle} that locally looks like $U\times \PP^{r-1}$ on affine opens trivializing $\cE$.
\end{exm}

\begin{defn}
    A morphism $\pi:X\to Y$ is \emph{projective} if $X \cong \Proj \sS_\bullet$ for some $\sS_\bullet$ as above. 
\end{defn}

\begin{Rem}
Hartshorne defines projective morphisms as $X\into Y\times \PP^n\to Y$, where the first map is a closed immersion. 
\end{Rem}


\subsection{Curves}

\begin{thm}
\label{CurvesCategoriesEquiv}
The following categories are equivalent:
\begin{enumerate}
\item Integral regular projective 1-dimensional $k$-varieties, and surjective $k$-morphisms.
\item Integral regular projective 1-dimensional $k$-varieties, and dominant $k$-morphisms.
\item Integral regular projective 1-dimensional $k$-varieties, and dominant rational maps.
\item Integral 1-dimensional $k$-varieties, and dominant rational maps.
\item The opposite category of finitely generated fields of transcendence degree 1 over $k$, and $k$-morphisms.
\end{enumerate}
\end{thm}


\section{Cohomology}

\subsection{Properties}

Let $X\to \Spec A$ be separated. (This isn't absolutely necessary.) We will define for each $k\ge 0$ a functor $H^k(X,-): \QCoh(X) \to A$-$\Mod$, such that:
\begin{itemize}
    \item $H^0(X,-) = \Gamma(X, -)$;
    \item Short exact sequences of QCoh sheaves gets sent to long exact sequences of $A$-modules;
    \item Let $\pi:X\to Y$ be a morphism of schemes, and $\cF\in \QCoh(X)$. Then there exist $\alpha_k: H^k(Y, \pi_\ast \cF) \to H^k(X, \cF)$, which are isomorphisms when $\pi$ is affine, that extend $\alpha^0: \Gamma(Y, \pi_\ast \cF) \to \Gamma(X, \cF)$. This gives, for $G\in \QCoh(Y)$, a composition 
    \[H^k(Y,G) \to H^k(Y, \pi_\ast \pi^\ast G) \to H^k(X, \pi^\ast G).\]

    \item If $X$ is covered by $n$ affine open charts, then $H^k(X,-) = 0$ if $k\ge n$. In particular, if $X$ is affine, then $H^1(X,-) = 0$ (which we recall from earlier).
    \item $H^k(X, \bigoplus \cF_j) = \bigoplus H^k(X,\cF_j)$.
\end{itemize}

A preview of what's to come:


\begin{thm}[cohomologies of $\cO(m)$]
    We have:
    \begin{itemize}
        \item $H^0(\PP^n_A, \cO(m)) = A^{\binom{n+m}{m}}$ if $m\ge 0$, and 0 if $m\le 0$;
        \item $H^n(\PP^n_A, \cO(m)) = A^{\binom{-m-1}{-m-1-n}}$ if $-m-1\ge n$, and 0 otherwise;
        \item All other cohomologies vanish.
    \end{itemize}
\end{thm}

\begin{thm}
    Let $X$ be projective over $A$, and $\cF$ a coherent sheaf. Then $\Gamma(X,\cF)$ is a finitely generated $A$-module.
\end{thm}

\begin{proof}
    We will show in fact that $H^k(X, \cF)$ are all finitely generated over $A$.

    Let $i: X\into \PP^n_A$ be a closed embedding, then $H^k(X, \cF) = H^k(\PP^n_A, i_\ast \cF)$. So we may WLOG assume $X = \PP^n_A$. Use descending induction on $k$. In the base cases $k\ge n+1$, the cohomologies all vanish.

    Recall that there exists a surjection $\cO(m)^{\oplus a} \to \cF\to 0$. Let $K$ be the kernel, and unwind to a long exact sequence. Suppose we want to show $H^n(X, \cF)$ is finitely generated. A segment of the long exact sequence reads:
    \[\dots \to H^n(\cO(m)^{\oplus a}) \to H^n(\cF) \to 0\]
    and since $H^n$ commutes with direct sums and by the explicit calculations, $H^n(\cO(m)^{\oplus a})$ is finitely generate, so $H^n(\cF)$ is as well. Suppose now we want to show this for $n-1$. Then
    \[\dots \to H^{n-1}(\cO(m)^{\oplus a}) \to H^{n-1}(\cF) \to H^n(K) \to \dots,\]
    and since both the left and right are finitely generated, so is the middle.
\end{proof}


\subsection{Definition}

Let $\sU = \{U_i\}_{i=1}^n$ be an affine cover of $X$, and let $\cF$ be a quasicoherent sheaf. Define the \v{C}ech complex
\[C^k_{\sU}(X, \cF) := \prod_{\substack{|I| = k+1 \\ I = \{i_0,\dots,i_k\} \subseteq [n]}} \Gamma(U_{i_0}\cap\dots\cap U_{i_k}, \cF)\]
with obvious differentials
\[d: C^k_{\sU}(X, \cF) \to C^{k+1}_{\sU}(X, \cF)\]
by alternatingly summing over the restriction maps. A short exact sequence of QCoh sheaves induces a short exact sequence of \v{C}ech complexes (this is where it is crucial that we're working with QCoh sheaves), which then induces the long exact sequence. It is then obvious that if $X$ is covered by $n$ affine open charts, then $H^k$ vanishes for $k\ge n$.

\begin{thm}
    Let $X$ be quasicompact and separated. The \v{C}ech cohomology is independent of the (finite) affine cover $\sU$.
\end{thm}


\begin{proof}
    The proof proceeds in several steps.

    Step 1: it suffices to show that the \v{C}ech complexes of $\{U_i\}_{i=1}^n$ and $\{U_i\}_{i=1}^{n+1}$ are quasi-isomorphic.

    Step 2: The kernel of the surjection $C_{\{U_i\}_{i=1}^{n+1}}(X,\cF) \onto C_{\{U_i\}_{i=1}^{n}}(X,\cF)$ is the chain complex whose $k$-th term is the product over all $I\subseteq[n+1]$ containing $n+1$, $|I|= k +1$. The goal is then to show that this is exact. But this is exactly the augmented \v{C}ech complex $C_{\{U_i\cap U_{n+1}\}_{i=1}^n}(U_{n+1},\cF)$. So it suffices to show that for affine schemes $X$, the \v{C}ech cohomology vanishes except at degree 0.

    Step 3: Suppose that $X$ is affine and $\{U_i\}$ cover $X$, and suppose $U_{n}$ already is $X$. Then the augmented \v{C}ech complex of $X$ surjects onto the augmented \v{C}ech complex of $U_1\cap\dots \cap U_{n-1}$, and the kernel is the \v{C}ech complex for $U_n$, which is just the \v{C}ech complex for $X$ shifted by one. So by the cohomology long exact sequence, the cohomology of the middle row vanishes.

    Step 4: In general, suppose $X$ is affine and $\{U_i\}$ cover $X$. Then there is an affine cover $D(f_j)$ where each $D(f_j)$ lies inside some $\{U_i\}$. Then the \v{C}ech complex localized at each $f_i$ is exact, so the original complex is exact as well. 
\end{proof}

More consequences of cohomology:

\begin{prop}
    Pushforwards of coherent sheaves by projective morphisms (of locally Noetherian schemes) is coherent.
\end{prop}

\begin{prop}
\label{proj+affine=finite}
    Suppose $Y$ is locally Noetherian. Then a morphism $\pi:X\to Y$ is projective and affine iff it is finite. 
\end{prop}

\begin{prop}
\label{proj+finitefiber=finite}
    Suppose $Y$ is Noetherian. Then a morphism $\pi:X\to Y$ is projective and has finite fiber iff it is finite. 
\end{prop}

\begin{prop}[fiber dimension of projective morphism is upper-semicontinuous]
\label{projFiberDimUppersemicontinuous}
    Let $\pi:X\to Y$ be projective, and let $Y$ be locallly Noetherian. Then the set $\{q\in Y: \dim \pi^{-1}(q) \ge k\}$ is Zariski-closed.
\end{prop}

\begin{thm}[Serre vanishing]
    Let $\cF$ be coherent on a projective $X/A$. Then for all $m\gg 0$, $H^i(X, \cF(m)) = 0$ for all $i>0$. 
\end{thm}

\subsection{Euler characteristic, Hilbert functions}

We work with a projective $k$-scheme $X$, and $\cF\in \Coh(X)$. The \emph{Euler characteristic} 
\[\chi(\cF) := \sum_{i \ge 0} \dim_k H^i(X, \cF).\]
For example, for $X=\PP^n$, $\cF = \cO(m)$, then
\[\chi(\cO(m)) = \frac{1}{n!}(m+1)(m+2)\dots (m+n)\]
for \emph{all} $m,n$. A general heuristic is that $\chi$ is better behaved than individual cohomology groups, and we study the individual cohomologies by proving vanishing theorems.

\begin{prop}
    Let $0\to \cF\to \cG\to \cH\to 0$ be an exact sequence of coherent sheaves. Then $\chi(\cG) = \chi(\cF) + \chi(\cH)$.
\end{prop}

Let $i:X\into \PP_k^N$ be a fixed embedding. Then by definition, $\cO_X(1) = i^\ast \cO_{\PP^N}(1)$. 

\begin{defn}
    The \emph{Hilbert function} of $\cF$ is defined by
    \[h_{\cF}(m) = \dim_k H^0(X, \cF(m)) =  \dim_k H^0(X, \cF\otimes \cO_X(1)^{\otimes m}).\]
\end{defn}

\begin{exm}
    Let $\cF = \cI_X$ be the ideal sheaf of $X$. Then we have an exact sequence
    \[0\to \cI_X \to \cO_{\PP^N} \to i_\ast \cO_X \to 0.\]
    Tensoring with $\cO_{\PP^N}(m)$, we get
    \[0\to \cI_X(m) \to \cO_{\PP^N}(m) \to (i_\ast \cO_X)(m) \to 0.\]
    By the projection formula, $(i_\ast \cO_X)(m) = i_\ast(\cO_X(m))$. Taking $\Gamma(\PP^n,-)$ gives us
    \[0\to H^0(\PP^N, \cI_X(m)) \to H^0(\PP^N, \cO(m)) \to H^0(X, \cO_X(m))\]
    where the last map is just restriction to $X$.
    So in other words, $H^0(\PP^N, \cI_X(m))$ should be interpreted as the degree-$m$ homogeneous polynomials in $x_0,\dots,x_N$ that vanish on $X$. In particular, it depends on the way $X$ is embedded into $\PP^N$.
\end{exm}



\begin{thm}
    The function $t\mapsto \chi(\cF(t))$ is a polynomial in $\QQ[t]$ whose degree is $\dim \Supp \cF$.
\end{thm}

Hence, by Serre vanishing, for $m\gg 0$, the Hilbert function is a polynomial, called the \emph{Hilbert polynomial} $p_{\cF}(m)$. In particular, the Hilbert polynomial $p_X(m)$ of $\cO_X$ is a polynomial of degree $\dim X$.


\begin{proof}
    (TODO)
\end{proof}

\begin{exm}
    Let $X = V(f)$ be a degree-$d$ hypersurface. Then 
    \[p_X(m) = p_{\PP^n}(m) - p_{\PP^n}(m-d) = \frac{1}{n!}((m+1)\dots(m+n) - (m+1-d)\dots(m+n-d)).\]
    In particular, its leading term is $\frac{d}{n!}m^{n-1}$.
\end{exm}

\begin{Rem}
In general, for a closed subscheme $X\into \PP^n$, its \emph{degree} is defined as the positive integer $a$ such that the leading coefficient of $p_X(t)$ is $\frac{a}{n!}$. Another piece of information is the constant term $p_X(0) = \chi(X, \cO_X)$. This is one minus the arithmetic genus.
\end{Rem}




\subsection{Riemman-Roch for line bundles on a regular projective curve}

Let $C$ be a regular projective curve over $k$ (not necessarily alg. closed), $D$ a Weil divisor. Recall that if $D = \sum a_p [p]$, then $\deg D = \sum a_p \deg p$.

\begin{thm}
    We have $\deg D = \chi(C, \cO(D)) - \chi(C, \cO_C)$.
\end{thm}

\begin{defn}
    For a line bundle $\cL$ on $C$, define its degree $\deg \cL = \chi(C, \cO(D)) - \chi(C, \cO_C)$. 
\end{defn}


\begin{defn}
    For a scheme $X$, the \emph{arithmetic genus} is defined to be $g = 1 - \chi(X,\cO_X)$. When $X$ is a integral projective curve over an algebraically closed field, it is true that $h^0(X,\cO_X) = 1$, so $h^1(X,\cO_X) = g$. 
\end{defn}



\subsection{Remarks on sheaf cohomology}

\begin{thm}[K\"unneth formula]
    Let $X,Y$ projective schemes over $k$, $\cF\in \QCoh(X)$, $\cG\in \QCoh(Y)$. Define $\cF\boxtimes \cG = \pi_1^\ast \cF\otimes \pi_2^\ast \cG$, where $\pi_1,\pi_2$ are projection maps from $X\times Y$. Then
    \[H^m(X\times Y, \cF\boxtimes \cG) = \bigoplus_{p+q=m} H^p(X, \cF)\otimes_k H^q(Y, \cG).\]
\end{thm}

\begin{thm}[cup product]
    There is a 
\end{thm}




\section{Curves of small genus}

We use the machinery of cohomology of line bundles to study curves of small genus. 

\begin{defn}
In this section, a \emph{curve} $C$ is a projective, geometrically integral, geometrically regular, dimension-1 scheme over a field $k$.
\end{defn}


\subsection{Preliminary tools}

\begin{defn}[degree of a finite morphism at a point]
Let $\pi:X\to Y$ be a finite morphism. Then $\pi_\ast \cO_X$ is a finite type quasicoherent sheaf, so we may consider the rank $d$ of $f_\ast\cO_X$ at a point $y\in Y$. We call $d$ the \emph{degree} of $\pi$ at $y$. Equivalently, the degree is $d = \dim_{\kappa(y)} \Gamma(\cO_{\pi^{-1}(y)}, \pi^{-1}(y))$ (just unwind the definition).
\end{defn}

\begin{Rem}
The degree of $\pi$ is upper-semicontinuous on $Y$.
\end{Rem}

\begin{lem}
Let $\pi:X\to Y$ be a finite morphism of Noetherian schemes, whose degree at every point of $Y$ is either 0 or 1. Then $\pi$ is a closed embedding.
\end{lem}


\begin{thm}[separating points and tangent vectors]
Let $k$ be algebraically closed. Let $\pi: X\to Y$ be a projective morphism of finite-type $k$-schemes that is injective on closed points and injective on tangent vectors at closed points. Then $\pi$ is a closed embedding.
\end{thm}

\begin{proof}
Since closed embeddings are affine-local on the target, we may WLOG $Y = \Spec B$. Since $\pi$ is projective, its fiber dimension is upper-semicontinuous on $Y$, so $\{y\in Y: \dim \pi^{-1}(y) \ge 1\}$ is closed. If it is nonempty, then it contains a closed point, which contradicts with injectivity. So the fibers are finite type and dimension 0 over the Spec of a field, hence finite. So $\pi$ is projective with finite fibers, hence finite (Theorem \ref{proj+finitefiber=finite}). 

Now, for any \emph{closed} point $y\in Y$, we claim that the degree of $\pi$ at $y$ is at most 1. Suppose $\pi^{-1}(y)$ is nonempty, then it contains 1 point $x$ that is finite over $\Spec k$, so it has to be $\Spec A$, where $A$ is a finite $k$-algebra with one prime ideal $\fm$. Then $k$ must be the residue field. Suppose for contradiction that $\dim_{k} A\neq 1$, then $A_\fm \neq k$. But $A_\fm = \cO_{\pi^{-1}(y), x} = \cO_{X,x}\otimes_{\cO_{Y,y}} k$, so $\fm_y\cO_{X,x}\neq \fm_x$. So $\fm_y/\fm_y^2 \to \fm_x/\fm_x^2$ is not surjective as maps of $k$-vector spaces, which contradicts $\pi$ being injective on tangent vectors, i.e. $(\fm_x/\fm_x^2)^\vee \to (\fm_y/\fm_y^2)^\vee$ being injective. So we conclude that the degree of $\pi$ at closed points is at most 1. But since the degree of $\pi$ is upper-semicontinuous, its degree at all points is at most 1. Hence we are done by the previous lemma.
\end{proof}

\begin{lem}
Suppose $\cL$ is a degree $2g-2$ line bundle, then $h^0(C, \cL) = g-1$ or $g$, with $h^0(C, \cL) = g$ iff $\cL = \omega_C$.
\end{lem}

\begin{thm}
Let $k$ be algebraically closed. Suppose $\cL$ is a line bundle on a curve $C$, and let $g = h^1(X, \cO_X)$ be the arithmetic genus of $C$.
\begin{itemize}
\item If $\deg \cL \ge 2g$, then $\cL$ is basepoint-free.
\item If $\deg \cL \ge 2g+1$, then $\cL$ is very ample (in fact, any basis of $\Gamma(C, \cL)$ gives a closed embedding $C\into \PP_k^{\deg \cL - g}$).
\end{itemize}
\end{thm}


\subsection{Genus 0}

\begin{exm}
The curve $x^2+y^2+z^2 = 0$ in $\PP^2_\RR$ has genus 0, and is not isomorphic to $\PP^1_\RR$.
\end{exm}

\begin{prop}
Any genus 0 curve $C$ with a $k$-point is isomorphic to $\PP^1_k$.
\end{prop}

\begin{prop}
All genus 0 curves can be described as conics in $\PP^2_k$.
\end{prop}


\begin{prop}
\label{deg1LineBundleGlobalSections}
Suppose $C$ is a curve not isomorphic to $\PP^1_k$, then any degree 1 line bundle $\cL$ has $h^0(C, \cL) < 2$. As a consequence, for degree-1 points $p,q$ on $C$, $\cO(p) \cong \cO(q)$ iff $p = q$.
\end{prop}

\subsection{Hyperelliptic curves}

Assume $k$ algebraically closed with characteristic not 2. 

\begin{defn}
A genus $g$ curve $C$ is \emph{hyperelliptic} if it admits a double cover (i.e. degree 2 finite morphism) $\pi: C\to \PP^1_k$ (which we may as well fix).
\end{defn}

Then the preimage of any closed point consists of either 1 or 2 points.

\begin{thm}[hyperelliptic Riemann-Hurwitz]
Let $C$ be a hyperelliptic curve with double cover $\pi: C\to \PP_k^1$. Then $\pi$ has $2g+2$ branch points (closed points $p\in \PP_k^1$ where $\pi^{-1}(p)$ is a single point).
\end{thm}


\begin{prop}
Let $p_1,\dots,p_r$ be distinct closed points in $\PP^1_k$. If $r$ is even, then there is precisely one double cover branched at those points. If $r$ is odd, then there are none.
\end{prop}

\begin{proof}
Suppose 0 and $\infty$ are distinct from $p_1,\dots,p_r$. Then all branch points are in $\AA^1$. Any double cover $C'\to \AA^1$ gives rise to a quadratic field extension $K/k(x)$, which must be Galois. Find $y\in K$ such that the nontrivial element $\sigma$ in the Galois group maps $y \mapsto -y$. Then $y^2\in k(x)$, so we can replace $y$ by an appropriate $k(x)$-multiple such that $y^2$ is a polynomial, monic with no repeated factors, say $y^2 = f(x) = x^N + a_{N-1}x^{N-1} + \dots + a_0$. This is a curve $C'_0$ in $\AA^2$, and by the Jacobian criterion, this curve is regular. Thus $C'_0$ and $C'$ are both normalizations of $\AA^1$ in $k[x](y)$, hence isomorphic. Because the branch points are $p_1,\dots,p_r$, we conclude that $f(x) = (x-p_1)\dots(x-p_r)$.

In the projective situation, we simply do the same for $k[u]$, $u = x^{-1}$, which gives rise to the curve $C''$ defined by $z^2 = (u-\frac{1}{p_1})\dots(u - \frac{1}{p_r})$. So the double cover $C\to \PP^1$ has to be glued using $C'$ and $C''$. Thus, in $K(C)$, we must have $z^2 = u^rf(1/u) = f(x)/x^r = y^2/x^r$. If $r$ is even, then there is a unique way to glue, i.e. identifying $z = y/x^{r/2}$. If $r$ is odd, $x$ does not have a square root in $k(x)[y]/(y^2 - f(x))$, so there is no way to glue $C'$ and $C''$ together compatibly.
\end{proof}


\begin{proof}[Proof of hyperelliptic Riemann-Hurwitz]
We now have an explicit description of $\pi: C\to \PP^1_k$, in terms of covering it by two affine opens. Writing down the \v{C}ech complex then easily tells us that $g = h^1(C, \cO_C) = \frac{r}{2}-1$, as desired.
\end{proof}


\begin{prop}
Suppose $g\ge 2$. If $\cL$ corresponds to a hyperelliptic cover $C\to \PP^1$, then $\cL^{\otimes (g-1)} \cong \omega_C$.
\end{prop}

\begin{proof}
Compose the hyperelliptic map with the $(g-1)$-th Veronese embedding
\[C\to \PP^1 \to \PP^{g-1}\]
then the pullback of $\cO_{\PP^{g-1}}(1)$ along this composition is $\cL^{\otimes (g-1)}$. The pullback $H^0(\PP^{g-1}, \cO(1)) \to H^0(C, \cL^{\otimes (g-1)})$ is injective: if a hyperplane $s\in H^0(\PP^{g-1}, \cO(1))$ is pulled back to 0, then $s$ vanishes on all of the image of $C$, so the image of $C$ (a rational normal curve) is contained in a hyperplane, which is impossible. So $\cL^{\otimes (g-1)}$ is a degree $2g-2$ line bundle that has at least $g$ linearly independent sections, so it is equal to $\omega_C$.
\end{proof}

\begin{prop}
\label{hyperellipticLineBundle}
A curve $C$ of genus at least 1 is hyperelliptic iff it has a degree 2 line bundle $\cL$ with $h^0(C, \cL) = 2$.
\end{prop}

\begin{proof}
Suppose $\cL$ is a degree 2 line bundle with $h^0(C,\cL)\ge 2$. We claim $h^0(C, \cL) = 2$. Suppose otherwise. Consider a closed point $p$, and the exact sequence $0\to \cO(-p) \to \cO_C\to \cO|_p \to 0$. Tensoring with $\cL$ gives $0\to \cL(-p) \to \cL \to \cL|_p \to 0$. Writing down the long exact sequence gives $h^0(C, \cL(-p)) + 1 \ge h^0(C, \cL) \ge 3$, so $h^0(C, \cL(-p)) \ge 2$. But $\cL(-p)$ has degree 1, so this contradicts with Proposition \ref{deg1LineBundleGlobalSections}.  So $h^0(C,\cL) = 2$. Let $s_1,s_2$ be linearly independent sections, we claim that this is basepoint-free. Suppose $\div(s_1) = p + q_1$, $\div(s_2) = p+q_2$. Then $\cO(q_1) = \cL(-p) = \cO(q_2)$, which implies $q_1 = q_2$, so $s_1/s_2$ has no zeros and no poles and therefore constant, which contradicts them being linearly independent.

Now, return to the original problem. Suppose $C$ is hyperelliptic, then the pullback of $\cO_{\PP^1}(1)$ is a degree 2 line bundle with at least 2 sections, so by our discussion above it has exactly 2 sections. Conversely, suppose $\cL$ is a degree 2 bundle with 2 sections, then it is basepoint-free and thus gives a map to $\PP^1$, which has degree 2.
\end{proof}


\subsection{Genus 1: elliptic curves}

\subsection{Genus 2}

We claim that in this case all curves are hyperelliptic. Let $C$ be a curve of genus $g = 2$. Then $\omega_C$ has degree $2g-2 = 2$, and has 2 sections. By Proposition \ref{hyperellipticLineBundle}, it is basepoint-free and gives a double cover to $\PP^1$. Conversely, any double cover gives a degree 2 line bundle with 2 sections, which must be $\omega_C$.

\subsection{Genus 3}

\begin{prop}[canonical embedding]
Let $k$ be algebraically closed. Suppose $C$ is not hyperelliptic, then $\omega_C$ gives a closed embedding $C\into \PP^{g-1}$. 
\end{prop} 

\subsection{Genus 4}

\subsection{Genus 5}




\end{document}