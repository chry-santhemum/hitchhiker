\documentclass[11pt]{amsart}
\usepackage{scorpius}
\usepackage{multicol}

\title{\'Etale Cohomology}
\author{Atticus Wang}
\date{Winter 2022}

\begin{document}

\maketitle

\tableofcontents

\newpage


\section{Algebra}

\subsection{Flat modules}


\begin{prop}
    Let $A$ be a ring, $M$ an $A$-module. TFAE:
    \begin{enumerate}
        \item $M$ is flat;
        \item $\Tor_i^A(M,N) = 0$ for any $A$-module $N$ and $i\ge 1$;
        \item $\Tor_i^A(M,N) = 0$ for any finitely generated $A$-module $N$ and $i\ge 1$;
        \item $\Tor_1^A(M,N) = 0$ for any $A$-module $N$;
        \item $\Tor_1^A(M,N) = 0$ for any finitely generated $A$-module $N$;
        \item $\Tor_1^A(M,A/I) = 0$ for any ideal $I\subseteq A$;
        \item $\Tor_1^A(M,A/I) = 0$ for any finitely generated ideal $I\subseteq A$;
        \item For any ideal $I\subseteq A$, the map $I\otimes_A M \to M$ is injective;
        \item For any finitely generated ideal $I\subseteq A$, the map $I\otimes_A M \to M$ is injective.
    \end{enumerate}
\end{prop}

\begin{proof}
    Obviously:
    \[
    \begin{tikzcd}[column sep=small, row sep = small]
        (1) \arrow[r] & (2) \arrow[r] \arrow[d] & (4)  \arrow[d] & (6) \arrow[r] \arrow[d] & (8) \arrow[l] \\
        & (3) \arrow[r] & (5) \arrow[ru] & (7) \arrow[r] & (9). \arrow[l]
    \end{tikzcd}
    \]
    The remaining implications:

    $(4) \implies (1)$: For any short exact sequence $0\to Q\to P\to N\to 0$, we have the Tor long exact sequence
    \[\dots \to \Tor_1^A(M,N) \to M\otimes Q\to M\otimes P\to M\otimes N \to 0,\]
    so $0\to M\otimes Q\to M\otimes P\to M\otimes N \to 0$ is exact.

    $(5) \implies (4)$: Let $N$ be an $A$-module. We use the fact that $N = \varinjlim N'$, where $N'$ ranges among the finitely generated submodules of $N$, ordered by inclusion. This is a filtered colimit, which is exact (AB5) and commutes with left adjoints, such as tensor products. So for any short exact sequence
    $0 \to Q\to P\to M\to 0$ that ends with $M$, tensoring with $N$ is exact. Now we take $P$ to be free (therefore flat), so $\Tor_1^A(N,P) = 0$. Then the Tor exact sequence reads
    \[\dots \to 0\to \Tor_1^A(N,M) \to Q\otimes N \to P\otimes N \to M\otimes N \to 0,\]
    so $\Tor_1^A(N,M) \to Q$ is injective and its image is zero. So $\Tor_1^A(N,M) = 0$ as desired.

    $(6)\implies (5)$: Consider a finitely generated $N$. Then there exists a filtration
    \[0 = N_0 \subset N_1\subset\dots \subset N_n = N,\]
    where each $N_i/N_{i-1}$ is generated by one element, i.e. isomorphic as $A$-module to $A/I$ for some ideal $I$. Induct on $i$ and we wish to show $\Tor_1^A(M,N_i) = 0$. The base case $i = 1$ is clear. For the induction step, we have the exact sequence
    \[0\to N_{i-1} \to N_i \to N_i/N_{i-1} \to 0,\]
    and by the Tor long exact sequence, $\Tor_1^A(M,N_i) = 0$.

    $(7)\implies (6)$: Use the fact that $A/I = \varinjlim A/I'$ where $I'$ ranges among the finitely generated ideals contained in $I$, ordered by inclusion. Then we can mimic the argument in the implication $(5)\implies (4)$.
\end{proof}

\begin{prop}
    Let $M$ be flat, then tensoring with $M$ commutes with intersections. \qed
\end{prop}

\begin{prop}[flatness and localizations]
    Let $A$ be a ring, $S\subset A$ a multiplicative subset. Then:
    \begin{enumerate}[(i)]
        \item $S^{-1}A$ is flat over $A$;
        \item Let $M$ be flat over $A$, then $S^{-1}M$ is flat over $S^{-1}A$;
        \item Suppose $A\to B$ is a ring homomorphism that sends $S$ to a subset of a multiplicative subset $T\subseteq B$, and let $N$ be a $B$-module. If $N$ is flat over $A$, then $T^{-1}N$ is flat over $S^{-1}A$;
        \item Suppose $A\to B$ is a ring homomorphism, and $N$ is a $B$-module. If $N_{\fm}$ is flat over $A$ for every maximal ideal $\fm\subset B$, then $N$ is flat over $A$.
    \end{enumerate}
\end{prop}

\begin{proof}
    (iii) Notice that $T^{-1}N\otimes_A \bullet = T^{-1}(N\otimes_A \bullet)$ is an exact functor, so $T^{-1}N$ is flat over $A$. So $S^{-1}T^{-1}N = T^{-1}N$ is flat over $S^{-1}A$.

    (iv) Injectivity is a local property.
\end{proof} 


\begin{prop}[flatness and torsion-free]
    Let $A$ be a ring.
    \begin{enumerate}[(i)]
        \item If $a\in A$ is a non-zerodivisor, and $M$ is a flat $A$-module, then $M\to M$ given by $m\mapsto am$ is injective; in particular, if $A$ is a domain, then $M$ is torsion-free.
        \item Let $A$ be a Dedekind domain, then any torsion-free $A$-module is flat.
    \end{enumerate}
\end{prop}

\begin{proof}
    (i) Because $a$ is not a zero-divisor, the map $A\to A_a$ is injective, and so is $M\to M\times_A A_a$ since $M$ is flat. Suppose $am = 0$ for some $m\in M$, then $m\mapsto m\otimes 1 = am \otimes a^{-1} = 0$, so $m = 0$ as well by injectivity.

    (ii) Suppose $M$ is torsion-free. It suffices to show that $M_{\fm}$ is flat over $A_{\fm}$ for every maximal ideal $\fm\subset A$, i.e. we may assume that $A$ is a DVR. Let $I\subseteq A$ be any ideal, then $I$ is principal, say generated by $r$, and the map $A\to I$ given by $1\mapsto r$ is an isomorphism of $A$-modules. So $M\to I\otimes_A M$, $m\mapsto r\otimes m$ is an isomorphism. Composing this with the natural map $f: I\otimes_A M \to M$, $r\otimes m \mapsto rm$, gives us the map $M\to M$, $m\mapsto rm$, which is injective since $M$ is torsion-free. So $f$ is injective as well, which shows that $M$ is flat.
\end{proof}



\subsection{Faithfully flat modules}


\begin{prop}
    Let $A$ be a ring, $M$ an $A$-module. TFAE:
    \begin{enumerate}
        \item The functor $N\mapsto M\otimes N$ is exact and faithful;
        \item Any sequence $N'\to N\to N''$ is exact iff $M\otimes N' \to M\otimes N \to M\otimes N''$ is exact;
        \item $M$ is flat, and $M\otimes N = 0$ implies $N = 0$;
        \item $M$ is flat, and $M/\fm M\neq 0$ for any maximal ideal $\fm$ of $A$. \qed
    \end{enumerate}
\end{prop}

If any of the following holds, we say $M$ is \emph{faithfully flat} over $A$.


\begin{cor}
\label{cor1.2.2}
    Let $A\to B$ be a map of local rings that maps the maximal ideal of $A$ into the maximal ideal of $B$. Then if a nonzero, finitely generated $B$-module $M$ is flat over $A$, it is faithfully flat over $A$.
\end{cor}


\begin{prop}
\label{prop1.2.3}
    Let $A\to B$ be a map of rings. If there exists a $B$-module $M$ faithfully flat over $A$, then $\Spec B\to \Spec A$ is onto.
\end{prop}

\begin{proof}
    The fiber over $\fp\subset A$ is $\Spec B\otimes_A A_{\fp}/\fp A_{\fp}$. Since $M\otimes_A A_{\fp}/\fp A_{\fp}$ is faithfully flat over $A_{\fp}/\fp A_{\fp}$, it is a nonzero $B\otimes_A A_{\fp}/\fp A_{\fp}$-module, so $B\otimes_A A_{\fp}/\fp A_{\fp} \neq 0$.
\end{proof}



\begin{cor}
    Let $A\to B$ be a map of rings. Suppose there exists a finitely generated $B$-module $M$ faithfully flat over $A$, whose support is $\Spec B$. Then for any $\fp\in \Spec A$, if $\fq$ is minimal among those containing $\fp B$, then $\fq^c = \fp$.
\end{cor}


\begin{proof}
    By \ref{cor1.2.2}, $M_{\fq}$ is faithfully flat over $A_{\fq^c}$. By \ref{prop1.2.3}, $\Spec B_{\fq} \to \Spec A_{\fq^c}$ is onto. Then by minimality of $\fq$, $\fq B_{\fq}$ is the preimage of $\fp A_{\fq^c}$, so $\fp = \fq^c$ as desired.
\end{proof}


\begin{prop}
    Let $\phi:A\to B$ be a map of rings. TFAE:
    \begin{enumerate}
        \item $B$ is faithfully flat over $A$;
        \item $B$ is flat over $A$, and $\phi^*: \Spec B\to \Spec A$ is surjective;
        \item $B$ is flat over $A$, and for any maximal $\fm\subset A$, there exists a maximal $\fn\subset B$ with $\fm = \fn^c$;
        \item $B$ is flat, and for any $A$-module $M$, $M\to M\otimes_A B$ is injective;
        \item For any ideal $I$ of $A$, $I\otimes_A B \to B$ is injective, and $\phi^{-1}(IB) = I$.
        \item $\phi$ is injective and $\coker \phi$ is flat over $A$.
    \end{enumerate}
\end{prop}

\begin{proof}
It is clear that $(1)\implies (2)\implies (3)$.

$(3)\implies (1)$: It suffices to show that $B/\fm B \neq 0$ for any maximal $\fm$ of $A$. Pick $\fn\subset B$ such that $\fn^c = \fm$, then there is a surjection $B/\fm B \to B/\fn\neq 0$.

$(1)\implies (4)$: Since $B$ is faithfully flat, it suffices to show $M\otimes_A B \to M\otimes_A B \otimes_A B$ is injective. But this has a left inverse $M\otimes_A B \otimes_A B \to M\otimes_A B$ given by $m\otimes b_1\otimes b_2 \mapsto m\otimes b_1b_2$.

$(4)\implies (5)$: Since $A/I \to (A/I)\otimes_A B = B/IB$ is injective, $\phi^{-1}(IB) \subseteq I$, so $\phi^{-1}(IB) = I$.

$(5)\implies (3)$: We know $B$ is flat, and $\phi^{-1}(\fm B) = \fm$, so any maximal ideal $\fn$ containing $\fm B$ pulls back to $\fn^c = \fm$.

$(4)\implies (6)$: Putting $M = A$, we see that $\phi$ is injective. Let $M$ be any $A$-module. The long exact sequence reads
\[0\to \Tor_1^A(B,M) \to \Tor_1^A(\coker \phi, M)\to M\to M\otimes_A B. \tag{$\ast$}\]
Since $B$ is flat, $\Tor_1^A(B,M) = 0$. Since $M\to M\otimes_A B$ is injective, we conclude that
\[\Tor_1^A(\coker \phi,M) = 0,\]
which implies that $\coker\phi$ is flat.

$(6)\implies (4)$: This time $(\ast)$ tells us that $\Tor_1^A(B,M)=0$ and $M\to M\otimes_A B$ is injective.
\end{proof}


\begin{prop}[faithful flatness and completions]
Let $A$ be Noetherian, and let $I\subset A$ be an ideal. Then the $I$-adic completion $\wh{A}$ is flat over $A$, and it is faithfully flat iff $I\subseteq \rad(A)$. \qed
\end{prop}

\begin{prop}
    Let $A$ be a ring, $I\subset A$ an ideal, and $M$ an $A$-module. If either
    \begin{itemize}
        \item $I$ is nilpotent, or
        \item $A$ is Noetherian, $I\subset \rad(A)$, and $M$ is finitely generated,
    \end{itemize}
    then TFAE:
    \begin{enumerate}
        \item $M$ is free;
        \item $M/IM$ is free over $A/I$, and $\Tor_1^A(M,A/I) = 0$;
        \item $M/IM$ is free over $A/I$, and 
        \[(M/IM)\otimes_{A/I} \pr{\bigoplus_{n\ge 0} I^n/I^{n+1}} \to \bigoplus_{n\ge 0} I^nM/I^{n+1}M\]
        is an isomorphism.
    \end{enumerate}
\end{prop}

\begin{proof}
    
\end{proof}



\subsection{Local criteria for flatness}


\subsection{Additive categories}

\begin{defn}
    An \emph{additive category} is a category $\sC$ where:
    \begin{itemize}
        \item Finite products and coproducts exist;
        \item A zero object exists;
        \item For any objects $A,B\in \sC$, $\Hom(A,B)$ has the structure of an abelian group, and composition of morphisms is bilinear.
    \end{itemize}
\end{defn}

\begin{prop}
    In an additive category, $A\oplus B$ is isomorphic to $A\times B$. \qed
\end{prop}

\begin{defn}
    A functor $F:\sC\to \sC'$ between additive categories is an \emph{additive functor} if $F(u+v) = F(u) + F(v)$ for morphisms $u,v$.
\end{defn}

\begin{prop}
    Additive functors send the zero object to the zero object.
\end{prop}

\begin{proof}
    An object $A\in \sC$ in an additive category is the zero object if and only if $\id_A = 0_A$, and both are preserved by an additive functor.
\end{proof}

\begin{defn}[kernel, cokernel, image, coimage]
    Given $u:A\to B$ in an additive category $\sC$, the \emph{kernel} $\ker(u)$, if it exists, is an equivalence class of monomorphisms $\ker: \ker(u) \to A$, such that any $C\to A \to B$ is zero iff $C\to A$ factors through $\ker(u)\to A$. It is unique if it exists.

    Similarly, the \emph{cokernel} $\coker(u)$ can be defined, and is a quotient object of $B$.

    Finally, define the \emph{image} $\im(u) = \ker(\coker(u))$, and the \emph{coimage} $\coim(u) = \coker(\ker(u))$.
\end{defn}

\begin{prop}
    If both $\im(u)$ and $\coim(u)$ exist, then there is a natural morphism 
    \[\bar{u}: \coim(u) \to \im(u),\]
    such that $u:A\to B$ factors through
    $A\onto \coim(u) \xto{\bar{u}} \im(u) \into B$. \qed
\end{prop}


\subsection{Abelian categories}

\begin{defn}
    An \emph{abelian category} $\sC$ is an additive category that satisfies:
    \begin{itemize}
        \item (AB1) All kernels and cokernels exist;
        \item (AB2) For any $u:A\to B$, $\bar{u}:\coim(u) \to \im(u)$ is an isomorphism.
    \end{itemize}
\end{defn}

\begin{prop}
    For an object $A\in \sC$ in an abelian category, the set of subobjects of $A$ are in bijection with the set of quotient objects of $A$, given by:
    \[[u:B\into A] \longmapsto [A\onto \coker(u)],\]
    \[[v:A\onto B] \longmapsto [\ker(v) \into A].\]
    Further, the subobjects of $A$ form a lattice: for $A_1,A_2\into A$,
    \[A_1\cup A_2 = \im(A_1\oplus A_2 \to A),\]
    \[A_1\cap A_2 = \ker(A\to A/A_1 \times A/A_2),\]
    and similarly do the quotient objects of $A$. \qed
\end{prop}

\begin{prop}
    A map $u:A\to B$ is mono iff $\ker(u) = 0$, and $u$ is epi iff $\coker(u) = 0$. \qed
\end{prop}

\begin{prop}
    In an abelian category, mono and epi together implies isomorphism. \qed
\end{prop}


\begin{prop}
    The sequence $0\to A\to B\to C$ is exact iff 
    \[0\to \Hom(M,A) \to \Hom(M,B)\to \Hom(M,C)\] is exact for all $M$. \qed
\end{prop}

\begin{prop}
    Let $\sC$ be any category, $\sC'$ be an abelian category, then $Hom(\sC, \sC')$ is an abelian category (with exactness pointwise). \qed
\end{prop}

\begin{thm}[Freyd-Mitchell embedding theorem]
    Let $\sC$ be a small abelian category, then there exists a unital ring $R$ (not necessarily commutative) and a fully faithful exact functor $F: \sC\to R$-$\Mod$.
\end{thm}



\subsection{Injective objects}

\begin{defn}
    In an abelian category $\sC$, an object $M$ is \emph{injective} if the contravariant functor $A\mapsto \Hom(A,M)$ is exact. (It is automatically left exact; right exactness is the same as saying that for any subobject $A'\into A$, any morphism $A'\to M$ extends to $A\to M$.)
\end{defn}

\begin{defn}
    An abelian category $\sC$ has \emph{enough injectives} if for each $A\in \sC$, there exists a mono $A\into M$, where $M$ is injective.
\end{defn}


\subsection{Grothendieck categories}


\begin{defn}
    A collection of objects $\{Z_i\}_{i\in I}$ in $\sC$ is a \emph{family of generators} if for each $A\in \sC$, $B\into A$, $B\neq A$, there exists $i\in I$ and a morphism $Z_i\to A$ that does not factor through $B$.
\end{defn}

\begin{defn}
    A \emph{Grothendieck category} $\sC$ is an abelian category that has a family of generators, and satisfies the following two axioms:
    \begin{itemize}
        \item (AB3) Arbitrary coproducts exist.
        \item (AB5) Assume AB3, and filtered colimits of short exact sequences are exact. Equivalently, for a filtered family of subobjects $A_i\into A$, $\varinjlim A_i = \sum A_i$.
    \end{itemize}
\end{defn}


\begin{prop}
    Suppose $\sC$ is an abelian category that satisfies (AB3). TFAE:
    \begin{enumerate}
        \item $\{Z_i\}$ is a family of generators;
        \item $Z = \bigcup_i Z_i$ is a generator;
        \item For each $A\in \sC$, there exists an epi $\bigoplus Z \onto A$.
    \end{enumerate}
\end{prop}


\begin{thm}[Grothendieck]
    Let $\sC$ be a Grothendieck category, then $\sC$ has enough injectives.
\end{thm}

\begin{exm}
    $\Ab$, $R$-$\Mod$, $\Sh(X)$, $\QCoh(V)$, etc.
\end{exm}

\begin{prop}
    Let $\sC$ be any category, $\sC'$ be an abelian category.
    \begin{enumerate}[(i)]
        \item If $\sC'$ satisfies (AB5), then so does $Hom(\sC, \sC')$.
        \item If $\sC'$ satisfies (AB3) and has generators, then so does $Hom(\sC, \sC')$.
    \end{enumerate}
\end{prop}

\begin{proof}
    Item (i) is not hard to show pointwise. For (ii), given any object $Z\in \sC'$ and $A\in \sC$, define an object $Z_A\in Hom(\sC, \sC')$ by
    \[B \mapsto \bigcup_{\Hom(A,B)} Z,\]
    with the obvious morphisms. Then observe that $\Hom_{Hom(\sC, \sC')}(Z_A, F) \cong \Hom_{\sC'}(Z, F(A))$ naturally. From this, it is not hard to show that if $Z$ is a generator of $\sC'$, then the $Z_A$'s form a family of generators for $Hom(\sC, \sC')$.
\end{proof}


\subsection{Derived functors}

\begin{defn}[$\partial$-functors]
    Let $\sC$ be abelian, $\sC'$ additive. A (covariant) $\partial$-functor $\sC\to \sC'$ is:
    \begin{itemize}
        \item a system of additive functors $T^i: \sC\to \sC'$ ($i\ge 0$), and
        \item connecting morphisms $\delta: T^i(A'') \to T^{i+1}(A')$, for every $i\ge 0$ and each short exact $0\to A'\to A\to A''\to 0$ in $\sC$,
    \end{itemize}
    satisfying:
    \begin{itemize}
        \item Given a map of short exact sequences
        \[
        \begin{tikzcd}
            0 \arrow[r] & A' \arrow[r] \arrow[d] & A \arrow[r] \arrow[d] & A'' \arrow[r] \arrow[d] & 0 \\
            0 \arrow[r] & B' \arrow[r] & B \arrow[r] & B'' \arrow[r] & 0,
        \end{tikzcd}
        \]
        the diagram
        \[
        \begin{tikzcd}
            T^i(A'') \arrow[r, "\delta"] \arrow[d] & T^{i+1}(A') \arrow[d] \\
            T^i(B'') \arrow[r, "\delta"] & T^{i+1}(B')
        \end{tikzcd}
        \]
        commutes;
        \item Given an exact sequence $0\to A'\to A \to A''\to 0$, the sequence
        \[0 \to T^0(A') \to T^0(A) \to T^0(A'') \xto{\delta} T^1(A') \to\dots\]
        is a chain complex.
    \end{itemize}
    When $\sC'$ is abelian as well, the $\partial$-functor is called \emph{exact} if the above chain complex is exact.
\end{defn}

\begin{defn}
    A \emph{morphism} of two $\partial$-functors $T^i, T'^i$ is a system of natural transformations $f^i: T^i\to T'^i$ that commute naturally with $\partial$.
\end{defn}

\begin{defn}[universal]
    A $\partial$-functor $T = (T^i): \sC \to \sC'$ is \emph{universal} if for each $\partial$-functor $T' = (T'^i)$ and each natural transformation $f^0: T^0\to T'^0$, there is a unique extension to a morphism of $\partial$-functors $T\to T'$.
\end{defn}

\begin{defn}[effaceable]
    An additive covariant functor $F: \sC\to \sC'$ is \emph{effaceable} if for each object $A\in \sC$, there is a monomorphism $u:A\to M$ in $\sC$ such that $F(u) = 0$.
\end{defn}

\begin{prop}
    Let $\sC$ be an abelian category with enough injectives, then $F:\sC \to \sC'$ is effaceable iff $F(M) = 0$ for all injective $M$. \qed
\end{prop}


\begin{thm}
    Let $\sC, \sC'$ be abelian categories, and $T = (T^i): \sC\to \sC'$ be an exact $\partial$-functor. Then if each $T^i$ is effaceable for $i>0$, then $T$ is universal. If, in addition, $\sC$ has enough injectives, then the converse is also true.
\end{thm}

\begin{defn}[right derived functors]
    Let $F:\sC\to \sC'$ be a left exact additive covariant functor between abelian categories. Then its right derived functors $R^iF$ ($i\ge 0$) is the (unique) universal exact $\partial$-functor extending $F$.
\end{defn}

\begin{thm}
    When $\sC$ has enough injectives, right derived functors exist for every left exact additive covariant functor $F$.
\end{thm}

\begin{proof}
    For $A\in \sC$, consider an injective resolution
    \[0 \to A\to M^0 \to M^1\to M^2 \to\dots\]
    Then $R^i F(A)$ is defined as the $i$th cohomology of
    \[0 \to F(M^0) \to F(M^1) \to F(M^2) \to\dots.\]
    This is functorial and does not depend on the particular injective resolution chosen, because any two resolutions extending the same map are chain homotopic. Also, $R^iF(M) = 0$ for $i>0$ and $M$ injective, because of the injective resolution $0\to M\to M\to 0$, which shows that $(R^iF)$ is universal.
    
    It remains to check that this is exact. Given a short exact sequence $0\to A'\to A\to A''\to 0$, take injective resolutions $0\to A'\to M'^i$ and $0\to A'' \to M''^i$, then we can construct an injective resolution $0\to A \to M^i$, where $M^i = M'^i \oplus M''^i$, such that $0\to M'^i \to M^i\to M''^i\to 0$ is exact (horseshoe lemma). Applying $F$, each $0\to F(M'^i) \to F(M^i)\to F(M''^i)\to 0$ is then exact as well, which gives a desired long exact sequence.
\end{proof}


\subsection{Spectral sequences}

\begin{prop}[five-term exact sequence]
For a cohomological spectral sequence $E_2^{p,q} \implies E^{p+q}$, the sequence
\[0 \to E_2^{1,0} \to E^{1} \to E_2^{0,1} \to E_2^{2,0} \to E^{2}\]
is exact.
\end{prop}

\begin{thm}[Grothendieck spectral sequence]
    Let $\sC, \sC'$ be abelian categories with enough injectives, and $\sC''$ another abelian category. Let $F: \sC\to \sC'$, $G: \sC' \to \sC''$ be left exact covariant additive functors, and suppose $F$ maps injective objects to $G$-acyclic objects (ones for which $R^iG$ is zero for $i>0$). Then for each $A\in \sC$, there is a spectral sequence
    \[E_2^{p,q} = R^pG(R^qF(A)) \implies E^{p+q} = R^{p+q}(G\circ F)(A),\]
    and this is functorial in $A$.
\end{thm}

\begin{proof}
    (TODO)
\end{proof}



\subsection{Limits and colimits}

\begin{prop}
    If an abelian category satisfies (AB3), then it has arbitrary colimits, and colimit is right exact. \qed
\end{prop}


\begin{prop}
    Right (resp. left) adjoints commute with limits (resp. colimits). \qed
\end{prop}


\begin{defn}[pseudofiltered and filtered]
    A category $\sI$ is \emph{pseudofiltered} if it satisfies:
    \begin{itemize}
        \item (PS1) Each $i\to j$, $i\to j'$ can be extended to $j\to k$, $j'\to k$, such that the square commutes;
        \item (PS2) Each $f,g: i\to j$ can be extended to $h:j\to k$ such that $h\circ f = h\circ g$.
    \end{itemize}
    It is \emph{filtered} if for any two objects $j,j'$, there exists an object $k$ and morphisms $j\to k$, $j'\to k$.
\end{defn}


\begin{defn}
    A full subcategory $\sB$ of a category $\sA$ is \emph{final} if any object $A\in \sA$ has a morphism $A\to B$, where $B\in \sB$.
\end{defn}


\begin{prop}
    Let $F: \sI\to \sC$ be a functor where $\sI$ satisfies (PS1), and $\sJ$ be a final subcategory of $\sI$. Then the natural map
    \[\varinjlim F \to \varinjlim F|_{\sJ}\]
    is an isomorphism. In particular, if $\sI$ has a final object $\infty$, then $\varinjlim F \cong F(\infty)$.
\end{prop}


\subsection{Sites}


\begin{defn}[sites]
    A \emph{site} consists of:
    \begin{itemize}
        \item a category $\sC$;
        \item a collection $\cov(\sC)$ of \emph{coverings}, i.e. families of morphisms $\{U_i\to U\}_{i\in I}$,
    \end{itemize}
    satisfying:
    \begin{itemize}
        \item Given a covering $\{U_i\to U\}_{i\in I}$, and any morphism $V\to U$, the fiber products $U_i\times_U V$ exist and $\{U_i\times_U V\to V\}_{i\in I}$ is a covering as well;
        \item If $\{U_i\to U\}_{i\in I}$ and $\{U_{ij}\to U_i\}_{j\in J_i}$ are covering families, then so is $\{U_{ij}\to U\}_{i\in I, j\in J_i}$;
        \item Any isomorphism $\{V\to U\}$ is a covering.
    \end{itemize}
\end{defn}

\begin{defn}[morphisms of sites]
    A morphism of sites $(\sC, \cov(\sC)) \to (\sC', \cov(\sC'))$ is a functor $F$ of categories, such that:
    \begin{itemize}
        \item For any covering $\{U_i\to U\}$ in $\cov(\sC)$, $\{F(U_i)\to F(U)\} \in \cov(\sC')$;
        \item Given a covering $\{U_i\to U\}_{i\in I}$, and any morphism $V\to U$, the maps $f(U_i\times_U V) \to f(U_i) \times_{f(U)} f(V)$ are isomorphisms.
    \end{itemize}
\end{defn}




\begin{defn}[sheaves on sites]
    Let $\sD$ be a category that admits arbitrary products. A $\sD$-valued \emph{presheaf} on a site $(\sC, \cov(\sC))$ is a contravariant functor $F: \sC^{op}\to \sD$. It is a \emph{sheaf} if for every covering $\{U_i\to U\}$,
    \[0\to F(U)\to \prod_i F(U_i) \rightrightarrows \prod_{i,j} F(U_i\times_U U_j)\]
    is exact. (This makes sense when $\sD = \Set, \Ab, R$-$\Mod$, etc.)

    A morphism of (pre)sheaves is a natural transformation of functors.
\end{defn}


Fix a site $T = (\sC, \cov(\sC))$. The category of abelian presheaves on $T$ is denoted by $\sP$, and the category of abelian sheaves on $T$ is denoted by $\sS$, which is a full subcategory of $\sP$.



\begin{defn}[universal effective epi]
    Let $\sC$ be a category with fiber products. An epi $f:U\to V$ is an \emph{effective epimorphism} if for any $Z$, 
    \[0\to \Hom(V,Z)\to \Hom(U,Z) \rightrightarrows \Hom(U\times_V U, Z)\]
    is exact. It is an \emph{universal effective} epimorphism if any pullback is effective as well.

    More generally, a \emph{family of effective epimorphisms} is a family $\{U_i\to V\}$ such that for any $Z$, 
    \[0\to \Hom(V,Z)\to \prod_i \Hom(U_i, Z) \rightrightarrows \prod_{i,j}\Hom(U_i\times_V U_j, Z)\]
    is exact. It is a family of \emph{universal effective} epimorphisms if any pullback is effective as well.
\end{defn}

\begin{prop}
    Let $\{U_i\to U\},\{U_{ij}\to U_i\}$ be families of universal effective epimorphisms, then so is $\{U_{ij}\to U\}$. \qed
\end{prop}




\begin{defn}[canonical topology]
    Let $\sC$ be a category with fiber products. The \emph{canonical topology} is a site whose coverings are the families of universal effective epimorphisms. 
\end{defn}

\begin{prop}
    With the canonical topology, every representable presheaf of sets (ones of form $U\mapsto \Hom(U,Z)$) is a sheaf. Moreover, the canonical topology is the finest topology in which all representable presheaves of sets are sheaves. \qed
\end{prop}

\subsection{Canonical topology on the category of left $G$-sets}

Let $\sC$ be the category of left $G$-sets with $G$-maps as morphisms, and equip it with the canonical Grothendieck topology.


\begin{prop}
    A family $\{U_i\to U\}$ is in $\cov(\sC)$ iff the images of $U_i$ cover $U$. \qed
\end{prop}



\begin{prop}
\label{G_sets_equiv_sheaves_sets}
    The category of left $G$-sets is in equivalence with the category of sheaves of sets on $\sC$, where the equivalence is given by
    \begin{align*}
        S &\mapsto \Hom(\bullet, S) \\
        F(G) &\mapsfrom F
    \end{align*}
    where $F(G)$ is a left $G$-set by: for $x\in F(G)$, $gx$ is defined as the image of $x$ under the morphism $F(G)\to F(G)$ induced by the map $G\to G$, $h\mapsto hg$.
\end{prop}

\begin{proof}
    The key part is constructing isomorphisms
    \[F(H) \to \Hom_{\sC}(H, F(G))\]
    functorial in $H$. Consider the covering $\{\phi_h: G_h\to H\}$, where each $G_h$ is a copy of the $G$-set $G$, and $\phi_h$ maps $1_G$ to $h$. Since $F$ is a sheaf,
    \[0 \to F(H) \to \prod_{h\in H} F(G_h) \rightrightarrows \prod_{h_1,h_2\in H} F(G_{h_1}\times_H G_{h_2})\]
    is exact. It is not hard to verify that, for an element $(x_h)_h\in \prod F(G_h)$,
    \[(x_h)_h \in \im(F(H)\to \prod F(G_h)) \implies x_{gh} = gx_h \implies (x_h)_h\in \ker(\prod F(G_h) \to \prod F(G_{h_1}\times_H G_{h_2})),\]
    so all implications are reversible. This gives a natural isomorphism between $F(H)$ and its image in $\prod F(G_h)$, which is the set of $G$-maps $H\to F(G)$.
\end{proof}

\begin{cor}
    The category of left $G$-modules is in equivalence with the category of sheaves of abelian groups on $\sC$. \qed
\end{cor}



\subsection{Canonical topology on the category of continuous $G$-sets}

Let $G$ be a profinite group.

\begin{prop}
    The open normal subgroups $H$ of $G$ form a neighborhood basis of $1$, and $G\cong \varprojlim G/H$.
\end{prop}

A \emph{continuous $G$-set} is a $G$-set $U$ whose action $G\times U\to U$ is continuous ($U$ equipped with the discrete topology).

\begin{prop}
    TFAE:
    \begin{enumerate}
        \item $U$ is a continuous $G$-set;
        \item For every $u\in U$, $\operatorname{Stab}(u)$ is open;
        \item $U = \bigcup U^H$, where $H$ ranges among open normal subgroups of $G$.
    \end{enumerate}
\end{prop}

Consider the category $\sC$ of continuous $G$-sets and $G$-maps, with the canonical topology. As before:

\begin{prop}
    A family $\{U_i\to U\}$ is in $\cov(\sC)$ iff the images of $U_i$ cover $U$. \qed
\end{prop}

\begin{prop}
    The category of continuous $G$-sets is in equivalence with the category of sheafs of sets on $\sC$, where the equivalence is given by
    \begin{align*}
        U &\mapsto \Hom(\bullet, U) \\
        \varinjlim F(G/H) &\mapsfrom F
    \end{align*}
    where $\varinjlim F(G/H)$ is a continuous $G$-set as usual.
\end{prop}


\begin{proof}
We will repeatedly use the argument in Proposition \ref{G_sets_equiv_sheaves_sets}. The nontrivial part is to give a natural isomorphism $F(U)\cong \Hom_G(U, \varinjlim F(G/H))$.

First, using the covering $\{U^H \to U\}$, we may identify $F(U)\cong \varprojlim F(U^H)$.

Next, fix an open normal subgroup $H$. Using the covering $\{G/H\to U^H\}$ sending 1 to each element in $U^H$, we may identify
\[F(U^H) \cong \Hom_{G/H}(U^H, F(G/H)\}.\]
Next, we wish to show that 
\[\Hom_{G/H}(U^H, F(G/H) \cong \Hom_G(U^H, \varinjlim_{H'\subseteq H} F(G/H')). \tag{$\ast$}\] 
This is because, given a fixed $H'\subset H$, $\{G/H'\to G/H\}$ is a covering, so $F(G/H)$ is identified with $F(G/H')^{H/H'}$, so the map $F(G/H) \to \varinjlim F(G/H')$ identifies $F(G/H)$ with $\varinjlim F(G/H')^H$, which proves ($\ast$). Putting everything together:
\begin{align*}
    F(U) &\cong \varprojlim F(U^H) \\
    &\cong \varprojlim \Hom_{G/H}(U^H, F(G/H)) \\
    &\cong \varprojlim \Hom_{G}(U^H, \varinjlim F(G/H')) \\
    &\cong \Hom_G(\varinjlim U^H, \varinjlim F(G/H')) \\
    &\cong \Hom_G(U, \varinjlim F(G/H)),
\end{align*}
as desired.
\end{proof}



\begin{cor}
The category of continuous $G$-modules is in equivalence with the category of sheafs of abelian groups on $\sC$.
\end{cor}





\subsection{\v{C}ech cohomology}


Let $T$ be a site, $\sP$ the abelian category of presheaves of abelian groups on $T$. It satisfies (AB5) and has generators, so it has enough injectives, so right derived functors exist for every left exact covariant additive functor $F:\sP\to \Ab$. Also, exactness is verified pointwise.

\begin{prop}
    All colimits exist in $\sP$ and are constructed pointwise. Colimits are additive and right exact, and are exact if they are pseudofiltered (AB5).
\end{prop}

\begin{defn}
    Let $\{U_i\to U\}$ be a covering. Define a functor
    \begin{align*}
        H^0(\{U_i\to U\}, \bullet): \sP&\to \Ab \\
        F &\mapsto \ker(\prod F(U_i) \rightrightarrows \prod F(U_i\times_U U_j)).
    \end{align*}
    Then it is left exact and additive, so we may define $R^qH^0(\{U_i\to U\}, \bullet) =: H^q(\{U_i\to U\}, \bullet)$, the \emph{$q$-th \v{C}ech cohomology group associated to $\{U_i\to U\}$ with values in $F$}.
\end{defn}


\begin{thm}
    Let $C^{\bullet}(\{U_i\to U\}, F)$ be the \v{C}ech cochain, then its $q$-th cohomology group can be canonically identified with $H^q(\{U_i\to U\}, F)$.
\end{thm}


\begin{proof}
    It is sufficient to show that the $q$-th cohomologies $\wt{H}^q(\{U_i\to U\}, F)$ of $C^{\bullet}(\{U_i\to U\}, F)$ form a \emph{universal} $\partial$-functor extending $H^0 = \wt{H}^0$, which in turn follows from each $\wt{H}^q$ being \emph{effaceable}, for $q\ge 1$, i.e. kills all injective objects. Let $F$ be an injective sheaf. Let $Z_U: V\mapsto \bigcup_{\Hom(V, U)} \ZZ$ be the generators of $\sP$, which satisfy $\Hom(Z_U, F) \cong \Hom(\ZZ, F(U)) \cong F(U)$. Then
    \[C^q(\{U_i\to U\}, F) \cong \Hom(\bigoplus_{i_0,\dots,i_q} Z_{U_{i_0}\times_U \dots \times_U U_{i_q}}, F).\]
    Since $F$ is injective, it suffices to show that the complex
    \[\dots \to \bigoplus_{i,j} Z_{U_i \times_U U_j}(V) \to \bigoplus_i Z_{U_i}(V) \to 0 \tag{$\ast$}\]
    is exact, for all $V$. Fix an arbitrary map $\phi: V\to U$, we denote $S = \coprod_i \Hom_{\phi}(V, U_i)$, where $\Hom_{\phi}(V, U_i)$ consists of morphisms that commute with $\phi$ and $U_i\to U$. Then to show ($\ast$) is exact, it suffices to show that
    \[\dots \to \bigoplus_{S\times S} \ZZ \to \bigoplus_{S} \ZZ\to 0\]
    is exact. But the identity on this chain complex is null-homotopic, so its homology groups are all zero, i.e. is exact.
\end{proof}



\begin{defn}
    A \emph{refinement map} of coverings $\{U_j'\to U\}_{j\in J} \to \{U_i\to U\}_{i\in I}$ consists of a map $\eps: J\to I$ of index sets, and $U$-morphisms $f_j:U_j' \to U_{\eps(j)}$.
\end{defn}

Each refinement map induces a map of \v{C}ech cohomology groups in the opposite direction, which is $\partial$-functorial. Thus we may define:

\begin{defn}[\v{C}ech cohomology]
    Let $U$ be an object, $F\in \sP$ an abelian presheaf. Then the \emph{$q$-th \v{C}ech cohomology of $U$ with values in $F$} is defined as
    \[\check{H}^q(U, F) = \varinjlim_{\{U_i\to U\}} H^q(\{U_i\to U\}, F).\]
\end{defn}

\begin{thm}
    The functor $F\mapsto \check{H}^0(U, F)$ is left exact and additive, and its right derived functors are $\check{H}^q(U, F)$.
\end{thm}

\begin{proof}
    It is sufficient to show that $\varinjlim$ takes exact sequences of functors of form $H^q(\bullet, F)$ to exact sequences in $\Ab$. To do this, we first prove the following lemma:

    \begin{lem}
    \label{unique_homology_from_refinement}
    Let $(f,\eps),(g,\eta)$ be two refinement maps $\{U_j'\to U\} \to \{U_i\to U\}$, then they induce the same maps
    \[H^q(\{U_i\to U\}, F) \to H^q(\{U_j'\to U\}, F).\]
    \end{lem}

    \begin{proof}
        Let $h: \prod F(U_{i_0}\times_U U_{i_1}) \to \prod F(U_j)$ be the ``homotopy'' map induced by maps $U_j\to U_{\eps(j)}\times_U U_{\eta(j)}$. Then the maps $f^0,g^0: \prod F(U_i) \to F(U_j')$ satisfy $f_0 - g_0 = h\circ d$, so they induce the same map in the zeroth cohomology, so they induce the same map in all cohomologies by universality.
    \end{proof}

Back to the theorem: the lemma tells us that instead of taking the colimit across the category of coverings with all refinement maps as morphisms, we may as well consider the poset of all coverings, ignoring the different refinement maps. This is now a filtered category: given coverings $\{U_i\to U\}, \{U_j'\to U\}$, by the axioms of a site, $\{U_i\times_U U_j' \to U\}$ is a covering as well. So taking the colimit is now exact and we are done.
\end{proof}



\subsection{Functors $f_p$ and $f^p$}

Let $f: T\to T'$ be a functor between the underlying categories of two sites. (It does not have to be a morphism of sites, aka a \emph{continuous functor}.) Let $\sP, \sP'$ be the categories of abelian presheaves on $F, F'$.

\begin{defn}
    Given an abelian presheaf $F'$ on $T'$, we may define an abelian presheaf $f^pF'$ on $T$ by $U\mapsto F'(f(U))$. This is an additive, exact functor $f^p: \sP'\to \sP$ that commutes with colimits.
\end{defn}

\begin{prop}
    The functor $f^p$ has a left adjoint $f_p: \sP\to \sP'$.
\end{prop}

\begin{proof}
    First, we define the presheaf $f_pF$. Let $U'\in T'$, then consider the category $\sI_{U'}$ of pairs $(U, \phi)$ where $U\in T$ and $\phi:U'\to f(U)$ is a morphism. Define
    \[f_pF(U') = \varinjlim F(U)\]
    where the colimit is taken across all $(U, \phi)$ as above. Let $\phi':U'\to V'$ be a morphism, there is an induced functor $\sI_{V'}\to \sI_{U'}$, hence a morphism $f_pF(V')\to f_pF(U')$.

    It remains to show that
    \[\Hom(f_pF, G') \cong \Hom(F, f^pG')\]
    functorially, which is routine.
\end{proof}

\begin{cor}
\label{map_injective_to_injective}
    If $f_p$ is exact, then $f^p$ maps injectives to injectives. \qed
\end{cor}

\begin{cor}
\label{represent_fp}
    If $F\in \sP$ is represented by $Z\in T$, i.e. $F(U) = \Hom(U, Z)$, then $f_pF$ is represented by $f(Z)$. \qed
\end{cor}

\begin{exm}
    Taking $T$ the site with only one object and one arrow, and $T'$ any site, let $i:T\to T'$ map the singular object to $U\in T'$. Then $\sP = \Ab$, and $i^p: \sP' \to \sP$ maps $F$ to $F(U)$. Conversely, given an abelian group $A$ and $V\in T'$, $i_pA(V) = \bigoplus_{\Hom(V,U)}(A)$. This is exact, so we conclude that if $F$ is an injective sheaf, then $F(V)$ is injective for all $V\in T'$. 
\end{exm}



\subsection{Sheafification}

Let $T$ be a site, $\sP$ be the category of abelian presheaves on $T$, and $\sS$ be the category of abelian sheaves. Let $i:\sS\to \sP$ be the embedding functor.

\begin{thm}
    The functor $i$ has a left adjoint, the sheafification functor $\sP\to \sS$.
\end{thm}

\begin{proof}
    Consider a functor $\nmid: \sP\to \sP$, sending
    \[F\mapsto F^{\nmid}: F^{\nmid}(U) = \check{H}^0(U, F).\]
    it is routine to verify that $F^{\nmid}$ is an abelian presheaf, $\nmid$ is indeed a functor, and there is a canonical morphism $F\to F^{\nmid}$. Now, observe that any morphism $F\to G$, where $G$ is a sheaf, factors uniquely as $F\to F^{\nmid}\to G$. Uniqueness can be seen by noting that if $F\to G$ is the zero map, then so are the induced maps $H^0(\{U_i\to U\}, F) \to H^0(\{U_i\to U\}, G) = G(U)$, and we can simply pass to the colimit.

    This finishes the proof of adjointness, provided that $F^{\nmid}$ is a sheaf. Unfortunately, this is not always true, but it is indeed true that $(F^{\nmid})^{\nmid} =: F^\sharp$ is a sheaf, which we prove in the next proposition. Intuitively, the correct functor should replace a global section with the collection of local sections that agree \emph{locally} on their overlaps, hence the need to sheafify in two steps.
\end{proof}



\begin{prop}
     A presheaf $F$ is \emph{separated} if $F(U)\to \prod F(U_i)$ is injective for each covering $\{U_i\to U\}$. Then:
     \begin{enumerate}[(i)]
         \item If $F$ is any presheaf, $F^{\nmid}$ is separated.
         \item If $F$ is separated, then $F^{\nmid}$ is a sheaf, and $F\to F^{\nmid}$ is an monomorphism.
     \end{enumerate}
\end{prop}


\begin{proof}
    Item (i) is routine. For (ii), we first show that $F\to F^{\nmid}$ is a monomorphism, i.e. $F(U) = H^0(\{U\to U\}, F)\to \check{H}^0(U, F)$ is injective. In fact, it suffices to show that for any refinement of coverings $\{V_j\to U\}\to \{U_i\to U\}$, $H^0(\{U_i\to U\}, F)\to H^0(\{V_j\to U\}, F)$ is injective. Say $s$ is in the kernel. Consider the covering $\{V_j\times_U U_i\to U\}$ and refinement maps $\{V_j\times_U U_i\to U\} \xto{pr_2} \{U_i\to U\}$, and $\{V_j\times_U U_i\to U\} \xto{pr_1} \{V_j\to U\} \to \{U_i\to U\}$. By \ref{unique_homology_from_refinement}, the two induce the same maps in $H^0$, so $s$ is mapped to 0 in $H^0(\{V_j\times_U U_i \to U\}, F)$. This map is given by the restriction of
    \[\prod F(U_i) \to \prod F(V_j\times_U U_i),\]
    which is injective since $F$ is separated.

    Now, we show that $F^{\nmid}$ is a sheaf. Suppose $s = (s_i)\in \prod_i F^{\nmid}(U_i)$ is in the kernel. Pick representing elements $s_i \in H^0(\{U_{ik}\to U_i\}, F)$. Then we have that the images of $s_i,s_j$ in $H^0(U_i\times_U U_j, F)$ agree, so they agree in some common refinement of $\{U_{ik}\times_U U_j\to U_i\times_U U_j\}$ and $\{U_{jl}\times_U U_i \to U_i\times_U U_j\}$. In fact, by the injectivity proven in the above paragraph, this means that they agree in any common refinement, such as $H^0(\{U_{ik}\times_U U_{jl} \to U_i\times_U U_j\}, F) \subseteq \prod_{k,l} F(U_{ik}\times_U U_{jl})$. Now, define the element $t\in H^0(\{U_{ik} \to U\}, F) \into F^{\nmid}(U)$ by $t_i = s_i \in \prod_k F(U_{ik})$, which lies in the kernel by the above reasoning. This shows that $F^{\nmid}$ is a sheaf.
\end{proof}

\begin{cor}
    An abelian presheaf $F$ is a sheaf iff for each covering $\{U_i\to U\}$, there is a refinement $\{U_j'\to U\}$ such that
    \[0 \to F(U) \to \prod F(U_j') \to \prod F(U_{j_1}' \times_U U_{j_2}') \tag{$\ast$}\]
    is exact.
\end{cor}

\begin{proof}
    The coverings which satisfy $(\ast)$ then forms a final subcategory of all coverings, so the colimit restricted to these coverings is the same as the colimit over all coverings. So $F\to F^{\nmid}$ is an isomorphism, so $F$ is a sheaf.
\end{proof}


\subsection{The category of abelian sheaves}

\begin{thm}
    The category $\sS$ of abelian sheaves on a site $T$ is a Grothendieck category, and therefore has enough injectives.
\end{thm}

\begin{proof}
    First, $\sS$ is an additive category as a full subcategory of $\sP$ that contains 0.

    Next, we construct the kernels and cokernels of a morphism $F\to G$. The kernel $K = K^\sharp$ is constructed pointwise and can be easily verified to be a sheaf. The cokernel $C^\sharp$ is defined to be the sheafification of the presheaf cokernel $C$, and this satisfies the universal property by the adjunction.

    The image $I^\sharp$ is defined similarly by sheafifying the presheaf image $I$. Since $0\to I\to G\to C \to 0$ is exact, so is $0\to I^\sharp \to G^\sharp \to C^\sharp$ (left exactness of $\check{H}^0$). So $I^\sharp = \ker(\coker(F^\sharp \to G^\sharp))$ in $\sS$. To show that this is isomorphic to the coimage $J^\sharp$, let $J$ be the presheaf coimage. Then $u:J\to I$ is an isomorphism. So $u^\sharp: J^\sharp \to I^\sharp$, which coincides with the natural map from coimage to image, is an isomorphism.

    Next, we show that $\sS$ satisfies (AB3). Let $F_i$ be a family of sheaves, and let $F$ be their pointwise, presheaf direct sum. Again by the adjunction, $F^\sharp$ is the sheaf direct sum.

    Next, we show that $\sS$ satisfies (AB5). Let $A_i\into B$ be a filtered family of subobjects, and we wish to show $\sum A_i = \varinjlim A_i$. Let $A = \sum A_i$ in $\sP$, then $A^\sharp = \sum A_i$ in $\sS$, since sheafification commutes with direct sums and images. In the AB5 category $\sP$, there is a unique extension $A\to B$. This induces a unique extension $A^\sharp \to B$, once again by the adjunction. This shows (AB5).

    Finally, we show that $\sS$ has a set of generators. In fact, since the presheaves $Z_U \in \sP$ generate $\sP$, given a monomorphism of sheaves $A\into B$, there exists $Z_U\to B$ that does not factor through $A$. Then the induced $Z_U^\sharp \to B$ does not factor through $A$ either. So $Z_U^\sharp$ is a set of generators.
\end{proof}


\begin{prop}
    The sheafification functor $\sharp: \sP \to \sS$ is exact.
\end{prop}

\begin{proof}
    As a left adjoint, it is clearly right exact. Also, $i\circ \sharp$ is left exact, thus so is $\sharp$.
\end{proof}


\subsection{Sheaf cohomology}

Let $T$ be a site, $\sS$ the abelian category of sheaves of abelian groups on $T$. It satisfies (AB5) and has generators, so it has enough injectives, so right derived functors exist for every left exact covariant additive functor $F:\sS\to \Ab$. In particular, consider the section functor $\Gamma_U: \sS\to \Ab$ given by $F\mapsto F(U)$, which is left exact.

\begin{defn}
    Define the $q$-th \emph{sheaf cohomology} 
    \[H^q(U, F):= R^q\Gamma_U(F).\]
\end{defn}

\begin{exm}[group cohomology]
    Let $G$ be a group, $A$ a left $G$-module, and $e$ the one-element left $G$-set. Then $\Gamma_e(\Hom_G(\bullet, A)) = \Hom_G(e, A) \cong A^G$. So
    \[H^q(e, \Hom_G(\bullet, A)) \cong H^q(G, A)\]
    is the usual group cohomology. Conversely, given any $G$-set $X$, we may write it as the disjoint union of $G$-orbits $X_i$, then $X_i\cong G/H_i$ as $G$-sets. Then
    \[H^q(X, \Hom_G(\bullet, A)) \cong \prod_i H^q(G/H_i, \Hom_G(\bullet, A)) \cong \prod_i H^q(H_i, A).\]
\end{exm}


\subsection{\v{C}ech-to-derived functor spectral sequence}

Let $T$ be a site, $\sP$ the category of abelian presheaves on $T$, and $\sS$ the category of abelian sheaves on $T$. The composition functor 
\[\sS \xto{i} \sP \xto{\check{H}^0(U, \bullet)} \Ab\] 
is equal to $\Gamma_U$. For $F\in \sS$, let $\cH^q(F) := R^qi(F)$ be the derived functors of $i$.


\begin{prop}
    For each $U\in T$, we have, canonically, 
    \[\cH^q(F)(U) = H^q(U, F).\]
\end{prop}

\begin{proof}
    Taking $q = 0$, we have $H^0(U, F) = F(U) = \cH^0(F)(U)$, so it is sufficient to show that $H^q(\bullet, F)$ (which are easily verified to be presheaves on $T$) form a universal $\partial$-functor $\sS\to \sP$. But both exactness and effaceability follow from the definition.
\end{proof}


\begin{prop}
\label{sheafification_vanish}
    For each abelian sheaf $F$, $\cH^q(F)^{\nmid} = 0$ for $q\ge 1$.
\end{prop}

\begin{proof}
    We know $\cH^q(F)^{\nmid} \to \cH^q(F)^\sharp$ is a monomorphism, so it suffices to show that $\cH^q(F)^\sharp = 0$ for $q\ge 1$. Apply the Grothendieck spectral sequence to the composition of functors $\id_{\sS} = \sharp \circ i$.
\end{proof}


\begin{thm}[\v{C}ech-to-derived functor spectral sequence]
Let $F$ be an abelian sheaf.
    \begin{enumerate}[(i)]
        \item For each covering $\{U_i\to U\}$, there is a spectral sequence
        \[E_2^{p,q} = H^p(\{U_i\to U\}, \cH^q(F)) \implies H^{p+q}(U, F);\]
        \item For each $U\in T$, there is a spectral sequence
        \[E_2^{p,q} = \check{H}^p(U, \cH^q(F)) \implies H^{p+q}(U, F).\]
    \end{enumerate}
\end{thm}

\begin{proof}
    To apply the Grothendieck spectral sequence, we have to show that injective sheaves are $G$-acyclic in the category of presheaves, where $G = H^0(\{U_i\to U\},\bullet)$ or $\check{H}^0(U,\bullet)$. Because $\sharp$ is exact, $i$ maps injectives to injectives (\ref{map_injective_to_injective}), which are $G$-acyclic for any additive left exact functor. 
\end{proof}


We get edge morphisms $H^p(\{U_i\to U\}, F)\to H^p(U, F)$ and $\check{H}^p(U,F)\to H^p(U,F)$.

\begin{cor}
\label{edge_morphism_is_iso}
    Let $\{U_i\to U\}$ be a covering, and $F$ an abelian sheaf such that 
    \[H^q(U_{i_0}\times_U \dots \times_U U_{i_r}, F) = 0\]
    for all $q\ge 1$. Then the edge morphisms $H^p(\{U_i\to U\}, F)\to H^p(U, F)$ are isomorphisms for all $p\ge 0$.
\end{cor}

\begin{proof}
    The given data implies that $H^p(\{U_i\to U\}, \cH^q(F)) = 0$ for all $p\ge 0, q\ge 1$, so the edge morphisms are isomorphisms.
\end{proof}

\begin{prop}
    The edge morphism $\check{H}^p(U,F)\to H^p(U,F)$ is isomorphic for $p=0,1$ and injective for $p=2$.
\end{prop}

\begin{proof}
    Using \ref{sheafification_vanish},  the five-term exact sequence rewrites as:
    \[0\to \check{H}^1(U, F) \to H^1(U,F) \to 0 \to \check{H}^2(U, F) \to H^2(U,F)\]
    which proves the proposition.
\end{proof}


\subsection{Flasque sheaves}

\begin{defn}
    An abelian sheaf $F$ on a site $T$ is \emph{flasque} (or \emph{flabby}) if for all $q\ge 1$ and all coverings $\{U_i\to U\}$, $H^q(\{U_i\to U\}, F) = 0$.
\end{defn}

\begin{prop}
    The following are true about flasque sheaves:
    \begin{enumerate}[(i)]
        \item Let $0\to F'\to F\to F''\to 0$ be exact in $\sS$. If $F'$ is flasque, then it is exact in $\sP$.
        \item Let $0\to F'\to F\to F''\to 0$ be exact in $\sS$. If $F',F$ are flasque, then so is $F''$.
        \item If $F\oplus G$ is flasque, so is $F$.
        \item Injective abelian sheaves are flasque. \qed
    \end{enumerate}
\end{prop}

\begin{cor}
    For an abelian sheaf $F$, TFAE:
    \begin{enumerate}
        \item $F$ is flasque;
        \item For all $q\ge 1$, $\cH^q(F) = 0$.
    \end{enumerate}
\end{cor}

\begin{proof}
    $(1)\implies (2)$: Let $0\to F\to M^0 \xto{f^0} M^1 \xto{f^1} M^2 \xto{f^2}\dots$ be an injective resolution in $\sS$, we wish to show it is exact in $\sP$. Split it into short exact sequences:
    \[0\to F\to M^0 \to \ker(f^1) \to 0\]
    \[0\to \ker(f^1) \to M^1 \to \ker(f^2) \to 0\]
    \[\dots\]
    Then by induction, each of $\ker(f^i)$ are flasque, and all these short exact sequences are exact in $\sP$ as well. Thus the long sequence is exact in $\sP$ too.

    $(2)\implies (1)$: By \ref{edge_morphism_is_iso}, the edge morphisms $H^q(\{U_i\to U\}, F)\to H^q(U, F)$ are isomorphisms.
\end{proof}

\begin{cor}
    Flasque resolutions can be used to compute sheaf cohomology.
\end{cor}

\begin{proof}
    The key is that flasque sheaves are \emph{$i$-acyclic}, by the previous corollary. So suppose we have an acyclic resolution $0\to F\to M^i$. This splits into short exact sequences $0\to K^i \to M^i \to K^{i+1}\to 0$, where $K^i = \ker(M^i\to M^{i+1})$. Its long exact sequence reads $0\to H^q(K^{i+1})\to H^{q+1}(K^i)\to 0$, since $M^i$ are acyclic. So by induction, $H^q(F) \cong H^q(K^0) \cong H^1(K^{q-1})$, which by the long exact sequence
    \[0\to K^{q-1}\to M^{q-1}\to K^{q}\to H^1(K^{q-1}) \to 0\]
    is equal to $K^q/\im(M^{q-1}\to K^q) \cong H^q(0\to M^i)$ in the category of presheaves.
\end{proof}

\begin{exm}
    Every abelian sheaf is flasque iff $i:\sS\to \sP$ is exact. This occurs, for example, when $T$ is the site of sets with the canonical topology.
\end{exm}


\subsection{Functors $f_s$ and $f^s$}


Let $f:T\to T'$ be a map of sites. Define $\sP, \sS, \sP', \sS'$. Define the two functors $f_s, f^s$ as:
\[f_s = \sharp' \circ f_p\circ i: \sS\to \sS',\]
\[f^s = \sharp \circ f^p \circ i': \sS'\to \sS.\]
It is clear that $f^s = f^p\circ i'$.

\begin{prop}
    $f_s$ is left adjoint to $f^s$. If, moreover, $f_s$ is exact, then $f^s$ maps injectives to injectives. \qed 
\end{prop}

\begin{exm}[direct and inverse image]
    Let $T,T'$ be the sites of open sets of two topological spaces $X,X'$, and let $\pi:X'\to X$ be a continuous map, which induces a map of sites $f:T\to T'$. Then $f^s$ is called the \emph{direct image} functor, and $f_s$ the \emph{inverse image} functor.
\end{exm}

\begin{exm}[$G$-sets]
\label{fsForG-sets}
    Let $T,T'$ be the canonical topologies on the category of left $G,G'$-sets. Let $\pi: G'\to G$ be a homomorphism of groups, which induces a map of sites $f:T\to T'$. Denote $\pi_\ast = f^s, \pi^\ast = f_s$. Identifying abelian sheaves with $G$-modules, we can write explicitly
    \[\pi_\ast(A') = \Hom_{G'}(G, A')\]
    as a $G$-module by $(ga)(h) = a(hg)$, and
    \[\pi^\ast(A) = A\]
    as a $G'$-module (cf. \ref{represent_fp}). In the case $G'\subseteq G$, the module $\pi_\ast(A') = \Hom_{G'}(G, A')$ is called the \emph{co-induced} $G$-module $\operatorname{CoInd}_{G'}^{G}A$. The adjunction then translates to half of Frobenius reciprocity.
\end{exm}


\begin{exm}[continuous $G$-sets]
    Let $G,G'$ be profintie groups, and $T,T'$ be the canonical topologies on the category of smooth left $G,G'$-sets. Let $\pi: G'\to G$ be a smooth homomorphism of groups, which induces a map of sites $f:T\to T'$. Denote $\pi_\ast = f^s, \pi^\ast = f_s$. Identifying abelian sheaves with continuous $G$-modules, we can write explicitly
    \[\pi_\ast(A') = \Hom_{G'}^{cts}(G,A') = \varinjlim \Hom_{G'}(G/H, A')\]
    and
    \[\pi^\ast(A) = A.\]
\end{exm}


\begin{prop}
\label{fs_exact}
    Suppose $T,T'$ have final objects and finite fiber products, and $f:T\to T'$ preserves them. Then $f_s$ is exact.
\end{prop}

\begin{proof}
    It is sufficient to show $f_p$ is left exact, i.e. given a fixed $U'\in T$, the functor $\sP\to \Ab$, $F\mapsto f_pF(U')$, is left exact. Let $\cI$ be the category of pairs $(U, \phi)$, where $\phi:U'\to f(U)$ is a morphism in $T'$. Then $f_pF(U') = \varinjlim_{(U,\phi)} F(U)$ taken over the category $\cI^{op}$, so it suffices to show that $\cI^{op}$ is pseudofiltered. In fact, it is filtered, and this follows from the assumptions on final objects and fiber products.
\end{proof}

Consequently, in all three examples above, $f^s$ maps injective objects to injective objects.



\subsection{The Leray spectral sequence}


Let $f:T\to T'$ be a map of sites, then $f^s: \sS'\to \sS$ is left exact, so right derived functors $R^qf^s$ exist.


\begin{prop}
\label{1.22.1}
    The following diagram commutes:
    \[
    \begin{tikzcd}
        \sS' \arrow[r, "R^qf^s"] \arrow[d, "\cH^q"] & \sS \\
        \sP' \arrow[r, "f^p"] & \sP \arrow[u, "\sharp"].
    \end{tikzcd}
    \]
    In other words, given an abelian sheaf $F'$ on $T'$, $R^qf^sF'$ is the sheafification of the presheaf $U\mapsto H^q(f(U), F')$ on $T$.
\end{prop}


\begin{proof}
    Applying the Grothendieck spectral sequence to $f^s = (\sharp\circ f^p)\circ (i')$, we have a spectral sequence
    \[E_2^{p,q} = R^p(\sharp\circ f^p)(\cH^q(F')) \implies R^{p+q}f^sF'.\]
    For $p > 0$, $E_2^{p,q} = 0$ since $\sharp\circ f^p$ is exact. So $(f^p\cH^q(F'))^\sharp = E_2^{0,q} = R^qf^sF'$.
\end{proof}


\begin{prop}
    The functor $f^s$ maps flasque objects in $\sS'$ to flasque objects in $\sS$. (Contrast this with the fact that if $f_s$ is exact, then $f^s$ maps injectives to injectives.) \qed
\end{prop}


\begin{cor}
    Let $T''\xto{g} T \xto{f} T'$ be maps of sites, then $f^s$ maps flabby sheaves to $g^s$-acyclic sheaves. \qed
\end{cor}

Consequently, we can apply the Grothendieck spectral sequence:

\begin{thm}[Leray spectral sequence]
    Let $T''\xto{g} T \xto{f} T'$ be maps of sites, and $F'$ an abelian sheaf on $T$. Then there is a spectral sequence
    \[E_2^{p,q} = R^pg^s(R^qf^sF') \implies R^{p+q}(fg)^s(F'). \tag{$\ast$}\]
    In particular, taking $T''$ to be the site with one object $\ast$ and one morphism, and let $g$ map $\ast$ to $U\in T$, the Leray spectral sequence reads
    \[E_2^{p,q} = H^p(U, R^qf^sF') \implies H^{p+q}(f(U), F'). \tag{$\ast\ast$}\]
\end{thm}

The edge morphisms read
\begin{align*}
E_2^{p,0} = H^p(U, f^sF') \to &H^p(f(U), F')\\
&H^q(f(U), F') \to R^qf^s(F')(U) = E_2^{0,q}
\end{align*}
The latter can be interpreted as the sheafification map in \ref{1.22.1}. 

\begin{exm}
    Let $\pi:G'\to G$ be a homomorphism of groups, $U = \{e\}$ the one-element $G$-set, and identify the category of $G'$-modules with the category of abelian sheaves on $T_{G'}$. Then given a $G'$-module $A'$, ($\ast\ast$)  reads
    \[E_2^{p,q} = H^p(G, R^q\pi_\ast (A')) \implies H^{p+q}(G', A').\]
    Here $\pi_\ast$ is the functor as in \ref{fsForG-sets}.
\end{exm}

\begin{exm}[Hochschild-Serre spectral sequence]
    Let $H \trianglelefteq G$ be a normal subgroup, and $\pi: G\to G/H$ the natural homomorphism. Then for each left $G$-module $A$,
    \[\pi_\ast A = \Hom_{G}(G/H, A) \cong A^H,\]
    so $R^q\pi_\ast(A) = H^q(H, A)$, where we identify $A$ with the abelian sheaf $\Hom_G(\bullet, A)$. Then the Leray spectral sequence in the previous example reads
    \[E_2^{p,q} = H^p(G/H, H^q(H, A)) \implies H^{p+q}(G, A).\]
    The edge morphisms $H^p(G/H, A^H)\to H^p(G,A)$ are called \emph{inflations}, and the edge morphisms $H^q(G,A)\to H^q(H,A)^{G/H}$ are called \emph{restrictions}. The five-term exact sequence reads
    \[0\to H^1(G/H, A^H) \to H^1(G,A) \to H^1(H,A)^{G/H}\to H^2(G/H, A^H)\to H^2(G,A),\]
    where the second-to-last map is also called the \emph{transgression}.
\end{exm}

\begin{exm}[Shapiro's lemma]
    Let $\pi:H\to G$ be an inclusion. Then $\pi_\ast$ is exact, so $E_2^{p,q} = 0$ for $q\ge 1$. Consequently, the edge morphism $H^p(G, \operatorname{CoInd}_H^G(A)) \to H^p(H, A)$ is an isomorphism.
\end{exm}

\begin{exm}[Tate cohomology]
(TODO: example 3.7.11 in Tamme)
\end{exm}



\subsection{Localization}

Let $T$ be a site, $Z\in T$ an object, then there is naturally a site $T/Z$ on the category of $Z$-objects. The map $i: T/Z\to T$ is then a map of sites.

\begin{lem}
\label{is_exact}
    The functor $i^s$ is exact.
\end{lem}

\begin{proof}
    We know from \ref{1.22.1} that $R^qi^sF = (i^p\cH^q(F))^\sharp$. From \ref{sheafification_vanish}, $\cH^q(F)^\sharp = 0$, so it suffices to show that $i^p$ commutes with $\nmid$, which is easy to check.
\end{proof}

\begin{cor}
\label{localization_same_cohomology}
    There are natural isomorphisms
    \[H^p(U\to Z, i^sF) \cong H^p(U, F)\]
    given any abelian sheaf $F$ on $T$, and any object $U\to Z$ in $T/Z$.
\end{cor}

\begin{proof}
    Applying the Leray spectral sequence, we get
    \[E_2^{p,q} = H^p(U\to Z, R^qi^sF) \implies H^{p+q}(U, F).\]
    But $R^qi^s = 0$ when $q\ge 1$, so the edge morphism $H^p(U\to Z, i^sF)\to H^p(U,F)$ is isomorphic.
\end{proof}

\begin{exm}
    Let $T_G$ be the canonical topology on left $G$-sets. Let $H\le G$ be a subgroup, then the left cosets $G/H$ is an object in $T$, and in fact the functor $T_G/(G/H) \to T_H$ given by $[A\xto{\phi} G/H]\mapsto \phi^{-1}(1_GH)$ is an equivalence of sites.
\end{exm}



\subsection{Comparison lemma}

\begin{thm}[comparison lemma]
\label{comparison_lem}
    Let $i:T'\to T$ be a map of sites, satisfying that:
    \begin{itemize}
        \item $i$ is fully faithful (and therefore $T'$ is equivalent to a full subcategory of $T$);
        \item A covering $\{U_i\to U\}$ of $T$, where all $U_i$ and $U$ are in $T'$, is a covering in $T'$;
        \item Each object $U\in T$ admits a covering $\{U_i\to U\}$, where $U_i\in T'$.
    \end{itemize}
    Then $i^s$ and $i_s$ are quasi-inverses.
\end{thm}

\begin{proof}
    We will show that the unit $\eta: \id_{\sS'} \to i^s\circ i_s$ and the counit $\eps: i_s\circ i^s\to \id_{\sS}$ are natural isomorphisms. (TODO)
\end{proof}


\begin{cor}
    Let $i:T'\to T$ be a map of sites, satisfying that:
    \begin{itemize}
        \item $i$ is fully faithful;
        \item Any covering $\{U_i\to U\}$ of $U\in T'$, where $U_i\in T$, admits a refinement $\{U_j'\to U\}$ where $U_j' \in T'$.
    \end{itemize}
    Then $\eta: \id_{S'}\to i^s\circ i_s$ is a natural isomorphism, and $i^s$ is exact.
\end{cor}

\begin{proof}
    The proof of exactness of $i^s$ is similar to \ref{is_exact}, since the second condition tells us that $i^p$ commutes with $\nmid$.
\end{proof}

\begin{cor}
\label{two_natural_maps}
    Let $i:T'\to T$ be a map of sites, satisfying the two conditions in the previous corollary. Let $U\in T'$ and $F,F'$ be abelian sheaves on $T,T'$, then we have natural isomorphisms
    \[H^p(T';U, i^s F) \to H^p(T;U, F)\]
    and
    \[H^p(T';U, F') \to H^p(T;U, i_sF').\]
\end{cor}

\begin{proof}
    The former comes from the Leray spectral sequence
    \[E_2^{p,q} = H^p(T'; U, R^qi^sF) \implies H^{p+q}(T; U, F).\]
    Since $i^s$ is exact, the edge morphisms $H^p(T'; U, i^sF)\to H^p(T;U,F)$ are isomorprhisms.

    The latter comes from the composite
    \[H^p(T'; U, F') \to H^p(T'; U, i^si_s F') \to H^p(T; U, i_sF')\]
    where the two maps are both isomorphisms by the previous corollary.
\end{proof}

\begin{exm}
    Let $G$ be a profinite group, $T_G$ the canonical topology on continuous $G$-sets, and $T_G'$ the canonical topology on \emph{finite} continuous $G$-sets. Then it is easy to see that $i: T_G'\to T_G$ satisfies the three conditions in the comparison lemma \ref{comparison_lem}: each continuous $G$-set $U$ can be covered by the orbits $Gu$ for $u\in U$, which are finite since the stabilizer of $u$ is open. 
\end{exm}


\subsection{Noetherian topology}

\begin{defn}
    Let $T$ be a site. An object $U$ is \emph{quasicompact} if for any cover $\{U_i\to U\}_{i\in I}$, there exists a finite subset $I'\subseteq I$ such that $\{U_i\to U\}_{i\in I'}$ is still a cover.

    We call $T$ \emph{Noetherian} if every object is quasicompact.
\end{defn}

\begin{exm}
    Let $X$ be a topological space, and $T$ the site of open sets. Then $X$ is a Noetherian space iff $T$ is Noetherian.
\end{exm}

Let $T$ be a site. Then we may define a site $T^f$ allowing only the finite coverings. Let $i:T^f\to T$ be the identity map. Clearly, $i^s$ is fully faithful.

\begin{prop}
    Let $T$ be Noetherian. Then the following are true:
    \begin{enumerate}[(i)]
        \item $i^s$ is an equivalence of categories.
        \item There are $\delta$-functorial isomorphisms $H^q(T^f; U, i^sF) \cong H^q(T; U, F)$ for any abelian sheaf $F$ on $T$.
        \item Flasque sheaves on $T$ can be checked on finite covers.
    \end{enumerate}
\end{prop}

Let $T$ be a site, and $F_i$ a family of abelian sheaves on $T$ indexed by some category $\cI$. There are natural morphisms
\[\varinjlim H^q(U, F_i) \to H^q(U, \varinjlim F_i) \tag{$\ast$}\]
which are not isomorphisms in general. However:

\begin{thm}
    If $T$ is Noetherian and $\cI$ is pseudofiltered, then the map $(\ast)$ is an isomorphism.
\end{thm}

\begin{proof}
    (TODO)
\end{proof}

For example, we obtain that $\varinjlim$ commutes with arbitrary direct sums.



\newpage

\section{Algebraic Geometry}


\subsection{Conventions}

All rings are Noetherian, all schemes are locally Noetherian.

\subsection{Chevalley's theorem}



\subsection{Normalization}

Let $X$ be an integral scheme, $K = K(X)$, $L/K$ an algebraic extension.

\begin{prop}
    There exists a normal scheme $\wt{X}$ with function field $L$ and a dominant morphism $f: \wt{X} \to X$, such that:
    \begin{itemize}
        \item For any affine open $U\subseteq X$, $\Gamma(f^{-1}(U), \cO_{X'})$ is the integral closure of $\Gamma(U,\cO_X)$ in $L$;
        \item (universal property) Any dominant map from a normal scheme $g:Y\to X$ factors uniquely through $f$.
    \end{itemize}
\end{prop}


\begin{proof}
    In the case $X = \Spec A$ is affine, define $\wt{X} = \Spec \wt{A}$, where $\wt{A}$ is the integral closure of $A$ in $L$. In general, for every affine open $U_i$, we have $f_i: \wt{U_i}\to U_i$. To glue these together, we show that $f_i^{-1}(U_i\cap U_j) \cong f_j^{-1}(U_i\cap U_j)$ for $i,j$. By the universal property, the map $f_i^{-1}(U_i\cap U_j) \to U_i\cap U_j \subseteq U_j$ factors uniquely through $\wt{U_j}$, and the image is a subset of $f_j^{-1}(U_i\cap U_j)$. By the universal property again, the map in the other way is the inverse of this map.    
\end{proof}

\begin{prop}[finiteness of integral closure]
    Suppose $A$ is a Noetherian domain, $K = \Frac A$, $L/K$ finite, $B$ the integral closure of $A$ in $L$. Then $B$ is a finitely generated $A$-module if either of the following holds:
    \begin{itemize}
        \item $A$ is integrally closed and $L/K$ is separable;
        \item $A$ is a finite type $k$-algebra. \qed
    \end{itemize}
\end{prop}

\subsection{Finite morphisms}

\begin{prop}[finite implies proper]
\label{finite_implies_proper}
    Any finite morphism $f:Y\to X$ is separated, finite type, and universally closed.
\end{prop}

\begin{proof}
    Properness is affine-local on the target, so assume $X, Y$ are affine. Then the first two requirements are obvious; the third follows from the going-up theorem and the fact that finite morphisms are stable under base change.
\end{proof}

\begin{prop}
    Let $f: X\to \Spec k$ be finite type, then TFAE:
    \begin{enumerate}
        \item $X=\Spec A$ where $A$ is Artinian;
        \item $X$ is finite and discrete;
        \item $X$ is discrete;
        \item $f$ is finite.
    \end{enumerate}
\end{prop}

\begin{proof}
    (1) $\implies$ (2): Artinian rings have finitely many prime ideals, and all primes are maximal.

    (3) $\implies$ (2): Discrete plus quasicompact implies finite. 
    
    (2) $\implies$ (4): For any affine open $\Spec A\subseteq X$, $A$ is Noetherian and has dimension 0, so $A$ is Artinian, so it is uniquely the product of Artinian local rings. So $X$ is affine ($\cO_X(X)$ is the product of the Artinian local rings corresponding to each point of $X$).

    (4) $\implies$ (1): Say $X = \Spec A$. Then $A$ is a finitely generated $k$-module, so $\dim A = \trdeg \Frac A = 0$, so $A$ is Artinian.
\end{proof}


\subsection{Quasifinite morphisms}

\begin{defn}
    A morphism of schemes $f:Y\to X$ is \emph{quasifinite} if it is finite-type and has finite fibers. An $A$-algebra $B$ is \emph{quasifinite} if it is finite-type and for all prime ideals $\fp\subset A$, $B\times_A \kappa(\fp)$ is a finite $\kappa(\fp)$-module (i.e. $\Spec B\to \Spec A$ is quasifinite).
\end{defn}

Since finite morphisms have finite fibers, finite implies quasifinite.

\begin{prop}
    Quasifiniteness is stable under composition and base change, and any immersion is quasifinite. \qed
\end{prop}

It follows that any open subscheme of a finite morphism is quasifinite. Conversely:



\begin{thm}[Zariski's main theorem, Grothendieck's form: EGA IV$_3$, Thm 8.12.6]
    Let $X$ be quasicompact, then any separated, quasifinite morphism $f:Y\to X$ factors into $Y\to Y'\to X$, where $Y\to Y'$ is an open embedding, and $Y'\to X$ is finite.
\end{thm}


\begin{cor}[proper and quasifinite implies finite]
    Let $X$ be quasicompact, then any proper, quasifinite morphism $f:Y\to X$ is finite.
\end{cor}

\begin{proof}
    Let $f = g\circ f'$ where $f':Y\to Y'$ is an open immersion and $g:Y'\to X$ is finite. Since $g$ is finite, it is separated (\ref{finite_implies_proper}), so $\Delta_g$ is a closed immersion, which is proper. So $\Delta_g$ and $f$ are both proper, so by cancellation theorem, $f'$ is proper as well, so its image is closed. Therefore, $f'$ is a closed immersion, hence finite as well, so $f$ is finite.
\end{proof}



\begin{exr}
    Let $f:Y\to X$ be separated and finite type, $X$ irreducible. If the fiber over the generic point $\eta\in X$ is finite, then there is a nonempty open $U\subseteq X$ such that $f$ is finite over $U$.
\end{exr}

\begin{proof}
    (TODO)
\end{proof}

\newpage


\section{\'Etale Cohomology}

\subsection{\'Etale morphisms}


\begin{defn}
    A locally finitely presented morphism $f:X\to Y$ is called \emph{\'etale} if one of the following equivalent conditions hold:
    \begin{itemize}
        \item $f$ is flat and unramified.
        \item For each affine scheme $Y'\to Y$ and each closed subscheme $Y_0'$ of $Y'$ defined by a nilpotent ideal, $\Hom_Y(Y',X) \to \Hom_Y(Y_0',X)$ is a bijection.
    \end{itemize}
\end{defn}

\begin{prop}
    The following facts are true about \'etale morphisms:
    \begin{itemize}
        \item Open immersions are \'etale;
        \item If $f:X\to Y$ and $g:Y\to Z$ are \'etale, then so is $g\circ f$;
        \item If $f:X\to X'$ and $g:Y\to Y'$ are \'etale over $S$, then so is $f\times_S g: X\times_S Y\to X'\times_S Y'$.
        \item If $g:Y\to Z$ and $g\circ f: X\to Z$ are \'etale, then so is $f$.
    \end{itemize}
\end{prop}

\begin{proof}
    (TODO)
\end{proof}

\subsection{The \'etale site}


\begin{defn}
    The \emph{(small) \'etale site} $X_{\et}$ of $X$ is defined by:
    \begin{itemize}
        \item underlying category: \'etale $X$-schemes
        \item coverings: surjective families.
    \end{itemize}
    Denote by $\wt{X}_{\et}$ the category of abelian sheaves on $X_{\et}$.
\end{defn}

There is a map of sites $\eps: X_{\operatorname{Zar}} \to X_{\et}$ from the Zariski site to the \'etale site. Applying the Leray spectral sequence, we get
\[E_2^{p,q} = H^p_{\operatorname{Zar}}(X, R^q\eps^s(F))\implies H^{p+q}_{\et}(X, F)\]
for each abelian sheaf $F$ on $X_{\et}$.


\subsection{Direct and inverse image functors}

Let $f:X\to Y$ be a morphism of schemes. Then this induces a map of sites $f_{\et}:Y_{\et} \to X_{\et}$. So we may define
\[f_\ast = (f_{\et})^s: \wt{X}_{\et} \to \wt{Y}_{\et}\]
\[f^\ast = (f_{\et})_s: \wt{Y}_{\et} \to \wt{X}_{\et}\]
which are called the direct image and inverse image, respectively. More explicitly:
\[(f_\ast F)(Y') = F(Y'\times_Y X)\]
\[(f^\ast G)(X') = \varinjlim_{(Y', \phi)} G(Y')\]
where the colimit ranges through all $X$-morphisms $\phi: X'\to Y'\times_Y X$, or equivalently, all $Y$-morphisms $X'\to Y'$. In fact, $f^\ast$ is exact by \ref{fs_exact}. So we conclude that:

\begin{prop} 
The following are true about $f_\ast$ and $f^\ast$:
\begin{enumerate}[(i)]
    \item $f^\ast$ is left adjoint to $f_\ast$;
    \item $f_\ast$ is left exact and maps injectives to injectives;
    \item $f^\ast$ is exact and commutes with colimits. \qed
\end{enumerate}
\end{prop}

Let $Y'\in Y_{\et}, F\in \wt{X}_{\et}$. The Leray spectral sequence reads:
\[E_2^{p,q} = H^p_{\et}(Y', R^qf_\ast(F)) \implies H^{p+q}(Y'\times_Y X, F).\]
As in \ref{two_natural_maps}, we obtain for $F \in \wt{X}_{\et}, G\in \wt{Y}_{\et}$, natural morphisms
\[H^p_{\et}(Y', f_\ast F)\to H^p_{\et}(Y'\times_Y X, F)\]
and
\[H^p_{\et}(Y', G)\to H^p_{\et}(Y'\times_Y X, f^\ast G)\]
obtained by composing $H^p_{\et}(Y',G)\to H^p_{\et}(Y', f_\ast f^\ast G)\to H^p(Y'\times_Y X, f^\ast G)$.

In general, let $f:X\to Y$, $g:Y\to Z$ be morphisms of schemes. Let $F\in \wt{X}_{\et}$, then we have
\[E_2^{p,q} = R^pg_\ast(R^qf_\ast(F)) \implies R^{p+q}(gf)_\ast (F).\]
The edge morphisms can be easily read.

\begin{defn}[base-change morphism]
Let
\[
\begin{tikzcd}
    X' \arrow[d, "v'"] \arrow[r, "f'"] & Y' \arrow[d, "v"] \\
    X \arrow[r, "f"] & Y
\end{tikzcd}
\]
be a commutative square of schemes. Let $F$ be an abelian sheaf on $X_{\et}$. We consider the composition
\begin{align*}
    R^pf_\ast(F) &\to R^pf_\ast(v'_\ast v'^\ast F) \\
    &\to R^p(fv')_\ast(v'^\ast F) = R^p(vf')_\ast(v'^\ast F) \\
    &\to v_\ast(R^pf'_\ast v'^\ast F),
\end{align*}
whose corresponding morphism under the adjunction is
\[v^\ast(R^pf_\ast(F)) \to R^pf'_\ast(v'^\ast(F)).\]
This is called the \emph{base change morphism}, which is functorial in $F$.
\end{defn}


\begin{defn}[restriction of a sheaf]
Let $f:X'\to X$ be \'etale. Then $X'_{\et}$ is naturally identified with $X_{\et}/X'$ as sites. Let $F\in \wt{X}_{\et}$. Define $F|_{X'} = f^\ast F$ as an abelian sheaf on $X'_{\et}$; this is the \emph{restriction} of $F$ on $X'_{\et}$. It is not hard to see that $F|_{X'}(U) = F(U)$ for $U\to X'$ \'etale.
\end{defn}

\begin{cor}[cf. \ref{localization_same_cohomology}]
    There are canonical isomorphisms
    \[H^q(X_{\et}; X', F) \cong H^q(X'_{\et}; X', F/X').\]
\end{cor}


\subsection{The restricted \'etale site}

\begin{defn}
    A morphism of schemes $f:X\to Y$ is \emph{finitely presented} if it is locally finitely presented and qcqs. Define the \emph{restricted \'etale site} $X_{\etfp}$ as the category of finitely presented \'etale $X$-schemes, together with surjective covers.
\end{defn}

\begin{prop}
    Let $X$ be quasicompact, then $X_{\etfp}$ is a Noetherian site.
\end{prop}

\begin{proof}
    Let $X'\to X$ be finitely presented and \'etale. Because $X$ is quasicompact, so is $X'$. Because \'etale morphisms are open, if $\{X_i\to X'\}$ cover $X'$, a finite subset cover $X$.
\end{proof}

There is an obvious map of sites $i: X_{\etfp}\to X_{\et}$. The functors $i^s, i_s$ are also denoted $\res, \ext$.


\begin{prop}
    If $X$ is quasi-separated, then $\res, \ext$ are quasi-inverses.
\end{prop}

\begin{proof}
    To apply the comparison lemma, it suffices to show that for any \'etale $X$-scheme $X'$, there exists a cover by finitely presented 
    \'etale $X$-schemes $X_i$.
    
    Let $f:X'\to X$ be the structure morphism. Let $x\in X'$ be a point, then there exists an affine open neighborhood $U =\Spec A$ of $f(x)$, whose preimage $f^{-1}(U)$ is covered by spectrums of finitely presented $A$-algebras. One of these, say $V \subseteq X'$, contains $x$. Then $f|_{V}: V\to X$ is finitely presented, because it is the composition $V\to U\subseteq X$ of a finitely presented morphism and a quasicompact (this uses $X$ quasiseparated) open immersion, which is also finitely presented.
\end{proof}

\begin{cor}
    Let $X$ be qcqs, then $H^q_{\et}(X,\bullet)$ commutes with pseudofiltered colimits, e.g. direct sums.
\end{cor}


\subsection{The case $X=\Spec k$}

The setup is as follows. Let $k$ be a field. Let $\bar{k}$ be the separable closure of $k$, so that $\bar{k}/k$ is Galois. Let $G = \Gal(\bar{k}/k)$. Let $X'$ be a $k$-scheme, and $X'(\bar{k})$ the set of $\bar{k}$-points of $X$, which corresponds to pairs $(x'\in X', \phi: \kappa(x')\into \bar{k})$. There is a natural $G$-action on $X'(\bar{k})$, and for an open subgroup $H\le G$, $X'(\bar{k})^H = X'(\bar{k}^H)$, where $\bar{k}^H/k$ is finite by infinite Galois theory. Furthermore, $X'(\bar{k})$ is a continuous $G$-set, since $X'(\bar{k}) = \bigcup_{H} X'(\bar{k}^H)$ (here $\kappa(x)/k$ is finite by nullstellensatz).


\begin{thm}
    The functor $X'\mapsto X'(\bar{k})$ is an equivalence of sites $(\Spec k)_{\et}$ and $T_G$ (with the canonical topology).
\end{thm}

\begin{proof}
    (TODO)
\end{proof}

\begin{cor}
\label{specket_equiv}
    There is an equivalence of categories between $\wt{(\Spec k)}_{\et}$ and the category of continuous $G$-modules, given by
    \[F\mapsto \varinjlim F(G/H) = \varinjlim F(\Spec k')\]
    as $k'$ ranges among the finite (normal) subextensions of $\bar{k}/k$.
\end{cor}

The RHS is also the stalk $F_P$ at the geometric $\bar{k}$-point $P = \Spec \bar{k}\to \Spec k$, cf. \ref{stalks}. 

\begin{cor}
    Let $F$ be an abelian sheaf on $\wt{(\Spec k)}_{\et}$, then there are $\partial$-functorial isomorphisms
    \[H^q_{\et}(\Spec k, F) \to H^q(G, \varinjlim F(\Spec k'))\]
    where the right side is Galois cohomology.
\end{cor}


\begin{cor}
\label{section_exact}
    Let $k$ be separably closed, then $F\mapsto F(\Spec k)$ is an equivalence of categories $\wh{(\Spec k)}_{\et}$ and $\Ab$, hence additive and exact.
\end{cor}



\subsection{Representable sheaves on $X_{\et}$}

\begin{prop}
    Let $F$ be a presheaf of sets on $X_{\et}$. Then to verify $F$ is a sheaf, it suffices to verify it for the following two types of coverings:
    \begin{itemize}
        \item $\{U_i'\to X'\}$, where each map is an open embedding (i.e. ``usual'' coverings by open sets)
        \item $\{Y'\to X'\}$, a single surjective morphism of affine schemes. \qed
    \end{itemize}
\end{prop}

\begin{thm}
    The coverings in $X_{\et}$ are families of universal effective epimorphisms in the category of $X$-schemes.
\end{thm}

The converse is false; in other words, the \'etale topology is coarser than the canonical topology. However, when $X = \Spec k$, the two topologies agree.

\begin{proof}
    The key part is to show that a surjective $X$-morphism of affine schemes is effective. This follows from a general result in faithfully flat descent theory. 
\end{proof}



\begin{cor}
    For each $X$-scheme $Z$, the functor $X'\mapsto \Hom_X(X',Z)$ is a sheaf of sets.
\end{cor}

\begin{prop}
    Let $f:Y\to X$ be a morphism of schemes. If $Z$ is an \'etale $X$-scheme, then
    \[f^\ast\Hom_X(\bullet, Z) \to \Hom_Y(\bullet, Z\times_X Y)\]
    is an isomorphism.
\end{prop}

\begin{defn}[group schemes]
    A \emph{group scheme} over a scheme $X$ is an $X$-scheme $G$, together with either of the following equivalent data:
    \begin{itemize}
        \item a contravariant functor $Z\mapsto \Hom_X(Z, G)$ from schemes over $X$ to $\Grp$;
        \item a triple of morphisms $\mu:G\times_X G\to G$, $e: X\to G$, and $i:G\to G$, satisfying associativity, identity, and inverse axioms.
    \end{itemize}
\end{defn}

For a (commutative) group scheme $G$ over $X$, let $G_X$ denote the sheaf on $X_{\et}$ represented by $G$, which is a sheaf of (abelian) groups.

\begin{exm}
    Some examples of group schemes:
    \begin{itemize}
        \item the additive group $\GG_a = \Spec \ZZ[t]\times_{\ZZ} X$, and the functor sends $X'\mapsto \cO_{X'}(X')$;
        \item the multiplicative group $\GG_m = \Spec \ZZ[t,t^{-1}]\times_{\ZZ} X$, and the functor sends $X'\mapsto \cO_{X'}(X')^{\times}$;
        \item the $n$-th roots of unity $\mu_n = \Spec \ZZ[t]/(t^n-1) \times_{\ZZ} X$, sending $X'\mapsto \{s\in \cO_{X'}(X'): s^n = 1\}$.
    \end{itemize}
\end{exm}

We have the following exact sequence of abelian sheaves:
\[0\to \mu_n \to \GG_m \xto{n} \GG_m\]
where the last map is raising to the $n$-th power.

\begin{exm}
    The constant sheaf $\ul{A}_X$, given an abelian group $A$, is defined as the sheafification of the abelian presheaf $X'\mapsto A$. Then one can verify that
    \[\ul{A}_X(X') = \Hom_X(X', \coprod_A X) = \Hom_{\Top}(X', A).\]
    When the connected components of $X'$ are open (e.g. $X'$ is locally Noetherian), this is the same as $\prod A$ over its connected components, but in general this is \href{https://math.stackexchange.com/questions/3178359/constant-sheaves-on-the-%C3%A9tale-site-of-a-scheme}{not true}. In addition, it is clear that $\Hom(\ul{A}_X, F) \cong \Hom(A, F(X))$.

    Consider the constant sheaf $\ul{\ZZ/(n)}_X$ on $X_{\et}$. The isomorphisms $\ul{\ZZ/(n)}\to \mu_n$ correspond to primitive roots of unities the global section of $X$. 
    
    When $n$ is invertible on $X$ (equivalently, $n$ is coprime to the characteristics of residue fields at every point), we have that $\mu_n$ and $\ul{\ZZ/(n)}$ are \emph{locally isomorphic}, i.e. for every $X'$, there exists a cover $\{X_i'\to X'\}$ in $X_{\et}$ where $\mu_n|_{X_i'} \cong \ul{\ZZ/(n)}|_{X_i'}$. In fact, given $X' = \Spec A$, consider $Y' = \Spec B$, $B = A[x]/(x^n - 1)$. Then $Y'\to X'$ is surjectove flat and unramified, hence an \'etale cover.
\end{exm}


\subsection{\'Etale cohomology of $(\GG_a)_X$}

\begin{prop}
    Let $M$ be a quasicoherent sheaf of $\cO_X$-modules on $X$. Then
    \[X'\mapsto \Gamma(X', M\otimes_{\cO_X} \cO_X')\]
    is an abelian sheaf on $X_{\et}$, denoted by $M_{\et}$. \qed
\end{prop}

The functor $M\mapsto M_{\et}$, from the category of quasicoherent $\cO_X$-modules to the category of abelian sheaves on $X_{\et}$, is additive and left exact (recall that $X'\to X$ is flat).

\begin{thm}
    Let $M$ be a quasicoherent $\cO_X$-module. The edge morphisms
    \[H^p_{\operatorname{Zar}}(X,M) \to H^p_{\et}(X, M_{\et})\]
    of the Leray spectral sequence
    \[E_2^{p,q} = H^p_{\operatorname{Zar}}(X,R^q\eps^s(M_{\et})) \implies H^{p+q}_{\et}(X,M_{\et})\]
    are isomorphisms.
\end{thm}

\begin{proof}
    As usual, it suffices to show that $R^q\eps^s(M_{\et}) = 0$ for $q\ge 1$, where $\eps: X_{\operatorname{Zar}} \to X_{\et}$.
    
    Assume first $X$ is affine. Let $T$ be the full subcategory of $X_{\et}$ consisting of affine schemes. By the comparison theorem, $H^q(X_{\et}; X, M_{\et}) \cong H^q(T; X, M_{\et})$. We claim that $M_{\et}$ is flasque on $T$, which would imply what we wanted. Since $T$ is Noetherian, flasque sheaves can be checked on finite covers, which can be further reduced to covers consisting of one single morphism $\{\Spec B\to \Spec A\}$. In this case, $M$ is an $A$-module, so the \v{C}ech complex goes
    \[0\to M\to M\otimes_A B \to M\otimes_A B\otimes_A B\to\dots\]
    which is exact since $A\to B$ is faithfully flat (the Amitsur complex).

    In general, let $X$ be any scheme. For any $X'\to X$ \'etale, 
    \[H^q(X_{\et}; X', M_{\et}) \cong H^q(X'_{\et}; X', M_{\et}|_{X'}) \cong H^q(X'_{\et}; X', (M|_{X'})_{\et}).\]
    So $(R^q\eps^sM_{\et})|_{X'} = R^q\eps^s(M|_{X'})_{\et}$. Taking $X'$ to be affine opens of $X$, we have shown that $R^q\eps^sM_{\et}$ is zero when restricted to all affine opens, so it is zero.
\end{proof}

\begin{cor}
    If $X$ is affine, then $H^p_{\et}(X, M_{\et}) = 0$ for $p\ge 1$. In particular, taking $M = \cO_X$ itself, we have $H^p_{\et}(X, (\GG_a)_X) = 0$.
\end{cor}



\subsection{The Artin-Schreier sequence}

Let $X$ be a scheme with prime characteristic $p$. This means the following equivalent things:
\begin{itemize}
    \item $\characteristic \Gamma(X, \cO_X) = p$;
    \item $\characteristic \Gamma(U, \cO_X) = p$ for every open $U\subseteq X$;
    \item $\characteristic \cO_{X,x} = p$ at every point $x\in X$;
    \item $X$ is an $\FF_p$-scheme.
\end{itemize}
Consider the constant sheaf $\ul{\ZZ/(p)}$ on $X_{\et}$. The unit global section gives a morphism of sheaves
\[\ul{\ZZ/(p)} \to \GG_a\]
which is easily verified to be injective. Let
\[F: \GG_a\to \GG_a\]
be the Frobenius.

\begin{thm}
    The sequence
    \[0\to \ul{\ZZ/(p)} \to \GG_a\xto{F-\id} \GG_a \to 0\]
    is exact, where $F-\id$ is the map $x\mapsto x^p-x$ on each $\GG_a(X') = \cO_{X'}(X')$. This is called the \emph{Artin-Schreier sequence} on $X$.
\end{thm}

\begin{proof}
    Exactness in the middle: suppose $s\in \cO_{X'}(X')$ such that $s^p = s$, then $X' = V(s^p-s) = V(s(s-1)\dots(s-p+1)) = \bigsqcup V(s-i)$, so each closed subscheme $V(s-i)$ is open as well. So we conclude that $s$ is in the image of the constant sheaf.

    Surjectivity: Consider $s\in \GG_a(X') = \cO_{X'}(X')$. It suffices to show that there is a cover $\{X_i'\to X'\}$ in $X_{\et}$, such that each $s_i = s|_{X_i'}\in \cO_{X_i'}(X_i')$ is of the form $t_i^p-t_i$ for $t_i\in \cO_{X_i'}(X_i')$. It suffices to show this for $X' = \Spec A$ affine. Let $Y' = \Spec B$, where $B = A[t]/(t^p-t-s)$. Since $B$ is free over $A$, $Y'\to X'$ is flat and surjective. It is unramified since $(t^p-t-s)' = -1$. This completes the proof.
\end{proof}

The Artin-Schreier sequence then gives the following long exact sequence:
\begin{align*}
    0 & \to H^0(X, \ul{\ZZ/(p)}) \to H^0(X, \cO_X) \to H^0(X, \cO_X) \\
    &\to H^1(X, \ul{\ZZ/(p)}) \to H^1(X, \cO_X) \to H^1(X, \cO_X) \\
    &\to H^2(X, \ul{\ZZ/(p)}) \to\dots
\end{align*}
from which we obtain:

\begin{cor}
    There is an exact sequence
    \[0\to \frac{H^0(X,\cO_X)}{(F-\id)H^0(X,\cO_X)} \to H^1(X, \ul{\ZZ/(p)}) \to H^1(X, \cO_X)^{F}\to 0.\]
\end{cor}

\begin{cor}
    When $X= \Spec A$ is affine of characteristic $p$, 
    \[H^q(X, \ul{\ZZ/(p)}) = \begin{cases}
    A/(F-\id)A & \text{if } q = 1;\\
    0 & \text{if } q\ge 2.
    \end{cases}\]
\end{cor}

\begin{cor}[see \href{https://math.stackexchange.com/questions/3917041/global-sections-of-integral-proper-k-scheme-is-finite-field-extension-of-k}{here}]
    Suppose $k$ is separably closed of characteristic $p$, and $X$ is a reduced, proper $k$-scheme. Then
    \[H^1(X, \ul{\ZZ/(p)}) \cong H^1(X,\cO_X)^F.\]
\end{cor}

\subsection{\'Etale cohomology of $(\GG_m)_X$}


\begin{thm}[Hilbert's Theorem 90]
    There is a canonical isomorphism
    \[H^1_{\et}(X, \GG_m) \cong \Pic(X),\]
    where $\Pic(X)$ is the Picard group of $X$, i.e. $H_{\operatorname{Zar}}^1(X, \cO_X^{\times})$.
\end{thm}

\begin{proof}
    Using the five-term exact sequence associated to $\eps:X_{\operatorname{Zar}}\to X_{\et}$, it suffices to show that $R^1\eps^s(\GG_m)_X = 0$. (TODO)
\end{proof}

We remark that by the usual Hilbert 90, $H^1_{\et}(\Spec k, \GG_m) = H^1(\Gal(k), (k^{\sep})^{\times}) = 0$. Another way to view this is that the above theorem tells us that $H^1_{\et}(\Spec k, \GG_m) \cong \Pic(\Spec k)$, which is trivial because $\Spec k$ is just a point.


\begin{defn}[Brauer group]
    The \emph{Brauer group} of a field $k$ is defined as 
    \[H^2_{\et}(\Spec k, \GG_m) = H^2(\Gal(k), (k^{\sep})^{\times}).\]
\end{defn}


\subsection{The Kummer sequence}

\begin{thm}
    Let $X$ be a scheme, and $n$ is invertible on $X$. Then there is an exact sequence
    \[0\to \mu_n \to \GG_m\xto{s\mapsto s^n} \GG_m \to 0,\]
    called the \emph{Kummer sequence} on $X$.
\end{thm}

\begin{proof}
    This is essentially the same as Artin-Schreier, except that we use the observation that $\Spec A[t]/(t^n - s) \to \Spec A$ is etale for any ring $A$ in which $n$ is invertible.
\end{proof}


Denote, for an abelian group $A$, $_nA = \ker(a\mapsto na)$ and $A_n = \coker(a\mapsto na)$. Then we obtain from the Kummer sequence that:

\begin{cor}
    There is an exact sequence
    \[0\to H^0(X, \cO_X^{\times})_n \to H^1(X, \mu_n) \to\thinspace_n\Pic(X)\to 0.\]
\end{cor}

\begin{cor}
    If $X = \Spec A$ for $A$ local in which $n$ is invertible,
    \[H^1(X,\mu_n) \cong A^{\times}/(A^{\times})^n.\]
\end{cor}

\begin{cor}
    Suppose $k$ is separably closed with characteristic coprime to $n$, and $X$ is a reduced, proper $k$-scheme. Then
    \[H^1(X, \mu_n) \cong \thinspace _n\Pic(X).\]
\end{cor}


\subsection{The sheaf of divisors on $X_{\et}$}

Let $X$ be a Noetherian scheme, so that it has finitely many irreducible components. Recall that $K = K(X)$, the ring of rational functions on $X$, is defined as the set of rational maps $X\to \AA^1_{\ZZ}$, which has a natural ring structure. \href{https://stacks.math.columbia.edu/tag/01RR}{In this case}, $K(X)$ is naturally isomorphic to the product $\prod \cO_{X,\eta}$ of stalks at the generic points of each irreducible component.

Let $j:\Spec K \to X$ be the natural map, which induces a natural map of abelian sheaves $(\GG_m)_X\to j_\ast (\GG_m)_{K}$. If $X$ has no embedded points (TODO: why necessary?), then $j$ is dominant, then so are the $\Spec K\times_X X'\to X'$ (\'etale implies open), so we conclude that $(\GG_m)_X\to j_\ast (\GG_m)_{K}$ is injective. Therefore, we may define a sheaf $\Div_X$ by the short exact sequence
\[0\to (\GG_m)_X\to j_\ast (\GG_m)_K\to \Div_X\to 0.\]
Intuitively, these are formal sums of codimension-1 subschemes modulo the principal ones.

Applying the long exact sequence associated to $\eps: X_{\operatorname{Zar}}\to X_{\et}$, we obtain
\[0\to \cO_X^{\times} \to \cK_X^{\times} \to \eps^s\Div_X\to R^1\eps^s(\GG_m)_X,\]
where $\cK_X$ is the sheaf of rational functions on $X$: it is the sheafification of the presheaf mapping each open $U\subseteq X$ to $S^{-1}\Gamma(U, \cO_X)$, where $S$ is the set of elements in $\Gamma(U, \cO_X)$ that are non-zerodivisors in all $\cO_{X,u}$, $u\in U$. Because $R^1\eps^s(\GG_m)_X = 0$, $\eps^s\Div_X$ is the usual sheaf of divisors in the Zariski topology.

If $f:X'\to X$ is also finite type, then $X'$ is Noetherian and has no embedded components as well. In this case, applying $f^s$, we get $\Div_X|_{X'} = \Div|_{X'}$.

By EGA IV, there is a canonical morphism
\[\Div_X\to \bigoplus_{x} (i_x)_\ast \ul{\ZZ}\]
where $x$ ranges among the points where the local rings have dimension 1. If $X$ is regular, then this is an isomorphism.

Since $X$ is qcqs, \'etale cohomology commutes with direct sums, so
\[H^1_{\et}(X, \Div_X) \cong \bigoplus_x H^1_{\et}(X, (i_x)_\ast \ul{\ZZ}).\]

\begin{lem}
    Let $X$ be a scheme, $x\in X$, $i_x: \Spec \kappa(x) \to X$, then
    \[H^1_{\et}(X, (i_x)_\ast \ul{A}) = 0,\]
    where $A$ is any torsion-free abelian group.
\end{lem}

\begin{proof}
    The group $H^1_{\et}(X, (i_x)_{\ast}\ul{A})$ injects into $H^1(\Spec \kappa(x), \ul{A}) = H^1(\Gal(\kappa(x)), A)$, which is the group of continuous maps $f:\Gal(\kappa(x))\to A$, which there is none: suppose $f^{-1}(1_A) = H$ is an open subgroup, then $[G:H]$ is finite, which contradicts with the fact that $A$ has no torsion.
\end{proof}

Consequently, $H^1_{\et}(X, \Div_X) = 0$ for $X$ a regular Noetherian scheme.

\begin{lem}
    Let $X$ be a scheme, $x\in X$, $i_x: \Spec \kappa(x) \to X$, then
    \[R^1(i_x)_\ast(\GG_m)_{\kappa(x)} = 0.\]
\end{lem}

\begin{proof}
    This is the sheafification of $H^1(X'\times_X \Spec \kappa(x), (\GG_m)_{\kappa(x)})$, which is zero by Hilbert 90.
\end{proof}

\begin{cor}
    Let $X$ be a regular Noetherian scheme. There is an injection
    \[H^2_{\et}(X, (\GG_m)_X)\into \prod \Br(K_i),\]
    where $K_i$ runs through the fields of rational functions on each irreducible component.
\end{cor}

\begin{proof}
    This is just the composition
    \[H^2_{\et}(X,(\GG_m)_X) \into H^2_{\et}(X, j_\ast(\GG_m)_K) \into \prod H^2(\Spec K_i, (\GG_m)_{K_i}) = \prod \Br(K_i),\]
    where the first injection is because of $H^1_{\et}(X,\Div_X) = 0$, and the second is because of the previous lemma.
\end{proof}

\begin{cor}
    Let $X$ be a regular algebraic curve over a separably closed field $k$, then 
    \[H^2(X, \GG_m) = 0.\]
\end{cor}

\begin{proof}
    By Tsen's theorem, $\Br(K_i)$ all vanish, therefore so does $H^2(X, \GG_m)$.
\end{proof}

\subsection{Stalks}
\label{stalks}

Let $P$ be a geometric point of $X$, i.e. a morphism $P = \Spec\Omega\to X$, where $\Omega$ is separably closed. By \ref{specket_equiv}, the category of abelian sheaves on $P_{\et}$ is equivalent to $\Ab$ via $F\mapsto F(P)$.

\begin{defn}
    The \emph{stalk} of an abelian sheaf $F$ on $X_{\et}$ at the geometric point $u:P\to X$ is $F_P := u^\ast F(P)$.
\end{defn}

\begin{exm}
    Let $F$ be represented by an \'etale group scheme $G$, then $u^\ast F$ is represented by $G\times_X P$. So $F_{X,P} = \Hom_P(P, G\times_X P) = \Hom_X(P, G)$, i.e. the stalk $F_{X,P}$ consists of all $\Omega$-points of $G$. 
\end{exm}

\begin{prop}
    The following are true about stalks:
    \begin{enumerate}[(i)]
        \item The functor $F\mapsto F_P$ from $\wt{X}_{\et}$ to $\Ab$ is exact.
        \item If $v:P'\to P$ is a morphism of geometric points of $X$, then
        $F_{P'}\cong F_P$.
        \item Let $f:X\to Y$ be a map of schemes, then for any abelian sheaf $F$ on $Y_{\et}$, $(f^\ast F)_P \cong F_P$.
    \end{enumerate}
\end{prop}

\begin{proof}
    (i) Both $u^\ast$ and $\Gamma_P$ are exact (\ref{section_exact}).
    Note that the isomorphism in (ii) is not canonical.
\end{proof}

There is a similar definition of stalks as a colimit: consider the category of ``\'etale neighborhoods of $P$'', which consists of pairs $(X', u')$, where $X'$ is \'etale over $X$, and $u':P\to X'$ is an $X$-morphism. In fact, by definition of the presheaf functor $f_p$, we see that $u^\ast F$ is the sheafification of the presheaf $(u_{\et})_pF$, which maps $P\mapsto \varinjlim_{(X', u')} F(X')$. Therefore, there is a canonical map
\[\varinjlim_{(X',u')} F(X') \to F_P. \tag{$\ast$}\]

\begin{prop}
    The above map $(\ast)$ is an isomorphism.
\end{prop}

\begin{proof}
    In fact, it is clear that any abelian presheaf on $P_{\et}$ is a sheaf.
\end{proof}

\begin{prop}
    More generally, let $G$ be any presheaf on $X_{\et}$, then
    \[\varinjlim_{(X',u')} G(X') \to G_P^\sharp\]
    is an isomorphism.
\end{prop}

\begin{proof}
    It suffices to show the following: let $f:T\to T'$ be a map of sites, and $G$ a presheaf on $T$. Then $(f_pG)^\sharp \cong f_s(G^\sharp)$. This follows from adjunction.
\end{proof}


\begin{prop}
    Mono, epi, and isomorphisms  between abelian sheaves on $X_{\et}$ can be checked at the level of stalks at each point. \qed
\end{prop}

\begin{cor}
    A global section $s\in F(X)$ is zero iff it is zero at each stalk.
\end{cor}

\begin{proof}
    A global section is the same as a map of sheaves $\ul{\ZZ}\to F$.
\end{proof}

\begin{defn}[support]
    Let $s\in F(X)$, then its \emph{support} is
    \[\Supp(s) = \{x\in X: s_{\bar{x}}\neq 0\}.\]
    This is Zariski-closed: suppose $s_{\bar{x}} = 0$, then there is an \'etale $X$-scheme $X'$, with $\phi:X'\to X$, such that $s|_{X'} = 0$. Then for any point $y\in \phi(X')$, which is a Zariski-open set, $s_{\bar{y}} = 0$.

    The \emph{support} of the sheaf $F$ is \emph{the closure of}
    \[\Supp(F) = \{x\in X: F_{\bar{x}} \neq 0\}.\]
\end{defn}


\subsection{Henselian rings}


\begin{defn}
    A local ring $(A,\fm, k)$ is \emph{Henselian} if it satisfies the following equivalent conditions:
    \begin{enumerate}
        \item Hensel's lemma holds for $A$: given a monic polynomial $f\in A[x]$ and its image $\bar{f}\in k[x]$, if $\bar{f} = \bar{g}\circ \bar{h}$ with coprime monic polynomials $\bar{g},\bar{h}$, then there exist $g,h\in A[x]$ such that $f = g\circ h$ and $g,h$ reduce to $\bar{g},\bar{h}\in k[x]$.

        \item For any finite $A$-algebra $B$, the canonical map $B\to \prod_{\fn} B_{\fn}$ is an isomorphism, where $\fn$ runs through all (finitely many) maximal ideals of $B$.

        \item[(2')] For any $A$-algebra $B$ of form $B = A[x]/(f)$, where $f$ is monic, the canonical map $B\to \prod_{\fn} B_{\fn}$ is an isomorphism.

        \item For any $A$-algebra $B$, if $B = C_{\fp}$ for some finitely generated $A$-algebra $C$ and $\fp$ above $\fm$, and if $B/\fm B$ is finite over $k$, then $B$ is finite over $A$.

        \item For any $A$-algebra $B$, if $B = C_{\fp}$ for an \'etale $A$-algebra $C$ and $\fp$ above $\fm$, and if $k = B/\fm B$, then $B = A$.

        \item[(4')] Let $S = \Spec A$, $s\in S$ the closed point of $S$. If $X$ is \'etale over $S$ and $x\in X$ is a $k$-rational point (meaning $\kappa(x) = k$) on the fiber of $s$, then there exists a (unique) section $u:S\to X$ with $u(s) = x$.
    \end{enumerate}
\end{defn}

\begin{exm}
    Complete Hausdorff local rings are Henselian.
\end{exm}

\begin{thm}
    Let $(A,\fm,k)$ be a Henselian local ring. The map $B \mapsto B/\fm B$ gives an equivalence between the category of finite \'etale algebras over $A$ and the category of finite \'etale algebras over $k$.
\end{thm}

\begin{defn}[Henselization]
    For any local ring $A$, there is a Henselian ring $A^h$ with a local ring map $A\to A^h$, such that any local ring map $A\to B$, with $B$ Henselian, factors through $A^h$. This is explicitly
    \[A^h = \varinjlim B\]
    where $B$ runs through all $C_{\fp}$ for $C/A$ \'etale, $\fp$ above $\fm$, and $B/\fp B = k$.
\end{defn}

\begin{prop}
    For any local ring $A$, $A^h$ is faithfully flat over $A$, $\fm A^h$ is its maximal ideal, and $A^h/\fm A^h \cong k$. Furthermore, $A$ is Noetherian if and only if $A^h$ is.
\end{prop}


\subsection{Strictly local rings}

\begin{defn}
    A local ring $(A,\fm, k)$ is \emph{strictly local} if it satisfies the following equivalent definitions:
    \begin{enumerate}
        \item $A$ is Henselian, and $k$ is separably closed.
        \item $A$ is Henselian, and all finite \'etale $A$-algebras are trivial (finite products of $A$).
        \item If $A\to B$ is a local homomorphism to the localization of some \'etale $A$-algebra, then it is an isomorphism.
        \item[(3')] If $X\to S = \Spec A$ is \'etale and $x$ is a point on the fiber of the closed point $s$, there exists a (unique) section $u:S\to X$ mapping $s\mapsto x$.
    \end{enumerate}
\end{defn}

\begin{defn}[strict Henselization]
    For any local ring $A$ and a fixed embedding $k\into \Omega$, there exists a strictly local ring $A^{hs}$ with a local ring map $A\to A^{hs}$, such that:
    \begin{itemize}
        \item $A^{hs}$ is faithfully flat over $A$;
        \item $\fm A^{hs}$ is its maximal ideal, and $A^{hs}/\fm A^{hs} \into \Omega$ is the separable closure of $k$ in $\Omega$;
        \item Any local ring map $A\to B$ with $B$ strictly local and $k(B)\into\Omega$ fixed factors through $A^{hs}$.
    \end{itemize}
    This is explicitly
    \[A^{hs} = \varinjlim B\]
    where $B$ runs through all $C_{\fp}$ for $C/A$ \'etale, $\fp$ above $\fm$, together with an embedding $k(B)\to \Omega$.
\end{defn}


\begin{exm}
    Let $X$ be a scheme, $P:\Spec \Omega\to X$ a geometric point. Then $\cO_{X,P}$ is the strict Henselization of $\cO_{X,x}$ with respect to $\kappa(x)\into\Omega$.
\end{exm}



\end{document}